\chapter{Mutable data structures}
\label{c:mutable}

The definition of a sum or product type may be annotated to allow physical 
(destructive) update on data structures of that type. This is the main 
feature of the {\em imperative programming} style. Writing values into memory locations is the fundamental mechanism of imperative languages 
such as Pascal. The Lisp language, while mostly functional, also provides
the dangerous functions {\tt rplaca} and {\tt rplacd} to
physically modify lists. Mutable structures are required to implement
many efficient algorithms. They are also very convenient to represent
the current state of a state machine.

\section{User-defined mutable data structures}

Assume we want to define a type {\tt person} as in the previous chapter.
Then, it seems natural to allow a person to change his/her age, job and the
city that person lives in, but {\em not} his/her name.
We can do this by annotating some labels in the type definition of
{\tt person} by the {\tt mutable}\ikwd{mutable@\verb`mutable`} keyword:
\begin{caml_example}
type person =
    {Name: string; mutable Age: int;
     mutable Job: string; mutable City: string};;
\end{caml_example}
We can build values of type {\tt person} in the very same way as before:
\begin{caml_example}
let jean =
    {Name="Jean"; Age=23; Job="Student"; City="Paris"};;
\end{caml_example}
But now, the value {\tt jean} may be physically modified in the fields
specified to be {\tt mutable} in the definition (and {\em only} in these
fields).

We can modify the field {\tt Field} of an expression \verb"<expr1>" in order
to assign it the value of \verb"<expr2>" by using the following construct:
\begin{verbatim}
<expr1>.Field <- <expr2>
\end{verbatim}
For example; if we want {\tt jean} to become one year older, we would write:
\begin{caml_example}
jean.Age <- jean.Age + 1;;
\end{caml_example}
Now, the value {\tt jean} has been modified into:
\begin{caml_example}
jean;;
\end{caml_example}
We may try to change the {\tt Name} of {\tt jean}, but we won't succeed: the
typecheker will not allow us to do that.
\begin{caml_example}
jean.Name <- "Paul";;
\end{caml_example}
It is of course possible to use such constructs in functions as in:
\begin{caml_example}
let get_older ({Age=n; _} as p) = p.Age <- n + 1;;
\end{caml_example}
\ikwd{as@\verb`as`}
In that example, we named {\tt n} the current {\tt Age} of the argument, but
we also named {\tt p} the argument. This is an {\em alias} pattern: it
saves us the bother of writing:
\begin{caml_example}
let get_older p =
    match p with {Age=n} -> p.Age <- n + 1;;
\end{caml_example}
Notice that in the two previous expressions, we did not specify all
fields of the record {\tt p}.
Other examples would be:
\begin{caml_example}
let move p new_city = p.City <- new_city
and change_job p j = p.Job <- j;;
change_job jean "Teacher"; move jean "Cannes";
get_older jean; jean;;
\end{caml_example}
We used the ``{\tt ;}'' character between the different changes we
imposed to {\tt jean}. This is the {\em sequencing} of evaluations: it
permits to evaluate successively several expressions, discarding the
result of each (except the last one). This construct becomes useful in
the presence of {\em side-effects} such as physical modifications and
input/output, since we want to explicitly specify the order in which
they are performed.

\section{The {\tt ref} type}

The {\tt ref} type is the predefined type of mutable indirection
cells.  It is present in the Caml system for reasons of compatibility
with earlier versions of Caml. The {\tt ref} type could be defined as
follows (we don't use the {\tt ref} name in the following definition
because we want to preserve the original {\tt ref} type):
\begin{caml_example}
type 'a reference = {mutable Ref: 'a};;
\end{caml_example}
Example of building a value of type {\tt ref}:
\begin{caml_example}
let r = ref (1+2);;
\end{caml_example}
The {\tt ref} identifier is syntactically presented as a sum data
constructor.  The definition of {\tt r} should be read as ``let {\tt
r} be a reference to the value of {\tt 1+2}''.  The value of {\tt r}
is nothing but a memory location whose contents can be overwritten.

We consult a reference (i.e. read its memory location) with the ``{\tt
!}'' symbol:
\begin{caml_example}
!r + 1;;
\end{caml_example}

We modify values of type {\tt ref} with the {\tt :=} infix function:
\begin{caml_example}
r:=!r+1;;
r;;
\end{caml_example}
Some primitives are attached to the {\tt ref} type, for example:
\begin{caml_example}
incr;;
decr;;
\end{caml_example}
which increments (resp. decrements) references on integers.

\section{Arrays}

Arrays are modifiable data structures. They belong to the
parameterized type \verb|'a vect|. Array expressions are bracketed by
\verb'[|' and \verb'|]', and elements are separated by semicolons:
\begin{caml_example}
let a = [| 10; 20; 30 |];;
\end{caml_example}
The length of an array is returned by with the function \verb|vect_length|:
\begin{caml_example}
vect_length a;;
\end{caml_example}


\subsection{Accessing array elements}

Accesses to array elements can be done using the following syntax:
\begin{caml_example}
a.(0);;
\end{caml_example}
or, more generally: $e_1${\tt .(}$e_2${\tt )},
where $e_1$ evaluates to an array and $e_2$ to an integer.
Alternatively, the function \verb|vect_item| is provided:
\begin{caml_example}
vect_item;;
\end{caml_example}
The first element of an array is at index 0. Arrays are useful because
accessing an element is done in constant time: an array is a
contiguous fragment of memory, while accessing list elements takes
linear time.

\subsection{Modifying array elements}

Modification of an array element is done with the construct:
\begin{center}
$e_1${\tt .(}$e_2${\tt )} \verb|<-| $e_3$
\end{center}
where $e_3$ has the same type as the elements of the array $e_1$. The
expression $e_2$ computes the index at which the modification will occur.

As for accessing, a function for modifying array elements is also
provided:
\begin{caml_example}
vect_assign;;
\end{caml_example}
For example:
\begin{caml_example}
a.(0) <- (a.(0)-1);;
a;;
vect_assign a 0 ((vect_item a 0) - 1);;
a;;
\end{caml_example}

\section{Loops: {\tt while} and {\tt for}}
\ikwd{while@\verb`while`}
\ikwd{for@\verb`for`}

Imperative programming (i.e. using side-effects such as physical
modification of data structures) traditionally makes use of sequences
and explicit loops. Sequencing evaluation in Caml Light is done by
using the semicolon ``\verb";"''. Evaluating expression $e_1$,
discarding the value returned, and then evaluating $e_2$ is written:
\begin{center}
$e_1$ {\tt ;} $e_2$
\end{center}
If $e_1$ and $e_2$ perform side-effects, this construct ensures
that they will be performed in the specified order (from left to
right). In order to emphasize sequential side-effects, instead of
using parentheses around sequences, one can use {\tt begin} and {\tt
end}, as in:\ikwd{begin@\verb`begin`}\ikwd{end@\verb`end`}
\begin{caml_example}
let x = ref 1 in
  begin
     x := !x + 1;
     x := !x * !x
  end;;
\end{caml_example}
The keywords {\tt begin} and {\tt end} are equivalent to opening and
closing parentheses. The program above could be written as:
\begin{caml_example}
let x = ref 1 in
  (x := !x + 1; x := !x * !x);;
\end{caml_example}

Explicit loops are not strictly necessary {\em per se}: a recursive
function could perform the same work. However, the usage of an
explicit loop locally emphasizes a more imperative style. Two loops
are provided:
\begin{itemize}
\item {\it while}: {\tt while} $e_1$ {\tt do} $e_2$ {\tt done}
evaluates $e_1$ which must return a boolean expression, if $e_1$
return {\tt true}, then $e_2$ (which is usually a sequence) is
evaluated, then $e_1$ is evaluated again and so on until $e_1$
returns {\tt false}.

\item{\it for}: two variants, increasing and decreasing
\begin{itemize}
\item {\tt for} $v$\verb|=|$e_1$ {\tt to} $e_2$ {\tt do} $e_3$ {\tt done}
\item {\tt for} $v$\verb|=|$e_1$ {\tt downto} $e_2$ {\tt do} $e_3$ {\tt done}
\end{itemize}
where $v$ is an identifier. Expressions $e_1$ and $e_2$ are the bounds
of the loop: they must evaluate to integers. In the case of the
increasing loop, the expressions $e_1$ and $e_2$ are evaluated
producing values $n_1$ and $n_2$ : if $n_1$ is strictly greater than
$n_2$, then nothing is done.  Otherwise, $e_3$ is evaluated $(n_2 -
n_1)+1$ times, with the variable $v$ bound successively to $n_1$, $n_1
+1$, \ldots, $n_2$.

The behavior of the decreasing loop is similar: if $n_1$ is strictly
smaller than $n_2$, then nothing is done. Otherwise, $e_3$ is
evaluated $(n_1 - n_2)+1$ times with $v$ bound to successive values
decreasing from $n_1$ to $n_2$.
\end{itemize}
Both loops return the value \verb|()| of type {\tt unit}.
\begin{caml_example}
for i=0 to (vect_length a) - 1 do a.(i) <- i done;;
a;;
\end{caml_example}

\section{Polymorphism and mutable data structures}

There are some restrictions concerning polymorphism and mutable data
structures.
One cannot enclose polymorphic objects inside mutable data structures.
\begin{caml_example}
let r = ref [];;
\end{caml_example}
The reason is that once the type of {\tt r}, {\tt ('a list) ref}, has
been computed, it cannot be changed. But the value of {\tt r} can be
changed: we could write:
\begin{verbatim}
r:=[1;2];;
\end{verbatim}
and now, {\tt r} would be a reference on a list of numbers while its type
would still be {\tt ('a list) ref}, allowing us to write:
\begin{verbatim}
r:= true::!r;;
\end{verbatim}
making {\tt r} a reference on {\tt [true; 1; 2]}, which is an illegal
Caml object.

Thus the Caml typechecker imposes that modifiable data structures appearing
at toplevel must possess monomorphic types (i.e. not polymorphic).



\section*{Exercises}

\begin{exo}\label{Annot:4}
Give a mutable data type defining the Lisp type of lists and define
the four functions {\tt car}, {\tt cdr}, {\tt rplaca} and {\tt rplacd}.
\end{exo}

\begin{exo}\label{Annot:5}
Define a \verb"stamp" function, of type \verb"unit -> int", that
returns a fresh integer each time it is called. That is, the first
call returns 1; the second call returns 2; and so on.
\end{exo}

\begin{exo}\label{Annot:6}
Define a \verb|quick_sort| function on arrays of floating point
numbers following the {\em quicksort} algorithm \cite{Quicksort}.
Information about the {\em quicksort} algorithm can be found
in~\cite{SedgewickPascal}, for example.
\end{exo}
