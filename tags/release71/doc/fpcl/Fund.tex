\chapter{Basic concepts}
\label{c:basicconcepts}

We examine in this chapter some fundamental concepts which we will
use and study in the following chapters. Some of them are specific to the
interface with the Caml language (toplevel, global definitions) while others
are applicable to any functional language.

\section{Toplevel loop}

The first contact with Caml is through its interactive
aspect\footnote{Caml Light implementations also possess a batch
compiler usable to compile files and produce standalone applications;
this will be discussed in chapter~\ref{c:standalone}.}. When running
Caml on a computer, we enter a {\em toplevel loop} where Caml waits
for input from the user, and then gives a response to what has been
entered.

The beginning of a Caml Light session looks like this (assuming
\verb|$| is the shell prompt on the host machine):
\begin{verbatim}
$camllight
>       Caml Light version 0.6

#
\end{verbatim}
On the PC version, the command is called {\tt caml}.
\def\sharp{{\tt\#}}

The ``{\sharp}'' character is Caml's prompt. When this character
appears at the beginning of a line in an actual Caml Light session,
the toplevel loop is waiting for a new toplevel phrase to be entered.

Throughout this document, the phrases starting by the {\sharp}
character represent legal input to Caml. Since this document has been
pre-processed by Caml Light and these lines have been effectively
given as input to a Caml Light process, the reader may reproduce by
him/herself the session contained in each chapter (each chapter of the
first two parts contains a different session, the third part is a
single session). Lines starting with the ``\verb|>|'' character are
Caml Light error messages. Lines starting by another
character are either Caml responses or (possibly) illegal input. The
input is printed in typewriter font ({\tt like this}), and output is
written using slanted typewriter font ({\tt \sl like that}).

{\bf Important:} Of course, when developing non-trivial programs, it
is preferable to edit programs in files and then to include the files
in the toplevel, instead of entering the programs directly.
Furthermore, when debugging, it is very useful to {\em trace} some
functions, to see with what arguments they are called, and what result
they return.  See the reference manual \cite{CamlLightDoc} for a
description of these facilities.

\section{Evaluation: from expressions to values}

Let us enter an arithmetic expression and see what is Caml's response:
\begin{caml_example}
1+2;;
\end{caml_example}
The expression ``\verb|1+2|'' has been entered, followed by
``\verb|;;|'' which represents the end of the current toplevel phrase. When
encountering ``\verb|;;|'', Caml enters the type-checking (more precisely {\em
type synthesis}) phase, and prints the inferred type for the expression. Then, it compiles code for the expression, executes it and, finally, prints the result.

In the previous example, the result of evaluation is printed
as ``\verb|3|'' and the type is ``\verb|int|'': the type of integers.

The process of evaluation of a Caml expression can be seen as transforming
this expression until no further transformation is allowed. These
transformations must preserve semantics. For example, if the expression
``\verb|1+2|'' has the mathematical object 3 as semantics, then the
result ``\verb|3|'' must have the same semantics.
The different phases of the Caml evaluation process are:

\begin{itemize}
\item parsing (checking the syntactic legality of input);
\item type synthesis;
\item compiling;
\item loading;
\item executing;
\item printing the result of execution.
\end{itemize}
%
Let us consider another example: the application of the successor function to
\verb|2+3|. The expression
\verb|(function x -> x+1)|\ikwd{function@\verb`function`}
should be read as ``the function that, given {\tt x}, returns
\verb|x+1|''. The juxtaposition of this expression to \verb|(2+3)| is
{\em function application}.
\begin{caml_example}
(function x -> x+1) (2+3);;
\end{caml_example}
There are several ways to reduce that value before obtaining the result
\verb"6". Here are two of them (the expression being reduced at each
step is underlined):
$$\begin{array}{c@{\qquad}c}
\mbox{\tt (function x -> x+1)} ~ \underline{\mbox{\tt (2+3)}} &
        \underline{\mbox{\tt (function x -> x+1) (2+3)}} \\
\downarrow & \downarrow \\
\underline{\mbox{\tt (function x -> x+1) 5}} &
        \underline{\mbox{\tt (2+3)}} ~ \mbox{\tt + 1} \\
\downarrow & \downarrow \\
\underline{\mbox{\tt 5+1}} &
        \underline{\mbox{\tt 5+1}} \\
\downarrow & \downarrow \\
\tt 6 & \tt 6
\end{array}$$

The transformations used by Caml during evaluation
cannot be described in this chapter, since
they imply knowledge about compilation of Caml programs and machine
representation of Caml values. However, since the basic control structure
of Caml is function application, it is quite easy to give an idea of
the transformations involved in the Caml evaluation process by using the
kind of rewriting we used in the last example.
The evaluation of the (well-typed) application $e_1 (e_2)$, where $e_1$
and $e_2$ denote arbitrary expressions, consists in the following
steps:
\begin{itemize}
\item Evaluate $e_2$, obtaining its value $v$.
\item
 Evaluate $e_1$ until it becomes a functional value. Because of the well-typing
 hypothesis, $e_1$ must represent a function from some type $t_1$ to some
$t_2$, and the type of $v$ is $t_1$.
We will write ({\tt function x ->} $e$) for the result of the evaluation
 of $e_1$. It denotes the mathematical object usually written as:
 \begin{center}
 $\begin{array}{ll}
  f:& t_1 \rightarrow t_2\\
    & x \mapsto e \mbox{ (where, of course, $e$ may depend on $x$)}
  \end{array}$
 \end{center}
N.B.: We do not evaluate $e$ before we know the value of $x$.
\item Evaluate $e$ where $v$ has been substituted for all occurrences of
\verb|x|. We then get the final value of the original expression. The result
is of type $t_2$.
\end{itemize}
%
In the previous example, we evaluate:
\begin{itemize}
\item \verb|2+3| to \verb|5|;
\item \verb|(function x -> x+1)| to itself (it is already a function body);
\item \verb|x+1| where \verb|5| is substituted for \verb|x|, i.e. evaluate
\verb|5+1|, getting \verb|6| as result.
\end{itemize}
%
It should be noticed that Caml uses call-by-value: arguments are
evaluated before being passed to functions. The relative evaluation
order of the functional expression and of the argument expression is
implementation-dependent, and should not be relied upon. The Caml
Light implementation evaluates arguments before functions
(right-to-left evaluation), as shown above; the original Caml
implementation evaluates functions before arguments (left-to-right evaluation).



\section{Types}


Types and their checking/synthesis are crucial to functional
programming:  they provide an indication about the correctness of
programs.

The universe of Caml values is partitioned into {\em types}.
A type represents a collection of values.
For example, \verb|int| is the type of integer numbers, and \verb|float| is the type of floating-point numbers.
Truth values belong to the \verb|bool|
type. Character strings belong to the \verb|string| type. %
%
Types are divided into two classes:
\begin{itemize}
\item
  Basic types (\verb|int|, \verb|bool|, \verb|string|, \ldots).
\item 
  Compound types such as functional types. For example,
  the type of functions from integers to integers is
  denoted by \verb|int -> int|. The type of functions from boolean values to
  character strings is written \verb|bool -> string|. %
The type of pairs
  whose first component is an integer and whose second component is a boolean value
  is written \verb|int * bool|.
\end{itemize}
%
In fact, any combination of the types above (and even more!) is possible.
This could be written as:
\begin{quote}
\begin{tabular}{lrl}
BasicType & ::= & {\tt int} \\
      & $|$ & {\tt bool} \\
      & $|$ & {\tt string} \\
\\
Type & ::= & BasicType \\
     & $|$ & Type \verb|->| Type\\
     & $|$ & Type {\tt *} Type
\end{tabular}
\end{quote}
Once a Caml toplevel phrase has been entered in the computer, the Caml
process starts analyzing it. First of all, it performs {\em
syntax analysis}, which consists in checking whether the phrase
belongs to the language or not. For example, here is a syntax
error\footnote{Actually, lexical analysis takes place before syntax
analysis and {\em lexical errors} may occur, detecting for instance
badly formed character constants.} (a parenthesis is missing):
\begin{caml_example}
(function x -> x+1 (2+3);;
\end{caml_example}
The carets ``\verb|^^^|'' underline the location where the error was
detected.

The second analysis performed is {\em type analysis}: the system
attempts to assign a type to each subexpression of the phrase, and to
synthesize the type of the whole phrase.  If type analysis fails (i.e.
if the system is unable to assign a sensible type to the phrase), then
a type error is reported and Caml waits for another input, rejecting
the current phrase.  Here is a type error (cannot add a number to a
boolean):
\begin{caml_example}
(function x -> x+1) (2+true);;
\end{caml_example}
The rejection of ill-typed phrases is called {\em strong typing}.
Compilers for weakly typed languages (C, for example) would instead
issue a warning message and continue their work at the risk of
getting a ``{\tt Bus error}'' or ``{\tt Illegal instruction}''
message at run-time. 

Once a sensible type has been deduced for the expression, then the
evaluation process (compilation, loading and execution) may
begin.

Strong typing enforces writing clear and well-structured programs.
Moreover, it adds security and increases the speed of program development,
since most typos and many conceptual errors are trapped and signaled
by the type analysis.
Finally, well-typed programs do not need dynamic type tests (the addition
function does not need to test whether or not its arguments are numbers),
hence static type analysis allows for more efficient machine code.

\section{Functions}

The most important kind of values in functional programming are
functional values. Mathematically, a function $f$ of type $A \rightarrow B$ is
a rule of correspondence which associates with each element of type $A$ a
unique member of type $B$.

If $x$ denotes an element of $A$, then we will write $f(x)$ for the
application of $f$ to $x$. Parentheses are often useless (they are
used only for grouping subexpressions), so we could also write $(f
(x))$ as well as $(f ((x)))$ or simply $f~x$. The value of $f~x$ is
the unique element of $B$ associated with $x$ by the rule of
correspondence for $f$.

The notation $f(x)$ is the one normally employed in mathematics to
denote functional application. However, we shall be careful not to
confuse a function with its application. We say ``the function $f$
with formal parameter $x$'', meaning that $f$ has been defined by:
\[f: x \mapsto e \]
or, in Caml, that the body of $f$ is something like
\verb|(function x -> ...)|.
Functions are values as other values.  In particular, functions may be
passed as arguments to other functions, and/or returned as result. For
example, we could write:
\begin{caml_example}
function f -> (function x -> (f(x+1) - 1));;
\end{caml_example}
That function takes as parameter a function (let us call it \verb|f|) and
returns another function which, when given an argument (let us call it
\verb|x|), will return the predecessor of the value of the application
\verb|f(x+1)|.

The type of that function should be read as:
\verb|(int -> int) -> (int -> int)|.


\section{Definitions}


It is important to give names to values.
We have already seen some named values: we
called them {\em formal parameters}. In the expression
\verb|(function x -> x+1)|, the name \verb|x| is a formal parameter.
Its name is irrelevant: changing it into another one
does not change the meaning of the
expression.
We could have written that function \verb|(function y -> y+1)|.

If now we apply this function to, say, \verb|1+2|, we will evaluate
the expression \verb|y+1| where \verb|y| is the value of \verb|1+2|.
Naming {\tt y} the value of \verb|1+2| in \verb|y+1| is written as:
\ikwd{let@\verb`let`}
\begin{caml_example}
let y=1+2 in y+1;;
\end{caml_example}
This expression is a legal Caml phrase, and the \verb|let|
construct is indeed widely used in Caml programs.

The \verb|let| construct introduces {\em local bindings of values to
identifiers}. They are {\em local} because the scope of {\tt y} is
restricted to the expression \verb|(y+1)|. The identifier {\tt y} kept
its previous binding (if any) after the evaluation of the ``{\tt let}
\ldots {\tt in} \ldots'' construct. We can check that \verb|y| has not
been globally defined by trying to evaluate it:
\begin{caml_example}
y;;
\end{caml_example}
Local bindings using {\tt let} also introduce {\em sharing} of
(possibly time-consuming) evaluations. When evaluating ``{\tt let
x=}$e_1$ {\tt in} $e_2$'', $e_1$ gets evaluated only once.  All
occurrences of {\tt x} in $e_2$ access the {\em value} of $e_1$ which
has been computed once.  For example, the computation of:
\begin{verbatim}
let big = sum_of_prime_factors 35461243
in big+(2+big)-(4*(big+1));;
\end{verbatim}
will be less expensive than:
\begin{verbatim}
  (sum_of_prime_factors 35461243)
+ (2 + (sum_of_prime_factors 35461243))
- (4 * (sum_of_prime_factors 35461243 + 1));;
\end{verbatim}
The \verb|let| construct also has  a global form for toplevel declarations, as
in:
\begin{caml_example}
let successor = function x -> x+1;;
let square = function x -> x*x;;
\end{caml_example}
When a value has been declared at toplevel, it is of course available during
the rest of the session.
\begin{caml_example}
square(successor 3);;
square;;
\end{caml_example}

When declaring a functional value, there are some alternative syntaxes
available.
For example we could have declared the \verb|square| function
by:
\begin{caml_example}
let square x = x*x;;
\end{caml_example}
or (closer to the mathematical notation) by:
\begin{caml_example}
let square (x) = x*x;;
\end{caml_example}
All these definitions are equivalent.


\section{Partial applications}


A {\em partial application} is the application of a function to some
but not all of its arguments.
Consider, for example, the function {\tt f} defined by:
\begin{caml_example}
let f x = function y -> 2*x+y;;
\end{caml_example}
Now, the expression \verb|f(3)| denotes the function which when given an
argument \verb|y| returns the value of \verb|2*3+y|. The application
\verb|f(x)| is called a {\em partial application}, since {\tt f} waits for
two successive arguments, and is applied to only one. The value of {\tt f(x)}
is still a function.

We may thus define an addition function by:
\begin{caml_example}
let addition x = function y -> x+y;;
\end{caml_example}
This can also be written:
\begin{caml_example}
let addition x y = x+y;;
\end{caml_example}
We can then define the successor function by:
\begin{caml_example}
let successor = addition 1;;
\end{caml_example}
Now, we may use our \verb|successor| function:
\begin{caml_example}
successor (successor 1);;
\end{caml_example}

\section*{Exercises}

\begin{exo}\label{Fund:1}
Give examples of functions with the following types:
\begin{enumerate}
\item \verb|(int -> int) -> int|
\item \verb|int -> (int -> int)|
\item \verb|(int -> int) -> (int -> int)|
\end{enumerate}
\end{exo}
%
\begin{exo}\label{Fund:2}
We have seen that the names of formal parameters are meaningless. It is then
possible to rename \verb"x" by \verb"y" in \verb"(function x -> x+x)". What
should we do in order to rename \verb"x" in \verb"y" in
\begin{verbatim}
(function x -> (function y -> x+y))
\end{verbatim}
Hint: rename \verb"y" by \verb"z" first. Question: why?
\end{exo}
%
\begin{exo}\label{Fund:3}
Evaluate the following expressions (rewrite them until no longer possible):
\begin{verbatim}
let x=1+2 in ((function y -> y+x) x);;
let x=1+2 in ((function x -> x+x) x);;
let f1 = function f2 -> (function x -> f2 x)
in let g = function x -> x+1
   in f1 g 2;;
\end{verbatim}
\end{exo}
