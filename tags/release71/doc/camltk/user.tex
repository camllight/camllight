\chapter{User's Manual}
{\bf This manual is a simple tutorial for \camltk }.  When writing your own
programs, you should consult the reference manual
(chapter~\ref{chap:ref}), as well as the \tk\ man pages.  Whenever the
reference manual is unsufficient, you can check the \verb|Widgets.src| file
(which syntax is found in chapter~\ref{chap:impl}) since this file gives the
mappings between \tk\ and \caml\ for functions and data. 

\section{Modules}
The \camltk\ interface is provided as a \caml\ library, composed of the
following modules:

\begin{itemize}
\item \verb|tk|, containing the initialisation functions, some frequently
used functions, and all types manipulated by the library.

\item a module for each widget class (e.g. \verb|button|). The creation
function and the commands specific to this class are contained in the
module. The module bears the same name as the widget class, except for
\verb|toplevel| widgets which are defined in module \verb|toplevelw|.

\item various other modules grouping functions of a same family
(e.g. \verb|focus|, \verb|winfo|). 

\item the \verb|textvariable| module, for the \verb|-textvariable| option
support. 

\item \verb|protocol| and \verb|support|, which contain the internals
of the interface. For example, the abstract type of widgets is defined in
\verb|support|, as well as some widget naming utilities.
\end{itemize} 

All modules required by the \camltk\ interface are in the \verb|tklib.zo|
library archive.

The organization of the interface is designed so that, in general, you
should \verb|#open| only the \verb|tk| module, and for the 
rest of the functions use the \verb|modulename__symbol| syntax (e.g.
\verb|text__insert|).
This convention is preferred because some functions (such as \verb|create|
and \verb|configure|) exist in many different modules. In this context,
referring to functions with exact qualified names makes programs clearer and
less error-prone.


\section{Basics}
\begin{latexonly}
\begin{figure}
\caption{The {\tt helloworld} example}
\label{fig:helloworld}
\begin{center}
\leavevmode
\epsfbox{helloworld-dump.ps}
\end{center}
\end{figure}
\end{latexonly}

\htmladdimg{helloworld.gif}

As usual with toolkits, the simplest way to learn how to write programs is
by copy-paste-ing from available source code. In general, a \caml\ program
using the \tk\ toolkit will have the following structure :

\input helloworld.tex

The first step is to initialize the interface with \verb|OpenTk|, and this
returns the {\em toplevel} widget.
Then, the program creates some widgets, as children of the {\em toplevel}
widget. These widgets form the user interface of your application. In this
simple case, we have only one button, whose behaviour is to print a simple
message on the standard output. Finally, the program enters the
\verb|MainLoop|, and user-interaction takes
place. Figure~\ref{fig:helloworld} shows the interface of this program.


\section{Example}
\begin{latexonly}
\begin{figure}
\caption{The {\tt addition} example}
\label{fig:addition}
\begin{center}
\leavevmode
\epsfbox{addition-dump.ps}
\end{center}
\end{figure}
\end{latexonly}

\htmladdimg{addition.gif}

Here is a complete working example with a few widgets and callbacks.
This program may be found in the \verb|test| directory of the distribution,
under the name \verb|addition.ml|.

\input addition.tex

The \verb|ScrollCommand| callback has the property of being called each time
the insertion cursor blinks in an \verb|entry| widget. This property is
responsible for the ``electric'' behavior of this program. Another approach
would have been to associate an action to the event ``key return has been 
pressed''. Figure~\ref{fig:addition} shows the interface of this program.

\section{Initialisation}
A program using the \camltk\ interface must call one of the three following
functions :
\begin{verbatim}
value OpenTk : unit -> Widget
and OpenTkClass : string -> Widget
and OpenTkDisplayClass : string -> string -> Widget
\end{verbatim}
Each of these functions returns the initial toplevel widget ("." in \tcl\tk).
\verb|OpenTkClass| lets you specify the resource class of the toplevel
widget, and \verb|OpenTkDisplayClass| lets you specify the display and the
resource class.

During the initialisation, the file \verb|$HOME/.camltkrc|, if it exists,
will be loaded by \tcl\tk. This might be the place where to write some
special customisations in \tcl, such as changing auto-load path, \ldots 

\section{Widgets}
The description of widget commands and options is given in the reference
manual (chapter~\ref{chap:ref}). The semantics of commands is however not
described there, so the reader should also refer to the \verb|man| pages of
the \tk\ distribution. 
As of this release, all widgets of \tk\ 3.6 are supported, although some
commands might not be implemented. Remember also that only few of them have
been tested.

\section{Bindings}
\tk\ offers, in addition to widget-specific callbacks, a general facility
for binding functions to events. (bind(n)).
Bindings involves the following types, defined in the \verb|tk| module:
\input bindtypes.tex

Binding are manipulated with the function
\begin{verbatim}
bind: Widget -> (Modifier list * XEvent) list -> BindAction -> unit
\end{verbatim} 

The first argument of type \verb|Widget| is naturally the widget for which
this binding is defined.
The second argument of type \verb|(Modifier list * XEvent) list| is the 
succession of events that will trigger the callback.
The last argument is the action : \verb|BindRemove| will remove a binding,
\verb|BindSet| will set a binding, and \verb|BindExtend| will extend a
binding.

In many cases, a callback needs information about the event that
triggered it, such as the precise key that was pressed, the position of the
mouse, etc. Therefore, a callback is of type \verb|EventInfo -> unit|, 
and the callback will be invoked with argument the record containing
event information. Since event information can be quite large,
the user has to specify, as an \verb|EventField list|, which fields in this
record are relevant for this particular callback. 
The other fields will {\bf not} contain accurate information. 
Check also the \tk\ documentation for finding out which fields are valid for a
given event.

\paragraph{Example}
\begin{verbatim}
bind myCanvas [[],Motion] (BindSet([Ev_MouseX; Ev_MouseY], mouse_moved)) ;;
\end{verbatim} 
will trigger the callback \verb|mouse_moved| when the event \verb|Motion|
occurs in the \verb|myCanvas| widget. The function \verb|mouse_moved| may
assume that the  information in the fields {\tt Ev\_MouseX, Ev\_MouseY} of
its argument contain accurate information.

\paragraph{Text and Canvas bindings}
\mbox{}\\

Bindings on objets in \verb|text| and \verb|canvas| widgets obey the same
rules, and the functions are 
\begin{verbatim}
tk__text_tag_bind : Widget -> TextTag -> 
      (Modifier list * XEvent) list -> BindAction -> unit
tk__canvas_bind : Widget -> TagOrId -> 
      (Modifier list * XEvent) list -> BindAction -> unit
\end{verbatim} 
Additional arguments are respectively the {\em tag} and the {\em tag or id}
for which the binding is defined.

\paragraph{Class bindings}
\mbox{}\\

Binding for classes of widgets are managed with
\begin{verbatim}
tk__class_bind : string -> 
      (Modifier list * XEvent) list -> BindAction -> unit
\end{verbatim} 
The first argument should be a widget class name (e.g. \verb|"Button"|).
Widget class names in \camltk\ are the exact same names as in \tk.
With \tk\ 3.6, a class binding is {\bf masked} by a widget specific binding.

\section{Text variables}
Text variables are available in \camltk. However, they are not physically
shared between \caml\ and \tk, that is, \tk\ cannot directly modify
a \caml\ variable.
Instead, setting or consulting a \verb|textvariable| requires a transaction
between \caml\ and \tk. However, the "sharing" properties of 
text variables in \tk\ are preserved. For example, if a
text variable is used to physically share the content of an {\tt entry} with
the text of a {\tt label}, then this sharing will effectively occur in the
\tk\ world, whether your \caml\ program is concerned with the actual
contents of the text variable or not. 


\section{Exiting}
The function
\begin{verbatim}
tk__CloseTk : unit -> unit
\end{verbatim} 
will terminate the main interaction loop (\verb|tk__MainLoop|) and
destroy all widgets.

% You should always try to call \verb|CloseTk| before exiting. Otherwise, in
% the {\em two-process} architecture, the slave {\tt wish} may stay around for
% a while, as well as named pipes created by the communication protocol in
% \verb|/tmp|. In particular, something like  
% \begin{verbatim}
% signal SIGINT (Signal_handle (function _ -> CloseTk(); exit 0))
% \end{verbatim} 
% should help your application behave nicely. Naturally your own cleanup
% should be inserted also.

\section{Errors}
The \camltk\ interface may raise the following exceptions:
\begin{description}
\item[TkError of string]: this exception is raised when a \tcl\tk\ command
fails during an evaluation. Normally, static typing in \caml, in addition to
some run-time verifications should prevent this situation from happening.
However, some errors are due to the external environment (e.g.
\verb|selection__get|) and cannot be prevented. The exception carries the
\tcl\tk\ error message.

\item[IllegalWidgetType of string]:
this exception is raised when a widget command is applied to a widget of a
different or unappropriate class (e.g. \verb|button__configure| applied to
a {\em scrollbar} widget). The exception carries the class of the faulty
widget.

\item[Invalid\_argument of string]:
this exception is raised  when some data cannot be exchanged between \caml\
and \tk. This situation occurs for example when an illegal option is used
for creating a widget. However, this error may also be caused by a faulty or
incomplete description of a widget, widget command or type.
\end{description} 

\section{More examples}
The \verb|test| directory of the distribution contains several test
examples. 
The \verb|books-examples| directory contains translations in \camltk\ of
several examples from \cite{ouster94}.

%The \verb|tktop| directory contains a more advanced example, although
%not very robust.

{\bf All examples included in this distribution  should not be considered as
working applications. They are provided for documentation purposes only}.

\section{The Caml Browser}
\begin{latexonly}
\begin{figure}
\caption{The Caml Browser}
\begin{center}
\leavevmode
\epsfbox{browser.ps}
\end{center}
\end{figure}
\end{latexonly}

\htmladdimg{browser.gif"}

In the directory {\tt browser} you will find a yet more complete example,
featuring many different kinds of widgets. The browser mimicks the 
\caml\ toplevel behaviour: you can open and close modules ({\tt \#open} and
{\tt \#close} directives), or add a library path ({\tt \#directory} directive).
You can enter a symbol (value, type constructor, value constructor,
exception), and see its definition. You can also see the complete interface
of a module by double-clicking on it. 
Remember that the browser will show you a pretty-printed version
of compiled interfaces ({\tt .zi} files), so you will not see the comments
that were in the source file.

The browser can also display source files. Hypertext navigation is also
available, but does not reflect the compiler's semantics,
as the \verb|#open| directives in the source file are not taken into
account.

When available, the browser also attemps to load {\tt TAGS} file, created by
the {\tt mletags} program in the \caml\ contributions. It is therefore
possible to navigate in program sources. However, note that the browser
implements only a simplified version of Emacs's {\tt tag} mechanism.

\section{Compilation}
In the following, we refer to \verb|TCLLIBDIR| (resp. \verb|TKLIBDIR|) as
an environment variable containing the directory where \verb|libtcl.a|
(resp. \verb|libtk.a|) is located. Consult your system administrator if you
don't have this information. 

Moreover, we assume a ``standard installation'' of \caml\ and \camltk\ from
the distribution. This means that all \camltk\ library files are installed
in the same directory as standard \caml\ library files.

The usual commands for compiling and linking are:
\begin{verbatim}
$ camlc -c addition.ml
$ camlc -custom -o addition tklib.zo addition.zo \
        -ccopt -L$TCLLIBDIR -ccopt -L$TKLIBDIR \
        -lcamltk -lcaml -ltk -ltcl -lX11
\end{verbatim}
Linking is a bit complex, because we need to tell the C compiler/linker
where to find the \verb|libtk.a| and \verb|libtcl.a| libraries, and to link
with all required pieces. Your linker might complain during this step.
Most problems have been encountered on {\bf Solaris} systems, but we also
know of successful installations on {\bf Solaris}.
\begin{itemize}
\item \verb|main| being defined twice. The problem here is that we are
dealing with three librairies (caml, tcl, tk) containing a \verb|main|
function. However, the linker should choose the first one, that is, the
\verb|main| of the caml library. At this time, the only solution we can
propose is to built alternate versions of the tcl and tk librairies, where
the \verb|main| function has been removed.

\item undefined symbol \verb|_Tcl_AppInit|. This function is referenced only
in the \verb|main| function of the tcl library. Thus, since this \verb|main|
is actually dead code, \verb|Tcl_AppInit| should not be required. A
workaround is to add the following lines
\begin{verbatim}
int Tcl_AppInit(Tcl_Interp *interp)
{
  return 0;
}
\end{verbatim} 
to the file \verb|libsupport/camltk.c|.
\end{itemize} 


\subsection{Toplevels}
The distribution also installs a toplevel featuring the \camltk\
interface: \verb|camltktop|.
This toplevel may be run with: 
\begin{verbatim}
$ camllight camltktop
\end{verbatim} 
Note however that the usage of the toplevel is awkward (for the time
being), because callbacks cannot be executed until you enter MainLoop, and
you cannot easily leave MainLoop.

\section{Extensions}
\camltk\ has already one extension (i.e. a feature not normally available in
\tk\ 3.6). This extension allows the association of callbacks to Unix input
file descriptors.

\subsection{File descriptor callbacks}
A callback can be associated to a file descriptor using the following
primitive:
\begin{verbatim}
tk__add_fileinput : file_descr -> (unit -> unit) -> unit
\end{verbatim} 
When some input is available on the file descriptor specified by the first
argument, then the callback (second argument) is called.
\begin{verbatim}
tk__remove_fileinput : file_descr -> unit
\end{verbatim} 
Removes the file descriptor callback. 

It is the programmer's responsability
to check for end of file, and to remove the callback if the file descriptor
is closed. Be aware that calling the \verb|update| function will potentially
destroy the sequentiality of read operations. Finally, note that during the
invocation of your callback, only one call to \verb|read| is guaranteed to
be non-blocking.

\section{Translating \tk\ idioms}
If you are a seasoned \tk\ programmer, you will find out that some \tk\
idioms have a different form in \camltk\ .

\subsection{Widgets}
First of all, widget creation is more ``functional''. One does not specify
the name of the created widget (except for \verb|toplevel| widgets), but
only its parent. The name is allocated by the library, and a handle to the
widget is returned to you. Then, widgets are not ``procedures''. They are
objects, and must be passed to widget manipulation functions.
For example,
\begin{verbatim}
button .myf.bok -title "Ok" -relief raised
\end{verbatim} 
is translated by (assuming that \verb|myf| is the parent widget \verb|.myf|)
\begin{verbatim}
let b = button__create myf [Title "Ok"; Relief Raised] in
...
\end{verbatim} 
Then, 
\begin{verbatim}
.myf.bok configure -state disabled
\end{verbatim} 
would be in \caml:
\begin{verbatim}
button__configure b [State Disabled]
\end{verbatim} 

This is more in the spirit of ML, but unfortunately eliminates the
possibility to specify configuration of widgets by resources based on widget 
names. It you absolutely want to use resources then you may call the
alternate \verb|create_named| function:
\begin{verbatim}
let b = button__create_named myf "bok" [Title "Ok"; Relief Raised] in
...
\end{verbatim} 
Assuming that \verb|myf| is the widget of path \verb|.myf|, then \verb|b|
will have path \verb|.myf.bok|. As in \tk, it is your responsibility to use
valid identifiers for path elements.

When widgets are mutually recursive (through callbacks, for example when
linking a scrollbar to a text widget), one should first create the widgets,
and then set up the callbacks by calling \verb|configure| on the widgets.
This is in contrast with \tcl\tk\ where one may refer to a procedure that
has not yet been defined.

Partially applied widget commands (such as redisplay commands) translate
quite well to \caml, with possibly some wrapping required due to 
value constructors.

\subsection{How to find out the \caml\ name of \tk\ keywords and functions}
Normally, a \caml\ command for a widget class will have the same name as the
corresponding \tk\ command. However, in some cases, for example when a same
command is used to read and set values, or has a variable number of type
incompatible arguments, \camltk\ will have several versions of the command
with different names.

For the time being the best information sources are:
\begin{itemize}
\item  the .mli files in the \verb|lib| directory (they also appear in
chapter~\ref{chap:ref}). This will give you at least the functions for each
widget class. 

\item  the \verb|Widgets.src| file of the distribution. 

\end{itemize} 

\section{Debugging}
Some \tk\ functions may have been improperly
implemented in the \camltk\ library. This may cause undue \tk\ errors
(exception \verb|TkError|, or sometimes \verb|Invalid_Argument|). To
facilitate the debugging, you can set the 
Unix environment variable \verb|CAMLTKDEBUG| to any value before launching
your program. This will allow you to see all
data transferred between the \caml\ and the \tk\ processes. Since this data is
essentially \tcl\tk\ code, you need a basic knowledge of this language to
understand what is going on.
It is also possible to trigger debugging by setting the boolean reference
\verb|protocol__debug| to true (or false to remove debugging).
