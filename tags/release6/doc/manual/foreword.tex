\chapter*{Foreword}
\markboth{Foreword}{}

This manual documents the release 0.6 of the Caml Light system. It is
organized as follows.
\begin{itemize}
\item Part~\ref{p:gettingstarted}, ``Getting started'', explains how
to install Caml Light on your machine.
\item Part~\ref{p:refman}, ``The Caml Light language reference
manual'', is the reference description of the Caml Light language.
\item Part~\ref{p:library}, ``The Caml Light library'', describes the
modules provided in the standard library.
\item Part~\ref{p:commands}, ``The Caml Light commands'', documents
the Caml Light compiler, toplevel system, and programming utilities.
\item Part~\ref{p:appendix}, ``Appendix'', contains a short bibliography, an
index of all identifiers defined in the standard library, and an
index of Caml Light keywords.
\end{itemize}

\section*{Conventions}

The Caml Light system comes in several versions: for Unix machines,
for Macintoshes, and for PCs. The parts of this manual that are
specific to one version are presented as shown below:

\begin{unix} This is material specific to the Unix version.
\end{unix}

\begin{mac} This is material specific to the Macintosh version.
\end{mac}

\begin{pcbase} This is material specific to the 8086 (and 80286) PC version.
\end{pcbase}

\begin{pcthree} This is material specific to the 80386 PC version.
\end{pcthree}

\begin{pc} This is material specific to the two PC versions.
\end{pc}

\section*{License}

The Caml Light system is copyright \copyright\ 1989, 1990, 1991, 1992,
1993 Institut National de Recherche en Informatique et en Automatique
(INRIA).  INRIA holds all ownership rights to the Caml Light system.
See the file {\machine COPYRIGHT} in the distribution for the copyright notice.

The Caml Light system can be freely copied, but not sold. More precisely,
INRIA grants any user of the Caml Light system the right to reproduce
it, provided that the copies are distributed free of charge and under
the conditions given in the {\machine COPYRIGHT} file.

\section*{Availability by FTP}

The complete Caml Light distribution resides on the machine
{\machine ftp.inria.fr}. The distribution files can be transferred by anonymous FTP:
\begin{center}
\begin{tabular}{ll}
Host:       & {\machine ftp.inria.fr} (Internet address {\machine 192.93.2.54}) \\
Login name: & {\machine anonymous} \\
Password:   & your e-mail address \\
Directory:  & {\machine lang/caml-light} \\
Files:      & see the index in file {\machine README}
\end{tabular}
\end{center}
