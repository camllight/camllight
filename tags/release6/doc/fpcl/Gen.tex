\def\Nat{{\bf N}}

\chapter{Functional languages}
\label{c:gen}

Programming languages are said to be {\em functional} when the basic
way of structuring programs is the notion of {\em function} and their
essential control structure is {\em function application}.  For
example, the Lisp language \cite{MacCarthy}, and more precisely its
modern successor Scheme \cite{SchemeReport,AbelsonSussman}, has been called
functional because it possesses these two properties.

However, we want the programming notion of function to be as close
as possible to the usual mathematical notion of function. In
mathematics, functions are ``first-class'' objects: they can be
arbitrarily manipulated. For example, they can be composed, and the
composition function is itself a function.

In mathematics, one would present the {\em successor} function in the
following way:
\[\begin{array}{ll}
\mbox{\em successor} : & \Nat ~ \longrightarrow ~ \Nat \\
                       & n ~ \longmapsto ~ n+1
\end{array}
\]
The functional composition could be presented as:
\[\begin{array}{ll}
\circ : & (A \Rightarrow B) \times (C \Rightarrow A)
                        ~ \longrightarrow ~ (C \Rightarrow B) \\
               & (f,g) ~ \longmapsto ~ (x~\longmapsto~f~(g~x))
\end{array}
\]
where $(A \Rightarrow B)$ denotes the space of functions from $A$ to $B$.

We remark here the importance of:
\begin{enumerate}
\item the notion of {\em type}; a mathematical function always possesses a
{\em domain} and a {\em codomain}. They will correspond to the programming
notion of type.
\item lexical binding: when we wrote the mathematical definition of
{\em successor}, we have assumed that the addition function $+$ had been
previously defined, mapping a pair of natural numbers to a natural
number; the meaning of the {\em successor} function is defined using
the {\em meaning} of the addition: whatever $+$ denotes in the future,
this {\em successor} function will remain the same.
\item the notion of {\em functional abstraction}, allowing to express
the behavior of $f \circ g$ as $(x~\longmapsto~f~(g~x))$, i.e.
the function which, when given some $x$, returns $f~(g~x)$.
\end{enumerate}
ML dialects (cf. below) respect these notions. But they also allow
non-functional programming styles, and, in this sense, they are
functional but not {\em purely functional}.

\section{History of functional languages}

Some historical points:

\begin{itemize}
\item 1930: Alonzo Church developed the $\lambda$-calculus \cite{Church} as an attempt to
           provide a basis for mathematics. The $\lambda$-calculus is a
           formal theory for functionality.
    The three basic constructs of
           the $\lambda$-calculus are:
           \begin{itemize}
              \item variable names (e.g. $x$, $y$,\ldots);
              \item application ($MN$ if $M$ and $M$ are terms);
              \item functional abstraction ($\lambda x . M$).
           \end{itemize}
           Terms of the $\lambda$-calculus represent functions. The pure
           $\lambda$-calculus has been proved inconsistent as a logical theory.
           Some {\em type systems} have been added to it in order to remedy
           this inconsistency.
\item 1958: Mac Carthy invented Lisp \cite{MacCarthy} whose programs have some similarities
           with terms of the $\lambda$-calculus.
    Lisp dialects have been recently
           evolving in order to be closer to modern functional languages
           (Scheme), but they still do not possess a type system.
\item 1965: P. Landin proposed the ISWIM \cite{ISWIM} language (for ``If You See What I Mean''),
           which is the precursor of languages of the ML family.
\item 1978: J. Backus introduced FP: a language of combinators \cite{FP} and a framework
           in which it is possible to reason about programs.
The main particularity of FP programs is that they have no variable names.
\item 1978: R. Milner proposes a language called ML \cite{ML}, intended to be the
           {\em metalanguage} of the LCF proof assistant (i.e. the language
           used to program the search of proofs). This language is inspired
           by ISWIM (close to $\lambda$-calculus) and possesses an original
           type system.
\item 1985: D. Turner proposed the Miranda \cite{Miranda85} programming language, which uses
           Milner's type system but where programs are submitted to {\em
           lazy evaluation}.
\end{itemize}
Currently, the two main families of functional languages are the ML and the
Miranda families.

\section{The ML family}

ML languages are based on a sugared\footnote{i.e. with a more user-friendly
syntax.}
version of $\lambda$-calculus.
Their evaluation regime is {\em call-by-value} (i.e. the argument is
evaluated before being passed to a function), and they use Milner's type
system.

The LCF proof system appeared in 1972 at Stanford (Stanford LCF). It
has been further developed at Cambridge (Cambridge LCF) where the ML
language was added to it.

From 1981 to 1986, a version of ML and its compiler was developed
in a collaboration between INRIA and Cambridge by G. Cousineau, G. Huet
and L. Paulson.

In 1981, L. Cardelli implemented a version of ML whose compiler generated
native machine code.

In 1984, a committee of researchers from the universities of Edinburgh
and Cambridge, Bell Laboratories and INRIA, headed by R. Milner worked
on a new extended language called Standard ML \cite{ProposalSML}. This core
language was completed by a module facility designed by D. MacQueen
\cite{SMLModules}.

Since 1984, the Caml language has been under design in a collaboration
between INRIA and LIENS\footnote{Laboratoire d'Informatique de l'Ecole
Normale Sup\'erieure, 45 Rue d'Ulm, 75005 Paris}). Its first release
appeared in 1987. The main implementors of Caml were Asc\'ander
Su\'arez, Pierre Weis and Michel Mauny.

In 1989 appeared Standard ML of New-Jersey, developed by Andrew Appel
(Princeton University) and David MacQueen (Bell Laboratories).

Caml Light is a smaller, more portable implementation of the core Caml
language, developed by Xavier Leroy since 1990.

\section{The Miranda family}

All languages in this family use {\em lazy evaluation} (i.e. the argument of a
function is evaluated if and when the function needs its value---arguments
are passed unevaluated to functions).
They also use Milner's type system.

Languages belonging to the Miranda family find their origin in the
SASL language \cite{SASL} (1976) developed by D. Turner. SASL and its
successors (KRC \cite{KRC} 1981, Miranda \cite{Miranda85} 1985 and
Haskell \cite{Haskell90} 1990) use {\em sets of mutually recursive
equations} as programs. These equations are written in a {\em script}
(collection of declarations) and the user may evaluate expressions
using values defined in this script.  LML (Lazy ML) has been developed
in G\"oteborg (Sweeden); its syntax is close to ML's syntax and it
uses a fast execution model: the G-machine \cite{GMachine}.

