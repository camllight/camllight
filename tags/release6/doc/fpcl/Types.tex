\chapter{User-defined types}
\label{c:udeftypes}

The user is allowed to define his/her own data types.
With this facility, there is no need to encode
the data structures that must be manipulated by a program into lists
(as in Lisp) or into arrays (as in Fortran). Furthermore, early
detection of type errors is enforced, since user-defined data types reflect
precisely the needs of the algorithms. %

Types are either:
\begin{itemize}
\item {\em product} types,
\item or {\em sum} types.
\end{itemize}

We have already seen examples of both kinds of types: the {\tt bool} and
{\tt list} types are sum types (they contain values with different shapes
and are defined and matched using several alternatives). The
cartesian product is an example of a product type: we know statically
the shape of values belonging to cartesian products.

In this chapter, we will see how to define and use new types in Caml.

%
\section{Product types}
\label{s:udefprodtypes}

Product types are {\em finite} {\em labeled} products of types.
They are a generalization of cartesian product.
Elements of product types are called {\em records}.

\subsection{Defining product types}

An example: suppose we want to define a data structure containing
information about individuals. We could define:
\begin{caml_example}
let jean=("Jean",23,"Student","Paris");;
\end{caml_example}
and use pattern-matching to extract any particular
information about the person {\tt jean}.
The problem with such usage of cartesian product is that a function
\verb"name_of" returning the name field of a value representing an individual
would have the same type as the general first projection for 4-tuples
(and indeed would be the same function). This type is not precise enough
since it allows for the application of the function to any 4-uple, and
not only to values such as {\tt jean}.

Instead of using cartesian product, we define a {\tt person} data
type:\ikwd{type@\verb`type`}
\begin{caml_example}
type person =
  {Name:string; Age:int; Job:string; City:string};;
\end{caml_example}
The type {\tt person} is the {\em product} of {\tt string}, {\tt int}, {\tt
string} and {\tt string}.
The field names provide type information and also documentation: it
is much easier to understand data structures such as {\tt jean} above than
arbitrary tuples.
%

We have {\em labels} (i.e. {\tt Name}, \ldots) to refer to components of
the products. The order of appearance of the products components is not
relevant: labels are sufficient to uniquely identify the components.
The Caml system finds a canonical order on labels
to represent and print record values. The order is always the order which
appeared in the definition of the type.
%

We may now define the individual {\tt jean} as:
\begin{caml_example}
let jean = {Job="Student"; City="Paris";
            Name="Jean"; Age=23};;
\end{caml_example}
This example illustrates the fact that order of labels is not
relevant.

\subsection{Extracting products components}

The canonical way of extracting product components is {\em pattern-matching}.
Pattern-matching provides a way to mention the shape of values and to give
(local) names to components of values.
In the following example, we name {\tt n} the numerical value contained in
the field {\tt Age} of the argument, and we choose to forget values
contained in other fields (using the \verb"_" character).%
\begin{caml_example}
let age_of = function
     {Age=n; Name=_; Job=_; City=_} -> n;;
age_of jean;;
\end{caml_example}
It is also possible to access the value of a single field, with the
{\tt .} (dot) operator:
\begin{caml_example}
jean.Age;;
jean.Job;;
\end{caml_example}
Labels always refer to the most recent type definition: when two
record types possess some common labels, then these labels always
refer to the most recently defined type. When using modules (see
section~\ref{s:modules}) this problem arises for types defined in
the same module. For types defined in different modules, the full name
of labels (i.e. with the name of the modules prepended) disambiguates
such situations.

\subsection{Parameterized product types}

It is important to be able to define parameterized types in order to
define {\em generic} data structures. The {\tt list} type is parameterized,
and this is the reason why we may build lists of any kind of values.
If we want to define the cartesian product as a Caml type, we need type
parameters because we want to be able to build cartesian product of {\em
any} pair of types.
\begin{caml_example}
type ('a,'b) pair = {Fst:'a; Snd:'b};;
let first x = x.Fst and second x = x.Snd;;
let p={Snd=true; Fst=1+2};;
first(p);;
\end{caml_example}
Warning: the {\tt pair} type is similar to the Caml cartesian product {\tt
*}, but it is a different type.
\begin{caml_example}
fst p;;
\end{caml_example}
Actually, any two type definitions produce different types. If we
enter again the previous definition:
\begin{caml_example}
type ('a,'b) pair = {Fst:'a; Snd:'b};;
\end{caml_example}
we cannot any longer extract the {\tt Fst} component of {\tt x}:
\begin{caml_example}
p.Fst;;
\end{caml_example}
since the label {\tt Fst} refers to the {\em latter} type {\tt pair} that we
defined, while {\tt p}'s type is the {\em former} {\tt pair}.

\section{Sum types}

A {\em sum} type is the {\em finite} {\em labeled} disjoint union of
several types.
A sum type definition defines a type as being the union of some other types.%

\subsection{Defining sum types}

Example: we want to have a type called {\tt identification} whose
values can be:
\begin{itemize}
\item either strings (name of an individual),
\item or integers (encoding of social security number as a pair of integers).
\end{itemize}
We then need a type containing {\em both} \verb|int * int| and
character strings.  We also want to preserve static type-checking, we
thus want to be able to distinguish pairs from character strings
even if they are injected in order to form a single type.

Here is what we would do:\ikwd{type@\verb`type`}
\begin{caml_example}
type identification = Name of string
                    | SS of int * int;;
\end{caml_example}
The type {\tt identification} is the labeled disjoint union of {\tt string}
and \verb|int * int|.
The labels {\tt Name} and {\tt SS} are {\em injections}.
They respectively inject \verb|int * int| and {\tt string} into a single type
{\tt identification}.

Let us use these injections in order to build new values:
\begin{caml_example}
let id1 = Name "Jean";;
let id2 = SS (1670728,280305);;
\end{caml_example}
Values {\tt id1} and {\tt id2} belong to the same type. They may for
example be put into lists as in:
\begin{caml_example}
[id1;id2];;
\end{caml_example}

Injections may possess one argument (as in the example above), or
none. The latter case corresponds\footnote{In Caml Light, there is no
implicit order on values of sum types.} to {\em enumerated types}, as
in Pascal. An example of enumerated type is:
\begin{caml_example}
type suit = Heart
          | Diamond
          | Club
          | Spade;;
Club;;
\end{caml_example}
The type {\tt suit} contains only 4 distinct elements.
Let us continue this example by defining a type for cards.
\begin{caml_example}
type card = Ace of suit
          | King of suit
          | Queen of suit
          | Jack of suit
          | Plain of suit * int;;
\end{caml_example}
The type {\tt card} is the disjoint union of:
\begin{itemize}
\item {\tt suit} under the injection {\tt Ace},
\item {\tt suit} under the injection {\tt King},
\item {\tt suit} under the injection {\tt Queen},
\item {\tt suit} under the injection {\tt Jack},
\item the product of {\tt int} and  {\tt suit} under the injection {\tt Plain}.
\end{itemize}
Let us build a list of cards:
\begin{caml_example}
let figures_of c = [Ace c; King c; Queen c; Jack c]
and small_cards_of c =
    map (function n -> Plain(c,n)) [7;8;9;10];;
figures_of Heart;;
small_cards_of Spade;;
\end{caml_example}

\subsection{Extracting sum components}

Of course, pattern-matching is used to extract sum components.
In case of sum types, pattern-matching uses dynamic tests for this
extraction.
The next example defines a function {\tt color} returning the name of
the color of the suit argument:%
\begin{caml_example}
let color = function Diamond -> "red"
                   | Heart -> "red"
                   | _ -> "black";;
\end{caml_example}
Let us count the values of cards (assume we are playing ``belote''):
\begin{caml_example}
let count trump = function
    Ace _        -> 11
  | King _       -> 4
  | Queen _      -> 3
  | Jack c       -> if c = trump then 20 else 2
  | Plain (c,10) -> 10
  | Plain (c,9)  -> if c = trump then 14 else 0
  | _             -> 0;;
\end{caml_example}

\subsection{Recursive types}

Some types possess a naturally recursive structure, lists, for
example. It is also the case for tree-like structures, since trees
have subtrees that are trees themselves.

Let us define a type for abstract syntax trees of a simple arithmetic
language\footnote{Syntax trees are said to be {\em abstract} because
they are pieces of {\em abstract syntax} contrasting with {\em
concrete syntax} which is the ``string'' form of programs: analyzing
(parsing) concrete syntax usually produces abstract syntax.}. An
arithmetic expression will be either a numeric constant, or a
variable, or the addition, multiplication, difference, or division of
two expressions.\ikwd{type@\verb`type`}
\begin{caml_example}
type arith_expr = Const of int
                | Var of string
                | Plus of args
                | Mult of args
                | Minus of args
                | Div of args
and args = {Arg1:arith_expr; Arg2:arith_expr};;
\end{caml_example}
The two types \verb|arith_expr| and \verb|args| are simultaneously
defined, and \verb|arith_expr| is recursive since its definition
refers to \verb|args| which itself refers to \verb|arith_expr|. As an
example, one could represent the abstract syntax tree of the
arithmetic expression ``\verb|x+(3*y)|'' as the Caml value:
\begin{caml_example}
Plus {Arg1=Var "x";
       Arg2=Mult{Arg1=Const 3; Arg2=Var "y"}};;
\end{caml_example}

The recursive definition of types may lead to types such that it is
hard or impossible to build values of these types.
For example:
\begin{caml_example}
type stupid = {Head:stupid; Tail:stupid};;
\end{caml_example}
Elements of this type are {\em infinite} data structures. Essentially,
the only way to construct one is:
\begin{caml_example}
let rec stupid_value =
    {Head=stupid_value; Tail=stupid_value};;
\end{caml_example}

Recursive type definitions should be {\em well-founded} (i.e. possess
a non-recursive case, or {\em base case}) in order to work well with
call-by-value.

\subsection{Parameterized sum types}

Sum types may also be parameterized.
Here is the definition of a type isomorphic to the {\tt list} type:
\begin{caml_example}
type 'a sequence = Empty
                 | Sequence of 'a * 'a sequence;;
\end{caml_example}

A more sophisticated example is the type of generic binary trees:
\begin{caml_example}
type ('a,'b) btree = Leaf of 'b
                   | Btree of ('a,'b) node
and ('a,'b) node = {Op:'a;
                    Son1: ('a,'b) btree;
                    Son2: ('a,'b) btree};;
\end{caml_example}
A binary tree is either a leaf (holding values of type {\tt 'b}) or a
node composed of an operator (of type {\tt 'a}) and two sons, both of them
being binary trees.

Binary trees can also be used to represent abstract trees for
arithmetic expressions (with only binary operators and only one kind
of leaves). The abstract
syntax tree \verb|t| of ``\verb|1+2|'' could be defined as:
\begin{caml_example}
let t = Btree {Op="+"; Son1=Leaf 1; Son2=Leaf 2};;
\end{caml_example}

Finally, it is time to notice that pattern-matching is not restricted to
function bodies, i.e. constructs such as:
\begin{center}
\begin{tabular}{rll}
\tt function & $P_1$ & \verb|->| $E_1$\\
    \verb'|' & \ldots\\
    \verb'|' & $P_n$ & \verb|->| $E_n$
\end{tabular}
\end{center}
but there is also a construct dedicated to pattern-matching of actual
values:
\ikwd{match@\verb`match`}
\begin{center}
\begin{tabular}{rll}
\tt match $E$ with & $P_1$ & \verb|->| $E_1$\\
          \verb'|' & \ldots\\
          \verb'|' & $P_n$ & \verb|->| $E_n$
\end{tabular}
\end{center}
which matches the value of the expression $E$ against each of the
patterns $P_i$, selecting the first one that matches, and giving
control to the corresponding expression. For example, we can match the
tree {\tt t} previously defined by:
\begin{caml_example}
match t with Btree{Op=x; Son1=_; Son2=_} -> x
           | Leaf l -> "No operator";;
\end{caml_example}

\subsection{Data constructors and functions}

One may ask: ``What is the difference between a sum data constructor and a
function?''.
At first sight, they look very similar. We assimilate constant data
constructors (such as {\tt Heart}) to constants. However, in Caml Light, sum data constructors with arguments are not presented as functions from the element types to the sum type:
\begin{caml_example}
Ace;;
\end{caml_example}
The reason is that a data constructor possesses particular properties
that a general function does not possess, and it is interesting to
understand these differences.  From the mathematical point of view, a
sum data constructor is known to be an {\em injection} while a Caml
function is a general function without further information.
A mathematical injection $f: A \rightarrow B$ admits an inverse function
$f^{-1}$ from its image $f(A) \subset B$ to $A$.

From the examples above, if we consider the {\tt King} constructor, then:
\begin{caml_example}
let king c = King c;;
\end{caml_example}
{\tt king} is the general function associated to the {\tt King}
constructor, and:
\begin{caml_example}
function King c -> c;;
\end{caml_example}
is the inverse function for {\tt king}.
It is a partial function, since pattern-matching may fail.

\subsection{Degenerate cases: when sums meet products}

What is the status of a sum type with a single case such as:
\begin{caml_example}
type counter1 = Counter of int;;
\end{caml_example}
Of course, the type {\tt counter1} is isomorphic to {\tt int}.
The injection \verb"function x -> Counter x" is a {\em total} function from {\tt int} to
{\tt counter1}. It is thus a {\em bijection}.

Another way to define a type isomorphic to {\tt int} would be:
\begin{caml_example}
type counter2 = {Counter: int};;
\end{caml_example}
The types {\tt counter1} and {\tt counter2} are isomorphic to {\tt int}.
They are at the same time sum and product types.
Their pattern-matching does not perform any run-time test.

The possibility of defining arbitrary complex data types permits the easy
manipulation of abstract syntax trees in Caml (such as the \verb"arith_expr"
type above). These abstract syntax trees are supposed to represent programs of
a language (e.g. a language of arithmetic expressions). These kind of
languages which are defined in Caml are called {\em object-languages} and
Caml is said to be their {\em metalanguage}.

\section{Summary}

\begin{itemize}
\item New types can be introduced in Caml.
\item Types may be {\em parameterized} by type variables. The syntax of type
parameters is:
\begin{verbatim}
<params> ::
          | <tvar>
          | ( <tvar> [, <tvar>]* )
\end{verbatim}
\item Types can be {\em recursive}.

\item Product types:
  \begin{itemize}
  \item Mathematical product of several types.
  \item The construct is:
     \begin{verbatim}
     type <params> <tname> =
                      {<Field>: <type>; ...}
     \end{verbatim}
     where the \verb"<type>"s may contain type variables appearing in
     \verb"<params>".
  \end{itemize}
\item Sum types:
  \begin{itemize}
  \item Mathematical disjoint union of several types.
  \item The construct is:
     \begin{verbatim}
     type <params> <tname> =
              <Injection> [of <type>] | ...
     \end{verbatim}
     where the \verb"<type>"s may contain type variables appearing in
     \verb"<params>".
  \end{itemize}
\end{itemize}

\section*{Exercises}

\begin{exo}\label{Types:1}
Define a function taking as argument a binary tree and returning a pair of
lists: the first one contains all operators of the tree, the second
one contains all its leaves.
\end{exo}
\begin{exo}\label{Types:2}
Define a function \verb"map_btree" analogous to the {\tt map} function
on lists.
The function \verb"map_btree" should take as arguments two functions {\tt f}
and {\tt g}, and a binary tree. It should return a new binary tree whose
leaves are the result of applying {\tt f} to the leaves of the tree
argument, and whose operators are the results of applying the {\tt g}
function to the operators of the argument.
\end{exo}
\begin{exo}\label{Types:3}
We can associate to the {\tt list} type definition an canonical
iterator in the following way. We give a functional interpretation
to the data constructors of the {\tt list} type.

We change the list constructors {\tt []} and {\tt ::} respectively
into a constant {\tt a} and an operator $\oplus$ (used as a prefix
identifier), and abstract with respect to these two operators, obtaining
the list iterator satisfying:
\begin{quote}
\tt list\_it $\oplus$ a [] = a\\
\tt list\_it $\oplus$ a ($x_1$::\ldots::$x_n$::[]) = $x_1$ $\oplus$ (\ldots $\oplus$ ($x_n$ $\oplus$ a)\ldots)
\end{quote}
Its Caml definition would be:
\begin{caml_example}
let rec list_it f a = 
        function [] -> a
               | x::l -> f x (list_it f a l);;
\end{caml_example}
As an example, the application of \verb|it_list| to the functional
composition and to its neutral element (the identity function),
computes the composition of lists of functions (try it!).

Define, using the same method, a canonical iterator over binary trees.
\end{exo}
