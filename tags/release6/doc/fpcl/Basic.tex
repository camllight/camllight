\chapter{Basic types}
\label{c:basic}

We examine in this chapter the Caml basic types.

\section{Numbers}

Caml Light provides two numeric types: integers (type \verb"int") and
floating-point numbers (type \verb"float"). Integers are limited to
the range $-2^{30} \ldots 2^{30}-1$. Integer arithmetic is taken modulo
$2^{31}$; that is, an integer operation that overflows does not raise an
error, but the result simply wraps around. Predefined operations
(functions) on integers include:
\begin{center}
\begin{tabular}{rl}
{\tt +} &addition\\
{\tt -} &subtraction and unary minus\\
{\tt *} &multiplication\\
{\tt /} &division\\
{\tt mod} &modulo
\end{tabular}
\end{center}

Some examples of expressions:
\begin{caml_example}
3 * 4 + 2;;
3 * (4 + 2);;
3 - 7 - 2;;
\end{caml_example}
There are precedence rules that make \verb"*" bind tighter than \verb"+", and so on. In doubt, write extra parentheses.
\par
So far, we have not seen the type of these arithmetic operations.
They all expect  two integer arguments; so, they are functions taking
one integer and returning a function from integers to integers.
The (functional) value of such infix identifiers may be obtained by taking
their {\em prefix} version.\ikwd{prefix@\verb`prefix`}
\begin{caml_example}
prefix +;;
prefix *;;
prefix mod;;
\end{caml_example}

As shown by their types, the infix operators \verb"+", \verb"*",
\ldots, do not work on floating-point values. A separate set of
floating-point arithmetic operations is provided:

\begin{center}
\begin{tabular}{rl}
{\tt +.} &addition\\
{\tt -.} &subtraction and unary minus\\
{\tt *.} &multiplication\\
{\tt /.} &division\\
\verb"sqrt" & square root \\
\verb"exp", \verb"log" & exponentiation and logarithm \\
\verb"sin", \verb"cos",\ldots & usual trigonometric operations
\end{tabular}
\end{center}

We have two conversion functions to go back and forth between integers
and floating-point numbers: \verb"int_of_float" and
\verb"float_of_int".

\begin{caml_example}
1 + 2.3;;
float_of_int 1 +. 2.3;;
\end{caml_example}

Let us now give some examples of numerical functions. We start by
defining some very simple functions on numbers: 
\begin{caml_example}
let square x = x *. x;;
square 2.0;;
square (2.0 /. 3.0);;
let sum_of_squares (x,y) = square(x) +. square(y);;
let half_pi = 3.14159 /. 2.0
in sum_of_squares(cos(half_pi), sin(half_pi));;
\end{caml_example}

We now develop a classical example: the computation of the root of a
function by Newton's method.
Newton's method can be stated as follows: if $y$ is an approximation to a
root of a function $f$, then:
\[
y - \frac{f(y)}{f'(y)}
\]
is a better approximation, where $f'(y)$ is the derivative of $f$
evaluated at $y$.
For example, with $f(x)=x^2 -a$, we have $f'(x)=2x$, and therefore:
\[
y-\frac{f(y)}{f'(y)} = y - \frac{y^2 -a}{2y} = \frac{y+\frac{a}{y}}{2}
\]

We can define a function {\tt deriv} for approximating the derivative of a
function at a given point by:
\begin{caml_example}
let deriv f x dx = (f(x+.dx) -. f(x)) /. dx;;
\end{caml_example}
Provided \verb|dx| is sufficiently small, this gives a reasonable estimate
of the derivative of $f$ at $x$.

The following function returns the absolute value of its floating
point number argument.  It uses the ``{\tt if} \ldots {\tt then}
\ldots {\tt else}''\ikwd{if@\verb`if`} conditional construct.
\begin{caml_example}
let abs x = if x >. 0.0 then x else -. x;;
\end{caml_example}
The main function, given below, uses three local functions. The first
one, {\tt until}, is an example of a {\em recursive} function: it
calls itself in its body, and is defined using a {\tt let rec}
construct ({\tt rec} shows that the definition is recursive). It takes
three arguments: a predicate {\tt p} on floats, a function {\tt
change} from floats to floats, and a float {\tt x}. If \verb|p(x)| is
false, then \verb"until" is called with last argument
\verb|change(x)|, otherwise, {\tt x} is the result of the whole call.
We will study recursive functions more closely later. The two other
auxiliary functions, {\tt satisfied} and {\tt improve}, take a float
as argument and deliver respectively a boolean value and a float. The
function {\tt satisfied} asks wether the image of its argument by {\tt
f} is smaller than {\tt epsilon} or not, thus testing wether {\tt y}
is close enough to a root of {\tt f}. The function {\tt improve}
computes the next approximation using the formula given below. The
three local functions are defined using a cascade of {\tt let}
constructs. The whole function is:
\begin{caml_example}
let newton f epsilon =
  let rec until p change x =
            if p(x) then x
            else until p change (change(x)) in
  let satisfied y = abs(f y) <. epsilon in
  let improve y = y -. (f(y) /. (deriv f y epsilon))
in until satisfied improve;;
\end{caml_example}
Some examples of equation solving:
\begin{caml_example}
let square_root x epsilon =
           newton (function y -> y*.y -. x) epsilon x
and cubic_root x epsilon =
           newton (function y -> y*.y*.y -. x) epsilon x;;
square_root 2.0 1e-5;;
cubic_root 8.0 1e-5;;
2.0 *. (newton cos 1e-5 1.5);;
\end{caml_example}

\section{Boolean values}

The presence of the conditional construct implies the presence of
boolean values.  The type \verb"bool" is composed of two values
\verb"true" and \verb"false".
\begin{caml_example}
true;;
false;;
\end{caml_example}
The functions with results of type \verb"bool" are often called {\em
predicates}.
Many predicates are predefined in Caml. Here are some of them:
\begin{caml_example}
prefix <;;
1 < 2;;
prefix <.;;
3.14159 <. 2.718;;
\end{caml_example}
There exist also \verb"<=", \verb">", \verb">=", and similarly
\verb"<=.", \verb">.", \verb">=.".

\subsection{Equality}

Equality has a special status in functional languages because of
functional values. It is easy to test equality of values of base types
(integers, floating-point numbers, booleans, \ldots):
\begin{caml_example}
1 = 2;;
false = (1>2);;
\end{caml_example}
But it is impossible, in the general case, to decide the equality of
functional values. Hence, equality stops on a run-time error when
it encounters functional values.
\begin{caml_example}
(fun x -> x) = (fun x -> x);;
\end{caml_example}
Since equality may be used on values of any type, what is its type?
Equality takes two arguments of the same type (whatever type it
is) and returns a boolean value.  The type of equality is a {\em
polymorphic type}, i.e. may take several possible forms.  Indeed, when
testing equality of two numbers, then its type is
\verb|int -> int -> bool|, and when testing equality of strings, its type is
\verb|string -> string -> bool|.
\begin{caml_example}
prefix =;;
\end{caml_example}
The type of equality uses {\em type variables}, written \verb|'a|,
\verb|`b|, etc. Any type can be substituted to type variables in order
to produces specific {\em instances} of types. For example,
substituting \verb|int| for \verb|'a| above produces
\verb|int -> int -> bool|, which is the type of the equality function
used on integer values.
\begin{caml_example}
1=2;;
(1,2) = (2,1);;
1 = (1,2);;
\end{caml_example}

\subsection{Conditional}
\ikwd{if@\verb`if`}

Conditional expressions are of the form ``${\tt
if}~e_{\rm{test}}~{\tt then}~e_1~{\tt else}~e_2$'', where
$e_{\rm{test}}$ is an expression of type {\tt bool} and $e_1$, $e_2$
are expressions possessing the same type. As an example, the logical
negation could be written:
\begin{caml_example}
let negate a = if a then false else true;;
negate (1=2);;
\end{caml_example}


\subsection{Logical operators}
\ikwd{or@\verb`or`}

The classical logical operators are available in Caml. Disjunction and
conjunction are respectively written {\tt or} and \verb|&|:
\begin{caml_example}
true or false;;
(1<2) & (2>1);;
\end{caml_example}
The operators \verb"&" and \verb"or" are not functions. They should
not be seen as regular functions, since they evaluate their second
argument only if it is needed. For example, the \verb"or" operator
evaluates its second operand only if the first one evaluates to
\verb"false".
The behavior of these operators may be described as follows:
\begin{center}
\begin{tabular}{lcl}
$e_1$ {\tt or} $e_2$ & \mbox{is equivalent to} &
                {\tt if} $e_1$ {\tt then true  else} $e_2$\\
$e_1$ \verb"&" $e_2$ & \mbox{is equivalent to} &
                {\tt if} $e_1$ {\tt then} $e_2$ {\tt else false}
\end{tabular}
\end{center}
Negation is predefined:\ikwd{not@\verb`not`}
\begin{caml_example}
not true;;
\end{caml_example}
The \verb|not| identifier receives a special treatment from the
parser: the application ``{\tt not f x}'' is recognized as ``{\tt not
(f x)}'' while ``{\tt f g x}'' is identical to ``{\tt (f g) x}''. This
special treatment explains why the functional value associated to {\tt not}
can be obtained only using the {\tt prefix} keyword:
\begin{caml_example}
prefix not;;
\end{caml_example}

\section{Strings and characters}

String constants (type \verb"string") are written between \verb|"|
characters (double-quotes). Single-character constants (type
\verb"char") are written between \verb|`| characters (backquotes).
The most used string operation is string concatenation, denoted by the
\verb"^" character.
\begin{caml_example}
"Hello" ^ " World!";;
prefix ^;;
\end{caml_example}
Characters are ASCII characters:
\begin{caml_example}
`a`;;
`\065`;;
\end{caml_example}
and can be built from their ASCII code as in:
\begin{caml_example}
char_of_int 65;;
\end{caml_example}
Other operations are available (\verb"sub_string", \verb|int_of_char|,
etc). See the Caml Light referencee manual \cite{CamlLightDoc} for
complete information.


\section{Tuples}

\subsection{Constructing tuples}

It is possible to combine values into tuples (pairs, triples, \ldots).
The {\em value constructor} for
tuples is the ``{\tt,}'' character (the comma). We will often use parentheses
around tuples in order to improve readability, but they are not strictly
necessary.
\begin{caml_example}
1,2;;
1,2,3,4;;
let p = (1+2, 1<2);;
\end{caml_example}
The infix ``{\tt *}'' identifier is the {\em type constructor} for tuples.
For instance, $t_1 {\tt *} t_2$ corresponds to the mathematical cartesian product of types $t_1$ and $t_2$.

We can build tuples from any values: the tuple value constructor is {\em
generic}.

\subsection{Extracting pair components}

{\em Projection} functions are used to extract components of tuples. For pairs, we have:
\begin{caml_example}
fst;;
snd;;
\end{caml_example}
They have polymorphic types, of course, since they may be applied to
any kind of pair.  They are predefined in the Caml system, but could
be defined by the user (in fact, the cartesian product itself could be
defined --- see section \ref{s:udefprodtypes}, dedicated to user-defined
product types):
\begin{caml_example}
let first (x,y) = x
and second (x,y) = y;;
first p;;
second p;;
\end{caml_example}
We say that \verb|first| is a {\em generic} value because it works
uniformly on several kinds of arguments (provided they are pairs). We
often confuse between ``generic'' and ``polymorphic'', saying that
such value is polymorphic instead of generic.


\section{Patterns and pattern-matching}

Patterns and pattern-matching play an important role in ML languages.
They appear in all real programs and are strongly related to types
(predefined or user-defined).

A {\em pattern} indicates the {\em shape} of a value.
Patterns are ``values with holes''.
A single variable (formal parameter) is a pattern (with no shape specified:
it matches any value).
When a value is {\em matched against} a pattern (this is called
{\em pattern-matching}), the pattern acts as a filter.
We already used patterns and pattern-matching in all the functions we wrote:
the function body {\tt (function x -> ...)} does (trivial) pattern-matching.
When applied to a value, the formal parameter {\tt x} gets bound to
that value.
The function body {\tt (function (x,y) -> x+y)} does also pattern-matching:
when applied to a value (a pair, because of well-typing hypotheses), the
{\tt x} and {\tt y} get bound respectively to the first and the second
component of that pair.
\par
All these pattern-matching examples were trivial ones, they did not involve any test:
\begin{itemize}
\item formal parameters such as {\tt x} do not impose any particular shape to the
value they are supposed to match;
\item pair patterns such as {\tt(x,y)} always match for typing reasons
(cartesian product is a {\em product type}).
\end{itemize}
However, some types do not guarantee such a uniform shape to their
values. Consider the {\tt bool} type: an element of type {\tt bool} is
either {\tt true} or {\tt false}.
%
%
If we impose to a value of type {\tt bool} to have the shape of {\tt true},
then pattern-matching may fail. Consider the following function:
\begin{caml_example}
let f = function true -> false;;
\end{caml_example}
The compiler warns us that the pattern-matching may fail (we did not
consider the {\tt false} case).

Thus, if we apply {\tt f} to a value that evaluates to {\tt true},
pattern-matching will succeed (an equality test is performed during
execution).
\begin{caml_example}
f (1<2);;
\end{caml_example}
But, if {\tt f}'s argument evaluates to {\tt false}, a run-time
error is reported:
\begin{caml_example}
f (1>2);;
\end{caml_example}
Here is a correct definition using pattern-matching:
\begin{caml_example}
let negate = function true -> false
                    | false -> true;;
\end{caml_example}
The pattern-matching has now two cases, separated by the ``{\verb"|"}''
character.
Cases are examined in turn, from top to bottom. An equivalent definition of
{\tt negate} would be:
\begin{caml_example}
let negate = function true -> false
                    | x -> true;;
\end{caml_example}
The second case now matches any boolean value (in fact, only {\tt false} since
{\tt true} has been caught by the first match case). When the second case is
chosen, the identifier {\tt x} gets bound to the argument of {\tt
negate}, and could be used in the right-hand part of the match case.
Since in our example we do not use the value of the argument in the
right-hand part of the second match case, another equivalent
definition of {\tt negate} would be:
\begin{caml_example}
let negate = function true -> false
                    | _ -> true;;
\end{caml_example}
Where ``\verb"_"'' acts as a formal paramenter (matches any value), but does
not introduce any binding: it should be read as ``anything else''.

As an example of pattern-matching, consider the following function
definition (truth value table of implication):
\begin{caml_example}
let imply = function (true,true) -> true
                   | (true,false) -> false
                   | (false,true) -> true
                   | (false,false) -> true;;
\end{caml_example}
Here is another way of defining {\tt imply}, by using variables:
\begin{caml_example}
let imply = function (true,x) -> x
                   | (false,x) -> true;;
\end{caml_example}
Simpler, and simpler again:
\begin{caml_example}
let imply = function (true,x) -> x
                   | (false,_) -> true;;
let imply = function (true,false) -> false
                   | _ -> true;;
\end{caml_example}
Pattern-matching is allowed on any type of value (non-trivial
pattern-matching is only possible on types with {\em data constructors}).

For example:
\begin{caml_example}
let is_zero = function 0 -> true | _ -> false;;
let is_yes = function "oui" -> true
                    | "si" -> true
                    | "ya" -> true
                    | "yes" -> true
                    | _ -> false;;
\end{caml_example}

\section{Functions}
\ikwd{function@\verb`function`}

The type constructor ``\verb"->"'' is predefined and cannot be defined
in ML's type system.
We shall explore in this section some further aspects of functions
and functional types.

\subsection{Functional composition}

Functional composition is easily definable in Caml. It is of course a
polymorphic function:
\begin{caml_example}
let compose f g = function x -> f (g (x));;
\end{caml_example}
The type of {\tt compose} contains no more constraints than the ones
appearing in the definition: in a sense, it is the {\em most general} type
compatible with these constraints.

These constraints are:
\begin{itemize}
\item the codomain of {\tt g} and the domain of {\tt f} must be the same;
\item {\tt x} must belong to the domain of {\tt g};
\item {\tt compose f g x} will belong to the codomain of {\tt f}.
\end{itemize}


Let's see \verb"compose" at work:
\begin{caml_example}
let succ x = x+1;;
compose succ list_length;;
(compose succ list_length) [1;2;3];;
\end{caml_example}

\subsection{Currying}
We can define two versions of the addition function:
\begin{caml_example}
let plus = function (x,y) -> x+y;;
let add = function x -> (function y -> x+y);;
\end{caml_example}
These two functions differ only in their way of taking their arguments.
The first one will take an argument belonging to a cartesian product,
the second one will take a number and return a function.
The {\tt add} function is said to be {\em the curried version} of {\tt
plus} (in honor of the logician Haskell Curry).


The currying transformation may be written in Caml under the form of a
higher-order function:
\begin{caml_example}
let curry f = function x -> (function y -> f(x,y));;
\end{caml_example}
Its inverse function may be defined by:
\begin{caml_example}
let uncurry f = function (x,y) -> f x y;;
\end{caml_example}
We may check the types of {\tt curry} and {\tt uncurry} on particular
examples:
\begin{caml_example}
uncurry (prefix +);;
uncurry (prefix ^);;
curry plus;;
\end{caml_example}

\section*{Exercises}
%
\begin{exo}\label{Basic:1}
Define functions that compute the surface area and the volume of
well-known geometric objects (rectangles, circles, spheres, \ldots).
\end{exo}
%
\begin{exo}\label{Basic:2}
What would happen in a language submitted to call-by-value with
recursion if there was no conditional construct ({\tt if ... then ... else ...})?
\end{exo}
\begin{exo}\label{Basic:3}
Define the {\tt factorial} function such that:
\[{\tt factorial}~n = n * (n-1) * \ldots * 2 * 1\]
Give two versions of {\tt factorial}: recursive and tail recursive.
\end{exo}
\begin{exo}\label{Basic:5}
Define the {\tt fibonacci} function that, when given a number $n$,
returns the $n$th Fibonacci number, i.e. the $n$th term $u_n$ of the
sequence defined by:
\[\begin{array}{l}
u_1 = 1\\
u_2 = 1\\
u_{n+2} = u_{n+1} + u_n
\end{array}\]
\end{exo}
\begin{exo}\label{Basic:6}
What are the types of the following expressions?
\begin{itemize}
\item {\tt uncurry compose}
\item {\tt compose curry uncurry}
\item {\tt compose uncurry curry}
\end{itemize}
\end{exo}
