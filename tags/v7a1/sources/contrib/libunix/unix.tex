\mysection{{\tt unix}: interface to the Unix system }

\label{s:unix}
\index{unix (module)@\verb`unix` (module)}%

\subsection*{Error report }\begin{verbatim}
type error =
    ENOERR
  | EPERM               (* Not owner *)
  | ENOENT              (* No such file or directory *)
  | ESRCH               (* No such process *)
  | EINTR               (* Interrupted system call *)
  | EIO                 (* I/O error *)
  | ENXIO               (* No such device or address *)
  | E2BIG               (* Arg list too long *)
  | ENOEXEC             (* Exec format error *)
  | EBADF               (* Bad file number *)
  | ECHILD              (* No children *)
  | EAGAIN              (* No more processes *)
  | ENOMEM              (* Not enough core *)
  | EACCES              (* Permission denied *)
  | EFAULT              (* Bad address *)
  | ENOTBLK             (* Block device required *)
  | EBUSY               (* Mount device busy *)
  | EEXIST              (* File exists *)
  | EXDEV               (* Cross-device link *)
  | ENODEV              (* No such device *)
  | ENOTDIR             (* Not a directory*)
  | EISDIR              (* Is a directory *)
  | EINVAL              (* Invalid argument *)
  | ENFILE              (* File table overflow *)
  | EMFILE              (* Too many open files *)
  | ENOTTY              (* Not a typewriter *)
  | ETXTBSY             (* Text file busy *)
  | EFBIG               (* File too large *)
  | ENOSPC              (* No space left on device *)
  | ESPIPE              (* Illegal seek *)
  | EROFS               (* Read-only file system *)
  | EMLINK              (* Too many links *)
  | EPIPE               (* Broken pipe *)
  | EDOM                (* Argument too large *)
  | ERANGE              (* Result too large *)
  | EWOULDBLOCK         (* Operation would block *)
  | EINPROGRESS         (* Operation now in progress *)
  | EALREADY            (* Operation already in progress *)
  | ENOTSOCK            (* Socket operation on non-socket *)
  | EDESTADDRREQ        (* Destination address required *)
  | EMSGSIZE            (* Message too long *)
  | EPROTOTYPE          (* Protocol wrong type for socket *)
  | ENOPROTOOPT         (* Protocol not available *)
  | EPROTONOSUPPORT     (* Protocol not supported *)
  | ESOCKTNOSUPPORT     (* Socket type not supported *)
  | EOPNOTSUPP          (* Operation not supported on socket *)
  | EPFNOSUPPORT        (* Protocol family not supported *)
  | EAFNOSUPPORT        (* Address family not supported by protocol family *)
  | EADDRINUSE          (* Address already in use *)
  | EADDRNOTAVAIL       (* Can't assign requested address *)
  | ENETDOWN            (* Network is down *)
  | ENETUNREACH         (* Network is unreachable *)
  | ENETRESET           (* Network dropped connection on reset *)
  | ECONNABORTED        (* Software caused connection abort *)
  | ECONNRESET          (* Connection reset by peer *)
  | ENOBUFS             (* No buffer space available *)
  | EISCONN             (* Socket is already connected *)
  | ENOTCONN            (* Socket is not connected *)
  | ESHUTDOWN           (* Can't send after socket shutdown *)
  | ETOOMANYREFS        (* Too many references: can't splice *)
  | ETIMEDOUT           (* Connection timed out *)
  | ECONNREFUSED        (* Connection refused *)
  | ELOOP               (* Too many levels of symbolic links *)
  | ENAMETOOLONG        (* File name too long *)
  | EHOSTDOWN           (* Host is down *)
  | EHOSTUNREACH        (* No route to host *)
  | ENOTEMPTY           (* Directory not empty *)
  | EPROCLIM            (* Too many processes *)
  | EUSERS              (* Too many users *)
  | EDQUOT              (* Disc quota exceeded *)
  | ESTALE              (* Stale NFS file handle *)
  | EREMOTE             (* Too many levels of remote in path *)
  | EIDRM               (* Identifier removed *)
  | EDEADLK             (* Deadlock condition. *)
  | ENOLCK              (* No record locks available. *)
  | ENOSYS              (* Function not implemented *)
  | EUNKNOWNERR
\end{verbatim}
\begin{comment}
 The type of error codes. 
\end{comment}
\begin{verbatim}
exception Unix_error of error * string * string
\end{verbatim}
\index{Unix_error (exception)@\verb`Unix_error` (exception)}%
\begin{comment}
 Raised by the system calls below when an error is encountered.
           The first component is the error code; the second component
           is the function name; the third component is the string parameter
           to the function, if it has one, or the empty string otherwise. 
\end{comment}
\begin{verbatim}
value error_message : error -> string
\end{verbatim}
\index{error_message@\verb`error_message`}%
\begin{comment}
 Return a string describing the given error code. 
\end{comment}
\begin{verbatim}
value handle_unix_error : ('a -> 'b) -> 'a -> 'b
\end{verbatim}
\index{handle_unix_error@\verb`handle_unix_error`}%
\begin{comment}
 \verbhandle_unix_error f x applies \verbf to \verbx and returns the result.
           If the exception \verbUnix_error is raised, it prints a message
           describing the error and exits with code 2. 
\end{comment}
\subsection*{Interface with the parent process }\begin{verbatim}
value environment : unit -> string vect
\end{verbatim}
\index{environment@\verb`environment`}%
\begin{comment}
 Return the process environment, as an array of strings
           with the format ``variable=value''. See also \verbsys__getenv. 
\end{comment}
\subsection*{Process handling }\begin{verbatim}
type process_status =
    WEXITED of int
  | WSIGNALED of int * bool
  | WSTOPPED of int
\end{verbatim}
\begin{comment}
 The termination status of a process. \verbWEXITED means that the
           process terminated normally by \verbexit; the argument is the return
           code. \verbWSIGNALED means that the process was killed by a signal;
           the first argument is the signal number, the second argument
           indicates whether a ``core dump'' was performed. \verbWSTOPPED means
           that the process was stopped by a signal; the argument is the
           signal number. 
\end{comment}
\begin{verbatim}
type wait_flag =
    WNOHANG
  | WUNTRACED
\end{verbatim}
\begin{comment}
 Flags for \verbwaitopt and \verbwaitpid.
           \verbWNOHANG means do not block if no child has
           died yet, but immediately return with a pid equal to 0.
           \verbWUNTRACED means report also the children that receive stop
           signals. 
\end{comment}
\begin{verbatim}
value execv : string -> string vect -> unit
\end{verbatim}
\index{execv@\verb`execv`}%
\begin{comment}
 \verbexecv prog args execute the program in file \verbprog, with
           the arguments \verbargs, and the current process environment. 
\end{comment}
\begin{verbatim}
value execve : string -> string vect -> string vect -> unit
\end{verbatim}
\index{execve@\verb`execve`}%
\begin{comment}
 Same as \verbexecv, except that the third argument provides the
           environment to the program executed. 
\end{comment}
\begin{verbatim}
value execvp : string -> string vect -> unit
\end{verbatim}
\index{execvp@\verb`execvp`}%
\begin{comment}
 Same as \verbexecv, except that the program is searched in the path. 
\end{comment}
\begin{verbatim}
value fork : unit -> int
\end{verbatim}
\index{fork@\verb`fork`}%
\begin{comment}
 Fork a new process. The returned integer is 0 for the child
           process, the pid of the child process for the parent process. 
\end{comment}
\begin{verbatim}
value wait : unit -> int * process_status
\end{verbatim}
\index{wait@\verb`wait`}%
\begin{comment}
 Wait until one of the children processes die, and return its pid
           and termination status. 
\end{comment}
\begin{verbatim}
value waitopt : wait_flag list -> int * process_status
\end{verbatim}
\index{waitopt@\verb`waitopt`}%
\begin{comment}
 Same as \verbwait, but takes a list of options to avoid blocking,
           or report also stopped children. 
\end{comment}
\begin{verbatim}
value waitpid : wait_flag list -> int -> process_status
\end{verbatim}
\index{waitpid@\verb`waitpid`}%
\begin{comment}
 Same as \verbwaitopt, but waits for the process whose pid is given. 
\end{comment}
\begin{verbatim}
value system : string -> process_status
\end{verbatim}
\index{system@\verb`system`}%
\begin{comment}
 Execute the given command, wait until it terminates, and return
           its termination status. The string is interpreted by the shell
           \verb/bin/sh and therefore can contain redirections, quotes, variables,
           etc. The result \verbWEXITED 127 indicates that the shell couldn't
           be executed. 
\end{comment}
\begin{verbatim}
value getpid : unit -> int
\end{verbatim}
\index{getpid@\verb`getpid`}%
\begin{comment}
 Return the pid of the process. 
\end{comment}
\begin{verbatim}
value getppid : unit -> int
\end{verbatim}
\index{getppid@\verb`getppid`}%
\begin{comment}
 Return the pid of the parent process. 
\end{comment}
\begin{verbatim}
value nice : int -> int
\end{verbatim}
\index{nice@\verb`nice`}%
\begin{comment}
 Change the process priority. The integer argument is added to the
           ``nice'' value. (Higher values of the ``nice'' value mean
           lower priorities.) Return the new nice value. 
\end{comment}
\subsection*{Basic file input/output }\begin{verbatim}
type file_descr
\end{verbatim}
\begin{comment}
 The abstract type of file descriptors. 
\end{comment}
\begin{verbatim}
value stdin : file_descr
value stdout : file_descr
value stderr : file_descr
\end{verbatim}
\index{stdin@\verb`stdin`}%
\index{stdout@\verb`stdout`}%
\index{stderr@\verb`stderr`}%
\begin{comment}
 File descriptors for standard input, standard output and
           standard error. 
\end{comment}
\begin{verbatim}
type open_flag =
    O_RDONLY                            (* Open for reading *)
  | O_WRONLY                            (* Open for writing *)
  | O_RDWR                              (* Open for reading and writing *)
  | O_NDELAY                            (* Open in non-blocking mode *)
  | O_APPEND                            (* Open for append *)
  | O_CREAT                             (* Create if nonexistent *)
  | O_TRUNC                             (* Truncate to 0 length if existing *)
  | O_EXCL                              (* Fail if existing *)
\end{verbatim}
\begin{comment}
 The flags to \verbopen. 
\end{comment}
\begin{verbatim}
type file_perm == int
\end{verbatim}
\begin{comment}
 The type of file access rights. 
\end{comment}
\begin{verbatim}
value open : string -> open_flag list -> file_perm -> file_descr
\end{verbatim}
\index{open@\verb`open`}%
\begin{comment}
 Open the named file with the given flags. Third argument is
           the permissions to give to the file if it is created. Return
           a file descriptor on the named file. 
\end{comment}
\begin{verbatim}
value close : file_descr -> unit
\end{verbatim}
\index{close@\verb`close`}%
\begin{comment}
 Close a file descriptor. 
\end{comment}
\begin{verbatim}
value read : file_descr -> string -> int -> int -> int
\end{verbatim}
\index{read@\verb`read`}%
\begin{comment}
 \verbread fd buff start len reads \verblen characters from descriptor
           \verbfd, storing them in string \verbbuff, starting at position \verbofs
           in string \verbbuff. Return the number of characters actually read. 
\end{comment}
\begin{verbatim}
value write : file_descr -> string -> int -> int -> int
\end{verbatim}
\index{write@\verb`write`}%
\begin{comment}
 \verbwrite fd buff start len writes \verblen characters to descriptor
           \verbfd, taking them from string \verbbuff, starting at position \verbofs
           in string \verbbuff. Return the number of characters actually
           written. 
\end{comment}
\subsection*{Interfacing with the standard input/output library (module io). }\begin{verbatim}
value in_channel_of_descr : file_descr -> in_channel
\end{verbatim}
\index{in_channel_of_descr@\verb`in_channel_of_descr`}%
\begin{comment}
 Create an input channel reading from the given descriptor. 
\end{comment}
\begin{verbatim}
value out_channel_of_descr : file_descr -> out_channel
\end{verbatim}
\index{out_channel_of_descr@\verb`out_channel_of_descr`}%
\begin{comment}
 Create an output channel writing on the given descriptor. 
\end{comment}
\begin{verbatim}
value descr_of_in_channel : in_channel -> file_descr
\end{verbatim}
\index{descr_of_in_channel@\verb`descr_of_in_channel`}%
\begin{comment}
 Return the descriptor corresponding to an input channel. 
\end{comment}
\begin{verbatim}
value descr_of_out_channel : out_channel -> file_descr
\end{verbatim}
\index{descr_of_out_channel@\verb`descr_of_out_channel`}%
\begin{comment}
 Return the descriptor corresponding to an output channel. 
\end{comment}
\subsection*{Seeking and truncating }\begin{verbatim}
type seek_command =
    SEEK_SET
  | SEEK_CUR
  | SEEK_END
\end{verbatim}
\begin{comment}
 Positioning modes for \verblseek. \verbSEEK_SET indicates positions
           relative to the beginning of the file, \verbSEEK_CUR relative to
           the current position, \verbSEEK_END relative to the end of the
           file. 
\end{comment}
\begin{verbatim}
value lseek : file_descr -> int -> seek_command -> int
\end{verbatim}
\index{lseek@\verb`lseek`}%
\begin{comment}
 Set the current position for a file descriptor 
\end{comment}
\begin{verbatim}
value truncate : string -> int -> unit
\end{verbatim}
\index{truncate@\verb`truncate`}%
\begin{comment}
 Truncates the named file to the given size. 
\end{comment}
\begin{verbatim}
value ftruncate : file_descr -> int -> unit
\end{verbatim}
\index{ftruncate@\verb`ftruncate`}%
\begin{comment}
 Truncates the file corresponding to the given descriptor
           to the given size. 
\end{comment}
\subsection*{File statistics }\begin{verbatim}
type file_kind =
    S_REG                               (* Regular file *)
  | S_DIR                               (* Directory *)
  | S_CHR                               (* Character device *)
  | S_BLK                               (* Block device *)
  | S_LNK                               (* Symbolic link *)
  | S_FIFO                              (* Named pipe *)
  | S_SOCK                              (* Socket *)
type stats =
  { st_dev : int;                       (* Device number *)
    st_ino : int;                       (* Inode number *)
    st_kind : file_kind;                (* Kind of the file *)
    st_perm : file_perm;                (* Access rights *)
    st_nlink : int;                     (* Number of links *)
    st_uid : int;                       (* User id of the owner *)
    st_gid : int;                       (* Group id of the owner *)
    st_rdev : int;                      (* Device minor number *)
    st_size : int;                      (* Size in bytes *)
    st_atime : int;                     (* Last access time *)
    st_mtime : int;                     (* Last modification time *)
    st_ctime : int }                    (* Last status change time *)
\end{verbatim}
\begin{comment}
 The informations returned by the \verbstat calls. 
\end{comment}
\begin{verbatim}
value stat : string -> stats
\end{verbatim}
\index{stat@\verb`stat`}%
\begin{comment}
 Return the information for the named file. 
\end{comment}
\begin{verbatim}
value lstat : string -> stats
\end{verbatim}
\index{lstat@\verb`lstat`}%
\begin{comment}
 Same as \verbstat, but in case the file is a symbolic link,
           return the information for the link itself. 
\end{comment}
\begin{verbatim}
value fstat : file_descr -> stats
\end{verbatim}
\index{fstat@\verb`fstat`}%
\begin{comment}
 Return the information for the file associated with the given
           descriptor. 
\end{comment}
\subsection*{Operations on file names }\begin{verbatim}
value unlink : string -> unit
\end{verbatim}
\index{unlink@\verb`unlink`}%
\begin{comment}
 Removes the named file 
\end{comment}
\begin{verbatim}
value rename : string -> string -> unit
\end{verbatim}
\index{rename@\verb`rename`}%
\begin{comment}
 \verbrename old new changes the name of a file from \verbold to \verbnew. 
\end{comment}
\begin{verbatim}
value link : string -> string -> unit
\end{verbatim}
\index{link@\verb`link`}%
\begin{comment}
 \verblink source dest creates a hard link named \verbdest to the file
           named \verbnew. 
\end{comment}
\subsection*{File permissions and ownership }\begin{verbatim}
type access_permission =
    R_OK                                (* Read permission *)
  | W_OK                                (* Write permission *)
  | X_OK                                (* Execution permission *)
  | F_OK                                (* File exists *)
\end{verbatim}
\begin{comment}
 Flags for the \verbaccess call. 
\end{comment}
\begin{verbatim}
value chmod : string -> file_perm -> unit
\end{verbatim}
\index{chmod@\verb`chmod`}%
\begin{comment}
 Change the permissions of the named file. 
\end{comment}
\begin{verbatim}
value fchmod : file_descr -> file_perm -> unit
\end{verbatim}
\index{fchmod@\verb`fchmod`}%
\begin{comment}
 Change the permissions of an opened file. 
\end{comment}
\begin{verbatim}
value chown : string -> int -> int -> unit
\end{verbatim}
\index{chown@\verb`chown`}%
\begin{comment}
 Change the owner uid and owner gid of the named file. 
\end{comment}
\begin{verbatim}
value fchown : file_descr -> int -> int -> unit
\end{verbatim}
\index{fchown@\verb`fchown`}%
\begin{comment}
 Change the owner uid and owner gid of an opened file. 
\end{comment}
\begin{verbatim}
value umask : int -> int
\end{verbatim}
\index{umask@\verb`umask`}%
\begin{comment}
 Set the process creation mask, and return the previous mask. 
\end{comment}
\begin{verbatim}
value access : string -> access_permission list -> unit
\end{verbatim}
\index{access@\verb`access`}%
\begin{comment}
 Check that the process has the given permissions over the named
           file. Raise \verbUnix_error otherwise. 
\end{comment}
\subsection*{File descriptor hacking }\begin{verbatim}
value fcntl_int : file_descr -> int -> int -> int
\end{verbatim}
\index{fcntl_int@\verb`fcntl_int`}%
\begin{comment}
 Interface to \verbfcntl in the case where the argument is an
           integer. The first integer argument is the command code;
           the second is the integer parameter. 
\end{comment}
\begin{verbatim}
value fcntl_ptr : file_descr -> int -> string -> int
\end{verbatim}
\index{fcntl_ptr@\verb`fcntl_ptr`}%
\begin{comment}
 Interface to \verbfcntl in the case where the argument is a pointer.
           The integer argument is the command code. A pointer to the string
           argument is passed as argument to the command. The string argument
           is usually set up with the functions from modules \verbpeek and
           \verbpoke. 
\end{comment}
\subsection*{Directories }\begin{verbatim}
value mkdir : string -> file_perm -> unit
\end{verbatim}
\index{mkdir@\verb`mkdir`}%
\begin{comment}
 Create a directory with the given permissions. 
\end{comment}
\begin{verbatim}
value chdir : string -> unit
\end{verbatim}
\index{chdir@\verb`chdir`}%
\begin{comment}
 Change the process working directory. 
\end{comment}
\begin{verbatim}
value rmdir : string -> unit
\end{verbatim}
\index{rmdir@\verb`rmdir`}%
\begin{comment}
 Remove an empty directory. 
\end{comment}
\begin{verbatim}
type dir_handle
\end{verbatim}
\begin{comment}
 The type of descriptors over opened directories. 
\end{comment}
\begin{verbatim}
value opendir : string -> dir_handle
\end{verbatim}
\index{opendir@\verb`opendir`}%
\begin{comment}
 Open a descriptor on a directory 
\end{comment}
\begin{verbatim}
value readdir : dir_handle -> string
\end{verbatim}
\index{readdir@\verb`readdir`}%
\begin{comment}
 Return the next entry in a directory 
\end{comment}
\begin{verbatim}
value rewinddir : dir_handle -> unit
\end{verbatim}
\index{rewinddir@\verb`rewinddir`}%
\begin{comment}
 Reposition the descriptor to the beginning of the directory 
\end{comment}
\begin{verbatim}
value closedir : dir_handle -> unit
\end{verbatim}
\index{closedir@\verb`closedir`}%
\begin{comment}
 Close a directory descriptor. 
\end{comment}
\subsection*{Pipes and redirections }\begin{verbatim}
value pipe : unit -> file_descr * file_descr
\end{verbatim}
\index{pipe@\verb`pipe`}%
\begin{comment}
 Create a pipe. The first component of the result is opened
           for reading, that's the exit to the pipe. The second component is
           opened for writing, that's the entrace to the pipe. 
\end{comment}
\begin{verbatim}
value dup : file_descr -> file_descr
\end{verbatim}
\index{dup@\verb`dup`}%
\begin{comment}
 Duplicate a descriptor. 
\end{comment}
\begin{verbatim}
value dup2 : file_descr -> file_descr -> unit
\end{verbatim}
\index{dup2@\verb`dup2`}%
\begin{comment}
 \verbdup2 fd1 fd2 duplicates \verbfd1 to \verbfd2, closing \verbfd2 if already
           opened. 
\end{comment}
\begin{verbatim}
value open_process_in: string -> in_channel
value open_process_out: string -> out_channel
value open_process: string -> in_channel * out_channel
\end{verbatim}
\index{open_process_in@\verb`open_process_in`}%
\index{open_process_out@\verb`open_process_out`}%
\index{open_process@\verb`open_process`}%
\begin{comment}
 High-level pipe and process management. These functions
           run the given command in parallel with the program,
           and return channels connected to the standard input and/or
           the standard output of the command. The command is interpreted
           by the shell \verb/bin/sh (cf. \verbsystem). Warning: writes on channels
           are buffered, hence be careful to call \verbflush at the right times
           to ensure correct synchronization. 
\end{comment}
\begin{verbatim}
value close_process_in: in_channel -> process_status
value close_process_out: out_channel -> process_status
value close_process: in_channel * out_channel -> process_status
\end{verbatim}
\index{close_process_in@\verb`close_process_in`}%
\index{close_process_out@\verb`close_process_out`}%
\index{close_process@\verb`close_process`}%
\begin{comment}
 Close channels opened by \verbopen_process_in, \verbopen_process_out
           and \verbopen_process, respectively, wait for the associated
           command to terminate, and return its termination status. 
\end{comment}
\subsection*{Symbolic links }\begin{verbatim}
value symlink : string -> string -> unit
\end{verbatim}
\index{symlink@\verb`symlink`}%
\begin{comment}
 \verbsymlink source dest creates the file \verbdest as a symbolic link
           to the file \verbsource. 
\end{comment}
\begin{verbatim}
value readlink : string -> string
\end{verbatim}
\index{readlink@\verb`readlink`}%
\begin{comment}
 Read the contents of a link. 
\end{comment}
\subsection*{Named pipes }\begin{verbatim}
value mkfifo : string -> file_perm -> unit
\end{verbatim}
\index{mkfifo@\verb`mkfifo`}%
\begin{comment}
 Create a named pipe with the given permissions. 
\end{comment}
\subsection*{Special files }\begin{verbatim}
value ioctl_int : file_descr -> int -> int -> int
\end{verbatim}
\index{ioctl_int@\verb`ioctl_int`}%
\begin{comment}
 Interface to \verbioctl in the case where the argument is an
           integer. The first integer argument is the command code;
           the second is the integer parameter. 
\end{comment}
\begin{verbatim}
value ioctl_ptr : file_descr -> int -> string -> int
\end{verbatim}
\index{ioctl_ptr@\verb`ioctl_ptr`}%
\begin{comment}
 Interface to \verbioctl in the case where the argument is a pointer.
           The integer argument is the command code. A pointer to the string
           argument is passed as argument to the command. The string argument
           is usually set up with the functions from modules \verbpeek and
           \verbpoke. 
\end{comment}
\subsection*{Polling }\begin{verbatim}
value select :
  file_descr list -> file_descr list -> file_descr list -> float ->
        file_descr list * file_descr list * file_descr list
\end{verbatim}
\index{select@\verb`select`}%
\begin{comment}
 Wait until some input/output operations become possible on
           some channels. The three list arguments are, respectively, a set
           of descriptors to check for reading (first argument), for writing
           (second argument), or for exceptional conditions (third argument).
           The fourth argument is the maximal timeout, in seconds; a
           negative fourth argument means no timeout (unbounded wait).
           The result is composed of three sets of descriptors: those ready
           for reading (first component), ready for writing (second component),
           and over which an exceptional condition is pending (third
           component). 
\end{comment}
\subsection*{Locking }\begin{verbatim}
type lock_command =
    F_ULOCK               (* Unlock a region *)
  | F_LOCK                (* Lock a region, and block if already locked *)
  | F_TLOCK               (* Lock a region, or fail if already locked *)
  | F_TEST                (* Test a region for other process' locks *)
\end{verbatim}
\begin{comment}
 Commands for \verblockf. 
\end{comment}
\begin{verbatim}
value lockf : file_descr -> lock_command -> int -> unit
\end{verbatim}
\index{lockf@\verb`lockf`}%
\begin{comment}
 \verblockf fd cmd size puts a lock on a region of the file opened
           as \verbfd. The region starts at the current read/write position for
           \verbfd (as set by \verblseek), and extends \verbsize bytes forward if
           \verbsize is positive, \verbsize bytes backwards if \verbsize is negative,
           or to the end of the file if \verbsize is zero. 
\end{comment}
\subsection*{Signals }\begin{verbatim}
type signal =
    SIGHUP              (* hangup *)
  | SIGINT              (* interrupt *)
  | SIGQUIT             (* quit *)
  | SIGILL              (* illegal instruction (not reset when caught) *)
  | SIGTRAP             (* trace trap (not reset when caught) *)
  | SIGABRT             (* used by abort *)
  | SIGEMT              (* EMT instruction *)
  | SIGFPE              (* floating point exception *)
  | SIGKILL             (* kill (cannot be caught or ignored) *)
  | SIGBUS              (* bus error *)
  | SIGSEGV             (* segmentation violation *)
  | SIGSYS              (* bad argument to system call *)
  | SIGPIPE             (* write on a pipe with no one to read it *)
  | SIGALRM             (* alarm clock *)
  | SIGTERM             (* software termination signal from kill *)
  | SIGURG              (* urgent condition on IO channel *)
  | SIGSTOP             (* sendable stop signal not from tty *)
  | SIGTSTP             (* stop signal from tty *)
  | SIGCONT             (* continue a stopped process *)
  | SIGCHLD             (* to parent on child stop or exit *)
  | SIGIO               (* input/output possible signal *)
  | SIGXCPU             (* exceeded CPU time limit *)
  | SIGXFSZ             (* exceeded file size limit *)
  | SIGVTALRM           (* virtual time alarm *)
  | SIGPROF             (* profiling time alarm *)
  | SIGWINCH            (* window changed *)
  | SIGLOST             (* resource lost (eg, record-lock lost) *)
  | SIGUSR1             (* user defined signal 1 *)
  | SIGUSR2             (* user defined signal 2 *)
\end{verbatim}
\begin{comment}
 The type of signals. 
\end{comment}
\begin{verbatim}
type signal_handler =
    Signal_default                      (* Default behavior for the signal *)
  | Signal_ignore                       (* Ignore the signal *)
  | Signal_handle of (unit -> unit)     (* Call the given function
                                           when the signal occurs. *)
\end{verbatim}
\begin{comment}
 The behavior on receipt of a signal 
\end{comment}
\begin{verbatim}
value kill : int -> signal -> unit
\end{verbatim}
\index{kill@\verb`kill`}%
\begin{comment}
 Send a signal to the process with the given process id. 
\end{comment}
\begin{verbatim}
value signal : signal -> signal_handler -> unit
\end{verbatim}
\index{signal@\verb`signal`}%
\begin{comment}
 Set the behavior to be taken on receipt of the given signal. 
\end{comment}
\begin{verbatim}
value pause : unit -> unit
\end{verbatim}
\index{pause@\verb`pause`}%
\begin{comment}
 Wait until a non-ignored signal is delivered. 
\end{comment}
\subsection*{Time functions }\begin{verbatim}
type process_times =
  { tms_utime : float;          (* User time for the process *)
    tms_stime : float;          (* System time for the process *)
    tms_cutime : float;         (* User time for the children processes *)
    tms_cstime : float }        (* System time for the children processes *)
\end{verbatim}
\begin{comment}
 The execution times (CPU times) of a process. 
\end{comment}
\begin{verbatim}
type tm =
  { tm_sec : int;                       (* Seconds 0..59 *)
    tm_min : int;                       (* Minutes 0..59 *)
    tm_hour : int;                      (* Hours 0..23 *)
    tm_mday : int;                      (* Day of month 1..31 *)
    tm_mon : int;                       (* Month of year 0..11 *)
    tm_year : int;                      (* Year - 1900 *)
    tm_wday : int;                      (* Day of week (Sunday is 0) *)
    tm_yday : int;                      (* Day of year 0..365 *)
    tm_isdst : bool }                   (* Daylight time savings in effect *)
\end{verbatim}
\begin{comment}
 The type representing wallclock time and calendar date. 
\end{comment}
\begin{verbatim}
value time : unit -> int
\end{verbatim}
\index{time@\verb`time`}%
\begin{comment}
 Return the current time since 00:00:00 GMT, Jan. 1, 1970,
           in seconds. 
\end{comment}
\begin{verbatim}
value gmtime : int -> tm
\end{verbatim}
\index{gmtime@\verb`gmtime`}%
\begin{comment}
 Convert a time in seconds, as returned by \verbtime, into a date and
           a time. Assumes Greenwich meridian time zone. 
\end{comment}
\begin{verbatim}
value localtime : int -> tm
\end{verbatim}
\index{localtime@\verb`localtime`}%
\begin{comment}
 Convert a time in seconds, as returned by \verbtime, into a date and
           a time. Assumes the local time zone. 
\end{comment}
\begin{verbatim}
value alarm : int -> int
\end{verbatim}
\index{alarm@\verb`alarm`}%
\begin{comment}
 Schedule a \verbSIGALRM signals after the given number of seconds. 
\end{comment}
\begin{verbatim}
value sleep : int -> unit
\end{verbatim}
\index{sleep@\verb`sleep`}%
\begin{comment}
 Stop execution for the given number of seconds. 
\end{comment}
\begin{verbatim}
value times : unit -> process_times
\end{verbatim}
\index{times@\verb`times`}%
\begin{comment}
 Return the execution times of the process. 
\end{comment}
\begin{verbatim}
value utimes : string -> int -> int -> unit
\end{verbatim}
\index{utimes@\verb`utimes`}%
\begin{comment}
 Set the last access time (second arg) and last modification time
           (third arg) for a file. Times are expressed in seconds from
           00:00:00 GMT, Jan. 1, 1970. 
\end{comment}
\subsection*{User id, group id }\begin{verbatim}
value getuid : unit -> int
\end{verbatim}
\index{getuid@\verb`getuid`}%
\begin{comment}
 Return the user id of the user executing the process. 
\end{comment}
\begin{verbatim}
value geteuid : unit -> int
\end{verbatim}
\index{geteuid@\verb`geteuid`}%
\begin{comment}
 Return the effective user id under which the process runs. 
\end{comment}
\begin{verbatim}
value setuid : int -> unit
\end{verbatim}
\index{setuid@\verb`setuid`}%
\begin{comment}
 Set the real user id and effective user id for the process. 
\end{comment}
\begin{verbatim}
value getgid : unit -> int
\end{verbatim}
\index{getgid@\verb`getgid`}%
\begin{comment}
 Return the group id of the user executing the process. 
\end{comment}
\begin{verbatim}
value getegid : unit -> int
\end{verbatim}
\index{getegid@\verb`getegid`}%
\begin{comment}
 Return the effective group id under which the process runs. 
\end{comment}
\begin{verbatim}
value setgid : int -> unit
\end{verbatim}
\index{setgid@\verb`setgid`}%
\begin{comment}
 Set the real group id and effective group id for the process. 
\end{comment}
\begin{verbatim}
value getgroups : unit -> int list
\end{verbatim}
\index{getgroups@\verb`getgroups`}%
\begin{comment}
 Return the list of groups to which the user executing the process
           belongs. 
\end{comment}
\begin{verbatim}
type passwd_entry =
  { pw_name : string;
    pw_passwd : string;
    pw_uid : int;
    pw_gid : int;
    pw_gecos : string;
    pw_dir : string;
    pw_shell : string }
\end{verbatim}
\begin{comment}
 Structure of entries in the \verbpasswd database. 
\end{comment}
\begin{verbatim}
type group_entry =
  { gr_name : string;
    gr_passwd : string;
    gr_gid : int;
    gr_mem : string vect }
\end{verbatim}
\begin{comment}
 Structure of entries in the \verbgroups database. 
\end{comment}
\begin{verbatim}
value getlogin : unit -> string
\end{verbatim}
\index{getlogin@\verb`getlogin`}%
\begin{comment}
 Return the login name of the user executing the process. 
\end{comment}
\begin{verbatim}
value getpwnam : string -> passwd_entry
\end{verbatim}
\index{getpwnam@\verb`getpwnam`}%
\begin{comment}
 Find an entry in \verbpasswd with the given name, or raise
           \verbNot_found. 
\end{comment}
\begin{verbatim}
value getgrnam : string -> group_entry
\end{verbatim}
\index{getgrnam@\verb`getgrnam`}%
\begin{comment}
 Find an entry in \verbgroup with the given name, or raise
           \verbNot_found. 
\end{comment}
\begin{verbatim}
value getpwuid : int -> passwd_entry
\end{verbatim}
\index{getpwuid@\verb`getpwuid`}%
\begin{comment}
 Find an entry in \verbpasswd with the given user id, or raise
           \verbNot_found. 
\end{comment}
\begin{verbatim}
value getgrgid : int -> group_entry
\end{verbatim}
\index{getgrgid@\verb`getgrgid`}%
\begin{comment}
 Find an entry in \verbgroup with the given group id, or raise
           \verbNot_found. 
\end{comment}
\subsection*{Internet addresses }\begin{verbatim}
type inet_addr
\end{verbatim}
\begin{comment}
 The abstract type of Internet addresses. 
\end{comment}
\begin{verbatim}
value inet_addr_of_string : string -> inet_addr
value string_of_inet_addr : inet_addr -> string
\end{verbatim}
\index{inet_addr_of_string@\verb`inet_addr_of_string`}%
\index{string_of_inet_addr@\verb`string_of_inet_addr`}%
\begin{comment}
 Conversions between string with the format \verbXXX.YYY.ZZZ.TTT
           and Internet addresses. \verbinet_addr_of_string raises \verbFailure
           when given a string that does not match this format. 
\end{comment}
\subsection*{Sockets }\begin{verbatim}
type socket_domain =
    PF_UNIX                             (* Unix domain *)
  | PF_INET                             (* Internet domain *)
\end{verbatim}
\begin{comment}
 The type of socket domains. 
\end{comment}
\begin{verbatim}
type socket_type =
    SOCK_STREAM                         (* Stream socket *)
  | SOCK_DGRAM                          (* Datagram socket *)
  | SOCK_RAW                            (* Raw socket *)
  | SOCK_SEQPACKET                      (* Sequenced packets socket *)
\end{verbatim}
\begin{comment}
 The type of socket kinds, specifying the semantics of
           communications. 
\end{comment}
\begin{verbatim}
type sockaddr =
    ADDR_UNIX of string
  | ADDR_INET of inet_addr * int
\end{verbatim}
\begin{comment}
 The type of socket addresses. \verbADDR_UNIX name is a socket
           address in the Unix domain; \verbname is a file name in the file
           system. \verbADDR_INET(addr,port) is a socket address in the Internet
           domain; \verbaddr is the Internet address of the machine, and
           \verbport is the port number. 
\end{comment}
\begin{verbatim}
type shutdown_command =
    SHUTDOWN_RECEIVE                    (* Close for receiving *)
  | SHUTDOWN_SEND                       (* Close for sending *)
  | SHUTDOWN_ALL                        (* Close both *)
\end{verbatim}
\begin{comment}
 The type of commands for \verbshutdown. 
\end{comment}
\begin{verbatim}
type msg_flag =
    MSG_OOB
  | MSG_DONTROUTE
  | MSG_PEEK
\end{verbatim}
\begin{comment}
 The flags for \verbrecv, \verbrecvfrom, \verbsend and \verbsendto. 
\end{comment}
\begin{verbatim}
value socket : socket_domain -> socket_type -> int -> file_descr
\end{verbatim}
\index{socket@\verb`socket`}%
\begin{comment}
 Create a new socket in the given domain, and with the
           given kind. The third argument is the protocol type; 0 selects
           the default protocol for that kind of sockets. 
\end{comment}
\begin{verbatim}
value socketpair :
        socket_domain -> socket_type -> int -> file_descr * file_descr
\end{verbatim}
\index{socketpair@\verb`socketpair`}%
\begin{comment}
 Create a pair of unnamed sockets, connected together. 
\end{comment}
\begin{verbatim}
value accept : file_descr -> file_descr * sockaddr
\end{verbatim}
\index{accept@\verb`accept`}%
\begin{comment}
 Accept connections on the given socket. The returned descriptor
           is a socket connected to the client; the returned address is
           the address of the connecting client. 
\end{comment}
\begin{verbatim}
value bind : file_descr -> sockaddr -> unit
\end{verbatim}
\index{bind@\verb`bind`}%
\begin{comment}
 Bind a socket to an address. 
\end{comment}
\begin{verbatim}
value connect : file_descr -> sockaddr -> unit
\end{verbatim}
\index{connect@\verb`connect`}%
\begin{comment}
 Connect a socket to an address. 
\end{comment}
\begin{verbatim}
value listen : file_descr -> int -> unit
\end{verbatim}
\index{listen@\verb`listen`}%
\begin{comment}
 Set up a socket for receiving connection requests. The integer
           argument is the maximal number of pending requests. 
\end{comment}
\begin{verbatim}
value shutdown : file_descr -> shutdown_command -> unit
\end{verbatim}
\index{shutdown@\verb`shutdown`}%
\begin{comment}
 Shutdown a socket connection. \verbSHUTDOWN_SEND as second argument
           causes reads on the other end of the connection to return
           an end-of-file condition.
           \verbSHUTDOWN_RECEIVE causes writes on the other end of the connection
           to return a closed pipe condition (\verbSIGPIPE signal). 
\end{comment}
\begin{verbatim}
value recv : file_descr -> string -> int -> int -> msg_flag list -> int
value recvfrom :
        file_descr -> string -> int -> int -> msg_flag list -> int * sockaddr
\end{verbatim}
\index{recv@\verb`recv`}%
\index{recvfrom@\verb`recvfrom`}%
\begin{comment}
 Receive data from an unconnected socket. 
\end{comment}
\begin{verbatim}
value send : file_descr -> string -> int -> int -> msg_flag list -> int
value sendto :
        file_descr -> string -> int -> int -> msg_flag list -> sockaddr -> int
\end{verbatim}
\index{send@\verb`send`}%
\index{sendto@\verb`sendto`}%
\begin{comment}
 Send data over an unconnected socket. 
\end{comment}
\subsection*{High-level network connection functions }\begin{verbatim}
value open_connection : sockaddr -> in_channel * out_channel
\end{verbatim}
\index{open_connection@\verb`open_connection`}%
\begin{comment}
 Connect to a server at the given address.
           Return a pair of buffered channels connected to the server.
           Remember to call \verbflush on the output channel at the right times
           to ensure correct synchronization. 
\end{comment}
\begin{verbatim}
value shutdown_connection : in_channel -> unit
\end{verbatim}
\index{shutdown_connection@\verb`shutdown_connection`}%
\begin{comment}
 ``Shut down'' a connection established with \verbopen_connection;
           that is, transmit an end-of-file condition to the server reading
           on the other side of the connection. 
\end{comment}
\begin{verbatim}
value establish_server : (in_channel -> out_channel -> 'a) -> sockaddr -> unit
\end{verbatim}
\index{establish_server@\verb`establish_server`}%
\begin{comment}
 Establish a server on the given address.
           The function given as first argument is called for each connection
           with two buffered channels connected to the client. A new process
           is created for each connection. The function \verbestablish_server
           never returns normally. 
\end{comment}
\subsection*{Host and protocol databases }\begin{verbatim}
type host_entry =
  { h_name : string;
    h_aliases : string vect;
    h_addrtype : socket_domain;
    h_addr_list : inet_addr vect }
\end{verbatim}
\begin{comment}
 Structure of entries in the \verbhosts database. 
\end{comment}
\begin{verbatim}
type protocol_entry =
  { p_name : string;
    p_aliases : string vect;
    p_proto : int }
\end{verbatim}
\begin{comment}
 Structure of entries in the \verbprotocols database. 
\end{comment}
\begin{verbatim}
type service_entry =
  { s_name : string;
    s_aliases : string vect;
    s_port : int;
    s_proto : string }
\end{verbatim}
\begin{comment}
 Structure of entries in the \verbservices database. 
\end{comment}
\begin{verbatim}
value gethostname : unit -> string
\end{verbatim}
\index{gethostname@\verb`gethostname`}%
\begin{comment}
 Return the name of the local host. 
\end{comment}
\begin{verbatim}
value gethostbyname : string -> host_entry
\end{verbatim}
\index{gethostbyname@\verb`gethostbyname`}%
\begin{comment}
 Find an entry in \verbhosts with the given name, or raise
           \verbNot_found. 
\end{comment}
\begin{verbatim}
value gethostbyaddr : inet_addr -> host_entry
\end{verbatim}
\index{gethostbyaddr@\verb`gethostbyaddr`}%
\begin{comment}
 Find an entry in \verbhosts with the given address, or raise
           \verbNot_found. 
\end{comment}
\begin{verbatim}
value getprotobyname : string -> protocol_entry
\end{verbatim}
\index{getprotobyname@\verb`getprotobyname`}%
\begin{comment}
 Find an entry in \verbprotocols with the given name, or raise
           \verbNot_found. 
\end{comment}
\begin{verbatim}
value getprotobynumber : int -> protocol_entry
\end{verbatim}
\index{getprotobynumber@\verb`getprotobynumber`}%
\begin{comment}
 Find an entry in \verbprotocols with the given protocol number,
           or raise \verbNot_found. 
\end{comment}
\begin{verbatim}
value getservbyname : string -> string -> service_entry
\end{verbatim}
\index{getservbyname@\verb`getservbyname`}%
\begin{comment}
 Find an entry in \verbservices with the given name, or raise
           \verbNot_found. 
\end{comment}
\begin{verbatim}
value getservbyport : int -> string -> service_entry
\end{verbatim}
\index{getservbyport@\verb`getservbyport`}%
\begin{comment}
 Find an entry in \verbservices with the given service number,
           or raise \verbNot_found. 
\end{comment}
\subsection*{Terminal interface }\begin{comment}
 The following functions implement the POSIX standard terminal
   interface. They provide control over asynchronous communication ports
   and pseudo-terminals. Refer to the \verbtermios man page for a
   complete description. 
\end{comment}
\begin{verbatim}
type terminal_io = {
\end{verbatim}
\begin{comment}
 Input modes: 
\end{comment}
\begin{verbatim}
    mutable c_ignbrk: bool;  (* Ignore the break condition. *)
    mutable c_brkint: bool;  (* Signal interrupt on break condition. *)
    mutable c_ignpar: bool;  (* Ignore characters with parity errors. *)
    mutable c_parmrk: bool;  (* Mark parity errors. *)
    mutable c_inpck: bool;   (* Enable parity check on input. *)
    mutable c_istrip: bool;  (* Strip 8th bit on input characters. *)
    mutable c_inlcr: bool;   (* Map NL to CR on input. *)
    mutable c_igncr: bool;   (* Ignore CR on input. *)
    mutable c_icrnl: bool;   (* Map CR to NL on input. *)
    mutable c_ixon: bool;    (* Recognize XON/XOFF characters on input. *)
    mutable c_ixoff: bool;   (* Emit XON/XOFF chars to control input flow. *)
\end{verbatim}
\begin{comment}
 Output modes: 
\end{comment}
\begin{verbatim}
    mutable c_opost: bool;   (* Enable output processing. *)
    mutable c_olcuc: bool;   (* Map lowercase to uppercase on output. *)
    mutable c_onlcr: bool;   (* Map NL to CR/NL on output. *)
    mutable c_ocrnl: bool;   (* Map CR to NL on output. *)
    mutable c_onocr: bool;   (* No CR output at column 0. *)
    mutable c_onlret: bool;  (* NL is assumed to perform as CR. *)
    mutable c_ofill: bool;   (* Use fill characters instead of delays. *)
    mutable c_ofdel: bool;   (* Fill character is DEL instead of NULL. *)
    mutable c_nldly: int;    (* Newline delay type (0-1). *)
    mutable c_crdly: int;    (* Carriage return delay type (0-3). *)
    mutable c_tabdly: int;   (* Horizontal tab delay type (0-3). *)
    mutable c_bsdly: int;    (* Backspace delay type (0-1). *)
    mutable c_vtdly: int;    (* Vertical tab delay type (0-1). *)
    mutable c_ffdly: int;    (* Form feed delay type (0-1). *)
\end{verbatim}
\begin{comment}
 Control modes: 
\end{comment}
\begin{verbatim}
    mutable c_obaud: int;    (* Output baud rate (0 means close connection).*)
    mutable c_ibaud: int;    (* Input baud rate. *)
    mutable c_csize: int;    (* Number of bits per character (5-8). *)
    mutable c_cstopb: int;   (* Number of stop bits (1-2). *)
    mutable c_cread: bool;   (* Reception is enabled. *)
    mutable c_parenb: bool;  (* Enable parity generation and detection. *)
    mutable c_parodd: bool;  (* Specify odd parity instead of even. *)
    mutable c_hupcl: bool;   (* Hang up on last close. *)
    mutable c_clocal: bool;  (* Ignore modem status lines. *)
\end{verbatim}
\begin{comment}
 Local modes: 
\end{comment}
\begin{verbatim}
    mutable c_isig: bool;    (* Generate signal on INTR, QUIT, SUSP. *)
    mutable c_icanon: bool;  (* Enable canonical processing
                                (line buffering and editing) *)
    mutable c_noflsh: bool;  (* Disable flush after INTR, QUIT, SUSP. *)
    mutable c_echo: bool;    (* Echo input characters. *)
    mutable c_echoe: bool;   (* Echo ERASE (to erase previous character). *)
    mutable c_echok: bool;   (* Echo KILL (to erase the current line). *)
    mutable c_echonl: bool;  (* Echo NL even if c_echo is not set. *)
\end{verbatim}
\begin{comment}
 Control characters: 
\end{comment}
\begin{verbatim}
    mutable c_vintr: char;   (* Interrupt character (usually ctrl-C). *)
    mutable c_vquit: char;   (* Quit character (usually ctrl-\). *)
    mutable c_verase: char;  (* Erase character (usually DEL or ctrl-H). *)
    mutable c_vkill: char;   (* Kill line character (usually ctrl-U). *)
    mutable c_veof: char;    (* End-of-file character (usually ctrl-D). *)
    mutable c_veol: char;    (* Alternate end-of-line char. (usually none). *)
    mutable c_vmin: int;     (* Minimum number of characters to read
                                before the read request is satisfied. *)
    mutable c_vtime: int;    (* Maximum read wait (in 0.1s units). *)
    mutable c_vstart: char;  (* Start character (usually ctrl-Q). *)
    mutable c_vstop: char    (* Stop character (usually ctrl-S). *)
  }
value tcgetattr: file_descr -> terminal_io
\end{verbatim}
\index{tcgetattr@\verb`tcgetattr`}%
\begin{comment}
 Return the status of the terminal referred to by the given
           file descriptor. 
\end{comment}
\begin{verbatim}
type setattr_when = TCSANOW | TCSADRAIN | TCSAFLUSH
value tcsetattr: file_descr -> setattr_when -> terminal_io -> unit
\end{verbatim}
\index{tcsetattr@\verb`tcsetattr`}%
\begin{comment}
 Set the status of the terminal referred to by the given
           file descriptor. The second argument indicates when the
           status change takes place: immediately (\verbTCSANOW),
           when all pending output has been transmitted (\verbTCSADRAIN),
           or after flushing all input that has been received but not
           read (\verbTCSAFLUSH). \verbTCSADRAIN is recommended when changing
           the output parameters; \verbTCSAFLUSH, when changing the input
           parameters. 
\end{comment}
\begin{verbatim}
value tcsendbreak: file_descr -> int -> unit
\end{verbatim}
\index{tcsendbreak@\verb`tcsendbreak`}%
\begin{comment}
 Send a break condition on the given file descriptor.
           The second argument is the duration of the break, in 0.1s units;
           0 means standard duration (0.25s). 
\end{comment}
\begin{verbatim}
value tcdrain: file_descr -> unit
\end{verbatim}
\index{tcdrain@\verb`tcdrain`}%
\begin{comment}
 Waits until all output written on the given file descriptor
           has been transmitted. 
\end{comment}
\begin{verbatim}
type flush_queue = TCIFLUSH | TCOFLUSH | TCIOFLUSH
value tcflush: file_descr -> flush_queue -> unit
\end{verbatim}
\index{tcflush@\verb`tcflush`}%
\begin{comment}
 Discard data written on the given file descriptor but not yet
           transmitted, or data received but not yet read, depending on the
           second argument: \verbTCIFLUSH flushes data received but not read,
           \verbTCOFLUSH flushes data written but not transmitted, and
           \verbTCIOFLUSH flushes both. 
\end{comment}
\begin{verbatim}
type flow_action = TCOOFF | TCOON | TCIOFF | TCION
value tcflow: file_descr -> flow_action -> unit
\end{verbatim}
\index{tcflow@\verb`tcflow`}%
\begin{comment}
 Suspend or restart reception or transmission of data on
           the given file descriptor, depending on the second argument:
           \verbTCOOFF suspends output, \verbTCOON restarts output,
           \verbTCIOFF transmits a STOP character to suspend input,
           and \verbTCION transmits a START character to restart input. 
\end{comment}
