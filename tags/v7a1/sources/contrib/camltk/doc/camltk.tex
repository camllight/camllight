\documentstyle[11pt,alltt,fullpage,html]{report}

\newcommand{\tcl}{{\sf Tcl}}
\newcommand{\tk}{{\sf Tk}}
\newcommand{\caml}{{\sf Caml}}
\newcommand{\camltk}{{\sf CamlTk}}
\newcommand{\wish}{{\sf wish}}
\newcommand{\thisrelease}{{\tt alpha3}}
\title{The \caml-\tk\ interface \\
       Release \thisrelease}
\author{Fran\c{c}ois Pessaux and Fran\c{c}ois Rouaix \\
           Projet Cristal, INRIA Rocquencourt \\
        {\small Francois.Pessaux@inria.fr, Francois.Rouaix@inria.fr}
        }
\date{November 1994}
\begin{document}
\maketitle


\section*{Introduction}
\tk\ is a GUI (graphical user-interface) library for the \tcl\ language.
Normally, \tk\ is controlled from \tcl. The purpose of the \camltk\
interface is to provide a \caml\ library to control \tk\ from \caml\
programs. Thus, \tk\ can be used to program user-interfaces in \caml\
without knowledge of the \tcl\ language.

\section*{Installation}

\begin{itemize}
\item This release runs only on Unix systems.
\item You need Caml Light 0.61.
\item You need to have \tcl\ and \tk\ installed. This release requires that
you have installed the includes and libraries, and \wish.
This release has been mostly tested with \tk\ 3.6.
\end{itemize} 

\begin{htmlonly}
First get the source
\htmladdnormallink{here}{ftp://ftp.inria.fr/lang/caml-light/camltk-alpha3.tar.Z},
and extract the archive in the {\tt contrib} directory of the Caml Light
distribution. 
Check 
\htmladdnormallink{here}
 {http://pauillac.inria.fr/\%7Erouaix/camltk-porting.html}
for problems reported since the distribution.
\end{htmlonly}

See the INSTALL file for instructions regarding the installation.


\section*{This documentation}
{\bf The present documentation is still sketchy, and some basic background
about toolkits, event-driven programming, and \tk\ is assumed.}
Documentation for \tk\ is currently available in \cite{ouster94},
\cite{welch94}, and in the {\tt man} pages of the \tk\ distribution.
The distribution also includes a first shot at an \caml\ browser, which may
help in finding out the types of functions available in the library as well as
the type definitions, all this in an {\em hypertext} tool.


What you will find here is:
\begin{description}
\item[The user's manual] : describes how to write \caml\ programs using the
\tk\ interface and to compile them. 
\item[The implementor's manual] : describes how the interface library is
generated from a high-level description. This part is of interest mainly for
expert \tk\ users who wish to extend the interface to support more
functions, more widgets, or simply to fix bugs in test versions.
\item[The Tentative reference manual] : is simply a the collection of
interface files from the library, restricted to functions (i.e. type
descriptions do not appear). 
\end{description}

\section*{Bug reports}
Before reporting a bug, please check the \verb|KnownBugs| file in the
distribution directory.
For the time being, please report problems and bugs directly to
\verb|Francois.Rouaix@inria.fr| instead of \caml\ mailing lists.
Don't hesitate to make comments and suggestions, or ask questions.
Contributions (description of missing stuff) are welcome.

%\section*{Thanks}

\tableofcontents
\chapter{User's Manual}
{\bf This manual is a simple tutorial for \camltk }.  When writing your own
programs, you should consult the partial reference manual, as well as the
\tk\ man pages.  Whenever the reference manual is unsufficient, you can
check the \verb|Widgets.src| file (which syntax is given in
chapter~\ref{chap:impl}) since it describes the mappings between
\tk\ and \caml\ for functions and data. 

\section{Modules}
The \camltk\ interface is provided as a \caml\ library, composed of the
following modules:

\begin{itemize}
\item \verb|tk|, containing the initialisation functions, some frequently
used functions, and all types manipulated by the library.

\item a module for each widget class (e.g. \verb|button|). The creation
function and the commands specific to this class are contained in the
module. The module bears the same name as the widget class, except for
\verb|toplevel| widgets which are defined in module \verb|toplevelw|.

\item various other modules grouping functions of a same family
(e.g. \verb|focus|, \verb|winfo|). 

\item the \verb|textvariable| module, for the \verb|-textvariable| option
support. 

\item \verb|protocol| and \verb|support|, which contain the internals
of the interface. For example, the abstract type of widgets is defined in
\verb|support|, as well as some widget naming utilities.
\end{itemize} 

The organization of the interface is designed so that, in general, you
should \verb|#open| only the \verb|tk| module, and for the 
rest of the functions use the \verb|modulename__symbol| syntax (e.g.
\verb|text__insert|).
This is because some functions (such as \verb|create| and \verb|configure|)
exist in many different modules, and it is more practical to refer to the
function with its exact qualified name.

All modules required by the \camltk\ interface are in the \verb|tklib.zo|
or \verb|tklib2.zo| library archives (only one of them is required,
depending on the architecture you choose, cf. section~\ref{sec:compiling}).

\section{Basics}
As usual with toolkits, the simplest way to learn how to write programs is
by copy-paste-ing from available source code. In general, a \caml\ program
using the \tk\ toolkit will have the following structure :

\begin{alltt}
\input{helloworld.ml}
\end{alltt}

The first step is to initialize the interface with \verb|OpenTk|, and this
returns the {\em toplevel} widget\footnote{Note that attempting to call a
\camltk\ function without previously having initialised the interface will
probably cause your program to core dump.}.
Then, the program creates some widgets, as children of the {\em toplevel}
widget forming the user interface of your application. In this simple case,
we have only one button, whose behaviour is to print a simple message on the
standard output. Finally, the program enters the \verb|MainLoop|, and
user-interaction takes place.

\section{Example}
Here is a complete working example with a few widgets and callbacks.
This program may be found in the \verb|test| directory of the distribution,
under the name \verb|addition.ml|.

\begin{alltt}
\input{addition.ml}
\end{alltt}

\section{Widgets}
The full description of widgets, their creation options, their commands, is
out of the scope of this document. An incomplete reference manual can be
found in chapter~\ref{chap:ref}.
The reader should also refer to the \verb|man| pages of the \tk\
distribution. 
As of this release, all widgets of \tk\ 3.6 are supported, although some
commands might not be implemented. Remember also that only few of them have
been tested.

\section{Bindings}
\tk\ offers, in addition to widget-specific callbacks, a general facility
for binding functions to events. (bind(n)).
In the \camltk\ interface, this is implemented by the function
\begin{verbatim}
bind: Widget -> (Modifier list * XEvent) list -> BindAction -> unit
\end{verbatim} 
with the following types

\input{bindtypes.mli}

The first argument of type \verb|Widget| is naturally the widget for which
this binding is defined.
The second argument of type \verb|(Modifier list * XEvent) list| is the 
succession of events that will trigger the callback.
The last argument is the action : \verb|BindRemove| will remove a binding,
\verb|BindSet| will set a binding, and \verb|BindExtend| will extend a
binding.
The argument of \verb|BindSet| and \verb|BindExtend| are an \verb|EventField
list| and the 
callback itself. 
In bindings, the callback needs some information about the event that
triggered it. This information is contained in the \verb|EventInfo| passed
to the callback, but the user has to specify which fields in this record are
relevant for this particular callback. The other fields do {\bf not} contain
accurate information. Check also the \tk\ documentation for finding out
which field is valid for a given event.

\paragraph{Example}
\begin{verbatim}
bind MyCan [[],Motion] 
   (BindSet([Ev_MouseX; Ev_MouseY;Ev_Place;Ev_SendEvent;Ev_KeySymString], 
            BindCmd)) ;;
\end{verbatim} 
will trigger the callback \verb|BindCmd| when the event \verb|Motion| occurs
in the \verb|MyCan| widget. The command \verb|BindCmd| may assume that the 
information in the fields {\tt Ev\_MouseX, Ev\_MouseY, Ev\_Place, Ev\_Place,
Ev\_SendEvent, Ev\_KeySymString} of its \verb|EventInfo| argument contain
accurate information.

Bindings for \verb|text| and \verb|canvas| widgets obey the same rules,
and the functions are 
\begin{verbatim}
tk__text_tag_bind : Widget -> TextTag -> 
      (Modifier list * XEvent) list -> BindAction -> unit
tk__canvas_bind : Widget -> TagOrId -> 
      (Modifier list * XEvent) list -> BindAction -> unit
\end{verbatim} 
Additional arguments are respectively the {\tt tag} and the {\tt tag or id}
for which the binding is defined.

Binding for classes of widgets are managed with
\begin{verbatim}
tk__class_bind : string -> 
      (Modifier list * XEvent) list -> BindAction -> unit
\end{verbatim} 
The first argument should be a widget class name (e.g. \verb|"Button"|).

\section{Exiting properly}
You should always try to call \verb|CloseTk| before exiting. Otherwise, in
the {\em two-process} architecture, the slave {\tt wish} may stay around for
a while, as well as named pipes created by the communication protocol in
\verb|/tmp|. In particular, something like  
\begin{verbatim}
signal SIGINT (Signal_handle (function _ -> CloseTk(); exit 0))
\end{verbatim} 
should help your application behave nicely. Naturally your own cleanup
should be inserted also.

\section{Errors}
The \camltk\ interface may raise the following exceptions:
\begin{description}
\item[TkError of string] this exception is raised when a \tcl\tk\ command fails
during an evaluation. Normally, static typing in \caml, in addition to some
run-time verifications should prevent this situation from happening.
However, some errors are due to the external environment (e.g.
\verb|selection__get|) and cannot be prevented. The exception carries the
\tcl\tk\ error message.

\item[IllegalWidgetType of string]
this exception is raised when a widget command is applied to a widget of a
different or unappropriate class (e.g. \verb|button__configure| applied to
a {\em scrollbar} widget). The exception carries the class of the faulty
widget.

\item[Invalid\_argument of string]
this exception is raised  when some data cannot be exchanged between \caml\
and \tk. This situation occurs for example when an illegal option is used
for creating a widget. However, this error may also be caused by a faulty or
incomplete description of a widget, widget command or type.
\end{description} 

\section{More examples}
The \verb|test| directory of the distribution contains several test
examples. 
The \verb|books-examples| directory contains translations in \camltk\ of
several examples from \cite{ouster94}.

%The \verb|tktop| directory contains a more advanced example, although
%not very robust.

{\bf All examples included in this distribution  should not be considered as
working applications. They are provided for documentation purposes only}.

\section{The Caml Browser}
In the directory {\tt browser} you fill find a yet more complete example,
featuring many different kinds of widgets. The browser mimicks the Caml
Light toplevel behaviour: you can open and close modules ({\tt \#open} and
{\tt \#close} directives), or add a library path ({\tt \#directory} directive).
You can enter a symbol (value, type constructor, value constructor,
exception), and see its definition. You can also see the complete interface
of a module by double-clicking on it. 
Remember that the browser will show you a pretty-printed version
of compiled interfaces ({\tt .zi} files), so you will not see the comments
that were in the source file.

The browser can also display source files. Hypertext navigation is also
available for source files, but does not reflect the compiler's semantics,
as the \verb|#open| directives in the source file are not taken into
account.


\section{Compilation}
\label{sec:compiling}
\subsection{Architecture}
In this release, there are two possible architectures for the \camltk\
interface. As in previous versions, there is a {\em two-process}
architecture, where \caml\ communicates to a slave \wish\ interpreter.
There is now a {\em single-process} architecture, where both \caml\ and \tk\
coexist in a single process.

{\bf Note: } because the {\em two-process} architecture is less portable and
less robust than the {\em single-process} architecture, the former may not
be supported in future versions.


%% Avec la 0.7, Camltk est installe avec le reste du bouzin
% In the following, we refer to \verb|CAMLTKDIR| as an environment variable
% pointing to the directory where the interface library has been
% installed, that is normally \verb|/usr/local/lib/caml-light/tk|.
In the following, we refer to \verb|TCLLIBDIR| (resp. \verb|TKLIBDIR|) as
an environment variable containing the directory where \verb|libtcl.a|
(resp. \verb|libtk.a|) is located. Consult your system administrator if you
don't have this information. Moreover, we assume a ``standard installation''
of \caml\ and \camltk from the distribution. This means, starting from
release \thisrelease, that all \camltk\ library files are installed in the
same directory as standard \caml\ library files.

\subsection{Compilation for Single-process architecture}
The library is named \verb|tklib.zo|.
The usual commands for compiling and linking are:
\begin{verbatim}
$ camlc -c addition.ml
$ camlc -custom -o addition tklib.zo addition.zo \
        -ccopt -L$TCLLIBDIR -ccopt -L$TKLIBDIR \
        -lcamltk -lcaml -ltk -ltcl -lX11
\end{verbatim}
Linking is a bit complex, because we need to tell the C compiler/linker
where to find the libtk.a and libtcl.a libraries, and to link with all
required pieces. Your linker might complain about \verb|main| being defined
twice. It this produces an error, you cannot use this architecture.
We will try to provide a bug-fix if this turns out to be a problem.

\subsection{Compilation for Two-process architecture}
The library is named \verb|tklib2.zo|.
The usual commands for compiling and linking are:
\begin{verbatim}
$ camlc -c addition.ml
$ camlc -custom -o addition unix.zo tklib2.zo addition.zo -lunix
\end{verbatim} 
 \verb|unix.zo| comes from the \verb|libunix| contrib, and is normally
installed in the same directory as standard \caml\ libraries. 
The commands are simpler because we rely on the external \verb|wish|
interpreter.

\subsection{Toplevels}
The distribution also contains two toplevels featuring the \camltk\
interface: \verb|camltktop1| and \verb|camltktop2|.
These toplevels may be run with: 
\begin{verbatim}
$ camllight camltktop1
$ camllight camltktop2
\end{verbatim} 
Note however that the usage of the toplevels is awkward (for the time
being), because callbacks cannot be executed until you enter MainLoop, and
you cannot easily leave MainLoop.

\section{Extensions}
\camltk\ has already one extension (i.e. a feature not normally available in
\tk\ 3.6). This extension allows the association of callbacks to Unix input
file descriptors.

\subsection{File descriptor callbacks}
A callback can be associated to a file descriptor using the following
primitive:
\begin{verbatim}
tk__add_fileinput : file_descr -> (unit -> unit) -> unit
\end{verbatim} 
When some input is available on the file descriptor specified by the first
argument, then the callback (second argument) is called.
\begin{verbatim}
tk__remove_fileinput : file_descr -> unit
\end{verbatim} 
Removes the file descriptor callback. It is the programmer's responsability
to check for end of file, and to remove the callback if the file descriptor
is closed. Be aware that calling the \verb|update| function will potentially
destroy the sequentiality of read operations.

\section{Translating \tk\ idioms}
If you are a seasoned \tk\ programmer, you will find out that some \tk\
idioms have a different form in \camltk\ .

\subsection{Widgets}
First of all, widget creation is more ``functional''. One does not specify
the name of the created widget (except for \verb|toplevel| widgets), but
only its parent. The name is allocated by the library, and a handle to the
widget is returned to you. Then, widgets are not ``procedures''. They are
objects, and must be passed to widget manipulation functions.
For example,
\begin{verbatim}
button .myf.bok -title "Ok" -relief raised
\end{verbatim} 
is translated by (assuming that \verb|myf| is the parent widget \verb|.myf|)
\begin{verbatim}
let b = button__create myf [Title "Ok"; Relief Raised] in
...
\end{verbatim} 
Then, 
\begin{verbatim}
.myf.bok configure -state disabled
\end{verbatim} 
would be in \caml:
\begin{verbatim}
button__configure b [State Disabled]
\end{verbatim} 

This is more in the spirit of ML, but unfortunately eliminates the
possibility to specify configuration of widgets by resources based on widget 
names. It you absolutely want to use resources then you may call the
alternate \verb|create_named| function:
\begin{verbatim}
let b = button__create_named myf "bok" [Title "Ok"; Relief Raised] in
...
\end{verbatim} 
Assuming that \verb|myf| is the widget of path \verb|.myf|, then \verb|b|
will have path \verb|.myf.bok|. As in \tk, it is your responsibility to use
valid identifiers for path elements.

When widgets are mutually recursive (through callbacks, for example when
linking a scrollbar to a text widget), one should first create the widgets,
and then set up the callbacks by calling \verb|configure| on the widgets.
This is in contrast with \tcl\tk\ where one may refer to a procedure that
has not yet been defined.

Partially applied widget commands (such as redisplay commands) translate
quite well to \caml, with possibly some wrapping required due to 
value constructors.

\subsection{Text variables}
\verb|TextVariables| are now available in \camltk. However,
\verb|textvariables| are not physically shared between \caml\ and \tk.
Instead, setting or consulting a \verb|textvariable| requires a transaction
between \caml\ and \tk. However, the "sharing" properties of 
\verb|textvariables| in \tk\ are preserved. For example, if the
\verb|textvariable| is  
used to physically share the content of an {\tt entry} with the text
of a {\tt label}, then this sharing will effectively occur in the \tk\
world, whether your \caml\ program is concerned with the actual
contents of the \verb|textvariable| or not. 

\subsection{How to find out the \caml\ name of \tk\ keywords and functions}
Since this documentation is not a reference manual, you have unfortunately
to make some efforts to find out what functions are available, and how the
datatypes are defined. For the time being the best methods are:
\begin{itemize}
\item Check the .mli files in the \verb|lib| directory (they also appear in
chapter~\ref{chap:ref}). This will give you at least the functions for each
widget class. For each function, the basic form of its \tk\ equivalent is
given in a comment. When a \tk\  function has several forms (e.g. variable
number of arguments), the \camltk\ interface will have different versions of
this function with different names (hopefully significative enough).

\item Check the \verb|Widgets.src| file of the distribution. 

\end{itemize} 

\section{Debugging}
Since this is a beta release, some \tk\ functions may have been improperly
implemented in the \camltk\ library. This may cause undue \tk\ errors
(exception \verb|TkError|, or sometimess \verb|Invalid\_Argument| ). To
facilitate the debugging, you can set the 
Unix environment variable \verb|CAMLTKDEBUG| to any value before launching
your program. This will allow you to see all
data transferred between the \caml\ and the \tk\ processes. Since this data is
essentially \tcl\tk\ code, you need a basic knowledge of this language to
understand what is going on.
It is also possible to trigger debugging by setting the boolean reference
\verb|protocol__debug| to true (or false to remove debugging).

\chapter{Implementor's Manual}
\label{chap:impl}
\tk\ is still evolving, and also offers numerous user-contributed
extensions. To ensure the compatibility of \camltk\ with this evolution,
we chose to divide the library in two parts:
\begin{itemize}
\item a protocol layer for communication between \caml\ and \tcl,
\item implementation of \tk\ commands generated from  a high-level
description of available widgets and functions.
\end{itemize} 

This chapter defines the syntax and semantics of the \camltk\ description
language, as well as some technical details of the protocol.

\section{The \camltk\ description language}
The language allows the specification of
\begin{description}
\item[widgets] of the standard \tk\ library, with options and associated
commands,
\item[functions] not associated with a specific widget class
\item[modules] grouping families of functions.
\end{description} 

A source file is simply a collection of entries, each entry describing an
element of the interface. The {\tt tkcompiler} provided in this distribution
takes the source file and produces \caml\ modules forming the interface
library.

\subsection{Types}
A type entry contains information to translate data from \caml\ to \tk\ 
and the converse. Each data type that needs to be exchanged by \caml\ and
\tk\ must be described by a type entry. For each type, a conversion function
from \caml\ to \tk\ is generated, as well as a parsing function from \tk\ to
\caml\ if required (because values of the type are returned by some
function in the library) and feasible.

A type is given by a set of value constructors, as in \caml. For each
constructor, one must give the \caml\ name, and optionally
\begin{itemize}
\item the \tk\ name of the constructor
\item the type of its arguments
\end{itemize} 

\paragraph{Example}
\begin{verbatim}
type State { 
   Normal "normal"
   Active "active"
   Disabled "disabled"
}
\end{verbatim} 
defines the \caml\ type
\begin{verbatim}
type State =
          Disabled
        | Active
        | Normal
;;
\end{verbatim} 
and the conversion function
\begin{verbatim}
let CAMLtoTKState = function
          Disabled -> "disabled"
        | Active -> "active"
        | Normal -> "normal"
;;
\end{verbatim} 

The \tcl\ source-code produced by the converter is the \tk\ name (if any)
followed by a white space and the \tk\ version of the argument (if any).
The parser can be generated only in restricted cases: zeroary constructors,
and at most one anonymous \verb|int| constructor and one anonymous
\verb|string| constructor (anonymous means that the \tk\ name is an empty
string). Otherwise, the compiler will produce a warning
message, and you have to write the parser yourself. 

\subsection{Subtypes}
As customary, types will be statically verified in \caml\ programs. However, in
order to reduce the number of value constructors required by the interface
(and associated naming problems),
we introduced a notion of {\em subtypes}. A subtype is a named subset of the
set of constructors of a type. In this case, type-checking (i.e. verifiying
that a constructor belongs to a subtype) will occur at run-time.

If a type requires subtyping, then one should not declare the type itself,
but instead define each of its subtypes. The compiler will consider the
whole set of constructors for the definition of the type. Note also that the
definition of a constructor may be omitted if it has  already been declared.

\paragraph{Example}
\begin{verbatim}
subtype Index(entry) {
   Number (int)
   End "end"
   Insert "insert"
   SelFirst "sel.first"
   SelLast "sel.last" 
}
subtype Index(listbox) {
   Number End 
}
\end{verbatim}
defines two subtypes (\verb|entry| and \verb|listbox|) of the type
\verb|Index|.  

Subtypes may be referred to (in argument types of functions) with the same
syntax: \verb|type(subtype)|.
\paragraph{Example}
\begin{verbatim}
   function () delete "delete" (Index(listbox), Index(listbox))
\end{verbatim} 

\subsection{Widgets}
A widget class description is composed of its name, the set of valid options
for the creation of widgets of this class, and the set of functions and
commands associated to the class.

Options descriptions follow the same syntax as value constructors in type or
subtype declarations, except for the presence of the \verb|option| keyword.
Actually, the options declared inside a widget description form a subtype
(with the name of the class) of the \verb|option| type. 
\verb|option| is not really hardwired in the compiler, in that it obeys the
same rules for subtyping as user-defined types. However the \verb|option|
type is implicitly used when producing the widget creation functions.

Function declarations are formed of
\begin{itemize}
\item the \verb|function| keyword
\item the result type of the function (inside parenthesis)
\item the \caml\ name of the function
\item the corresponding \tk\ name
\item the type of the arguments (inside parenthesis)
\end{itemize} 

The \caml\ function produced from this description always takes a widget as
first argument. The function is curryfied, although the notation seems to
imply that the function has only one tuple argument.


\paragraph{Example}
\begin{verbatim}
widget message {
   option Anchor
   option Background
   option Borderwidth
   option Cursor
   option Font
   option Foreground
   option PadX
   option PadY
   option Relief
   option Text
   option Width

   option Aspect "-aspect" (int)
   option Justify "-justify" (Justification)

   function () configure "configure" (option(message) list)
   function (string) configure_get "configure" ()
   }
\end{verbatim}
Note that the first options are given implicitly, since they belong to
``standard'' options, defined elsewhere. The two others are given in their
full form. 

Each widget class is compiled into a separate module (bearing the name of the
widget class). Besides the functions described in the entry, the compiler
produces a creation function of type \verb|Widget -> option list -> Widget|
(with, in fact, run-time verification of options, who must belong to the
subtype \verb|option(widgetname)|. The first argument is the parent of the
created widget.


\subsection{Modules}
A module is simply a set of functions to be grouped in a separate module.
\paragraph{Example}
\begin{verbatim}
module selection {
   function () clear "selection clear" (widget)
   function (string) get "selection get" ()
   }
\end{verbatim} 

\subsection{Builtins}
Some datatypes cannot be described in the syntax of the interface
description language, because they require custom converters. Thus it is
possible to write them directly in \caml\ and place 
them in a \verb|builtin_*.ml| file.
To avoid a warning message from the compiler, one may also declare the type
as external
\paragraph{Example}
\begin{verbatim}
type Units external     # builtin_GetPixel
\end{verbatim} 
For each builtin type \verb|foo|, one should provide a \verb|CAMLtoTKfoo|
function, of type \verb|foo -> string|, and, if data of this type is to be
returned by a function, a parser \verb|TKtoCAMLfoo|.
For example, the file \verb|builtin_GetPixel.ml| contains the \verb|Units|
type, used extensively in \tk\ for specifying distances or coordinates.
It requires both a pretty-printer and a parser (which emulates the
\verb|tkGetPixel| function in the \tk\ library).


\section{Compiling}
The interface description source is compiled with
\begin{verbatim}
$ tkcompiler filename
\end{verbatim}
(the file compiled defaults to \verb|Widgets.src| if \verb|filename| is
omitted).
The compiler requires the existence of a \verb|lib| subdirectory, where it
will produce \verb|tkgen.ml|, \verb|modules|, and various \caml\ files
corresponding to entries in the interface description source.

The compiler will report the following errors
\begin{description}
\item[lexical errors]
\item[syntax errors] see the grammar of the language below
\item[duplicate definitions]
\item[illegal implicit use of constructors]
\item[cyclic dependancy on types]
\end{description} 

Producing the library requires the presence of several files in the
\verb|lib| directory. For more details, check the \verb|Makefile| in
\verb|lib|. 

\section{Run-time support}
Apart from the modules produced by compiling the widget description file,
the \camltk\ interface uses low-level support for some basic data types and
communication protocol between \caml\ and \tcl.
The files produced by \verb|tkcompiler| are independant of the
architecture. They only rely on the functionalities of the low-level layers,
as described in their interfacs \verb|protocol.mli| and \verb|support.mli|.
The following sections describe the internals of these layers.

\subsection{Architecture}
There are two versions of the architecture.
\subsubsection{Two-process architecture}
The interface is implemented as a communication protocol between a \caml\
process and the {\tt wish} interpreter ({\tt wish} is a toplevel interactive
interpreter for  \tcl\ and \tk).
A \caml\ function of the library produces \tcl\tk\ code, and sends it to the
\wish\ interpreter. \wish\ sends callback information and results from
function calls to the \caml\ process.
All communications take place in Un*x pipes.
The support layer is implemented in the \verb|protocol2.ml|.
This version is considered obsolete and therefore not documented.

\subsubsection{Single-process architecture}
The support layer is implemented in the \verb|protocol1.ml|, and the
extensions are defined in \verb|camltk.c|. The rest of the documentation
applies to this architecture.

\subsection{Widget support}
\begin{description}
\item[Widget naming] :
in the contrary of the {\sf Tcl/Tk} approach, \camltk\ keeps the naming of
widgets internal to the library. Moreover, there is no particular data
associated to widgets in the \caml\ half of the world. A widget is
essentially its \tk\ path. The public functions are: 

\begin{verbatim}
value default_toplevel_widget : Widget
      	(* The default toplevel widget is ".", of type "toplevel" *)
;;
value new_toplevel_widget : string -> Widget
       (* the argument must be a valid window name: no 8bit char, start
          with a lowercase, ... *)
;;
\end{verbatim} 
And some functions used internally are
\begin{verbatim}
value widget_name : Widget -> string
      (* Return the name (tk "path") of a widget *)
;;
value new_widget_atom : string -> Widget -> Widget
and   new_named_widget : string -> Widget -> string -> Widget
      (* Abstract creation functions *)
;;
\end{verbatim} 

\item[Widget typing] :
\camltk\ attempts to check widget types before sending code to \tcl.
We chose to have a unique \caml\ type for widgets, \verb|Widget|, and
to check dynamically when necessary if a widget is of the required class.
Thus, internally, 
\begin{verbatim}
type Widget =
  Untyped of string
| Typed of string * string
;;
\end{verbatim} 
We also maintain a table associating {\em widget paths} to {\em Widgets}.
The reason for this table is that we may have to dynamically retype a widget
path returned by a \tcl\ function.

\item[Quoting strings] :
\tcl\ has several special characters. Thus, \caml\ strings must be
transformed by quoting these characters, to avoid their interpretation by
\tcl.
\end{description}


\subsection{Transferring control}
\caml\ has been extended with a new primitive \verb|tcl_eval| for evaluating
\tcl\ source phrases, and \tcl\ has been extended with a new command
\verb|camlcb| to call \caml\ (or more precisely a given \caml\ function).

The implementation of \verb|tcl_eval| is straightforward.

\verb|camlcb| relies on the \verb|callback| primitive of the \caml\
runtime. Currently, the \caml\ runtime does not allow storage of \caml\
values in C global variables, because global variables cannot be roots for
the garbage collector. The implementation of \verb|camlcb| uses the same
trick as signal handling.

\subsection{Other \camltk\ primitives}
The other primitives, written in C, added the \caml\ runtime deal with
\begin{description}
\item[initialising a \tcl\ interpreter] :  straightforward
\item[file descriptor callbacks] : the mechanism of file descriptor
callbacks is available in \tk, although not as \tcl\ commands. We simply
provide a \caml\ primitive to deal with this form of callbacks.
\end{description} 

\subsection{Transferring data}
We provide {\em extensible buffers} to simplify the production of the \tcl\
source code to be evaluated by \tcl.

\subsection{Callback handling}
Callbacks are \caml\ functions. They have to be invoked by \tcl, using the
\verb|camlcb| command. The link between \tcl\ and \caml\ is made through
a hash-table, associating an identifier (string) to a \caml\ function.

In order to avoid space leaks due to storing callback closures in the table,
we implement a memo-mechanism associating callbacks to widgets.
Except for class wide callbacks, each callback is defined for a given
widget. When this widget is destroyed, the callback cannot be invoked any
more, and we can remove it from the table. We thus provide a general binding
for all widgets, on event \verb|Destroy|, that will remove all callbacks
associated to a widget being destroyed.


% Not included in html version
\chapter{Reference Manual (incomplete)}
\label{chap:ref}
\section*{Module {\tt tk}}
Builtin functions are:
\begin{alltt}
value OpenTk : unit -> Widget;;
value OpenTkClass : string -> Widget;;

value bind : Widget -> (Modifier list * XEvent) list -> BindAction -> unit;;

value canvas_bind : Widget -> TagOrId 
     -> (Modifier list * XEvent) list -> BindAction -> unit;;

value text_tag_bind : Widget -> string 
     -> (Modifier list * XEvent) list -> BindAction -> unit;;

value add_fileinput : file_descr -> (unit -> unit) -> unit;;

value remove_fileinput : file_descr -> unit;;
\end{alltt}

\noindent
The public functions generated from {\tt Widgets.src} are:
\begin{alltt}
\input{../lib/tkgen.mli}
\end{alltt}

%% The generated modules
\include{inclmod}

%% Other builtin modules
\section*{Module {\tt textvariable}}
\begin{alltt}
value new : unit -> TextVariable;;

value set : TextVariable -> string -> unit;;

value get : TextVariable -> string;;
\end{alltt}


\begin{htmlonly}
\chapter{Reference manual}
{\Large WARNING: The on-line reference manual corresponds to the version
being developped NOW. It may not be accurate for the last released version}

\htmladdnormallink{This}{/bin/psearch/doc-camltk/library}
is the on-line version.
\end{htmlonly}



\bibliography{tk}
\bibliographystyle{plain}
\appendix
\chapter*{BNF syntax of the interface description language}
\begin{verbatim}
Type0 :
      TYINT
    | TYFLOAT
    | TYBOOL
    | TYCHAR
    | TYSTRING
    | WIDGET
    | IDENT

Type01 :
       Type0
     | IDENT LPAREN IDENT RPAREN
     | WIDGET LPAREN IDENT RPAREN
     | OPTION LPAREN IDENT RPAREN

Type1 :
      Type01
    | Type01 LIST

Type1list :
      Type1 COMMA Type1list
    | Type1

Typearg :
      LPAREN RPAREN
    | LPAREN Type1 RPAREN
    | LPAREN Type1list RPAREN

Type :
     Typearg
   | LPAREN FUNCTION Typearg RPAREN

Constructor :
     IDENT STRING
   | IDENT Type
   | IDENT STRING Type

AbbrevConstructor :
     Constructor
   | IDENT

Constructors :
     Constructor Constructors
   | Constructor

AbbrevConstructors :
     AbbrevConstructor AbbrevConstructors
   | AbbrevConstructor

Command :
     FUNCTION Typearg IDENT STRING Type

Option :
     OPTION IDENT STRING Type
   | OPTION IDENT

WidgetComponent :
     Command
   | Option

WidgetComponents :
     /* epsilon */
   | WidgetComponent WidgetComponents

ModuleComponents :
     /* epsilon */
   | Command ModuleComponents

Entry :
     WIDGET IDENT LBRACE WidgetComponents RBRACE
   | Command
   | TYPE IDENT LBRACE Constructors RBRACE
   | TYPE IDENT EXTERNAL
   | SUBTYPE OPTION LPAREN IDENT RPAREN LBRACE AbbrevConstructors RBRACE
   | SUBTYPE IDENT LPAREN IDENT RPAREN LBRACE AbbrevConstructors RBRACE  
   | MODULE IDENT LBRACE ModuleComponents RBRACE
   | EOF
\end{verbatim} 

\end{document}

