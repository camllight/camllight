\mysection{{\tt str}: regular expressions and high-level string processing }

\label{s:str}
\index{str (module)@\verb`str` (module)}%

\subsection*{Regular expressions }\begin{verbatim}
type regexp
\end{verbatim}
\begin{comment}
 The type of compiled regular expressions. 
\end{comment}
\begin{verbatim}
value regexp: string -> regexp
\end{verbatim}
\index{regexp@\verb`regexp`}%
\begin{comment}
 Compile a regular expression. The syntax for regular expressions
           is the same as in Gnu Emacs. The special characters are
           \verb$^.*+?[]\. The following constructs are recognized:
\\[\smallskipamount]          \verb.      matches any character except newline
\\[\smallskipamount]          \verb*      (postfix) matches the previous expression zero, one or
                    several times
\\[\smallskipamount]          \verb-      (postfix) matches the previous expression one or
                    several times
\\[\smallskipamount]          \verb?      (postfix) matches the previous expression once or
                    not at all
\\[\smallskipamount]          \verb[..]   character set; ranges are denoted with \verb-, as in \verba-z;
                    an initial \verb^, as in \verb^0-9, complements the set
\\[\smallskipamount]          \verb^      matches at beginning of line
\\[\smallskipamount]          \verb$      matches at end of line
\\[\smallskipamount]          \verb\|     (infix) alternative between two expressions
\\[\smallskipamount]          \verb\(..\) grouping and naming of the enclosed expression
\\[\smallskipamount]          \verb\1     the text matched by the first \verb\(...\) expression
                    (\verb\2 for the second expression, etc)
\\[\smallskipamount]          \verb\b     matches word boundaries
\\[\smallskipamount]          \verb\^, \$ quotes special characters. 
\end{comment}
\begin{verbatim}
value regexp_case_fold: string -> regexp
\end{verbatim}
\index{regexp_case_fold@\verb`regexp_case_fold`}%
\begin{comment}
 Same as \verbregexp, but the compiled expression will match text
           in a case-insensitive way: uppercase and lowercase letters will
           be considered equivalent. 
\end{comment}
\subsection*{String matching and searching }\begin{verbatim}
value string_match: regexp -> string -> int -> bool
\end{verbatim}
\index{string_match@\verb`string_match`}%
\begin{comment}
 \verbstring_match r s start tests whether the characters in \verbs
           starting at position \verbstart match the regular expression \verbr. 
\end{comment}
\begin{verbatim}
value search_forward: regexp -> string -> int -> int
\end{verbatim}
\index{search_forward@\verb`search_forward`}%
\begin{comment}
 \verbsearch_forward r s start searchs the string \verbs for a substring
           matching the regular expression \verbr. The search starts at position
           \verbstart and proceeds towards the end of the string.
           Return the position of the first character of the matched
           substring, or raise \verbNot_found if no substring matches. 
\end{comment}
\begin{verbatim}
value search_backward: regexp -> string -> int -> int
\end{verbatim}
\index{search_backward@\verb`search_backward`}%
\begin{comment}
 Same as \verbsearch_forward, but the search proceeds towards the
           beginning of the string. 
\end{comment}
\begin{verbatim}
value matched_string: string -> string
\end{verbatim}
\index{matched_string@\verb`matched_string`}%
\begin{comment}
 \verbmatched_string s returns the substring of \verbs that was matched
           by the latest \verbstring_match, \verbsearch_forward or \verbsearch_backward.
           The user must make sure that the parameter \verbs is the same string
           that was passed to the matching or searching function. 
\end{comment}
\begin{verbatim}
value match_beginning: unit -> int
value match_end: unit -> int
\end{verbatim}
\index{match_beginning@\verb`match_beginning`}%
\index{match_end@\verb`match_end`}%
\begin{comment}
 \verbmatch_beginning() returns the position of the first character
           of the substring that was matched by \verbstring_match,
           \verbsearch_forward or \verbsearch_backward. \verbmatch_end() returns
           the position of the character following the last character of
           the matched substring. 
\end{comment}
\begin{verbatim}
value matched_group: int -> string -> string
\end{verbatim}
\index{matched_group@\verb`matched_group`}%
\begin{comment}
 \verbmatched_group n s returns the substring of \verbs that was matched
           by the \verbnth group \verb\(...\) of the regular expression during
           the latest \verbstring_match, \verbsearch_forward or \verbsearch_backward.
           The user must make sure that the parameter \verbs is the same string
           that was passed to the matching or searching function. 
\end{comment}
\begin{verbatim}
value group_beginning: int -> int
value group_end: int -> int
\end{verbatim}
\index{group_beginning@\verb`group_beginning`}%
\index{group_end@\verb`group_end`}%
\begin{comment}
 \verbgroup_beginning n returns the position of the first character
           of the substring that was matched by the \verbnth group of
           the regular expression. \verbgroup_end n returns
           the position of the character following the last character of
           the matched substring. 
\end{comment}
\subsection*{Replacement }\begin{verbatim}
value global_replace: regexp -> string -> string -> string
\end{verbatim}
\index{global_replace@\verb`global_replace`}%
\begin{comment}
 \verbglobal_replace regexp s repl returns a string identical to \verbs,
           except that all substrings of \verbs that match \verbregexp have been
           replaced by \verbrepl. The replacement text \verbrepl can contain
           \verb\1, \verb\2, etc; these sequences will be replaced by the text
           matched by the corresponding group in the regular expression.
           \verb\0 stands for the text matched by the whole regular expression. 
\end{comment}
\begin{verbatim}
value replace_first: regexp -> string -> string -> string
\end{verbatim}
\index{replace_first@\verb`replace_first`}%
\begin{comment}
 Same as \verbglobal_replace, except that only the first substring
           matching the regular expression is replaced. 
\end{comment}
\begin{verbatim}
value global_substitute: regexp -> (string -> string) -> string -> string
\end{verbatim}
\index{global_substitute@\verb`global_substitute`}%
\begin{comment}
 \verbglobal_substitute regexp subst s returns a string identical
           to \verbs, except that all substrings of \verbs that match \verbregexp
           have been replaced by the result of function \verbsubst. The
           function \verbsubst is called once for each matching substring,
           and receives \verbs (the whole text) as argument. 
\end{comment}
\begin{verbatim}
value substitute_first: regexp -> (string -> string) -> string -> string
\end{verbatim}
\index{substitute_first@\verb`substitute_first`}%
\begin{comment}
 Same as \verbglobal_substitute, except that only the first substring
           matching the regular expression is replaced. 
\end{comment}
\subsection*{Splitting }\begin{verbatim}
value split: regexp -> string -> string list
\end{verbatim}
\index{split@\verb`split`}%
\begin{comment}
 \verbsplit r s splits \verbs into substrings, taking as delimiters
           the substrings that match \verbr, and returns the list of substrings.
           For instance, \verbsplit (regexp "[ \t]+") s splits \verbs into
           blank-separated words. 
\end{comment}
\begin{verbatim}
value bounded_split: regexp -> string -> int -> string list
\end{verbatim}
\index{bounded_split@\verb`bounded_split`}%
\begin{comment}
 Same as \verbsplit, but splits into at most \verbn substrings,
           where \verbn is the extra integer parameter. 
\end{comment}
\subsection*{Joining }\begin{verbatim}
value concat: string list -> string
\end{verbatim}
\index{concat@\verb`concat`}%
\begin{comment}
 Catenate a list of string. 
\end{comment}
\begin{verbatim}
value join: string -> string list -> string
\end{verbatim}
\index{join@\verb`join`}%
\begin{comment}
 Catenate a list of string. The first argument is a separator, which
           is inserted between the strings. 
\end{comment}
\subsection*{Extracting substrings }\begin{verbatim}
value string_before: string -> int -> string
\end{verbatim}
\index{string_before@\verb`string_before`}%
\begin{comment}
 \verbstring_before s n returns the substring of all characters of \verbs
           that precede position \verbn (excluding the character at 
           position \verbn). 
\end{comment}
\begin{verbatim}
value string_after: string -> int -> string
\end{verbatim}
\index{string_after@\verb`string_after`}%
\begin{comment}
 \verbstring_after s n returns the substring of all characters of \verbs
           that follow position \verbn (including the character at 
           position \verbn). 
\end{comment}
\begin{verbatim}
value first_chars: string -> int -> string
\end{verbatim}
\index{first_chars@\verb`first_chars`}%
\begin{comment}
 \verbfirst_chars s n returns the first \verbn characters of \verbs.
           This is the same function as \verbstring_before. 
\end{comment}
\begin{verbatim}
value last_chars: string -> int -> string
\end{verbatim}
\index{last_chars@\verb`last_chars`}%
\begin{comment}
 \verblast_chars s n returns the last \verbn characters of \verbs. 
\end{comment}
\subsection*{Formatting }\begin{verbatim}
value format: string -> 'a
\end{verbatim}
\index{format@\verb`format`}%
\begin{comment}
 Same as \verbfprintf and \verbprintf from the \verbprintf module,
           except that the result of the formatting is returned as a
           string instead of being written on a channel. The structure
           of the format string is described in \verbprintf. 
\end{comment}
