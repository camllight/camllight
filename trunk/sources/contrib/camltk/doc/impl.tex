\chapter{Implementor's Manual}
\label{chap:impl}
\tk\ is still evolving, and also offers numerous user-contributed
extensions. Thus, we chose to generate the library from a high-level
description of available widgets and functions. 
We describe in this chapter
the syntax and semantics of the description language.

\section{The interface description language}
The language allows the specification of
\begin{description}
\item[widgets] of the standard \tk\ library, with options and associated commands
\item[toplevel functions,] not associated with a specific widget class
\item[modules,] grouping families of functions.
\end{description} 

A source file is simply a collection of entries, each entry describing an
element of the interface. The {\tt tkcompiler} provided in this distribution
takes the source file and produces \caml\ modules forming the interface
library.

\section{Types}
A type entry contains information to translate data from \caml\ to \tk\ 
and the converse. Each data type that needs to be exchanged by \caml\ and
\tk\ must be described by a type entry. For each type, a conversion function
from \caml\ to \tk\ is generated, as well as a parsing function from \tk\ to
\caml\ if required (because values of the type are returned by some
function in the library).

A type is given by a set of value constructors, as in ML. For each
constructor, one must give the ML name, and optionally
\begin{itemize}
\item the \tk\ name of the constructor
\item the type of its arguments
\end{itemize} 

\paragraph{Example}
\begin{verbatim}
type State { 
   Normal "normal"
   Active "active"
   Disabled "disabled"
}
\end{verbatim} 
defines the \caml\ type
\begin{verbatim}
type State =
          Disabled
        | Active
        | Normal
;;
\end{verbatim} 
and the conversion function
\begin{verbatim}
let CAMLtoTKState = function
          Disabled -> "disabled"
        | Active -> "active"
        | Normal -> "normal"
;;
\end{verbatim} 

The \tk\ source-code produced from the ML value is the \tk\ name (if any)
followed by the \tk\ version of the argument (if any).
The parser can be generated only in restricted cases: zeroary constructors,
and at most one anonymous \verb|int| constructor and one anonymous
\verb|string| constructor (anonymous means that the \tk\ name is an empty
string). Otherwise, the compiler will produce a warning
message, and you have to write the parser yourself. 

\section{Subtypes}
As customary, types will be statically verified in ML programs. However, in
order to reduce the number of value constructors required by the interface
(and associated naming problems),
we introduced a notion of {\em subtypes}. A subtype is a named subset of the
set of constructors of a type. In this case, type-checking (i.e. verifiying
that a constructor belongs to a subtype) will occur at run-time.

If a type requires subtyping, then one should not declare the type itself,
but instead define each of its subtypes. The compiler will consider the
whole set of constructors for the definition of the type. Note also that the
definition of a constructor may be omitted if it has  already been declared.

\paragraph{Example}
\begin{verbatim}
subtype Index(entry) {
   Number (int)
   End "end"
   Insert "insert"
   SelFirst "sel.first"
   SelLast "sel.last" 
}
subtype Index(listbox) {
   Number End Insert 
}
\end{verbatim}
defines two subtypes (\verb|entry| and \verb|listbox|) of the type
\verb|Index|.  

Subtypes may be referred to (in argument types of functions) with the same
syntax: \verb|type(subtype)|.
\paragraph{Example}
\begin{verbatim}
   function () delete "delete" (Index(listbox), Index(listbox))
\end{verbatim} 

\section{Widgets}
A widget class description is composed of its name, the set of valid options
for the creation of widgets of this class, and the set of functions and
commands associated to the class.

Options descriptions follow the same syntax as value constructors in type or
subtype declarations, except for the presence of the \verb|option| keyword.
Actually, the options declared inside a widget description form a subtype
(with the name of the class) of the \verb|option| type. 
\verb|option| is not really hardwired in the compiler, in that it obeys the
same rules for subtyping as user-defined types. However the \verb|option|
type is implicitly used when producing the widget creation functions.

Function declarations are formed of
\begin{itemize}
\item the \verb|function| keyword
\item the result type of the function (inside parenthesis)
\item the ML name of the function
\item the corresponding \tk\ name
\item the type of the arguments (inside parenthesis)
\end{itemize} 

The ML function produced from this description always takes a widget as
first argument. The function is curryfied, although the notation seems to
imply that the function has only one tuple argument.


\paragraph{Example}
\begin{verbatim}
widget message {
   option Anchor
   option Background
   option Borderwidth
   option Cursor
   option Font
   option Foreground
   option PadX
   option PadY
   option Relief
   option Text
   option Width

   option Aspect "-aspect" (int)
   option Justify "-justify" (Justification)

   function () configure "configure" (option(message) list)
   function (string) configure_get "configure" ()
   }
\end{verbatim}
Note that the first options are given implicitly, since they belong to
``standard'' options, defined elsewhere. The two others are given in their
full form. 

Each widget class is compiled into a separate module (bearing the name of the
widget class). Besides the functions described in the entry, the compiler
produces a creation function of type \verb|Widget -> option list -> Widget|
(with, in fact, run-time verification of options, who must belong to the
subtype \verb|option(widgetname)|. The first argument is the parent of the
created widget.


\section{Modules}
A module is simply a set of functions to be grouped in a separate module.
\paragraph{Example}
\begin{verbatim}
module selection {
   function () clear "selection clear" (widget)
   function (string) get "selection get" ()
   }
\end{verbatim} 

\section{Builtins}
Some datatypes cannot be described in the syntax of the interface
description language, because they require custom pretty-printers and
parsers. Thus it is possible to write them directly in \caml\ and place
them in a \verb|builtin_*.ml| file.
To avoid a warning message from the compiler, one may also declare the type
as external
\paragraph{Example}
\begin{verbatim}
type Units external     # builtin_GetPixel
\end{verbatim} 
For each builtin type \verb|foo|, one should provide a \verb|CAMLtoTKfoo|
function, of type \verb|foo -> string|, and, if data of this type is to be
returned by a function, a parser \verb|TKtoCAMLfoo|.
For example, the file \verb|builtin_GetPixel.ml| contains the \verb|Units|
type, used extensively in \tk\ for specifying distances or coordinates.
It requires both a pretty-printer and a parser (which emulates the
\verb|tkGetPixel| function in the \tk\ library).


\section{Compiling}
The interface description source is compiled with
\begin{verbatim}
$ tkcompiler filename
\end{verbatim}
(the file compiled defaults to \verb|Widgets.src| if \verb|filename| is
omitted).
The compiler requires the existence of a \verb|lib| subdirectory, where it
will produce \verb|tkgen.ml|, \verb|modules|, and various \caml\ files
corresponding to entries in the interface description source.

The compiler will report the following errors
\begin{description}
\item[lexical errors]
\item[syntax errors] see the grammar of the language below
\item[duplicate definitions]
\item[illegal implicit use of constructors]
\item[cyclic dependancy on types]
\end{description} 

Producing the library requires the presence of several files in the
\verb|lib| directory. For more details, check the \verb|Makefile| in
\verb|lib|. 

\section{The architecture}
Note that the files produced by \verb|tkcompiler| are independant of the
architecture. They only rely on the \verb|protocol.mli| specification of the
lower layer of communication.
\subsection{Two-process architecture}
The interface is implemented as a communication protocol between a \caml\
process and the {\tt wish} interpreter ({\tt wish} is a toplevel interactive
interpreter for  \tcl\ and \tk).
A \caml\ function of the library produces \tcl\tk\ code, and sends it to the
\wish\ interpreter. \wish\ sends callback information and results from
function calls to the \caml\ process.
The support layer is implemented in the \verb|protocol2.ml|.


\subsection{Single-process architecture}
\caml\ has been extended with a new primitive for
evaluating \tcl\ source phrases, and \tcl\ has been extended to call \caml\
(or more precisely a given \caml function). 
The support layer is implemented in the \verb|protocol1.ml|, and the
extensions are defined in \verb|camltk.c|.
