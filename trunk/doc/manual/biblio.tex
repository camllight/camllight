\chapter{Further reading}

For the interested reader, we list below some references to books and
reports related (sometimes loosely) to Caml Light.

\section{Programming in ML}

The books below are programming courses taught in ML. Their main goal
is to teach programming, not to describe ML in full details --- though
most contain fairly good introductions to the ML language. Some of
those books use the Standard ML dialect instead of the Caml dialect,
so you will have to keep in mind the differences in syntax and in
semantics.

\begin{itemize}

\item Pierre Weis and Xavier Leroy. {\it Le langage Caml.}
Inter�ditions, 1993.

The natural companion to this manual, provided you read French. This
book is a step-by-step introduction to programming in Caml, and
presents many realistic examples of Caml programs.

\item Lawrence C.\ Paulson. {\it ML for the working programmer.}
Cambridge University Press, 1991.

An excellent introduction to programming in Standard ML. Develops a
theorem prover as a complete example. Contains a presentation of
the module system of Standard ML.

\item Chris Reade. {\em Elements of Functional Programming.}
Addison-Wesley, 1989.

An introduction to Standard ML, with sections on denotational
semantics and lambda-calculus.

\item Ryan Stansifer. {\em ML primer.} Prentice-Hall, 1992.

A short, but nice introduction to programming in Standard ML.

\item Th�r�se Accart Hardin and V�ronique Donzeau-Gouge Vigui�. {\em
Concepts et outils de la programmation. Du fonctionnel �
l'imp�ratif avec Caml et Ada.} Inter�ditions, 1992.

A first course in programming, that first introduces the main programming
notions in Caml, then shows them underlying Ada. Intended for
beginners; slow-paced for the others.

\item Ake Wikstrom. {\em Functional programming using Standard ML}.
Prentice-Hall, 1987.

A first course in programming, taught in Standard ML. Intended for
absolute beginners; slow-paced for the others.

\item Harold Abelson and Gerald Jay Sussman.
{\em Structure and Interpretation of Computer Programs.} The MIT
press, 1985.  (French translation: {\em Structure et interpr�tation
des programmes informatiques}, Inter�ditions, 1989.)

An outstanding course on programming, taught in Scheme, the modern
dialect of Lisp. Well worth reading, even if you are more interested
in ML than in Lisp.

\end{itemize}

\section{Descriptions of ML dialects}

The books and reports below are descriptions of various programming
languages from the ML family. They assume some familiarity with ML.

\begin{itemize}

\item Xavier Leroy and Pierre Weis. {\em Manuel de r�f�rence du
langage Caml.} Inter�ditions, 1993.

The French edition of the present reference manual and user's manual.

\item Robert Harper. {\em Introduction to Standard ML.} Technical
report ECS-LFCS-86-14, University of Edinburgh, 1986.

An overview of Standard ML, including the module system. Terse, but
still readable.

\item Robin Milner, Mads Tofte and Robert Harper. {\em The definition
of Standard ML.} The MIT press, 1990.

A complete formal definition of Standard ML, in the framework of
structured operational semantics. This book is probably the most
mathematically precise definition of a programming language ever
written. It is heavy on formalism and horribly terse, so
even readers that are thoroughly familiar with ML will have
major difficulties with it.

\item Robin Milner and Mads Tofte. {\em Commentary on Standard ML.}
The MIT Press, 1991.

A commentary on the book above, that attempts to explain the most
delicate parts and motivate the design choices. Easier to read than the
Definition, but still rather involving.

\item Guy Cousineau and G�rard Huet. {\em The CAML primer.} Technical
report~122, INRIA, 1990.

A short description of the original Caml system, from which Caml Light
has evolved. Some familiarity with Lisp is assumed.

\item Pierre Weis et al. {\em The CAML reference manual, version
2.6.1.} Technical report~121, INRIA, 1990.

The manual for the original Caml system, from which Caml Light
has evolved.  

\item Michael J.\ Gordon, Arthur J.\ Milner and Christopher P.\
Wadsworth. {\em Edinburgh LCF.} Lecture Notes in Computer Science
volume~78, Springer-Verlag, 1979.

This is the first published description of the ML language, at the
time when it was nothing more than the control language for the LCF
system, a theorem prover. This book is now obsolete, since the ML
language has much evolved since then; but it is still of historical
interest.

\item Paul Hudak, Simon Peyton-Jones and Philip Wadler. {\em
Report on the programming language Haskell, version 1.1.} Technical
report, Yale University, 1991.

Haskell is a purely functional language with lazy semantics that
shares many important points with ML (full functionality, polymorphic
typing), but has interesting features of its own (dynamic overloading,
also called type classes).

\section{Implementing functional programming languages}

The references below are intended for those who are curious to learn
how a language like Caml Light is compiled and implemented.

\item Xavier Leroy. {\em The ZINC experiment: an economical
implementation of the ML language.} Technical report~117, INRIA, 1990.
(Available by anonymous FTP on {\machine ftp.inria.fr}.)

A description of the ZINC implementation, the prototype ML
implementation that has evolved into Caml Light. Large parts of this
report still apply to the current Caml Light system, in particular the
description of the execution model and abstract machine. Other parts
are now obsolete. Yet this report still gives a complete overview of the
implementation techniques used in Caml Light.

\item Simon Peyton-Jones. {\em The implementation of functional
programming languages.} Prentice-Hall, 1987. (French translation:
{\em Mise en \oe uvre des langages fonctionnels de programmation},
Masson, 1990.)

An excellent description of the implementation of purely functional
languages with lazy semantics, using the technique known as graph
reduction. The part of the book that deals with the transformation
from ML to enriched lambda-calculus directly applies to Caml Light.
You will find a good description of how pattern-matching is compiled
and how types are inferred. The remainder of the book does not apply
directly to Caml Light, since Caml Light is not purely functional (it
has side-effects), has strict semantics, and does not use graph
reduction at all.

\item Andrew W.\ Appel. {\em Compiling with continuations.} Cambridge
University Press, 1992.

A complete description of an optimizing compiler for Standard ML,
based on an intermediate representation called continuation-passing
style. Shows how many advanced program optimizations can be applied to
ML. Not directly relevant to the Caml Light system, since Caml Light
does not use continuation-passing style at all, and makes little
attempts at optimizing programs.

\end{itemize}

\section{Applications of ML}

The following reports show ML at work in various, sometimes
unexpected, areas.

\begin{itemize}

\item Emmanuel Chailloux and Guy Cousineau. {\em The MLgraph primer.}
Technical report 92-15, �cole Normale Sup�rieure, 1992. (Available by
anonymous FTP on {\machine ftp.ens.fr}.)
%, r�pertoire {\machine biblio}, fichier
% {\machine liens-92-15.A4.300dpi.ps.Z}.)

Describes a Caml Light library that produces Postscript pictures
through high-level drawing functions.

\item Xavier Leroy. {\em Programmation du syst�me Unix en Caml Light.}
Technical report~147, INRIA, 1992. (Available by anonymous FTP on
{\machine ftp.inria.fr}.)
%, r�pertoire {\machine INRIA/publication}, fichier {\machine RT-0147.ps.Z}.)

A Unix systems programming course, demonstrating the use of the Caml
Light library that gives access to Unix system calls.

\item John H.\ Reppy. {\em Concurrent programming with events --- The
concurrent ML manual.} Cornell University, 1990.
(Available by anonymous FTP on {\machine research.att.com}.)
%, r�pertoire {\machine dist/ml}, fichier {\machine CML-0.9.8.tar.Z}.)

Concurrent ML extends Standard ML of New Jersey with concurrent
processes that communicate through channels and events.

\item Jeannette M. Wing, Manuel Faehndrich, J.\ Gregory Morrisett and
Scottt Nettles. {\em Extensions to Standard ML to support
transactions.} Technical report CMU-CS-92-132, Carnegie-Mellon
University, 1992. (Available by anonymous FTP on
{\machine reports.adm.cs.cmu.edu}.)
% , r�pertoire {\machine 1992}, fichier {\machine CMU-CS-92-132.ps}.)

How to integrate the basic database operations to Standard ML.

\item Emden R.\ Gansner and John H.\ Reppy. {\em eXene.} Bell Labs,
1991. (Available by anonymous FTP on {\machine research.att.com}.)
%, r�pertoire {\machine dist/ml}, fichier {\machine eXene-0.4.tar.Z}.)

An interface between Standard ML of New Jersey and the X Windows
windowing system.

%% \item Daniel de Rauglaudre. {\em X toolkit in Caml Light.} INRIA,
%% 1992. (Included in the Caml Light distribution.)
%% % Disponible par FTP anonyme sur
%% % {\machine ftp.inria.fr}, r�pertoire {\machine lang/caml-light}, fichier {\machine rt5.tar.Z}.)
%% 
%% An interface between Caml Light and the X Windows windowing system. 

\end{itemize}
