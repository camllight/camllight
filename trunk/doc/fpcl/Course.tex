\title{Functional programming \\ using Caml Light}
\author{Michel Mauny}
\date{January 1995}

\emergencystretch=50pt  % pour que TeX resolve les overfull hbox lui-meme
\setcounter{tocdepth}{1}        % Pour ne pas mettre les \subsection
                                % dans la table des matieres
\def\CAMLVERSION{0.7}
\newtheorem{exoth}{Exercise}[chapter]
\def\exo{\begin{exoth}\ifx\exofont\undefined\else\exofont\fi}
\def\endexo{\end{exoth}}
\def\Example{\paragraph{Example}}
\def\End{\par\noindent$\Box$\par}
\def\ikwd#1{}

\begin{document}

\maketitle

\cleardoublepage

\tableofcontents

\chapter{Introduction}

This document is a tutorial introduction to functional programming, and,
more precisely, to the usage of Caml Light.  It has been used to teach
Caml Light\footnote{The ``Caml Strong'' version of these notes is available
as an INRIA technical report~\cite{BDC}.} in different universities
and is intended for beginners. It contains numerous examples and
exercises, and absolute beginners should read it while sitting in
front of a Caml Light toplevel loop, testing examples and variations
by themselves.

After generalities about functional programming, some features
specific to Caml Light are described. ML type synthesis and a simple
execution model are presented in a complete example of prototyping a
subset of ML.

Part~I (chapters \ref{c:gen}--\ref{c:udeftypes}) may be skipped by
users familiar with ML. Users with experience in functional programming,
but unfamiliar with the ML dialects may skip the very first chapters and
start at chapter~\ref{c:udeftypes}, learning the Caml Light syntax
from the examples.  Part~I starts with some intuition about functions
and types and gives an overview of ML and other functional languages
(chapter~\ref{c:gen}). Chapter~\ref{c:basicconcepts} outlines the
interaction with the Caml Light toplevel loop and its basic objects.
Basic types and some of their associated primitives are presented in
chapter~\ref{c:basic}. Lists (chapter \ref{c:lists}) and user-defined
types (chapter \ref{c:udeftypes}) are structured data allowing for the
representation of complex objects and their easy creation and
destructuration.

While concepts presented in part~I are common (under one form or
another) to many functional languages, part~B (chapters
\ref{c:mutable}--\ref{c:standalone}) is dedicated to features
specific to Caml Light: mutable data structures (chapter
\ref{c:mutable}), exception handling (chapter \ref{c:exceptions}),
input/output (chapter \ref{c:basicio}) and streams and parsers (chapter
\ref{c:streams}) show a more imperative side of the language.
Standalone programs and separate compilation (chapter
\ref{c:standalone}) allow for modular programming and the creation of
standalone applications. Concise examples of Caml Light features are
to be found in this part.

Part~C (chapters \ref{c:ASL}--\ref{c:ASLcompiling}) is meant for
already experienced Caml Light users willing to know more about how
the Caml Light compiler synthesizes the types of expression and how
compilation and evaluation proceeds.  Some knowledge about first-order
unification is assumed. The presentation is rather informal, and is
sometimes terse (specially in the chapter about type synthesis). We
prototype a small and simple functional language (called ASL): we give
the complete prototype implementation, from the ASL parser to the
symbolic execution of code.  Lexing and parsing of ASL programs are
presented in chapter
\ref{c:ASL}, providing realistic usages of streams and parsers.
Chapter \ref{c:ASLsemantics} presents an untyped call-by-value
semantics of ASL programs through the definition of an ASL
interpreter. The encoding of recursion in untyped ASL is presented in
chapter \ref{c:ASLuntypedrecursion}, showing the expressive power of
the language. The type synthesis of functional programs is
demonstrated in chapter \ref{c:ASLtyping}, using destructive
unification (on first-order terms representing types) as a central
tool.  Chapter \ref{c:ASLcompiling} introduces the Categorical
Abstract Machine: a simple execution model for call-by-value
functional programs. Although the Caml Light execution model is
different from the one presented here, an intuition about the simple
compilation of functional languages can be found in this chapter.

\medskip

{\bf Warning:} The programs and remarks (especially contained in parts
B and C) might not be valid in Caml Light versions different from
\CAMLVERSION.

\part{Functional programming}
\def\Nat{{\bf N}}

\chapter{Functional languages}
\label{c:gen}

Programming languages are said to be {\em functional} when the basic
way of structuring programs is the notion of {\em function} and their
essential control structure is {\em function application}.  For
example, the Lisp language \cite{MacCarthy}, and more precisely its
modern successor Scheme \cite{SchemeReport,AbelsonSussman}, has been called
functional because it possesses these two properties.

However, we want the programming notion of function to be as close
as possible to the usual mathematical notion of function. In
mathematics, functions are ``first-class'' objects: they can be
arbitrarily manipulated. For example, they can be composed, and the
composition function is itself a function.

In mathematics, one would present the {\em successor} function in the
following way:
\[\begin{array}{ll}
\mbox{\em successor} : & \Nat ~ \longrightarrow ~ \Nat \\
                       & n ~ \longmapsto ~ n+1
\end{array}
\]
The functional composition could be presented as:
\[\begin{array}{ll}
\circ : & (A \Rightarrow B) \times (C \Rightarrow A)
                        ~ \longrightarrow ~ (C \Rightarrow B) \\
               & (f,g) ~ \longmapsto ~ (x~\longmapsto~f~(g~x))
\end{array}
\]
where $(A \Rightarrow B)$ denotes the space of functions from $A$ to $B$.

We remark here the importance of:
\begin{enumerate}
\item the notion of {\em type}; a mathematical function always possesses a
{\em domain} and a {\em codomain}. They will correspond to the programming
notion of type.
\item lexical binding: when we wrote the mathematical definition of
{\em successor}, we have assumed that the addition function $+$ had been
previously defined, mapping a pair of natural numbers to a natural
number; the meaning of the {\em successor} function is defined using
the {\em meaning} of the addition: whatever $+$ denotes in the future,
this {\em successor} function will remain the same.
\item the notion of {\em functional abstraction}, allowing to express
the behavior of $f \circ g$ as $(x~\longmapsto~f~(g~x))$, i.e.
the function which, when given some $x$, returns $f~(g~x)$.
\end{enumerate}
ML dialects (cf. below) respect these notions. But they also allow
non-functional programming styles, and, in this sense, they are
functional but not {\em purely functional}.

\section{History of functional languages}

Some historical points:

\begin{itemize}
\item 1930: Alonzo Church developed the $\lambda$-calculus \cite{Church} as an attempt to
           provide a basis for mathematics. The $\lambda$-calculus is a
           formal theory for functionality.
    The three basic constructs of
           the $\lambda$-calculus are:
           \begin{itemize}
              \item variable names (e.g. $x$, $y$,\ldots);
              \item application ($MN$ if $M$ and $M$ are terms);
              \item functional abstraction ($\lambda x . M$).
           \end{itemize}
           Terms of the $\lambda$-calculus represent functions. The pure
           $\lambda$-calculus has been proved inconsistent as a logical theory.
           Some {\em type systems} have been added to it in order to remedy
           this inconsistency.
\item 1958: Mac Carthy invented Lisp \cite{MacCarthy} whose programs have some similarities
           with terms of the $\lambda$-calculus.
    Lisp dialects have been recently
           evolving in order to be closer to modern functional languages
           (Scheme), but they still do not possess a type system.
\item 1965: P. Landin proposed the ISWIM \cite{ISWIM} language (for ``If You See What I Mean''),
           which is the precursor of languages of the ML family.
\item 1978: J. Backus introduced FP: a language of combinators \cite{FP} and a framework
           in which it is possible to reason about programs.
The main particularity of FP programs is that they have no variable names.
\item 1978: R. Milner proposes a language called ML \cite{ML}, intended to be the
           {\em metalanguage} of the LCF proof assistant (i.e. the language
           used to program the search of proofs). This language is inspired
           by ISWIM (close to $\lambda$-calculus) and possesses an original
           type system.
\item 1985: D. Turner proposed the Miranda \cite{Miranda85} programming language, which uses
           Milner's type system but where programs are submitted to {\em
           lazy evaluation}.
\end{itemize}
Currently, the two main families of functional languages are the ML and the
Miranda families.

\section{The ML family}

ML languages are based on a sugared\footnote{i.e. with a more user-friendly
syntax.}
version of $\lambda$-calculus.
Their evaluation regime is {\em call-by-value} (i.e. the argument is
evaluated before being passed to a function), and they use Milner's type
system.

The LCF proof system appeared in 1972 at Stanford (Stanford LCF). It
has been further developed at Cambridge (Cambridge LCF) where the ML
language was added to it.

From 1981 to 1986, a version of ML and its compiler was developed
in a collaboration between INRIA and Cambridge by G. Cousineau, G. Huet
and L. Paulson.

In 1981, L. Cardelli implemented a version of ML whose compiler generated
native machine code.

In 1984, a committee of researchers from the universities of Edinburgh
and Cambridge, Bell Laboratories and INRIA, headed by R. Milner worked
on a new extended language called Standard ML \cite{ProposalSML}. This core
language was completed by a module facility designed by D. MacQueen
\cite{SMLModules}.

Since 1984, the Caml language has been under design in a collaboration
between INRIA and LIENS\footnote{Laboratoire d'Informatique de l'Ecole
Normale Sup\'erieure, 45 Rue d'Ulm, 75005 Paris}). Its first release
appeared in 1987. The main implementors of Caml were Asc\'ander
Su\'arez, Pierre Weis and Michel Mauny.

In 1989 appeared Standard ML of New-Jersey, developed by Andrew Appel
(Princeton University) and David MacQueen (Bell Laboratories).

Caml Light is a smaller, more portable implementation of the core Caml
language, developed by Xavier Leroy since 1990.

\section{The Miranda family}

All languages in this family use {\em lazy evaluation} (i.e. the argument of a
function is evaluated if and when the function needs its value---arguments
are passed unevaluated to functions).
They also use Milner's type system.

Languages belonging to the Miranda family find their origin in the
SASL language \cite{SASL} (1976) developed by D. Turner. SASL and its
successors (KRC \cite{KRC} 1981, Miranda \cite{Miranda85} 1985 and
Haskell \cite{Haskell90} 1990) use {\em sets of mutually recursive
equations} as programs. These equations are written in a {\em script}
(collection of declarations) and the user may evaluate expressions
using values defined in this script.  LML (Lazy ML) has been developed
in G\"oteborg (Sweeden); its syntax is close to ML's syntax and it
uses a fast execution model: the G-machine \cite{GMachine}.


\chapter{Basic concepts}
\label{c:basicconcepts}

We examine in this chapter some fundamental concepts which we will
use and study in the following chapters. Some of them are specific to the
interface with the Caml language (toplevel, global definitions) while others
are applicable to any functional language.

\section{Toplevel loop}

The first contact with Caml is through its interactive
aspect\footnote{Caml Light implementations also possess a batch
compiler usable to compile files and produce standalone applications;
this will be discussed in chapter~\ref{c:standalone}.}. When running
Caml on a computer, we enter a {\em toplevel loop} where Caml waits
for input from the user, and then gives a response to what has been
entered.

The beginning of a Caml Light session looks like this (assuming
\verb|$| is the shell prompt on the host machine):
\begin{verbatim}
$camllight
>       Caml Light version 0.6

#
\end{verbatim}
On the PC version, the command is called {\tt caml}.
\def\sharp{{\tt\#}}

The ``{\sharp}'' character is Caml's prompt. When this character
appears at the beginning of a line in an actual Caml Light session,
the toplevel loop is waiting for a new toplevel phrase to be entered.

Throughout this document, the phrases starting by the {\sharp}
character represent legal input to Caml. Since this document has been
pre-processed by Caml Light and these lines have been effectively
given as input to a Caml Light process, the reader may reproduce by
him/herself the session contained in each chapter (each chapter of the
first two parts contains a different session, the third part is a
single session). Lines starting with the ``\verb|>|'' character are
Caml Light error messages. Lines starting by another
character are either Caml responses or (possibly) illegal input. The
input is printed in typewriter font ({\tt like this}), and output is
written using slanted typewriter font ({\tt \sl like that}).

{\bf Important:} Of course, when developing non-trivial programs, it
is preferable to edit programs in files and then to include the files
in the toplevel, instead of entering the programs directly.
Furthermore, when debugging, it is very useful to {\em trace} some
functions, to see with what arguments they are called, and what result
they return.  See the reference manual \cite{CamlLightDoc} for a
description of these facilities.

\section{Evaluation: from expressions to values}

Let us enter an arithmetic expression and see what is Caml's response:
\begin{caml_example}
1+2;;
\end{caml_example}
The expression ``\verb|1+2|'' has been entered, followed by
``\verb|;;|'' which represents the end of the current toplevel phrase. When
encountering ``\verb|;;|'', Caml enters the type-checking (more precisely {\em
type synthesis}) phase, and prints the inferred type for the expression. Then, it compiles code for the expression, executes it and, finally, prints the result.

In the previous example, the result of evaluation is printed
as ``\verb|3|'' and the type is ``\verb|int|'': the type of integers.

The process of evaluation of a Caml expression can be seen as transforming
this expression until no further transformation is allowed. These
transformations must preserve semantics. For example, if the expression
``\verb|1+2|'' has the mathematical object 3 as semantics, then the
result ``\verb|3|'' must have the same semantics.
The different phases of the Caml evaluation process are:

\begin{itemize}
\item parsing (checking the syntactic legality of input);
\item type synthesis;
\item compiling;
\item loading;
\item executing;
\item printing the result of execution.
\end{itemize}
%
Let us consider another example: the application of the successor function to
\verb|2+3|. The expression
\verb|(function x -> x+1)|\ikwd{function@\verb`function`}
should be read as ``the function that, given {\tt x}, returns
\verb|x+1|''. The juxtaposition of this expression to \verb|(2+3)| is
{\em function application}.
\begin{caml_example}
(function x -> x+1) (2+3);;
\end{caml_example}
There are several ways to reduce that value before obtaining the result
\verb"6". Here are two of them (the expression being reduced at each
step is underlined):
$$\begin{array}{c@{\qquad}c}
\mbox{\tt (function x -> x+1)} ~ \underline{\mbox{\tt (2+3)}} &
        \underline{\mbox{\tt (function x -> x+1) (2+3)}} \\
\downarrow & \downarrow \\
\underline{\mbox{\tt (function x -> x+1) 5}} &
        \underline{\mbox{\tt (2+3)}} ~ \mbox{\tt + 1} \\
\downarrow & \downarrow \\
\underline{\mbox{\tt 5+1}} &
        \underline{\mbox{\tt 5+1}} \\
\downarrow & \downarrow \\
\tt 6 & \tt 6
\end{array}$$

The transformations used by Caml during evaluation
cannot be described in this chapter, since
they imply knowledge about compilation of Caml programs and machine
representation of Caml values. However, since the basic control structure
of Caml is function application, it is quite easy to give an idea of
the transformations involved in the Caml evaluation process by using the
kind of rewriting we used in the last example.
The evaluation of the (well-typed) application $e_1 (e_2)$, where $e_1$
and $e_2$ denote arbitrary expressions, consists in the following
steps:
\begin{itemize}
\item Evaluate $e_2$, obtaining its value $v$.
\item
 Evaluate $e_1$ until it becomes a functional value. Because of the well-typing
 hypothesis, $e_1$ must represent a function from some type $t_1$ to some
$t_2$, and the type of $v$ is $t_1$.
We will write ({\tt function x ->} $e$) for the result of the evaluation
 of $e_1$. It denotes the mathematical object usually written as:
 \begin{center}
 $\begin{array}{ll}
  f:& t_1 \rightarrow t_2\\
    & x \mapsto e \mbox{ (where, of course, $e$ may depend on $x$)}
  \end{array}$
 \end{center}
N.B.: We do not evaluate $e$ before we know the value of $x$.
\item Evaluate $e$ where $v$ has been substituted for all occurrences of
\verb|x|. We then get the final value of the original expression. The result
is of type $t_2$.
\end{itemize}
%
In the previous example, we evaluate:
\begin{itemize}
\item \verb|2+3| to \verb|5|;
\item \verb|(function x -> x+1)| to itself (it is already a function body);
\item \verb|x+1| where \verb|5| is substituted for \verb|x|, i.e. evaluate
\verb|5+1|, getting \verb|6| as result.
\end{itemize}
%
It should be noticed that Caml uses call-by-value: arguments are
evaluated before being passed to functions. The relative evaluation
order of the functional expression and of the argument expression is
implementation-dependent, and should not be relied upon. The Caml
Light implementation evaluates arguments before functions
(right-to-left evaluation), as shown above; the original Caml
implementation evaluates functions before arguments (left-to-right evaluation).



\section{Types}


Types and their checking/synthesis are crucial to functional
programming:  they provide an indication about the correctness of
programs.

The universe of Caml values is partitioned into {\em types}.
A type represents a collection of values.
For example, \verb|int| is the type of integer numbers, and \verb|float| is the type of floating-point numbers.
Truth values belong to the \verb|bool|
type. Character strings belong to the \verb|string| type. %
%
Types are divided into two classes:
\begin{itemize}
\item
  Basic types (\verb|int|, \verb|bool|, \verb|string|, \ldots).
\item 
  Compound types such as functional types. For example,
  the type of functions from integers to integers is
  denoted by \verb|int -> int|. The type of functions from boolean values to
  character strings is written \verb|bool -> string|. %
The type of pairs
  whose first component is an integer and whose second component is a boolean value
  is written \verb|int * bool|.
\end{itemize}
%
In fact, any combination of the types above (and even more!) is possible.
This could be written as:
\begin{quote}
\begin{tabular}{lrl}
BasicType & ::= & {\tt int} \\
      & $|$ & {\tt bool} \\
      & $|$ & {\tt string} \\
\\
Type & ::= & BasicType \\
     & $|$ & Type \verb|->| Type\\
     & $|$ & Type {\tt *} Type
\end{tabular}
\end{quote}
Once a Caml toplevel phrase has been entered in the computer, the Caml
process starts analyzing it. First of all, it performs {\em
syntax analysis}, which consists in checking whether the phrase
belongs to the language or not. For example, here is a syntax
error\footnote{Actually, lexical analysis takes place before syntax
analysis and {\em lexical errors} may occur, detecting for instance
badly formed character constants.} (a parenthesis is missing):
\begin{caml_example}
(function x -> x+1 (2+3);;
\end{caml_example}
The carets ``\verb|^^^|'' underline the location where the error was
detected.

The second analysis performed is {\em type analysis}: the system
attempts to assign a type to each subexpression of the phrase, and to
synthesize the type of the whole phrase.  If type analysis fails (i.e.
if the system is unable to assign a sensible type to the phrase), then
a type error is reported and Caml waits for another input, rejecting
the current phrase.  Here is a type error (cannot add a number to a
boolean):
\begin{caml_example}
(function x -> x+1) (2+true);;
\end{caml_example}
The rejection of ill-typed phrases is called {\em strong typing}.
Compilers for weakly typed languages (C, for example) would instead
issue a warning message and continue their work at the risk of
getting a ``{\tt Bus error}'' or ``{\tt Illegal instruction}''
message at run-time. 

Once a sensible type has been deduced for the expression, then the
evaluation process (compilation, loading and execution) may
begin.

Strong typing enforces writing clear and well-structured programs.
Moreover, it adds security and increases the speed of program development,
since most typos and many conceptual errors are trapped and signaled
by the type analysis.
Finally, well-typed programs do not need dynamic type tests (the addition
function does not need to test whether or not its arguments are numbers),
hence static type analysis allows for more efficient machine code.

\section{Functions}

The most important kind of values in functional programming are
functional values. Mathematically, a function $f$ of type $A \rightarrow B$ is
a rule of correspondence which associates with each element of type $A$ a
unique member of type $B$.

If $x$ denotes an element of $A$, then we will write $f(x)$ for the
application of $f$ to $x$. Parentheses are often useless (they are
used only for grouping subexpressions), so we could also write $(f
(x))$ as well as $(f ((x)))$ or simply $f~x$. The value of $f~x$ is
the unique element of $B$ associated with $x$ by the rule of
correspondence for $f$.

The notation $f(x)$ is the one normally employed in mathematics to
denote functional application. However, we shall be careful not to
confuse a function with its application. We say ``the function $f$
with formal parameter $x$'', meaning that $f$ has been defined by:
\[f: x \mapsto e \]
or, in Caml, that the body of $f$ is something like
\verb|(function x -> ...)|.
Functions are values as other values.  In particular, functions may be
passed as arguments to other functions, and/or returned as result. For
example, we could write:
\begin{caml_example}
function f -> (function x -> (f(x+1) - 1));;
\end{caml_example}
That function takes as parameter a function (let us call it \verb|f|) and
returns another function which, when given an argument (let us call it
\verb|x|), will return the predecessor of the value of the application
\verb|f(x+1)|.

The type of that function should be read as:
\verb|(int -> int) -> (int -> int)|.


\section{Definitions}


It is important to give names to values.
We have already seen some named values: we
called them {\em formal parameters}. In the expression
\verb|(function x -> x+1)|, the name \verb|x| is a formal parameter.
Its name is irrelevant: changing it into another one
does not change the meaning of the
expression.
We could have written that function \verb|(function y -> y+1)|.

If now we apply this function to, say, \verb|1+2|, we will evaluate
the expression \verb|y+1| where \verb|y| is the value of \verb|1+2|.
Naming {\tt y} the value of \verb|1+2| in \verb|y+1| is written as:
\ikwd{let@\verb`let`}
\begin{caml_example}
let y=1+2 in y+1;;
\end{caml_example}
This expression is a legal Caml phrase, and the \verb|let|
construct is indeed widely used in Caml programs.

The \verb|let| construct introduces {\em local bindings of values to
identifiers}. They are {\em local} because the scope of {\tt y} is
restricted to the expression \verb|(y+1)|. The identifier {\tt y} kept
its previous binding (if any) after the evaluation of the ``{\tt let}
\ldots {\tt in} \ldots'' construct. We can check that \verb|y| has not
been globally defined by trying to evaluate it:
\begin{caml_example}
y;;
\end{caml_example}
Local bindings using {\tt let} also introduce {\em sharing} of
(possibly time-consuming) evaluations. When evaluating ``{\tt let
x=}$e_1$ {\tt in} $e_2$'', $e_1$ gets evaluated only once.  All
occurrences of {\tt x} in $e_2$ access the {\em value} of $e_1$ which
has been computed once.  For example, the computation of:
\begin{verbatim}
let big = sum_of_prime_factors 35461243
in big+(2+big)-(4*(big+1));;
\end{verbatim}
will be less expensive than:
\begin{verbatim}
  (sum_of_prime_factors 35461243)
+ (2 + (sum_of_prime_factors 35461243))
- (4 * (sum_of_prime_factors 35461243 + 1));;
\end{verbatim}
The \verb|let| construct also has  a global form for toplevel declarations, as
in:
\begin{caml_example}
let successor = function x -> x+1;;
let square = function x -> x*x;;
\end{caml_example}
When a value has been declared at toplevel, it is of course available during
the rest of the session.
\begin{caml_example}
square(successor 3);;
square;;
\end{caml_example}

When declaring a functional value, there are some alternative syntaxes
available.
For example we could have declared the \verb|square| function
by:
\begin{caml_example}
let square x = x*x;;
\end{caml_example}
or (closer to the mathematical notation) by:
\begin{caml_example}
let square (x) = x*x;;
\end{caml_example}
All these definitions are equivalent.


\section{Partial applications}


A {\em partial application} is the application of a function to some
but not all of its arguments.
Consider, for example, the function {\tt f} defined by:
\begin{caml_example}
let f x = function y -> 2*x+y;;
\end{caml_example}
Now, the expression \verb|f(3)| denotes the function which when given an
argument \verb|y| returns the value of \verb|2*3+y|. The application
\verb|f(x)| is called a {\em partial application}, since {\tt f} waits for
two successive arguments, and is applied to only one. The value of {\tt f(x)}
is still a function.

We may thus define an addition function by:
\begin{caml_example}
let addition x = function y -> x+y;;
\end{caml_example}
This can also be written:
\begin{caml_example}
let addition x y = x+y;;
\end{caml_example}
We can then define the successor function by:
\begin{caml_example}
let successor = addition 1;;
\end{caml_example}
Now, we may use our \verb|successor| function:
\begin{caml_example}
successor (successor 1);;
\end{caml_example}

\section*{Exercises}

\begin{exo}\label{Fund:1}
Give examples of functions with the following types:
\begin{enumerate}
\item \verb|(int -> int) -> int|
\item \verb|int -> (int -> int)|
\item \verb|(int -> int) -> (int -> int)|
\end{enumerate}
\end{exo}
%
\begin{exo}\label{Fund:2}
We have seen that the names of formal parameters are meaningless. It is then
possible to rename \verb"x" by \verb"y" in \verb"(function x -> x+x)". What
should we do in order to rename \verb"x" in \verb"y" in
\begin{verbatim}
(function x -> (function y -> x+y))
\end{verbatim}
Hint: rename \verb"y" by \verb"z" first. Question: why?
\end{exo}
%
\begin{exo}\label{Fund:3}
Evaluate the following expressions (rewrite them until no longer possible):
\begin{verbatim}
let x=1+2 in ((function y -> y+x) x);;
let x=1+2 in ((function x -> x+x) x);;
let f1 = function f2 -> (function x -> f2 x)
in let g = function x -> x+1
   in f1 g 2;;
\end{verbatim}
\end{exo}

\chapter{Basic types}
\label{c:basic}

We examine in this chapter the Caml basic types.

\section{Numbers}

Caml Light provides two numeric types: integers (type \verb"int") and
floating-point numbers (type \verb"float"). Integers are limited to
the range $-2^{30} \ldots 2^{30}-1$. Integer arithmetic is taken modulo
$2^{31}$; that is, an integer operation that overflows does not raise an
error, but the result simply wraps around. Predefined operations
(functions) on integers include:
\begin{center}
\begin{tabular}{rl}
{\tt +} &addition\\
{\tt -} &subtraction and unary minus\\
{\tt *} &multiplication\\
{\tt /} &division\\
{\tt mod} &modulo
\end{tabular}
\end{center}

Some examples of expressions:
\begin{caml_example}
3 * 4 + 2;;
3 * (4 + 2);;
3 - 7 - 2;;
\end{caml_example}
There are precedence rules that make \verb"*" bind tighter than \verb"+", and so on. In doubt, write extra parentheses.
\par
So far, we have not seen the type of these arithmetic operations.
They all expect  two integer arguments; so, they are functions taking
one integer and returning a function from integers to integers.
The (functional) value of such infix identifiers may be obtained by taking
their {\em prefix} version.\ikwd{prefix@\verb`prefix`}
\begin{caml_example}
prefix +;;
prefix *;;
prefix mod;;
\end{caml_example}

As shown by their types, the infix operators \verb"+", \verb"*",
\ldots, do not work on floating-point values. A separate set of
floating-point arithmetic operations is provided:

\begin{center}
\begin{tabular}{rl}
{\tt +.} &addition\\
{\tt -.} &subtraction and unary minus\\
{\tt *.} &multiplication\\
{\tt /.} &division\\
\verb"sqrt" & square root \\
\verb"exp", \verb"log" & exponentiation and logarithm \\
\verb"sin", \verb"cos",\ldots & usual trigonometric operations
\end{tabular}
\end{center}

We have two conversion functions to go back and forth between integers
and floating-point numbers: \verb"int_of_float" and
\verb"float_of_int".

\begin{caml_example}
1 + 2.3;;
float_of_int 1 +. 2.3;;
\end{caml_example}

Let us now give some examples of numerical functions. We start by
defining some very simple functions on numbers: 
\begin{caml_example}
let square x = x *. x;;
square 2.0;;
square (2.0 /. 3.0);;
let sum_of_squares (x,y) = square(x) +. square(y);;
let half_pi = 3.14159 /. 2.0
in sum_of_squares(cos(half_pi), sin(half_pi));;
\end{caml_example}

We now develop a classical example: the computation of the root of a
function by Newton's method.
Newton's method can be stated as follows: if $y$ is an approximation to a
root of a function $f$, then:
\[
y - \frac{f(y)}{f'(y)}
\]
is a better approximation, where $f'(y)$ is the derivative of $f$
evaluated at $y$.
For example, with $f(x)=x^2 -a$, we have $f'(x)=2x$, and therefore:
\[
y-\frac{f(y)}{f'(y)} = y - \frac{y^2 -a}{2y} = \frac{y+\frac{a}{y}}{2}
\]

We can define a function {\tt deriv} for approximating the derivative of a
function at a given point by:
\begin{caml_example}
let deriv f x dx = (f(x+.dx) -. f(x)) /. dx;;
\end{caml_example}
Provided \verb|dx| is sufficiently small, this gives a reasonable estimate
of the derivative of $f$ at $x$.

The following function returns the absolute value of its floating
point number argument.  It uses the ``{\tt if} \ldots {\tt then}
\ldots {\tt else}''\ikwd{if@\verb`if`} conditional construct.
\begin{caml_example}
let abs x = if x >. 0.0 then x else -. x;;
\end{caml_example}
The main function, given below, uses three local functions. The first
one, {\tt until}, is an example of a {\em recursive} function: it
calls itself in its body, and is defined using a {\tt let rec}
construct ({\tt rec} shows that the definition is recursive). It takes
three arguments: a predicate {\tt p} on floats, a function {\tt
change} from floats to floats, and a float {\tt x}. If \verb|p(x)| is
false, then \verb"until" is called with last argument
\verb|change(x)|, otherwise, {\tt x} is the result of the whole call.
We will study recursive functions more closely later. The two other
auxiliary functions, {\tt satisfied} and {\tt improve}, take a float
as argument and deliver respectively a boolean value and a float. The
function {\tt satisfied} asks wether the image of its argument by {\tt
f} is smaller than {\tt epsilon} or not, thus testing wether {\tt y}
is close enough to a root of {\tt f}. The function {\tt improve}
computes the next approximation using the formula given below. The
three local functions are defined using a cascade of {\tt let}
constructs. The whole function is:
\begin{caml_example}
let newton f epsilon =
  let rec until p change x =
            if p(x) then x
            else until p change (change(x)) in
  let satisfied y = abs(f y) <. epsilon in
  let improve y = y -. (f(y) /. (deriv f y epsilon))
in until satisfied improve;;
\end{caml_example}
Some examples of equation solving:
\begin{caml_example}
let square_root x epsilon =
           newton (function y -> y*.y -. x) epsilon x
and cubic_root x epsilon =
           newton (function y -> y*.y*.y -. x) epsilon x;;
square_root 2.0 1e-5;;
cubic_root 8.0 1e-5;;
2.0 *. (newton cos 1e-5 1.5);;
\end{caml_example}

\section{Boolean values}

The presence of the conditional construct implies the presence of
boolean values.  The type \verb"bool" is composed of two values
\verb"true" and \verb"false".
\begin{caml_example}
true;;
false;;
\end{caml_example}
The functions with results of type \verb"bool" are often called {\em
predicates}.
Many predicates are predefined in Caml. Here are some of them:
\begin{caml_example}
prefix <;;
1 < 2;;
prefix <.;;
3.14159 <. 2.718;;
\end{caml_example}
There exist also \verb"<=", \verb">", \verb">=", and similarly
\verb"<=.", \verb">.", \verb">=.".

\subsection{Equality}

Equality has a special status in functional languages because of
functional values. It is easy to test equality of values of base types
(integers, floating-point numbers, booleans, \ldots):
\begin{caml_example}
1 = 2;;
false = (1>2);;
\end{caml_example}
But it is impossible, in the general case, to decide the equality of
functional values. Hence, equality stops on a run-time error when
it encounters functional values.
\begin{caml_example}
(fun x -> x) = (fun x -> x);;
\end{caml_example}
Since equality may be used on values of any type, what is its type?
Equality takes two arguments of the same type (whatever type it
is) and returns a boolean value.  The type of equality is a {\em
polymorphic type}, i.e. may take several possible forms.  Indeed, when
testing equality of two numbers, then its type is
\verb|int -> int -> bool|, and when testing equality of strings, its type is
\verb|string -> string -> bool|.
\begin{caml_example}
prefix =;;
\end{caml_example}
The type of equality uses {\em type variables}, written \verb|'a|,
\verb|`b|, etc. Any type can be substituted to type variables in order
to produces specific {\em instances} of types. For example,
substituting \verb|int| for \verb|'a| above produces
\verb|int -> int -> bool|, which is the type of the equality function
used on integer values.
\begin{caml_example}
1=2;;
(1,2) = (2,1);;
1 = (1,2);;
\end{caml_example}

\subsection{Conditional}
\ikwd{if@\verb`if`}

Conditional expressions are of the form ``${\tt
if}~e_{\rm{test}}~{\tt then}~e_1~{\tt else}~e_2$'', where
$e_{\rm{test}}$ is an expression of type {\tt bool} and $e_1$, $e_2$
are expressions possessing the same type. As an example, the logical
negation could be written:
\begin{caml_example}
let negate a = if a then false else true;;
negate (1=2);;
\end{caml_example}


\subsection{Logical operators}
\ikwd{or@\verb`or`}

The classical logical operators are available in Caml. Disjunction and
conjunction are respectively written {\tt or} and \verb|&|:
\begin{caml_example}
true or false;;
(1<2) & (2>1);;
\end{caml_example}
The operators \verb"&" and \verb"or" are not functions. They should
not be seen as regular functions, since they evaluate their second
argument only if it is needed. For example, the \verb"or" operator
evaluates its second operand only if the first one evaluates to
\verb"false".
The behavior of these operators may be described as follows:
\begin{center}
\begin{tabular}{lcl}
$e_1$ {\tt or} $e_2$ & \mbox{is equivalent to} &
                {\tt if} $e_1$ {\tt then true  else} $e_2$\\
$e_1$ \verb"&" $e_2$ & \mbox{is equivalent to} &
                {\tt if} $e_1$ {\tt then} $e_2$ {\tt else false}
\end{tabular}
\end{center}
Negation is predefined:\ikwd{not@\verb`not`}
\begin{caml_example}
not true;;
\end{caml_example}
The \verb|not| identifier receives a special treatment from the
parser: the application ``{\tt not f x}'' is recognized as ``{\tt not
(f x)}'' while ``{\tt f g x}'' is identical to ``{\tt (f g) x}''. This
special treatment explains why the functional value associated to {\tt not}
can be obtained only using the {\tt prefix} keyword:
\begin{caml_example}
prefix not;;
\end{caml_example}

\section{Strings and characters}

String constants (type \verb"string") are written between \verb|"|
characters (double-quotes). Single-character constants (type
\verb"char") are written between \verb|`| characters (backquotes).
The most used string operation is string concatenation, denoted by the
\verb"^" character.
\begin{caml_example}
"Hello" ^ " World!";;
prefix ^;;
\end{caml_example}
Characters are ASCII characters:
\begin{caml_example}
`a`;;
`\065`;;
\end{caml_example}
and can be built from their ASCII code as in:
\begin{caml_example}
char_of_int 65;;
\end{caml_example}
Other operations are available (\verb"sub_string", \verb|int_of_char|,
etc). See the Caml Light referencee manual \cite{CamlLightDoc} for
complete information.


\section{Tuples}

\subsection{Constructing tuples}

It is possible to combine values into tuples (pairs, triples, \ldots).
The {\em value constructor} for
tuples is the ``{\tt,}'' character (the comma). We will often use parentheses
around tuples in order to improve readability, but they are not strictly
necessary.
\begin{caml_example}
1,2;;
1,2,3,4;;
let p = (1+2, 1<2);;
\end{caml_example}
The infix ``{\tt *}'' identifier is the {\em type constructor} for tuples.
For instance, $t_1 {\tt *} t_2$ corresponds to the mathematical cartesian product of types $t_1$ and $t_2$.

We can build tuples from any values: the tuple value constructor is {\em
generic}.

\subsection{Extracting pair components}

{\em Projection} functions are used to extract components of tuples. For pairs, we have:
\begin{caml_example}
fst;;
snd;;
\end{caml_example}
They have polymorphic types, of course, since they may be applied to
any kind of pair.  They are predefined in the Caml system, but could
be defined by the user (in fact, the cartesian product itself could be
defined --- see section \ref{s:udefprodtypes}, dedicated to user-defined
product types):
\begin{caml_example}
let first (x,y) = x
and second (x,y) = y;;
first p;;
second p;;
\end{caml_example}
We say that \verb|first| is a {\em generic} value because it works
uniformly on several kinds of arguments (provided they are pairs). We
often confuse between ``generic'' and ``polymorphic'', saying that
such value is polymorphic instead of generic.


\section{Patterns and pattern-matching}

Patterns and pattern-matching play an important role in ML languages.
They appear in all real programs and are strongly related to types
(predefined or user-defined).

A {\em pattern} indicates the {\em shape} of a value.
Patterns are ``values with holes''.
A single variable (formal parameter) is a pattern (with no shape specified:
it matches any value).
When a value is {\em matched against} a pattern (this is called
{\em pattern-matching}), the pattern acts as a filter.
We already used patterns and pattern-matching in all the functions we wrote:
the function body {\tt (function x -> ...)} does (trivial) pattern-matching.
When applied to a value, the formal parameter {\tt x} gets bound to
that value.
The function body {\tt (function (x,y) -> x+y)} does also pattern-matching:
when applied to a value (a pair, because of well-typing hypotheses), the
{\tt x} and {\tt y} get bound respectively to the first and the second
component of that pair.
\par
All these pattern-matching examples were trivial ones, they did not involve any test:
\begin{itemize}
\item formal parameters such as {\tt x} do not impose any particular shape to the
value they are supposed to match;
\item pair patterns such as {\tt(x,y)} always match for typing reasons
(cartesian product is a {\em product type}).
\end{itemize}
However, some types do not guarantee such a uniform shape to their
values. Consider the {\tt bool} type: an element of type {\tt bool} is
either {\tt true} or {\tt false}.
%
%
If we impose to a value of type {\tt bool} to have the shape of {\tt true},
then pattern-matching may fail. Consider the following function:
\begin{caml_example}
let f = function true -> false;;
\end{caml_example}
The compiler warns us that the pattern-matching may fail (we did not
consider the {\tt false} case).

Thus, if we apply {\tt f} to a value that evaluates to {\tt true},
pattern-matching will succeed (an equality test is performed during
execution).
\begin{caml_example}
f (1<2);;
\end{caml_example}
But, if {\tt f}'s argument evaluates to {\tt false}, a run-time
error is reported:
\begin{caml_example}
f (1>2);;
\end{caml_example}
Here is a correct definition using pattern-matching:
\begin{caml_example}
let negate = function true -> false
                    | false -> true;;
\end{caml_example}
The pattern-matching has now two cases, separated by the ``{\verb"|"}''
character.
Cases are examined in turn, from top to bottom. An equivalent definition of
{\tt negate} would be:
\begin{caml_example}
let negate = function true -> false
                    | x -> true;;
\end{caml_example}
The second case now matches any boolean value (in fact, only {\tt false} since
{\tt true} has been caught by the first match case). When the second case is
chosen, the identifier {\tt x} gets bound to the argument of {\tt
negate}, and could be used in the right-hand part of the match case.
Since in our example we do not use the value of the argument in the
right-hand part of the second match case, another equivalent
definition of {\tt negate} would be:
\begin{caml_example}
let negate = function true -> false
                    | _ -> true;;
\end{caml_example}
Where ``\verb"_"'' acts as a formal paramenter (matches any value), but does
not introduce any binding: it should be read as ``anything else''.

As an example of pattern-matching, consider the following function
definition (truth value table of implication):
\begin{caml_example}
let imply = function (true,true) -> true
                   | (true,false) -> false
                   | (false,true) -> true
                   | (false,false) -> true;;
\end{caml_example}
Here is another way of defining {\tt imply}, by using variables:
\begin{caml_example}
let imply = function (true,x) -> x
                   | (false,x) -> true;;
\end{caml_example}
Simpler, and simpler again:
\begin{caml_example}
let imply = function (true,x) -> x
                   | (false,_) -> true;;
let imply = function (true,false) -> false
                   | _ -> true;;
\end{caml_example}
Pattern-matching is allowed on any type of value (non-trivial
pattern-matching is only possible on types with {\em data constructors}).

For example:
\begin{caml_example}
let is_zero = function 0 -> true | _ -> false;;
let is_yes = function "oui" -> true
                    | "si" -> true
                    | "ya" -> true
                    | "yes" -> true
                    | _ -> false;;
\end{caml_example}

\section{Functions}
\ikwd{function@\verb`function`}

The type constructor ``\verb"->"'' is predefined and cannot be defined
in ML's type system.
We shall explore in this section some further aspects of functions
and functional types.

\subsection{Functional composition}

Functional composition is easily definable in Caml. It is of course a
polymorphic function:
\begin{caml_example}
let compose f g = function x -> f (g (x));;
\end{caml_example}
The type of {\tt compose} contains no more constraints than the ones
appearing in the definition: in a sense, it is the {\em most general} type
compatible with these constraints.

These constraints are:
\begin{itemize}
\item the codomain of {\tt g} and the domain of {\tt f} must be the same;
\item {\tt x} must belong to the domain of {\tt g};
\item {\tt compose f g x} will belong to the codomain of {\tt f}.
\end{itemize}


Let's see \verb"compose" at work:
\begin{caml_example}
let succ x = x+1;;
compose succ list_length;;
(compose succ list_length) [1;2;3];;
\end{caml_example}

\subsection{Currying}
We can define two versions of the addition function:
\begin{caml_example}
let plus = function (x,y) -> x+y;;
let add = function x -> (function y -> x+y);;
\end{caml_example}
These two functions differ only in their way of taking their arguments.
The first one will take an argument belonging to a cartesian product,
the second one will take a number and return a function.
The {\tt add} function is said to be {\em the curried version} of {\tt
plus} (in honor of the logician Haskell Curry).


The currying transformation may be written in Caml under the form of a
higher-order function:
\begin{caml_example}
let curry f = function x -> (function y -> f(x,y));;
\end{caml_example}
Its inverse function may be defined by:
\begin{caml_example}
let uncurry f = function (x,y) -> f x y;;
\end{caml_example}
We may check the types of {\tt curry} and {\tt uncurry} on particular
examples:
\begin{caml_example}
uncurry (prefix +);;
uncurry (prefix ^);;
curry plus;;
\end{caml_example}

\section*{Exercises}
%
\begin{exo}\label{Basic:1}
Define functions that compute the surface area and the volume of
well-known geometric objects (rectangles, circles, spheres, \ldots).
\end{exo}
%
\begin{exo}\label{Basic:2}
What would happen in a language submitted to call-by-value with
recursion if there was no conditional construct ({\tt if ... then ... else ...})?
\end{exo}
\begin{exo}\label{Basic:3}
Define the {\tt factorial} function such that:
\[{\tt factorial}~n = n * (n-1) * \ldots * 2 * 1\]
Give two versions of {\tt factorial}: recursive and tail recursive.
\end{exo}
\begin{exo}\label{Basic:5}
Define the {\tt fibonacci} function that, when given a number $n$,
returns the $n$th Fibonacci number, i.e. the $n$th term $u_n$ of the
sequence defined by:
\[\begin{array}{l}
u_1 = 1\\
u_2 = 1\\
u_{n+2} = u_{n+1} + u_n
\end{array}\]
\end{exo}
\begin{exo}\label{Basic:6}
What are the types of the following expressions?
\begin{itemize}
\item {\tt uncurry compose}
\item {\tt compose curry uncurry}
\item {\tt compose uncurry curry}
\end{itemize}
\end{exo}

\chapter{Lists}
\label{c:lists}

Lists represent an important data structure, mainly because of their success
in the Lisp language.
Lists in ML are {\it homogeneous\/}: a list cannot contain elements of
different types.
This may be annoying to new ML users, yet lists are not as fundamental
as in Lisp, since ML provides a facility for introducing new types allowing the
user to define more precisely the data structures needed by the
program (cf. chapter~\ref{c:udeftypes}).

\section{Building lists}

Lists are empty or non empty sequences of elements.
They are built with two {\it value constructors\/}:
\begin{itemize}
\item \verb"[]", the empty list (read: {\it nil\/});
\item \verb"::", the non-empty list constructor (read: {\it cons\/}). It takes an element $e$ and a list $l$,
and builds a new list whose first element ({\it head\/}) is $e$ and rest ({\it
tail\/}) is $l$.
\end{itemize}
The special syntax \verb"["$e_1$\verb";" \ldots\verb";" $e_n$ \verb"]"
builds the list whose elements are $e_1, \ldots, e_n$. It is
equivalent to $e_1$ \verb"::" $(e_2$ \verb"::" \ldots $(e_n$ \verb":: []"%
$)\ldots)$.


We may build lists of numbers:
\begin{caml_example}
1::2::[];;
[3;4;5];;
let x=2 in [1; 2; x+1; x+2];;
\end{caml_example}
Lists of functions:
\begin{caml_example}
let adds =
  let add x y = x+y
  in [add 1; add 2; add 3];;
\end{caml_example}
and indeed, lists of anything desired.


We may wonder what are the types of the list (data) constructors. The
empty list is a list of anything (since it has no element), it has
thus the type ``{\em list of anything}''. The list constructor
\verb|::| takes an element and a list (containing elements with the
same type) and returns a new list.  Here again, there is no type
constraint on the elements considered.

\begin{caml_example}
[];;
function head -> function tail -> head::tail;;
\end{caml_example}
We see here that the {\tt list} type is a {\it recursive} type.
The \verb"::" constructor receives two arguments; the second argument is itself a {\tt list}.

\section{Extracting elements from lists: pattern-matching}

We know how to build lists; we now show how to test emptiness of lists
(although  the equality predicate could be used for that) and extract
elements from lists (e.g. the first one).
We have used pattern-matching  on pairs, numbers or boolean
values. The syntax of patterns also includes list patterns. (We will
see that any data constructor can actually be used in a pattern.)
For lists, at least two cases have to be considered (empty, non empty):
\begin{caml_example}
let is_null = function [] -> true | _ -> false;;
let head = function x::_ -> x
                  | _ -> raise (Failure "head");;
let tail = function _::l -> l
                  | _ -> raise (Failure "tail");;
\end{caml_example}
The expression \verb|raise (Failure "head")| will produce a
run-time error when evaluated. In the definition of \verb"head" above,
we have chosen to forbid taking the head of an empty list. We could have
chosen \verb"tail []" to evaluate to \verb"[]", but we cannot return a value
for \verb"head []" without changing the type of the \verb"head" function.


We say that the types {\tt list} and {\tt bool} are {\it sum types},
because they are defined  with several alternatives:
\begin{itemize}
\item a list is either \verb"[]" or \verb"::" of \ldots
\item a boolean value is either \verb"true" or \verb"false"
\end{itemize}

Lists and booleans are typical examples of sum types. Sum types are
the only types whose values need run-time tests in order to be matched
by a non-variable pattern (i.e. a pattern that is not a single variable).


The cartesian product is a {\it product} type (only one alternative).
Product types do not involve run-time tests during pattern-matching,
because the type of their values suffices to indicate statically
what their structure is.

\section{Functions over lists}


We will see in this section the definition of some useful functions
over lists. These functions are of general interest, but are not
exhaustive. Some of them are predefined in the Caml Light system.
See also \cite{CAMLPrimer} or \cite{CAMLRefMan} for other examples of
functions over lists.

Computation of the length of a list:
\begin{caml_example}
let rec length = function [] -> 0
                        | _::l -> 1 + length(l);;
length [true; false];;
\end{caml_example}
Concatenating two lists:
\begin{caml_example}
let rec append =
     function [], l2 -> l2
            | (x::l1), l2 -> x::(append (l1,l2));;
\end{caml_example}
The \verb"append" function is already defined in Caml, and bound to the
infix identifier \verb"@".
\begin{caml_example}
[1;2] @ [3;4];;
\end{caml_example}
Reversing a list:
\begin{caml_example}
let rec rev = function [] -> []
                     | x::l -> (rev l) @ [x];;
rev [1;2;3];;
\end{caml_example}

The \verb"map" function applies a function on all the elements
of a list, and return the list of the function results. It
demonstrates full functionality (it takes a function
as argument), list processing, and polymorphism (note the sharing of type
variables between the arguments of \verb"map" in its type).
\begin{caml_example}
let rec map f =
    function [] -> []
           | x::l -> (f x)::(map f l);;
map (function x -> x+1) [1;2;3;4;5];;
map length [ [1;2;3]; [4;5]; [6]; [] ];;
\end{caml_example}



The following function is a list iterator. It takes a function $f$, a
base element $a$ and a list
\verb|[|$x_1$\verb|;|\ldots\verb|;|$x_n$\verb|]|. It computes:
$$
{\tt it\_list} ~ f ~ a ~ \verb|[|x_1\verb|;|\ldots\verb|;|x_n\verb|]| =
f~ (\ldots (f ~(f~ a~ x_1)~ x_2)~\ldots) x_n
$$
For instance, when applied to the curried addition function, to the
base element \verb"0", and to a list of numbers, it computes the sum
of all elements in the list. The expression:
\begin{itemize}
\item[] \verb|it_list (prefix +) 0 [1;2;3;4;5]|
\item[] evaluates to ((((0+1)+2)+3)+4)+5
\item[] i.e. to \verb|15|.
\end{itemize}
\begin{caml_example}
let rec it_list f a =
        function [] -> a
               | x::l -> it_list f (f a x) l;;
let sigma = it_list prefix + 0;;
sigma [1;2;3;4;5];;
it_list (prefix *) 1 [1;2;3;4;5];;
\end{caml_example}
The \verb|it_list| function is one of the many ways to iterate
over a list. For other list iteration functions, see \cite{CAMLPrimer}.

\section*{Exercises}

%
\begin{exo}\label{Lists:1}
Define the {\tt combine} function which, when given a pair of lists, returns a 
list of pairs such that:
\begin{verbatim}
combine ([x1;...;xn],[y1;...;yn]) = [(x1,y1);...;(xn,yn)]
\end{verbatim}
The function generates a run-time error if  the argument lists do not
have the same length.
\end{exo}
\begin{exo}\label{Lists:2}
Define a function which, when given a list, returns the list of all its 
sublists.
\end{exo}

\chapter{User-defined types}
\label{c:udeftypes}

The user is allowed to define his/her own data types.
With this facility, there is no need to encode
the data structures that must be manipulated by a program into lists
(as in Lisp) or into arrays (as in Fortran). Furthermore, early
detection of type errors is enforced, since user-defined data types reflect
precisely the needs of the algorithms. %

Types are either:
\begin{itemize}
\item {\em product} types,
\item or {\em sum} types.
\end{itemize}

We have already seen examples of both kinds of types: the {\tt bool} and
{\tt list} types are sum types (they contain values with different shapes
and are defined and matched using several alternatives). The
cartesian product is an example of a product type: we know statically
the shape of values belonging to cartesian products.

In this chapter, we will see how to define and use new types in Caml.

%
\section{Product types}
\label{s:udefprodtypes}

Product types are {\em finite} {\em labeled} products of types.
They are a generalization of cartesian product.
Elements of product types are called {\em records}.

\subsection{Defining product types}

An example: suppose we want to define a data structure containing
information about individuals. We could define:
\begin{caml_example}
let jean=("Jean",23,"Student","Paris");;
\end{caml_example}
and use pattern-matching to extract any particular
information about the person {\tt jean}.
The problem with such usage of cartesian product is that a function
\verb"name_of" returning the name field of a value representing an individual
would have the same type as the general first projection for 4-tuples
(and indeed would be the same function). This type is not precise enough
since it allows for the application of the function to any 4-uple, and
not only to values such as {\tt jean}.

Instead of using cartesian product, we define a {\tt person} data
type:\ikwd{type@\verb`type`}
\begin{caml_example}
type person =
  {Name:string; Age:int; Job:string; City:string};;
\end{caml_example}
The type {\tt person} is the {\em product} of {\tt string}, {\tt int}, {\tt
string} and {\tt string}.
The field names provide type information and also documentation: it
is much easier to understand data structures such as {\tt jean} above than
arbitrary tuples.
%

We have {\em labels} (i.e. {\tt Name}, \ldots) to refer to components of
the products. The order of appearance of the products components is not
relevant: labels are sufficient to uniquely identify the components.
The Caml system finds a canonical order on labels
to represent and print record values. The order is always the order which
appeared in the definition of the type.
%

We may now define the individual {\tt jean} as:
\begin{caml_example}
let jean = {Job="Student"; City="Paris";
            Name="Jean"; Age=23};;
\end{caml_example}
This example illustrates the fact that order of labels is not
relevant.

\subsection{Extracting products components}

The canonical way of extracting product components is {\em pattern-matching}.
Pattern-matching provides a way to mention the shape of values and to give
(local) names to components of values.
In the following example, we name {\tt n} the numerical value contained in
the field {\tt Age} of the argument, and we choose to forget values
contained in other fields (using the \verb"_" character).%
\begin{caml_example}
let age_of = function
     {Age=n; Name=_; Job=_; City=_} -> n;;
age_of jean;;
\end{caml_example}
It is also possible to access the value of a single field, with the
{\tt .} (dot) operator:
\begin{caml_example}
jean.Age;;
jean.Job;;
\end{caml_example}
Labels always refer to the most recent type definition: when two
record types possess some common labels, then these labels always
refer to the most recently defined type. When using modules (see
section~\ref{s:modules}) this problem arises for types defined in
the same module. For types defined in different modules, the full name
of labels (i.e. with the name of the modules prepended) disambiguates
such situations.

\subsection{Parameterized product types}

It is important to be able to define parameterized types in order to
define {\em generic} data structures. The {\tt list} type is parameterized,
and this is the reason why we may build lists of any kind of values.
If we want to define the cartesian product as a Caml type, we need type
parameters because we want to be able to build cartesian product of {\em
any} pair of types.
\begin{caml_example}
type ('a,'b) pair = {Fst:'a; Snd:'b};;
let first x = x.Fst and second x = x.Snd;;
let p={Snd=true; Fst=1+2};;
first(p);;
\end{caml_example}
Warning: the {\tt pair} type is similar to the Caml cartesian product {\tt
*}, but it is a different type.
\begin{caml_example}
fst p;;
\end{caml_example}
Actually, any two type definitions produce different types. If we
enter again the previous definition:
\begin{caml_example}
type ('a,'b) pair = {Fst:'a; Snd:'b};;
\end{caml_example}
we cannot any longer extract the {\tt Fst} component of {\tt x}:
\begin{caml_example}
p.Fst;;
\end{caml_example}
since the label {\tt Fst} refers to the {\em latter} type {\tt pair} that we
defined, while {\tt p}'s type is the {\em former} {\tt pair}.

\section{Sum types}

A {\em sum} type is the {\em finite} {\em labeled} disjoint union of
several types.
A sum type definition defines a type as being the union of some other types.%

\subsection{Defining sum types}

Example: we want to have a type called {\tt identification} whose
values can be:
\begin{itemize}
\item either strings (name of an individual),
\item or integers (encoding of social security number as a pair of integers).
\end{itemize}
We then need a type containing {\em both} \verb|int * int| and
character strings.  We also want to preserve static type-checking, we
thus want to be able to distinguish pairs from character strings
even if they are injected in order to form a single type.

Here is what we would do:\ikwd{type@\verb`type`}
\begin{caml_example}
type identification = Name of string
                    | SS of int * int;;
\end{caml_example}
The type {\tt identification} is the labeled disjoint union of {\tt string}
and \verb|int * int|.
The labels {\tt Name} and {\tt SS} are {\em injections}.
They respectively inject \verb|int * int| and {\tt string} into a single type
{\tt identification}.

Let us use these injections in order to build new values:
\begin{caml_example}
let id1 = Name "Jean";;
let id2 = SS (1670728,280305);;
\end{caml_example}
Values {\tt id1} and {\tt id2} belong to the same type. They may for
example be put into lists as in:
\begin{caml_example}
[id1;id2];;
\end{caml_example}

Injections may possess one argument (as in the example above), or
none. The latter case corresponds\footnote{In Caml Light, there is no
implicit order on values of sum types.} to {\em enumerated types}, as
in Pascal. An example of enumerated type is:
\begin{caml_example}
type suit = Heart
          | Diamond
          | Club
          | Spade;;
Club;;
\end{caml_example}
The type {\tt suit} contains only 4 distinct elements.
Let us continue this example by defining a type for cards.
\begin{caml_example}
type card = Ace of suit
          | King of suit
          | Queen of suit
          | Jack of suit
          | Plain of suit * int;;
\end{caml_example}
The type {\tt card} is the disjoint union of:
\begin{itemize}
\item {\tt suit} under the injection {\tt Ace},
\item {\tt suit} under the injection {\tt King},
\item {\tt suit} under the injection {\tt Queen},
\item {\tt suit} under the injection {\tt Jack},
\item the product of {\tt int} and  {\tt suit} under the injection {\tt Plain}.
\end{itemize}
Let us build a list of cards:
\begin{caml_example}
let figures_of c = [Ace c; King c; Queen c; Jack c]
and small_cards_of c =
    map (function n -> Plain(c,n)) [7;8;9;10];;
figures_of Heart;;
small_cards_of Spade;;
\end{caml_example}

\subsection{Extracting sum components}

Of course, pattern-matching is used to extract sum components.
In case of sum types, pattern-matching uses dynamic tests for this
extraction.
The next example defines a function {\tt color} returning the name of
the color of the suit argument:%
\begin{caml_example}
let color = function Diamond -> "red"
                   | Heart -> "red"
                   | _ -> "black";;
\end{caml_example}
Let us count the values of cards (assume we are playing ``belote''):
\begin{caml_example}
let count trump = function
    Ace _        -> 11
  | King _       -> 4
  | Queen _      -> 3
  | Jack c       -> if c = trump then 20 else 2
  | Plain (c,10) -> 10
  | Plain (c,9)  -> if c = trump then 14 else 0
  | _             -> 0;;
\end{caml_example}

\subsection{Recursive types}

Some types possess a naturally recursive structure, lists, for
example. It is also the case for tree-like structures, since trees
have subtrees that are trees themselves.

Let us define a type for abstract syntax trees of a simple arithmetic
language\footnote{Syntax trees are said to be {\em abstract} because
they are pieces of {\em abstract syntax} contrasting with {\em
concrete syntax} which is the ``string'' form of programs: analyzing
(parsing) concrete syntax usually produces abstract syntax.}. An
arithmetic expression will be either a numeric constant, or a
variable, or the addition, multiplication, difference, or division of
two expressions.\ikwd{type@\verb`type`}
\begin{caml_example}
type arith_expr = Const of int
                | Var of string
                | Plus of args
                | Mult of args
                | Minus of args
                | Div of args
and args = {Arg1:arith_expr; Arg2:arith_expr};;
\end{caml_example}
The two types \verb|arith_expr| and \verb|args| are simultaneously
defined, and \verb|arith_expr| is recursive since its definition
refers to \verb|args| which itself refers to \verb|arith_expr|. As an
example, one could represent the abstract syntax tree of the
arithmetic expression ``\verb|x+(3*y)|'' as the Caml value:
\begin{caml_example}
Plus {Arg1=Var "x";
       Arg2=Mult{Arg1=Const 3; Arg2=Var "y"}};;
\end{caml_example}

The recursive definition of types may lead to types such that it is
hard or impossible to build values of these types.
For example:
%% \begin{caml_example}
%% type stupid = {Head:stupid; Tail:stupid};;
%% \end{caml_example}
\begin{caml_example}
type stupid = {Next:stupid};;
\end{caml_example}
Elements of this type are {\em infinite} data structures. Essentially,
the only way to construct one is:
%% \begin{caml_example}
%% let rec stupid_value =
%%     {Head=stupid_value; Tail=stupid_value};;
%% \end{caml_example}
\begin{caml_example}
let rec stupid_value = {Next=stupid_value};;
\end{caml_example}

Recursive type definitions should be {\em well-founded} (i.e. possess
a non-recursive case, or {\em base case}) in order to work well with
call-by-value.

\subsection{Parameterized sum types}

Sum types may also be parameterized.
Here is the definition of a type isomorphic to the {\tt list} type:
\begin{caml_example}
type 'a sequence = Empty
                 | Sequence of 'a * 'a sequence;;
\end{caml_example}

A more sophisticated example is the type of generic binary trees:
\begin{caml_example}
type ('a,'b) btree = Leaf of 'b
                   | Btree of ('a,'b) node
and ('a,'b) node = {Op:'a;
                    Son1: ('a,'b) btree;
                    Son2: ('a,'b) btree};;
\end{caml_example}
A binary tree is either a leaf (holding values of type {\tt 'b}) or a
node composed of an operator (of type {\tt 'a}) and two sons, both of them
being binary trees.

Binary trees can also be used to represent abstract trees for
arithmetic expressions (with only binary operators and only one kind
of leaves). The abstract
syntax tree \verb|t| of ``\verb|1+2|'' could be defined as:
\begin{caml_example}
let t = Btree {Op="+"; Son1=Leaf 1; Son2=Leaf 2};;
\end{caml_example}

Finally, it is time to notice that pattern-matching is not restricted to
function bodies, i.e. constructs such as:
\begin{center}
\begin{tabular}{rll}
\tt function & $P_1$ & \verb|->| $E_1$\\
    \verb'|' & \ldots\\
    \verb'|' & $P_n$ & \verb|->| $E_n$
\end{tabular}
\end{center}
but there is also a construct dedicated to pattern-matching of actual
values:
\ikwd{match@\verb`match`}
\begin{center}
\begin{tabular}{rll}
\tt match $E$ with & $P_1$ & \verb|->| $E_1$\\
          \verb'|' & \ldots\\
          \verb'|' & $P_n$ & \verb|->| $E_n$
\end{tabular}
\end{center}
which matches the value of the expression $E$ against each of the
patterns $P_i$, selecting the first one that matches, and giving
control to the corresponding expression. For example, we can match the
tree {\tt t} previously defined by:
\begin{caml_example}
match t with Btree{Op=x; Son1=_; Son2=_} -> x
           | Leaf l -> "No operator";;
\end{caml_example}

\subsection{Data constructors and functions}

One may ask: ``What is the difference between a sum data constructor and a
function?''.
At first sight, they look very similar. We assimilate constant data
constructors (such as {\tt Heart}) to constants. Similarly, in Caml
Light, sum data constructors with arguments also possess a functional type:
\begin{caml_example}
Ace;;
\end{caml_example}
However, a data constructor possesses particular properties
that a general function does not possess, and it is interesting to
understand these differences.  From the mathematical point of view, a
sum data constructor is known to be an {\em injection} while a Caml
function is a general function without further information.
A mathematical injection $f: A \rightarrow B$ admits an inverse function
$f^{-1}$ from its image $f(A) \subset B$ to $A$.

From the examples above, if we consider the {\tt King} constructor, then:
\begin{caml_example}
let king c = King c;;
\end{caml_example}
{\tt king} is the general function associated to the {\tt King}
constructor, and:
\begin{caml_example}
function King c -> c;;
\end{caml_example}
is the inverse function for {\tt king}.
It is a partial function, since pattern-matching may fail.

\subsection{Degenerate cases: when sums meet products}

What is the status of a sum type with a single case such as:
\begin{caml_example}
type counter1 = Counter of int;;
\end{caml_example}
Of course, the type {\tt counter1} is isomorphic to {\tt int}.
The injection \verb"function x -> Counter x" is a {\em total} function from {\tt int} to
{\tt counter1}. It is thus a {\em bijection}.

Another way to define a type isomorphic to {\tt int} would be:
\begin{caml_example}
type counter2 = {Counter: int};;
\end{caml_example}
The types {\tt counter1} and {\tt counter2} are isomorphic to {\tt int}.
They are at the same time sum and product types.
Their pattern-matching does not perform any run-time test.

The possibility of defining arbitrary complex data types permits the easy
manipulation of abstract syntax trees in Caml (such as the \verb"arith_expr"
type above). These abstract syntax trees are supposed to represent programs of
a language (e.g. a language of arithmetic expressions). These kind of
languages which are defined in Caml are called {\em object-languages} and
Caml is said to be their {\em metalanguage}.

\section{Summary}

\begin{itemize}
\item New types can be introduced in Caml.
\item Types may be {\em parameterized} by type variables. The syntax of type
parameters is:
\begin{verbatim}
<params> ::
          | <tvar>
          | ( <tvar> [, <tvar>]* )
\end{verbatim}
\item Types can be {\em recursive}.

\item Product types:
  \begin{itemize}
  \item Mathematical product of several types.
  \item The construct is:
     \begin{verbatim}
     type <params> <tname> =
                      {<Field>: <type>; ...}
     \end{verbatim}
     where the \verb"<type>"s may contain type variables appearing in
     \verb"<params>".
  \end{itemize}
\item Sum types:
  \begin{itemize}
  \item Mathematical disjoint union of several types.
  \item The construct is:
     \begin{verbatim}
     type <params> <tname> =
              <Injection> [of <type>] | ...
     \end{verbatim}
     where the \verb"<type>"s may contain type variables appearing in
     \verb"<params>".
  \end{itemize}
\end{itemize}

\section*{Exercises}

\begin{exo}\label{Types:1}
Define a function taking as argument a binary tree and returning a pair of
lists: the first one contains all operators of the tree, the second
one contains all its leaves.
\end{exo}
\begin{exo}\label{Types:2}
Define a function \verb"map_btree" analogous to the {\tt map} function
on lists.
The function \verb"map_btree" should take as arguments two functions {\tt f}
and {\tt g}, and a binary tree. It should return a new binary tree whose
leaves are the result of applying {\tt f} to the leaves of the tree
argument, and whose operators are the results of applying the {\tt g}
function to the operators of the argument.
\end{exo}
\begin{exo}\label{Types:3}
We can associate to the {\tt list} type definition an canonical
iterator in the following way. We give a functional interpretation
to the data constructors of the {\tt list} type.

We change the list constructors {\tt []} and {\tt ::} respectively
into a constant {\tt a} and an operator $\oplus$ (used as a prefix
identifier), and abstract with respect to these two operators, obtaining
the list iterator satisfying:
\begin{quote}
\tt list\_it $\oplus$ a [] = a\\
\tt list\_it $\oplus$ a ($x_1$::\ldots::$x_n$::[]) = $x_1$ $\oplus$ (\ldots $\oplus$ ($x_n$ $\oplus$ a)\ldots)
\end{quote}
Its Caml definition would be:
\begin{caml_example}
let rec list_it f a = 
        function [] -> a
               | x::l -> f x (list_it f a l);;
\end{caml_example}
As an example, the application of \verb|it_list| to the functional
composition and to its neutral element (the identity function),
computes the composition of lists of functions (try it!).

Define, using the same method, a canonical iterator over binary trees.
\end{exo}

\part{Caml Light specifics}
\chapter{Mutable data structures}
\label{c:mutable}

The definition of a sum or product type may be annotated to allow physical 
(destructive) update on data structures of that type. This is the main 
feature of the {\em imperative programming} style. Writing values into memory locations is the fundamental mechanism of imperative languages 
such as Pascal. The Lisp language, while mostly functional, also provides
the dangerous functions {\tt rplaca} and {\tt rplacd} to
physically modify lists. Mutable structures are required to implement
many efficient algorithms. They are also very convenient to represent
the current state of a state machine.

\section{User-defined mutable data structures}

Assume we want to define a type {\tt person} as in the previous chapter.
Then, it seems natural to allow a person to change his/her age, job and the
city that person lives in, but {\em not} his/her name.
We can do this by annotating some labels in the type definition of
{\tt person} by the {\tt mutable}\ikwd{mutable@\verb`mutable`} keyword:
\begin{caml_example}
type person =
    {Name: string; mutable Age: int;
     mutable Job: string; mutable City: string};;
\end{caml_example}
We can build values of type {\tt person} in the very same way as before:
\begin{caml_example}
let jean =
    {Name="Jean"; Age=23; Job="Student"; City="Paris"};;
\end{caml_example}
But now, the value {\tt jean} may be physically modified in the fields
specified to be {\tt mutable} in the definition (and {\em only} in these
fields).

We can modify the field {\tt Field} of an expression \verb"<expr1>" in order
to assign it the value of \verb"<expr2>" by using the following construct:
\begin{verbatim}
<expr1>.Field <- <expr2>
\end{verbatim}
For example; if we want {\tt jean} to become one year older, we would write:
\begin{caml_example}
jean.Age <- jean.Age + 1;;
\end{caml_example}
Now, the value {\tt jean} has been modified into:
\begin{caml_example}
jean;;
\end{caml_example}
We may try to change the {\tt Name} of {\tt jean}, but we won't succeed: the
typecheker will not allow us to do that.
\begin{caml_example}
jean.Name <- "Paul";;
\end{caml_example}
It is of course possible to use such constructs in functions as in:
\begin{caml_example}
let get_older ({Age=n; _} as p) = p.Age <- n + 1;;
\end{caml_example}
\ikwd{as@\verb`as`}
In that example, we named {\tt n} the current {\tt Age} of the argument, but
we also named {\tt p} the argument. This is an {\em alias} pattern: it
saves us the bother of writing:
\begin{caml_example}
let get_older p =
    match p with {Age=n} -> p.Age <- n + 1;;
\end{caml_example}
Notice that in the two previous expressions, we did not specify all
fields of the record {\tt p}.
Other examples would be:
\begin{caml_example}
let move p new_city = p.City <- new_city
and change_job p j = p.Job <- j;;
change_job jean "Teacher"; move jean "Cannes";
get_older jean; jean;;
\end{caml_example}
We used the ``{\tt ;}'' character between the different changes we
imposed to {\tt jean}. This is the {\em sequencing} of evaluations: it
permits to evaluate successively several expressions, discarding the
result of each (except the last one). This construct becomes useful in
the presence of {\em side-effects} such as physical modifications and
input/output, since we want to explicitly specify the order in which
they are performed.

\section{The {\tt ref} type}

The {\tt ref} type is the predefined type of mutable indirection
cells.  It is present in the Caml system for reasons of compatibility
with earlier versions of Caml. The {\tt ref} type could be defined as
follows (we don't use the {\tt ref} name in the following definition
because we want to preserve the original {\tt ref} type):
\begin{caml_example}
type 'a reference = {mutable Ref: 'a};;
\end{caml_example}
Example of building a value of type {\tt ref}:
\begin{caml_example}
let r = ref (1+2);;
\end{caml_example}
The {\tt ref} identifier is syntactically presented as a sum data
constructor.  The definition of {\tt r} should be read as ``let {\tt
r} be a reference to the value of {\tt 1+2}''.  The value of {\tt r}
is nothing but a memory location whose contents can be overwritten.

We consult a reference (i.e. read its memory location) with the ``{\tt
!}'' symbol:
\begin{caml_example}
!r + 1;;
\end{caml_example}

We modify values of type {\tt ref} with the {\tt :=} infix function:
\begin{caml_example}
r:=!r+1;;
r;;
\end{caml_example}
Some primitives are attached to the {\tt ref} type, for example:
\begin{caml_example}
incr;;
decr;;
\end{caml_example}
which increments (resp. decrements) references on integers.

\section{Arrays}

Arrays are modifiable data structures. They belong to the
parameterized type \verb|'a vect|. Array expressions are bracketed by
\verb'[|' and \verb'|]', and elements are separated by semicolons:
\begin{caml_example}
let a = [| 10; 20; 30 |];;
\end{caml_example}
The length of an array is returned by with the function \verb|vect_length|:
\begin{caml_example}
vect_length a;;
\end{caml_example}


\subsection{Accessing array elements}

Accesses to array elements can be done using the following syntax:
\begin{caml_example}
a.(0);;
\end{caml_example}
or, more generally: $e_1${\tt .(}$e_2${\tt )},
where $e_1$ evaluates to an array and $e_2$ to an integer.
Alternatively, the function \verb|vect_item| is provided:
\begin{caml_example}
vect_item;;
\end{caml_example}
The first element of an array is at index 0. Arrays are useful because
accessing an element is done in constant time: an array is a
contiguous fragment of memory, while accessing list elements takes
linear time.

\subsection{Modifying array elements}

Modification of an array element is done with the construct:
\begin{center}
$e_1${\tt .(}$e_2${\tt )} \verb|<-| $e_3$
\end{center}
where $e_3$ has the same type as the elements of the array $e_1$. The
expression $e_2$ computes the index at which the modification will occur.

As for accessing, a function for modifying array elements is also
provided:
\begin{caml_example}
vect_assign;;
\end{caml_example}
For example:
\begin{caml_example}
a.(0) <- (a.(0)-1);;
a;;
vect_assign a 0 ((vect_item a 0) - 1);;
a;;
\end{caml_example}

\section{Loops: {\tt while} and {\tt for}}
\ikwd{while@\verb`while`}
\ikwd{for@\verb`for`}

Imperative programming (i.e. using side-effects such as physical
modification of data structures) traditionally makes use of sequences
and explicit loops. Sequencing evaluation in Caml Light is done by
using the semicolon ``\verb";"''. Evaluating expression $e_1$,
discarding the value returned, and then evaluating $e_2$ is written:
\begin{center}
$e_1$ {\tt ;} $e_2$
\end{center}
If $e_1$ and $e_2$ perform side-effects, this construct ensures
that they will be performed in the specified order (from left to
right). In order to emphasize sequential side-effects, instead of
using parentheses around sequences, one can use {\tt begin} and {\tt
end}, as in:\ikwd{begin@\verb`begin`}\ikwd{end@\verb`end`}
\begin{caml_example}
let x = ref 1 in
  begin
     x := !x + 1;
     x := !x * !x
  end;;
\end{caml_example}
The keywords {\tt begin} and {\tt end} are equivalent to opening and
closing parentheses. The program above could be written as:
\begin{caml_example}
let x = ref 1 in
  (x := !x + 1; x := !x * !x);;
\end{caml_example}

Explicit loops are not strictly necessary {\em per se}: a recursive
function could perform the same work. However, the usage of an
explicit loop locally emphasizes a more imperative style. Two loops
are provided:
\begin{itemize}
\item {\it while}: {\tt while} $e_1$ {\tt do} $e_2$ {\tt done}
evaluates $e_1$ which must return a boolean expression, if $e_1$
return {\tt true}, then $e_2$ (which is usually a sequence) is
evaluated, then $e_1$ is evaluated again and so on until $e_1$
returns {\tt false}.

\item{\it for}: two variants, increasing and decreasing
\begin{itemize}
\item {\tt for} $v$\verb|=|$e_1$ {\tt to} $e_2$ {\tt do} $e_3$ {\tt done}
\item {\tt for} $v$\verb|=|$e_1$ {\tt downto} $e_2$ {\tt do} $e_3$ {\tt done}
\end{itemize}
where $v$ is an identifier. Expressions $e_1$ and $e_2$ are the bounds
of the loop: they must evaluate to integers. In the case of the
increasing loop, the expressions $e_1$ and $e_2$ are evaluated
producing values $n_1$ and $n_2$ : if $n_1$ is strictly greater than
$n_2$, then nothing is done.  Otherwise, $e_3$ is evaluated $(n_2 -
n_1)+1$ times, with the variable $v$ bound successively to $n_1$, $n_1
+1$, \ldots, $n_2$.

The behavior of the decreasing loop is similar: if $n_1$ is strictly
smaller than $n_2$, then nothing is done. Otherwise, $e_3$ is
evaluated $(n_1 - n_2)+1$ times with $v$ bound to successive values
decreasing from $n_1$ to $n_2$.
\end{itemize}
Both loops return the value \verb|()| of type {\tt unit}.
\begin{caml_example}
for i=0 to (vect_length a) - 1 do a.(i) <- i done;;
a;;
\end{caml_example}

\section{Polymorphism and mutable data structures}

There are some restrictions concerning polymorphism and mutable data
structures.
One cannot enclose polymorphic objects inside mutable data structures.
\begin{caml_example}
let r = ref [];;
\end{caml_example}
The reason is that once the type of {\tt r}, {\tt ('a list) ref}, has
been computed, it cannot be changed. But the value of {\tt r} can be
changed: we could write:
\begin{verbatim}
r:=[1;2];;
\end{verbatim}
and now, {\tt r} would be a reference on a list of numbers while its type
would still be {\tt ('a list) ref}, allowing us to write:
\begin{verbatim}
r:= true::!r;;
\end{verbatim}
making {\tt r} a reference on {\tt [true; 1; 2]}, which is an illegal
Caml object.

Thus the Caml typechecker imposes that modifiable data structures appearing
at toplevel must possess monomorphic types (i.e. not polymorphic).



\section*{Exercises}

\begin{exo}\label{Annot:4}
Give a mutable data type defining the Lisp type of lists and define
the four functions {\tt car}, {\tt cdr}, {\tt rplaca} and {\tt rplacd}.
\end{exo}

\begin{exo}\label{Annot:5}
Define a \verb"stamp" function, of type \verb"unit -> int", that
returns a fresh integer each time it is called. That is, the first
call returns 1; the second call returns 2; and so on.
\end{exo}

\begin{exo}\label{Annot:6}
Define a \verb|quick_sort| function on arrays of floating point
numbers following the {\em quicksort} algorithm \cite{Quicksort}.
Information about the {\em quicksort} algorithm can be found
in~\cite{SedgewickPascal}, for example.
\end{exo}

\chapter{Escaping from computations: exceptions}
\label{c:exceptions}

In some situations, it is necessary to abort computations.
If we are
trying to compute the integer division of an integer {\tt n} by {\tt 0},
then we must escape from that embarrassing situation without returning any
result.

Another example of the usage of such an escape mechanism appears when we
want to define the {\tt head} function on lists:%
\begin{caml_example}
let head = function
     x::L -> x
   | [] -> raise (Failure "head: empty list");;
\end{caml_example}
We cannot give a regular value to the expression {\tt head []} without
losing the polymorphism of {\tt head}. We thus choose to escape: we {\em
raise an exception}.

\section{Exceptions}

An exception is a Caml value of the built-in type \verb"exn", very similar to a sum type. An exception:
\begin{itemize}
\item has a {\em name} ({\tt Failure} in our example),
\item and holds zero or one value ({\tt "head: empty list"} of type {\tt string} in the example).
\end{itemize}

Some exceptions are predefined, like {\tt Failure}. New 
exceptions can be defined with the following construct:
\ikwd{exception@\verb`exception`}
\begin{verbatim}
exception <exception name> [of <type>]
\end{verbatim}

Example:%
\begin{caml_example}
exception Found of int;;
\end{caml_example}
The exception {\tt Found} has been declared, and it carries integer
values. When we apply it to an integer, we get an exception value
(of type \verb"exn"):
\begin{caml_example}
Found 5;;
\end{caml_example}

\section{Raising an exception}

Raising an exception is done by applying the primitive function
\verb"raise" to a value of type \verb"exn":
\begin{caml_example}
raise;;
raise (Found 5);;
\end{caml_example}
Here is a more realistic example:
\begin{caml_example}
let find_index p =
  let rec find n =
    function [] -> raise (Failure "not found")
           | x::L -> if p(x) then raise (Found n)
                     else find (n+1) L
  in find 1;;
\end{caml_example}
The \verb"find_index" function always fails. It raises:
\begin{itemize}
\item {\tt Found n}, if there is an element {\tt x} of the list such that
      {\tt p(x)}, in this case {\tt n} is the index of {\tt x} in the list,
\item the {\tt Failure} exception if no such {\tt x} has been found.
\end{itemize}
Raising exceptions is more than an error mechanism: it is a programmable
control structure.
In the \verb"find_index" example, there was no error when raising the {\tt Found} exception: we
only wanted to quickly escape from the computation, since we found what we
were looking for.
This is why it must be possible to {\em trap} exceptions:
we want to trap possible errors, but we also want to get our result in the
case of the \verb"find_index function".

\section{Trapping exceptions}

Trapping exceptions is achieved by the following construct:
\ikwd{try@\verb`try`}
\begin{verbatim}
try <expression> with <match cases>
\end{verbatim}

This construct evaluates
\verb"<expression>". If no exception is raised during the evaluation, then
the result of the {\tt try} construct is the result of \verb"<expression>".
If an exception is raised during this evaluation, then the raised exception
is matched against the \verb"<match cases>". If a case matches, then control
is passed to it. If no case matches, then the exception is propagated
outside of the {\tt try} construct, looking for the enclosing {\tt try}.%

Example:
\begin{caml_example}
let find_index p L =
  let rec find n =
    function [] -> raise (Failure "not found")
           | x::L -> if p(x) then raise (Found n)
                     else find (n+1) L
  in
    try find 1 L with Found n -> n;;
find_index (function n -> (n mod 2) = 0) [1;3;5;7;9;10];;
find_index (function n -> (n mod 2) = 0) [1;3;5;7;9];;
\end{caml_example}
The \verb"<match cases>" part of the {\tt try} construct is a regular pattern matching on values of type \verb"exn".
It is thus possible to trap any exception
by using the \verb"_" symbol.
As an example, the following function traps any
exception raised during the application of its two arguments. Warning: the
\verb"_" will also trap interrupts from the keyboard such as control-C!
\begin{caml_example}
let catch_all f arg default =
       try f(arg) with _ -> default;;
\end{caml_example}
It is even possible to catch all exceptions, do something special (close or remove opened files, for example), and raise again that exception, to propagate it upwards.
\begin{caml_example}
let show_exceptions f arg =
        try f(arg) with x -> print_string "Exception raised!\n"; raise x;;
\end{caml_example}
In the example above, we print a message to the
standard output channel (the terminal), before raising again the trapped
exception.
\begin{caml_example}
catch_all (function x -> raise (Failure "foo")) 1 0;;
catch_all (show_exceptions (function x -> raise (Failure "foo"))) 1 0;;
\end{caml_example}

\section{Polymorphism and exceptions}

Exceptions must not be polymorphic for a reason similar to the one for
references (although it is a bit harder to give an example).
\begin{caml_example}
exception Exc of 'a list;;
\end{caml_example}
One reason is that the {\tt excn} type is not a parameterized type,
but one deeper reason is that if the exception {\tt Exc} is declared
to be polymorphic, then a function may raise {\tt Exc [1;2]}. There
might be no mention of that fact in the type inferred for the
function.  Then, another function may trap that exception, obtaining
the value {\tt [1;2]} whose real type is {\tt int list}.  But the only
type known by the typechecker is {\tt 'a list}: the {\tt try} form
should refer to the {\tt Exc} data constructor, which is known to be
polymorphic.  It may then be possible to build an ill-typed Caml value
{\tt [true; 1; 2]}, since the typechecker does not possess any further
type information than {\tt 'a list}.

The problem is thus the absence of static connection from exceptions
that are raised and the occurrences where they are trapped. Another
example would be the one of a function raising {\tt Exc} with an
integer or a boolean value, depending on some condition. Then, in that
case, when trying to trap these exceptions, we cannot decide wether
they will hold integers or boolean values.


\section*{Exercises}

\begin{exo}\label{Exc:1}
Define the function \verb"find_succeed" which given a function {\tt f} and a
list {\tt L} returns the first element of {\tt L} on which the application
of {\tt f} succeeds.
\end{exo}
\begin{exo}\label{Exc:2}
Define the function \verb"map_succeed" which given a function {\tt f} and a
list {\tt L} returns the list of the results of successful applications of
{\tt f} to elements of {\tt L}.
\end{exo}

\chapter{Basic input/output}
\label{c:basicio}

We describe in this chapter the Caml Light input/output model and some
of its primitive operations. More complete information about IO can
be found in the Caml Light manual \cite{CamlLightDoc}.

Caml Light has an imperative input/output model: an IO operation
should be considered as a side-effect, and is thus dependent on the
order of evaluation. IOs are performed onto {\em channels} with types
\verb|in_channel| and \verb|out_channel|. These types are {\em
abstract}, i.e. their representation is not accessible.

Three channels are predefined:
\begin{caml_example}
std_in;;
std_out;;
std_err;;
\end{caml_example}
They are the ``standard'' IO channels: \verb|std_in| is usually
connected to the keyboard, and printing onto \verb|std_out| and
\verb|std_err| usually appears on the screen.

\section{Printable types}

It is not possible to print and read every value. Functions, for
example, are typically not readable, unless a suitable string
representation is designed and reading such a representation
is followed by an interpretation computing the desired function.

We call {\em printable type} a type for which there are input/output
primitives implemented in Caml Light. The main printable types are:
\begin{itemize}
\item characters: type \verb|char|;
\item strings: type \verb|string|;
\item integers: type \verb|int|;
\item floating point numbers: type \verb|float|.
\end{itemize}
We know all these types from the previous chapters. Strings and
characters support a notation for escaping to ASCII codes or to
denote special characters such as newline:
\begin{caml_example}
`A`;;
`\065`;;
`\\`;;
`\n`;;
"string with\na newline inside";;
\end{caml_example}
The ``\verb|\|'' character is used as an escape and is
useful for non-printable or special characters.

Of course, character constants can be used as constant patterns:
\begin{caml_example}
function `a` -> 0 | _ -> 1;;
\end{caml_example}
On types such as {\tt char} that have a finite number of constant
elements, it may be useful to use {\em or-patterns}, gathering
constants in the same matching rule:
\begin{caml_example}
let is_vowel = function
  `a` | `e` | `i` | `o` | `u` | `y` -> true
| _ -> false;;
\end{caml_example}
The first rule is chosen if the argument matches one of the cases.
Since there is a total ordering on characters, the syntax of character
patterns is enriched with a ``\verb|..|'' notation:
\begin{caml_example}
let is_lower_case_letter = function
  `a`..`z` -> true
| _ -> false;;
\end{caml_example}
Of course, or-patterns and this notation can be mixed, as in:
\begin{caml_example}
let is_letter = function
  `a`..`z` | `A`..`Z` -> true
| _ -> false;;
\end{caml_example}

In the next sections, we give the most commonly used IO primitives on
these printable types. A complete listing of predefined IO operations
is given in \cite{CamlLightDoc}.

\section{Output}

Printing on standard output is performed by the following
functions:
\begin{caml_example}
print_char;;
print_string;;
print_int;;
print_float;;
\end{caml_example}
Printing is {\em buffered}, i.e. the effect of a call to a printing
function may not be seen immediately: {\em flushing} explicitly the
output buffer is sometimes required, unless a printing function
flushes it implicitly. Flushing is done with the {\tt flush}
function:
\begin{caml_example}
flush;;
print_string "Hello!"; flush std_out;;
\end{caml_example}
The \verb|print_newline| function prints a newline character and
flushes the standard output:
\begin{caml_example}
print_newline;;
\end{caml_example}
Flushing is required when writing standalone applications, in which
the application may terminate without all printing being done.
Standalone applications should terminate by a call to the {\tt exit}
function (from the {\tt io} module), which flushes all pending output on
\verb|std_out| and
\verb|std_err|.

Printing on the standard error channel \verb|std_err| is done with the
following functions:
\begin{caml_example}
prerr_char;;
prerr_string;;
prerr_int;;
prerr_float;;
\end{caml_example}
The following function prints its string argument followed by a
newline character to \verb|std_err| and then flushes \verb|std_err|.
\begin{caml_example}
prerr_endline;;
\end{caml_example}

\section{Input}

These input primitives flush the standard output and read from the
standard input:
\begin{caml_example}
read_line;;
read_int;;
read_float;;
\end{caml_example}
Because of their names and types, these functions do not need further
explanation.

\section{Channels on files}

When programs have to read from or print to files, it is necessary to
open and close channels on these files.

\subsection{Opening and closing channels}

Opening and closing is performed with the following functions:
\begin{caml_example}
open_in;;
open_out;;
close_in;;
close_out;;
\end{caml_example}
The \verb|open_in| function checks the existence of its filename
argument, and returns a new input channel on that file;
\verb|open_out| creates a new file (or truncates it to zero length if
it exists) and returns an output channel on that file. Both functions
fail if permissions are not sufficient for reading or writing.

\noindent
{\bf Warning:}
\begin{itemize}
\item Closing functions close their channel argument. Since
their behavior is unspecified on already closed channels, anything
can happen in this case!
\item Closing one of the standard IO channels (\verb|std_in|,
\verb|std_out|, \verb|std_err|) have unpredictable effects!
\end{itemize}

\subsection{Reading or writing from/to specified channels}

Some of the functions on standard input/output have corresponding
functions working on channels:
\begin{caml_example}
output_char;;
output_string;;
input_char;;
input_line;;
\end{caml_example}

\subsection{Failures}

The exception \verb|End_of_file| is raised when an input operation
cannot complete because the end of the file has been reached.
\begin{caml_example}
End_of_file;;
\end{caml_example}

The exception \verb|sys__Sys_error| (\verb|Sys_error| from the module
{\tt sys}) is raised when some manipulation of files is forbidden by
the operating system:
\begin{caml_example}
open_in "abracadabra";;
\end{caml_example}

The functions that we have seen in this chapter are sufficient for our
needs. Many more exist (useful mainly when working with files) and are
described in \cite{CamlLightDoc}.

\section*{Exercises}

\begin{exo}\label{IO:1}
Define a function \verb|copy_file| taking two filenames (of type {\tt
string}) as arguments, and copying the contents of the first file on
the second one. Error messages must be printed on \verb|std_err|.
\end{exo}

\begin{exo}\label{IO:2}
Define a function {\tt wc} taking a filename as argument and printing
on the standard output the number of characters and lines appearing
in the file. Error messages must be printed on \verb|std_err|.
\end{exo}

\noindent
{\bf Note:} it is good practice to develop a program in defining small
functions. A single function doing the whole work is usually harder to
debug and to read. With small functions, one can trace them and see
the arguments they are called on and the result they produce.

\chapter{Streams and parsers}\label{c:streams}

In the next part of these course notes, we will implement a small
functional language. Parsing valid programs of this language requires
writing a lexical analyzer and a parser for the language. For the
purpose of writing easily such programs, Caml Light provides a special
data structure: {\em streams}. Their main usage is to be interfaced to
input channels or strings and to be matched against {\em stream
patterns}.

\section{Streams}

Streams belong to an abstract data type: their actual representation
remains hidden from the user. However, it is still possible to build
streams either ``by hand'' or by using some predefined functions.

\subsection{The {\tt stream} type}

The type {\tt stream} is a parameterized type. One can build streams
of integers, of characters or of any other type. Streams receive a
special syntax, looking like the one for lists. The empty stream is
written:
\begin{caml_example}
[< >];;
\end{caml_example}
A non empty stream possesses elements that are written preceded by the
``\verb|'|'' (quote) character.
\begin{caml_example}
[< '0; '1; '2 >];;
\end{caml_example}
Elements that are not preceded by ``\verb|'|'' are {\em substreams}
that are expanded in the enclosing stream:
\begin{caml_example}
[< '0; [<'1;'2>]; '3 >];;
let s = [< '"abc" >] in [< s; '"def" >];;
\end{caml_example}
Thus, stream concatenation can be defined as:
\begin{caml_example}
let stream_concat s t = [< s; t >];;
\end{caml_example}
Building streams in this way can be useful while testing a parsing
function or defining a lexical analyzer (taking as argument a stream
of characters and returning a stream of tokens). Stream concatenation
{\em does not copy} substreams: they are simply put in the same
stream. Since (as we will see later) stream matching has a
destructive effect on streams (streams are physically ``eaten'' by
stream matching), parsing \verb|[< t; t >]| will in fact parse {\tt t}
only once: the first occurrence of {\tt t} will be consumed, and the
second occurrence will be empty before its parsing will be performed.

Interfacing streams with an input channel can be done with the
function:
\begin{caml_example}
stream_of_channel;;
\end{caml_example}
returning a stream of characters which are read from the channel
argument. The end of stream will coincide with the end of the file
associated to the channel.

In the same way, one can build the character stream associated to a
character string using:
\begin{caml_example}
stream_of_string;;
let s = stream_of_string "abc";;
\end{caml_example}

\subsection{Streams are lazily evaluated}

Stream expressions are submitted to {\em lazy evaluation}, i.e. they
are effectively build only when required. This is useful in that it
allows for the easy manipulation of ``interactive'' streams like the
stream built from the standard input. If this was not the case, i.e.
if streams were immediately completely computed, a program evaluating
``\verb|stream_of_channel std_in|'' would read everything up to an
end-of-file on standard input before giving control to the rest of the
program. Furthermore, lazy evaluation of streams allows for the
manipulation of infinite streams.  As an example, we can build the
infinite stream of integers, using side effects to show precisely when
computations occur:
\begin{caml_example}
let rec ints_from n =
   [< '(print_int n; print_char ` `; flush std_out; n);
      ints_from (n+1) >];;
let ints = ints_from 0;;
\end{caml_example}
We notice that no printing occurred and that the program terminates:
this shows that none of the elements have been evaluated and that the
infinite stream has not been built. We will see in the next section
that these side-effects will occur on demand, i.e. when tests will be
needed by a matching function on streams.

\section{Stream matching and parsers}

The syntax for building streams can be used for pattern-matching over
them. However, stream matching is more complex than the usual pattern
matching.

\subsection{Stream matching is destructive}

Let us start with a simple example:
\begin{caml_example}
let next = function [< 'x >] -> x;;
\end{caml_example}
The {\tt next} function returns the first element of its stream
argument, and fails if the stream is empty:
\begin{caml_example}
let s = [< '0; '1; '2 >];;
next s;;
next s;;
next s;;
next s;;
\end{caml_example}
We can see from the previous examples that the stream pattern
\verb|[< 'x >]| matches {\em an initial segment} of the stream. Such a
pattern must be read as ``the stream whose first element matches
{\tt x}''. Furthermore, once stream matching has succeeded, the
stream argument has been {\em physically modified} and does not contain
any longer the part that has been recognized by the {\tt next}
function.

If we come back to the infinite stream of integers, we can see that
the calls to {\tt next} provoke the evaluation of the necessary part
of the stream:
\begin{caml_example}
next ints; next ints; next ints;;
\end{caml_example}
Thus, successive calls to {\tt next} remove the first
elements of the stream until it becomes empty. Then, {\tt next} fails
when applied to the empty stream, since, in the definition of {\tt
next}, there is no stream pattern that matches an initial segment of the
empty stream.

It is of course possible to specify several stream patterns as in:
\begin{caml_example}
let next = function
  [< 'x >] -> x
| [< >] -> raise (Failure "empty");;
\end{caml_example}
Cases are tried in turn, from top to bottom.

Stream pattern components are not restricted to quoted patterns
(intended to match stream elements), but can be also function calls
(corresponding to non-terminals, in the grammar terminology).
Functions appearing as stream pattern components are intended to match
substreams of the stream argument: they are called on the actual
stream argument, and they are followed by a pattern which should match
the result of this call. For example, if we define a parser
recognizing a non  empty sequence of characters \verb|`a`|:
\begin{caml_example}
let seq_a =
    let rec seq = function
        [< '`a`; seq l >] -> `a`::l
      | [< >] -> []
    in function [< '`a`; seq l >] -> `a`::l;;
\end{caml_example}
we used the recursively defined function \verb|seq| inside the
stream pattern of the first rule. This definition should be read as:
\begin{itemize}
\item if the stream is not empty and if its first element matches
\verb|`a`|, apply \verb|seq| to the rest of the stream, let {\tt l}
be the result of this call and return \verb|`a`::l|,
\item otherwise, fail (raise \verb|Parse_failure|);
\end{itemize}
and \verb|seq| should be read in the same way (except that, since it
recognizes possibly empty sequences of \verb|`a`|, it never fails).

Less operationally, we can read it as: ``a non-empty sequence of
\verb|`a`| starts with an \verb|`a`|, and is followed by a possibly
empty sequence of \verb|`a`|.

Another example is the recognition of a non-empty sequence of \verb|`a`|
followed by a \verb|`b`|, or a \verb|`b`| alone:
\begin{caml_example}
let seq_a_b = function
  [< seq_a l; '`b` >] -> l@[`b`]
| [< '`b` >] -> [`b`];;
\end{caml_example}
Here, operationally, once an \verb|`a`| has been recognized, the first
matching rule is chosen. Any further mismatch (either from
\verb|seq_a| or from the last \verb|`b`|) will raise a
\verb|Parse_error| exception, and the whole parsing will fail. On the
other hand, if the first character is not an \verb|`a`|, \verb|seq_a|
will raise \verb|Parse_failure|, and the second rule
(\verb|[< '`b` >] -> ...|) will be tried.

This behavior is typical of predictive parsers.  Predictive parsing
is recursive-descent parsing with one look-ahead token. In other
words, a predictive parser is a set of (possibly mutually recursive)
procedures, which are selected according to the shape of (at most) the
first token.

\subsection{Sequential binding in stream patterns}

Bindings in stream patterns occur sequentially, in contrast with
bindings in regular patterns, which can be thought as occurring in
parallel. Stream matching is guaranteed to be performed from left to right.
For example, computing the sum of the elements of an integer stream
could be defined as:
\begin{caml_example}
let rec stream_sum n = function
  [< '0; (stream_sum n) p >] -> p
| [< 'x; (stream_sum (n+x)) p >] -> p
| [< >] -> n;;
stream_sum 0 [< '0; '1; '2; '3; '4 >];;
\end{caml_example}
The \verb|stream_sum| function uses its first argument as an
accumulator holding the sum computed so far. The call
\verb|(stream_sum (n+x))| uses {\tt x} which was bound in the stream
pattern component occurring at the left of the call.

{\bf Warning:} streams patterns are legal only in the {\tt function}
and {\tt match} constructs. The {\tt let} and other forms are
restricted to usual patterns. Furthermore, a stream pattern cannot
appear inside another pattern.

\section{Parameterized parsers}

Since a parser is a function like any other function, it can be
parameterized or be used as a parameter. Parameters used only in the
right-hand side of stream-matching rules simulate {\em inherited
attributes} of attribute grammars. Parameters used as parsers in
stream patterns allow for the implementation of {\em higher-order}
parsers. We will use the next example to motivate the introduction of
parameterized parsers.

\subsection{Example: a parser for arithmetic expressions}

Before building a parser for arithmetic expressions, we need a lexical
analyzer able to recognize arithmetic operations and integer
constants. Let us first define a type for tokens:
\begin{caml_example}
type token =
  PLUS | MINUS | TIMES | DIV | LPAR | RPAR
| INT of int;;
\end{caml_example}
Skipping blank spaces is performed by the {\tt spaces} function
defined as:
\begin{caml_example}
let rec spaces = function
  [< '` `|`\t`|`\n`; spaces _ >] -> ()
| [< >] -> ();;
\end{caml_example}
The conversion of a digit (character) into its integer value is done
by:
\begin{caml_example}
let int_of_digit = function
  `0`..`9` as c -> (int_of_char c) - (int_of_char `0`)
| _ -> raise (Failure "not a digit");;
\end{caml_example}
The ``{\tt as}'' keyword allows for naming a pattern: in this case,
the variable {\tt c} will be bound to the actual digit matched by
\verb|`0`..`9`|. Pattern built with {\tt as} are called {\em alias
patterns}.

For the recognition of integers, we already feel the need for a
parameterized parser. Integer recognition is done by the {\tt integer}
analyzer defined below. It is parameterized by a numeric value
representing the value of the first digits of the number:
\begin{caml_example}
let rec integer n = function
  [< ' `0`..`9` as c; (integer (10*n + int_of_digit c)) r >] -> r
| [< >] -> n;;
integer 0 (stream_of_string "12345");;
\end{caml_example}
We are now ready to write the lexical analyzer, taking a stream of
characters, and returning a stream of tokens. Returning a token stream
which will be explored by the parser is a simple, reasonably efficient
and intuitive way of composing a lexical analyzer and a parser.
\begin{caml_example}
let rec lexer s = match s with
  [< '`(`; spaces _ >] -> [< 'LPAR; lexer s >]
| [< '`)`; spaces _ >] -> [< 'RPAR; lexer s >]
| [< '`+`; spaces _ >] -> [< 'PLUS; lexer s >]
| [< '`-`; spaces _ >] -> [< 'MINUS; lexer s >]
| [< '`*`; spaces _ >] -> [< 'TIMES; lexer s >]
| [< '`/`; spaces _ >] -> [< 'DIV; lexer s >]
| [< '`0`..`9` as c; (integer (int_of_digit c)) n; spaces _ >]
                       -> [< 'INT n; lexer s >];;
\end{caml_example}
We assume there is no leading space in the input.

Now, let us examine the language that we want to recognize. We shall
have integers, infix arithmetic operations and parenthesized
expressions.
The BNF form of the grammar is:
\begin{verbatim}
Expr ::= Expr + Expr
       | Expr - Expr
       | Expr * Expr
       | Expr / Expr
       | ( Expr )
       | INT
\end{verbatim}
The values computed by the parser will be {\em abstract syntax trees}
(by contrast with {\em concrete syntax}, which is the input string or
stream). Such trees belong to the following type:
\begin{caml_example}
type atree =
  Int of int
| Plus of atree * atree
| Minus of atree * atree
| Mult of atree * atree
| Div of atree * atree;;
\end{caml_example}
The {\tt Expr} grammar is ambiguous. To make it unambiguous, we will
adopt the usual precedences for arithmetic operators and assume that
all operators associate to the left. Now, to use stream matching for
parsing, we must take into account the fact that matching rules are
chosen according to the behavior of the first component of each
matching rule. This is predictive parsing, and, using well-known
techniques, it is easy to rewrite the grammar above in such a way that
writing the corresponding predictive parser becomes trivial. These
techniques are described in~\cite{DragonBook}, and consist in adding
a non-terminal for each precedence level, and removing left-recursion.
We obtain:
\begin{verbatim}
Expr ::= Mult
       | Mult + Expr
       | Mult - Expr

Mult ::= Atom
       | Atom * Mult
       | Atom / Mult

Atom ::= INT
       | ( Expr )
\end{verbatim}
While removing left-recursion, we forgot about left associativity of
operators. This is not a problem, as long as we build correct abstract
trees.

Since stream matching bases its choices on the first component of
stream patterns, we cannot see the grammar above as a parser. We need
a further transformation, factoring common prefixes of grammar rules
(left-factor). We obtain:
\begin{verbatim}
Expr ::= Mult RestExpr

        RestExpr ::= + Mult RestExpr
                   | - Mult RestExpr
                   | (* nothing *)

Mult ::= Atom RestMult

        RestMult ::= * Atom RestMult
                   | / Atom RestMult
                   | (* nothing *)

Atom ::= INT
       | ( Expr )
\end{verbatim}
Now, we can see this grammar as a parser (note that the order of rules
becomes important, and empty productions must appear last). The shape
of the parser is:
\begin{verbatim}
let rec expr =
    let rec restexpr ? = function
        [< 'PLUS; mult ?; restexpr ? >] -> ?
      | [< 'MINUS; mult ?; restexpr ? >] -> ?
      | [< >] -> ?
in function [< mult e1; restexpr ? >] -> ?

and mult =
    let rec restmult ? = function
        [< 'TIMES; atom ?; restmult ? >] -> ?
      | [< 'DIV; atom ?; restmult ? >] -> ?
      | [< >] -> ?
in function [< atom e1; restmult ? >] -> ?

and atom = function
  [< 'INT n >] -> Int n
| [< 'LPAR; expr e; 'RPAR >] -> e
\end{verbatim}
We used question marks where parameters, bindings and results still
have to appear.  Let us consider the {\tt expr} function: clearly, as
soon as {\tt e1} is recognized, we must be ready to build the leftmost
subtree of the result. This leftmost subtree is either restricted to
{\tt e1} itself, in case {\tt restexpr} does not encounter any
operator, or it is the tree representing the addition (or subtraction)
of {\tt e1} and the expression immediately following the additive
operator. Therefore, {\tt restexpr} must be called with {\tt e1} as an
intermediate result, and accumulate subtrees built from its
intermediate result, the tree constructor corresponding to the
operator and the last expression encountered. The body of {\tt expr}
becomes:
\begin{verbatim}
let rec expr =
    let rec restexpr e1 = function
        [< 'PLUS; mult e2; restexpr (Plus (e1,e2)) e >] -> e
      | [< 'MINUS; mult e2; restexpr (Minus (e1,e2)) e >] -> e
      | [< >] -> e1
in function [< mult e1; (restexpr e1) e2 >] -> e2
\end{verbatim}
Now, {\tt expr} recognizes a product {\tt e1} (by {\tt mult}), and
applies \verb|(restexpr e1)| to the rest of the stream. According to
the additive operator encountered (if any), this function will apply
{\tt mult} which will return some {\tt e2}. Then the process continues
with \verb|Plus(e1,e2)| as intermediate result. In the end, a
correctly balanced tree will be produced (using the last rule of {\tt
restexpr}).

With the same considerations on {\tt mult} and {\tt restmult}, we can
complete the parser, obtaining:
\begin{caml_example}
let rec expr =
    let rec restexpr e1 = function
        [< 'PLUS; mult e2; (restexpr (Plus (e1,e2))) e >] -> e
      | [< 'MINUS; mult e2; (restexpr (Minus (e1,e2))) e >] -> e
      | [< >] -> e1
in function [< mult e1; (restexpr e1) e2 >] -> e2

and mult =
    let rec restmult e1 = function
        [< 'TIMES; atom e2; (restmult (Mult (e1,e2))) e >] -> e
      | [< 'DIV; atom e2; (restmult (Div (e1,e2))) e >] -> e
      | [< >] -> e1
in function [< atom e1; (restmult e1) e2 >] -> e2

and atom = function
  [< 'INT n >] -> Int n
| [< 'LPAR; expr e; 'RPAR >] -> e;;
\end{caml_example}
And we can now try our parser:
\begin{caml_example}
expr (lexer (stream_of_string "(1+2+3*4)-567"));;
\end{caml_example}

\subsection{Parameters simulating inherited attributes}

In the previous example, the parsers {\tt restexpr} and {\tt restmult}
take an abstract syntax tree {\tt e1} as argument and pass it down to the
result through recursive calls such as \verb|(restexpr
(Plus(e1,e2)))|. If we see such parsers as non-terminals ({\tt
RestExpr} from the grammar above) this parameter acts as an inherited
attribute of the non-terminal. Synthesized attributes are simulated by
the right hand sides of stream matching rules.

\subsection{Higher-order parsers}

In the definition of {\tt expr}, we may notice that the parsers {\tt
expr} and {\tt mult} on the one hand and {\tt restexpr} and {\tt
restmult} on the other hand have exactly the same structure. To
emphasize this similarity, if we define parsers for additive (resp.
multiplicative) operators by:
\begin{caml_example}
let addop = function
  [< 'PLUS >] -> (function (x,y) -> Plus(x,y))
| [< 'MINUS >] -> (function (x,y) -> Minus(x,y))
and multop = function
  [< 'TIMES >] -> (function (x,y) -> Mult(x,y))
| [< 'DIV >] -> (function (x,y) -> Div(x,y));;
\end{caml_example}
we can rewrite the {\tt expr} parser as:
\begin{caml_example}
let rec expr =
    let rec restexpr e1 = function
        [< addop f; mult e2; (restexpr (f (e1,e2))) e >] -> e
      | [< >] -> e1
in function [< mult e1; (restexpr e1) e2 >] -> e2

and mult =
    let rec restmult e1 = function
        [< multop f; atom e2; (restmult (f (e1,e2))) e >] -> e
      | [< >] -> e1
in function [< atom e1; (restmult e1) e2 >] -> e2

and atom = function
  [< 'INT n >] -> Int n
| [< 'LPAR; expr e; 'RPAR >] -> e;;
\end{caml_example}
Now, we take advantage of these similarities in order to define a
general parser for left-associative operators. Its name is
\verb|left_assoc| and is parameterized by a parser for operators and a
parser for expressions:
\begin{caml_example}
let rec left_assoc op term =
    let rec rest e1 = function
        [< op f; term e2; (rest (f (e1,e2))) e >] -> e
      | [< >] -> e1
    in function [< term e1; (rest e1) e2 >] -> e2;;
\end{caml_example}
Now, we can redefine {\tt expr} as:
\begin{caml_example}
let rec expr str = left_assoc addop mult str
and mult str = left_assoc multop atom str
and atom = function
  [< 'INT n >] -> Int n
| [< 'LPAR; expr e; 'RPAR >] -> e;;
\end{caml_example}
And we can now try our definitive parser:
\begin{caml_example}
expr (lexer (stream_of_string "(1+2+3*4)-567"));;
\end{caml_example}
Parameterized parsers are useful for defining general parsers such as
\verb|left_assoc| that can
be used with different instances. Another example of a useful general
parser is the {\tt star} parser defined as:
\begin{caml_example}
let rec star p = function
  [< p x; (star p) l >] -> x::l
| [< >] -> [];;
\end{caml_example}
The {\tt star} parser iterates zero or more times its argument {\tt p}
and returns the list of results. We still have to be careful in using
these general parsers because of the predictive nature of parsing. For
example, {\tt star p} will never successfully terminate if {\tt p}
has a rule for the empty stream pattern: this rule will make
the second rule of {\tt star} useless!

\subsection{Example: parsing a non context-free language}

As an example of parsing with parameterized parsers, we shall build
a parser for the language $\{wCw~ |~ w \in (A|B)^{*} \}$, which is
known to be non context-free.

First, let us define a type for this alphabet:
\begin{caml_example}
type token = A | B | C;;
\end{caml_example}
Given an input of the form $w {\tt C} w$, the idea for a parser
recognizing it is:
\begin{itemize}
\item first, to recognize the sequence $w$ with a parser {\tt wd} (for
{\em word definition}) returning information in order to build a
parser recognizing only $w$;
\item then to recognize {\tt C};
\item and to use the parser built at the first step to recognize the
sequence $w$.
\end{itemize}
The definition of {\tt wd} is as follows:
\begin{caml_example}
let rec wd = function
  [< 'A; wd l >] -> (function [< 'A >] -> "a")::l
| [< 'B; wd l >] -> (function [< 'B >] -> "b")::l
| [< >] -> [];;
\end{caml_example}
The {\tt wu} function (for {\em word usage}) builds a parser
sequencing a list of parsers:
\begin{caml_example}
let rec wu = function
  p::pl -> (function [< p x; (wu pl) l >] -> x^l)
| [] -> (function [< >] -> "");;
\end{caml_example}
The {\tt wu} function builds, from a list of parsers $p_i$, for
$i=1..n$, a single parser
\begin{center}
\verb|function [<|$p_1~x_1${\tt;}\ldots{\tt;}$p_n~x_n$\verb|>] -> [|$x_1${\tt;}\ldots{\tt;}$x_n$\verb|]|
\end{center}
which is the sequencing of parsers $p_i$.
The main parser {\tt w} is:
\begin{caml_example}
let w = function [< wd l; 'C; (wu l) r >] -> r;;
w [< 'A; 'B; 'B; 'C; 'A; 'B; 'B >];;
w [< 'C >];;
\end{caml_example}

In the previous parser, we used an intermediate list of parsers in
order to build the second parser. We can redefine {\tt wd} without
using such a list:
\begin{caml_example}
let w =
    let rec wd wr = function
        [< 'A; (wd (function [< wr r; 'A >] -> r^"a")) p >] -> p
      | [< 'B; (wd (function [< wr r; 'B >] -> r^"b")) p >] -> p
      | [< >] -> wr
    in function [< (wd (function [< >] -> "")) p; 'C; p str >] -> str;;
w [< 'A; 'B; 'B; 'C; 'A; 'B; 'B >];;
w [< 'C >];;
\end{caml_example}
Here, {\tt wd} is made local to {\tt w}, and takes as parameter {\tt
wr} (for {\em word recognizer}) whose initial value is the parser with
an empty stream pattern.  This parameter accumulates intermediate
results, and is delivered at the end of parsing the initial sequence
$w$. After checking for the presence of {\tt C}, it is used to parse
the second sequence $w$.

\section{Further reading}

A summary of the constructs over streams and of primitives over
streams is given in \cite{CamlLightDoc}.

An alternative to parsing with streams and stream matching are the
{\tt camllex} and {\tt camlyacc} programs.

A detailed presentation of streams and stream matching following
``predictive parsing'' semantics can be found
in~\cite{MaunydeRauglaudre92a}, where alternative semantics are given
with some possible implementations.

\section*{Exercises}

\begin{exo}\label{Streams:1}
Define a parser for the language of prefix arithmetic expressions
generated by the grammar:
\begin{verbatim}
Expr ::= INT
       | + Expr Expr
       | - Expr Expr
       | * Expr Expr
       | / Expr Expr
\end{verbatim}
Use the lexical analyzer for arithmetic expressions given above. The
result of the parser must be the integer resulting from the evaluation
of the arithmetic expression, i.e. its type must be:
\begin{center}
\verb`token -> int`
\end{center}
\end{exo}

\begin{exo}\label{Streams:2}
Enrich the type {\tt token} above with a constructor {\tt IDENT
of string} for identifiers, and enrich the lexical analyzer for it to
recognize identifiers built from alphabetic letters (upper or
lowercase). Length of identifiers may be limited.
\end{exo}

\chapter{Standalone programs and separate compilation\label{c:standalone}}

So far, we have used Caml Light in an interactive way. It is also
possible to program in Caml Light in a batch-oriented way: writing
source code in a file, having it compiled into an executable program,
and executing the program outside of the Caml Light environment.
Interactive use is great for learning the language and quickly testing
new functions. Batch use is more convenient to develop larger
programs, that should be usable without knowledge of Caml Light.

Note for Macintosh users: batch compilation is not available
in the standalone Caml Light application. It requires the MPW
environment (see the Caml Light manual).


\section{Standalone programs}

Standalone programs are composed of a sequence of phrases, contained
in one or several text files. Phrases are the same as at toplevel:
expressions, value declarations, type declarations, exception
declarations, and directives. When executing the stand-alone program
produced by the compiler, all phrases are executed in order. The
values of expressions and declared global variables are not printed,
however. A stand-alone program has to perform input and output
explicitly.

Here is a sample program, that prints the number of characters and the
number of lines of its standard input, like the \verb"wc" Unix
utility.

\begin{verbatim}
let chars = ref 0;;
let lines = ref 0;;
try
  while true do
    let c = input_char std_in in
      chars := !chars + 1;
      if c = `\n` then lines := !lines + 1 else ()
  done
with End_of_file ->
  print_int !chars; print_string " characters, ";
  print_int !lines; print_string " lines.\n";
  exit 0
;;
\end{verbatim}

The \verb"input_char" function reads the next character from an input
channel (here, \verb"std_in", the channel connected to standard input).
It raises exception \verb"End_of_file" when reaching the end of the
file. The \verb"exit" function aborts the process. Its argument is the
exit status of the process. Calling \verb"exit" is absolutely necessary to
ensure proper flushing of the output channels.

Assume this program is in file \verb"count.ml". To compile it, simply
run the \verb"camlc" command from the command interpreter:
\begin{verbatim}
camlc -o count count.ml
\end{verbatim}
The compiler produces an executable file \verb"count". You can now run
\verb"count" with the help of the "camlrun" command:
\begin{verbatim}
camlrun count < count.ml
\end{verbatim}
This should display something like:
\begin{verbatim}
306 characters, 13 lines.
\end{verbatim}
Under Unix, the \verb"count" file can actually be executed directly,
just like any other Unix command, as in:
\begin{verbatim}
./count < count.ml
\end{verbatim}
This also works under MS-DOS, provided the executable file is given
extension \verb".exe". That is, if we compile \verb"count.ml" as follows:
\begin{verbatim}
camlc -o count.exe count.ml
\end{verbatim}
we can run \verb"count.exe" directly, as in:
\begin{verbatim}
count.exe < count.ml
\end{verbatim}
See the reference manual for more information on \verb"camlc".

\section{Programs in several files}\label{s:modules}

It is possible to split one program into several source files,
separately compiled. This way, local changes do not imply a full
recompilation of the program. Let us illustrate that on the previous
example. We split it in two modules: one that implements integer
counters; another that performs the actual counting. Here is the first
one, \verb"counter.ml":
\begin{verbatim}
(* counter.ml *)
type counter = { mutable val: int };;
let new init = { val = init };;
let incr c = c.val <- c.val + 1;;
let read c = c.val;;
\end{verbatim}
Here is the source for the main program, in file \verb"main.ml".
\begin{verbatim}
(* main.ml *)
let chars = counter__new 0;;
let lines = counter__new 0;;
try
  while true do
    let c = input_char std_in in
      counter__incr chars;
      if c = `\n` then counter__incr lines else ()
  done
with End_of_file ->
  print_int (counter__read chars); print_string " characters, ";
  print_int (counter__read lines); print_string " lines.\n";
  exit 0
;;
\end{verbatim}
Notice that references to identifiers defined in module
\verb"counter.ml" are prefixed with the name of the module,
\verb"counter", and by \verb"__" (the ``long dash'' symbol: two
underscore characters). If we had simply entered \verb"new 0", for
instance, the compiler would have assumed \verb"new" is an identifier
declared in the current module, and issued an ``undefined identifier''
error.

Compiling this program requires two compilation steps, plus one final
linking step.
\begin{verbatim}
camlc -c counter.ml
camlc -c main.ml
camlc -o main counter.zo main.zo
\end{verbatim}
Running the program is done as before:
\begin{verbatim}
camlrun main < counter.ml
\end{verbatim}
The \verb"-c" option to \verb"camlc" means ``compile only''; that is,
the compiler should not attempt to produce a stand-alone executable
program from the given file, but simply an object code file (files
\verb"counter.zo", \verb"main.zo"). The final linking phases takes the
two \verb".zo" files and produces the executable \verb"main". Object
files must be linked in the right order: for each global identifier,
the module defining it must come before the modules that use it.

\medskip

Prefixing all external identifiers by the name of their defining
module is sometimes tedious. Therefore, the Caml Light compiler
provides a mechanism to omit the \verb"module__" part in external
identifiers. The system maintains a list of ``default'' modules, called
the currently opened modules, and whenever it encounters an identifier
without the \verb"module__" part, it searches through the opened modules,
to find one that defines this identifier. Searched modules always
include the module being compiled (searched first), and some library
modules of general use. In addition, two directives are provided to add
and to remove modules from the list of opened modules:
\begin{itemize}
\item \verb|#open "module";;| to add \verb"module" in front of the list;
\item \verb|#close "module";;| to remove \verb"module" from the list.
\end{itemize}
For instance, we can rewrite the \verb"main.ml" file above as:
\begin{verbatim}
#open "counter";;
let chars = new 0;;
let lines = new 0;;
try
  while true do
    let c = input_char std_in in
      incr chars;
      if c = `\n` then incr lines
  done
with End_of_file ->
  print_int (read chars);
  print_string " characters, ";
  print_int (read lines);
  print_string " lines.\n";
  exit 0
;;
\end{verbatim}
After the \verb|#open "counter"| directive, the identifier \verb"new"
automatically resolves to \verb"counters__new".

If two modules, say \verb"mod1" and \verb"mod2", define both a global
value \verb"f", then \verb"f" in a client module \verb"client"
resolves to \verb"mod1__f" if \verb"mod1" is opened but not \verb"mod2", or
if \verb"mod1" has been opened more recently than \verb"mod2". Otherwise, it resolves to \verb"mod2__f".
For instance, the two system modules \verb"int" and \verb"float" both
define the infix identifier
\verb"+". Both modules \verb"int" and \verb"float" are opened by
default, but \verb"int" comes first. Hence, \verb"x + y" is understood
as the integer addition, since \verb"+" is resolved to \verb"int__+".
But after the directive
\verb|#open "float";;|, module \verb"float" comes first, and the
identifier \verb"+" is resolved to \verb"float__+". 

\section{Abstraction}

Some globals defined in a module are not intended to be used outside
of this module. Then, it is good programming style not to export them
outside of the module, so that the compiler can check they are not used in
another module. Also, one may wish to export a data type abstractly,
that is, without publicizing the structure of the type. This ensures
that other modules cannot build or inspect objects of that type
without going through one of the functions on that type exported in
the defining module. This helps in writing clean,
well-structured programs.

The way to do that in Caml Light is to write an explicit interface, or
output signature, specifying those identifiers that are visible from
the outside. All other identifiers will remain local to the module.
For global values, their types must be given by hand. The interface is
contained in a file whose name is the module name, with extension
\verb".mli".

Here is for instance an interface for the \verb"counter" module, that
abstracts the type \verb"counter":
\begin{verbatim}
(* counter.mli *)
type counter;;        (* an abstract type *)
value new : int -> counter
  and incr : counter -> unit
  and read : counter -> int;;
\end{verbatim}

Interfaces must be compiled separately. However, once the interface for
module $A$ has been compiled, any module $B$ that uses $A$ can be
immediately compiled, even if the implementation of $A$ is not yet
compiled or even not yet written. Consider:
\begin{verbatim}
camlc -c counter.mli
camlc -c main.ml
camlc -c counter.ml
camlc -o main counter.zo main.zo
\end{verbatim}
The implementation \verb|main.ml| could be compiled before
\verb|counter.ml|. The only requirement for compiling \verb|main.ml|
is the existence of \verb|counter.zi|, the compiled interface of the
\verb|counter| module.

\section*{Exercises}

\begin{exo}\label{Modules:1}
Complete the \verb"count" command: it should be able to operate on
several files, given on the command line. Hint:
\verb"sys__command_line" is an array of strings, containing the
command-line arguments to the process.
\end{exo}

\part{A complete example}
\chapter{ASL: A Small Language}
\label{c:ASL}


We present in this chapter a simple language: ASL (A Small Language).
This language is basically the $\lambda$-calculus (the purely
functional kernel of Caml) enriched with a conditional construct. The
conditional must be a special construct, because our language will be
submitted to call-by-value: thus, the conditional cannot be a
function.
%

ASL programs are built up from numbers, variables, functional
expressions ($\lambda$-abstractions), applications and conditionals.
An ASL program consists of a global declaration of an identifier
getting bound to the value of an expression. The primitive functions that
are available are equality between numbers and arithmetic binary
operations.
The concrete syntax of ASL expressions can be described (ambiguously) as:
\begin{quote}
\begin{verbatim}
Expr ::= INT
       | IDENT
       | "if" Expr "then" Expr "else" Expr "fi"
       | "(" Expr ")"
       | "\" IDENT "." Expr
\end{verbatim}
and the syntax of declarations is given as:
\begin{verbatim}
Decl ::= "let" IDENT "be" Expr ";"
       | Expr ";"
\end{verbatim}
\end{quote}
Arithmetic binary operations will be written in prefix position and
will belong to the class {\tt IDENT}. The \verb|\| symbol will play
the role of the Caml keyword {\tt function}.

We start by defining the abstract syntax of ASL expressions and of ASL
toplevel phrases. Then we define a parser in order to produce abstract
syntax trees from the concrete syntax of ASL programs.

\section{ASL abstract syntax trees}

We encode variable names by numbers. These numbers represent the {\em
binding depth} of variables. For instance, the function of {\tt x}
returning {\tt x} (the ASL identity function) will be represented as:
%
\begin{verbatim}
Abs("x", Var 1)
\end{verbatim}
And the ASL application function which would be written in Caml:
\begin{verbatim}
(function f -> (function x -> f(x)))
\end{verbatim}
would be represented as:
\begin{verbatim}
Abs("f", Abs("x", App(Var 2, Var 1)))
\end{verbatim}
and should be viewed as the tree:
\begin{center}
\setlength{\unitlength}{24pt}
\begin{picture}(4,3)(0,0)
\put(1,3){\makebox(0,0){\tt Abs}}
\put(2,2){\makebox(0,0){\tt Abs}}
\put(0,2){\makebox(0,0){\tt "f"}}
\put(1,1){\makebox(0,0){\tt "x"}}
\put(3,1){\makebox(0,0){\tt App}}
\put(2,0){\makebox(0,0){\tt Var 2}}
\put(4,0){\makebox(0,0){\tt Var 1}}
\def\arrows(#1,#2){
  \put(#1,#2){\vector(-1,-1){0.6}}
  \put(#1,#2){\vector(1,-1){0.6}}}
\arrows(1,2.8)
\arrows(2,1.8)
\arrows(3,0.8)
\end{picture}
\end{center}
{\tt Var n} should be read as ``an occurrence of the variable bound by
the {\tt n}th abstraction node encountered when
going toward the root of the abstract syntax tree''. In our example, when
going from {\tt Var 2} to the root, the 2nd abstraction node we encounter
introduces the {\tt "f"} variable.

The numbers encoding variables in abstract syntax trees of functional
expressions are called ``De Bruijn\footnote{They have been proposed by
N.G.~De~Bruijn in \cite{DeBruijn} in order to facilitate the mechanical
treatment of $\lambda$-calculus terms.} numbers''. The characters that we attach
to abstraction nodes simply serve as documentation: they will not be used by
any of the semantic analyses that we will perform on the trees. The type of
ASL abstract syntax trees is defined by:
%
%
\begin{caml_example}
type asl = Const of int
         | Var of int
         | Cond of asl * asl * asl
         | App of asl * asl
         | Abs of string * asl

and top_asl = Decl of string * asl;;
\end{caml_example}

\section{Parsing ASL programs}

Now we come to the problem of defining a concrete syntax for ASL programs
and declarations.
%

The choice of the concrete aspect of the programs is
simply a matter of taste. The one we choose here is close to the syntax of
$\lambda$-calculus (except that we will use the {\em backslash} character
because there is no ``$\lambda$'' on our keyboards). We will use the {\em
curried} versions of equality and arithmetic functions. We will also use
a {\em prefix} notation (\`a la Lisp) for their application. We will write
``{\tt + (+ 1 2) 3}'' instead of ``{\tt (1+2)+3}''. The ``{\tt if $e_1$
then $e_2$ else $e_3$}'' construct will be written ``{\tt if $e_1$
then $e_2$ else $e_3$ fi}'', and will return the {\tt then} part when
$e_1$ is different from 0 (0 acts thus as falsity in ASL conditionals).

\subsection{Lexical analysis}
\label{s:ASLlexing}

The concrete aspect of ASL programs will be either declarations of the
form:
\begin{quote}
{\tt let} {\it identifier} {\tt be} {\it expression}{\tt ;}
\end{quote}
or:
\begin{quote}
{\it expression}{\tt ;}
\end{quote}
which will be understood as:
\begin{quote}
{\tt let it} {\tt be} {\it expression}{\tt ;}
\end{quote}
The tokens produced by the lexical analyzer will represent the
keywords {\tt let}, {\tt be}, {\tt if} and {\tt else}, the \verb|\|
binder, the dot, parentheses, integers, identifiers, arithmetic
operations and terminating semicolons. We reuse here most of the code
that we developed in chapter~\ref{c:streams} or in the answers to its
exercises.

Skipping blank spaces:
\begin{caml_example}
let rec spaces = function
  [< '` `|`\t`|`\n`; spaces _ >] -> ()
| [< >] -> ();;
\end{caml_example}
The type of tokens is given by:
\begin{caml_example}
type token = LET | BE | LAMBDA | DOT | LPAR | RPAR
           | IF | THEN | ELSE | FI | SEMIC
           | INT of int | IDENT of string;;
\end{caml_example}
Integers:
\begin{caml_example}
let int_of_digit = function
  `0`..`9` as c -> (int_of_char c) - (int_of_char `0`)
| _ -> raise (Failure "not a digit");;
let rec integer n = function
  [< ' `0`..`9` as c; (integer (10*n + int_of_digit c)) r >] -> r
| [< >] -> INT n;;
\end{caml_example}
We restrict ASL identifiers to be composed of lowercase letters, the
eight first being significative. An explanation about the {\tt ident}
function can be found in the chapter dedicated to the answers to
exercises (chapter~\ref{c:ans}). The function given here is slightly
different and tests its result in order to see wether it is a keyword
({\tt let}, {\tt be}, \ldots) or not:
\begin{caml_example}
let ident_buf = make_string 8 ` `;;
let rec ident len = function
  [< ' `a`..`z` as c;
     (if len >= 8 then ident len
      else begin
            set_nth_char ident_buf len c;
            ident (succ len)
           end) s >] -> s
| [< >] -> (match sub_string ident_buf 0 len
            with "let" -> LET
               | "be" -> BE
               | "if" -> IF
               | "then" -> THEN
               | "else" -> ELSE
               | "fi" -> FI
               | s -> IDENT s);;
\end{caml_example}
A reasonable lexical analyzer would use a hash table to recognize keywords
faster.

Primitive operations are recognized by the following function, which
also detects illegal operators and ends of input:
\begin{caml_example}
let oper = function
  [< '`+`|`-`|`*`|`/`|`=` as c >] -> IDENT(make_string 1 c)
| [< 'c >] -> prerr_string "Illegal character: ";
              prerr_endline (char_for_read c);
              raise (Failure "ASL parsing")
| [< >] -> prerr_endline "Unexpected end of input";
           raise (Failure "ASL parsing");;
\end{caml_example}
The lexical analyzer has the same structure as the one given in
chapter~\ref{c:streams} except that leading blanks are skipped.
\begin{caml_example}
let rec lexer str = spaces str;
match str with
  [< '`(`; spaces _ >]  -> [< 'LPAR; lexer str >]
| [< '`)`; spaces _ >]  -> [< 'RPAR; lexer str >]
| [< '`\\`; spaces _ >] -> [< 'LAMBDA; lexer str >]
| [< '`.`; spaces _ >]  -> [< 'DOT; lexer str >]
| [< '`;`; spaces _ >]  -> [< 'SEMIC; lexer str >]
| [< '`0`..`9` as c;
     (integer (int_of_digit c)) tok;
      spaces _ >]       -> [< 'tok; lexer str >]
| [< '`a`..`z` as c;
     (set_nth_char ident_buf 0 c; ident 1) tok;
     spaces _ >]        -> [< 'tok; lexer str >]
| [< oper tok; spaces _ >] -> [< 'tok; lexer str >]
;;
\end{caml_example}
The lexical analyzer returns a stream of tokens that the parser will
receive as argument.

\subsection{Parsing}

The final output of our parser will be abstract syntax trees of type
{\tt asl} or \verb"top_asl". This implies that we will detect unbound
identifiers at parse-time. In this case, we will raise the {\tt Unbound}
exception defined as:
\begin{caml_example}
exception Unbound of string;;
\end{caml_example}
We also need a function which will compute the binding depths of variables.
That function simply looks for the position of the first occurrence of a
variable name in a list. It will raise {\tt Unbound} if there is no such
occurrence.
%
\begin{caml_example}
let binding_depth s rho =
 let rec bind n = function
     []  -> raise (Unbound s)
  | t::l -> if s = t then Var n else bind (n+1) l
  in bind 1 rho
;;
\end{caml_example}
We also need a global environment, containing names of already bound
identifiers.  The global environment contains predefined names for the
equality and arithmetic functions. We represent the global environment
as a reference since each ASL declaration will augment it with a new
name.
\begin{caml_example}
let init_env =  ["+";"-";"*";"/";"="];;
let global_env = ref init_env;;
\end{caml_example}
We now give a parsing function for ASL programs. Blanks at the
beginning of the string are skipped.
\begin{caml_example}
let rec top = function
    [< 'LET; 'IDENT id; 'BE; expression e; 'SEMIC >] -> Decl(id,e)
  | [< expression e; 'SEMIC >] -> Decl("it",e)

and expression = function
    [< (expr !global_env) e >] -> e

and expr rho =
  let rec rest e1 = function
          [< (atom rho) e2; (rest (App(e1,e2))) e >] -> e
        | [< >] -> e1
  in function
       [< 'LAMBDA; 'IDENT id; 'DOT; (expr (id::rho)) e >] -> Abs(id,e)
     | [< (atom rho) e1; (rest e1) e2 >] -> e2

and atom rho = function
    [< 'IDENT id >] ->
        (try binding_depth id rho with Unbound s ->
              print_string "Unbound ASL identifier: ";
              print_string s; print_newline();
              raise (Failure "ASL parsing"))
  | [< 'INT n >] -> Const n
  | [< 'IF; (expr rho) e1; 'THEN; (expr rho) e2;
       'ELSE; (expr rho) e3; 'FI >] -> Cond(e1,e2,e3)
  | [< 'LPAR; (expr rho) e; 'RPAR >] -> e;;
\end{caml_example}
The complete parser that we will use reads a string, converts it
into a stream, and produces the token stream that is parsed:
\begin{caml_example}
let parse_top s = top(lexer(stream_of_string s));;
\end{caml_example}
Let us try our grammar (we do not augment the global environment at
each declaration: this will be performed after the semantic treatment
of ASL programs). We need to write double \verb|\| inside strings,
since \verb|\| is the string escape character.
\begin{caml_example}
parse_top "let f be \\x.x;";;
parse_top "let x be + 1 ((\\x.x) 2);";;
\end{caml_example}
Unbound identifiers and undefined operators are correctly detected:
\begin{caml_example}
parse_top "let y be g 3;";;
parse_top "f (if 0 then + else - fi) 2 3;";;
parse_top "^ x y;";;
\end{caml_example}

\chapter{Untyped semantics of ASL programs}
\label{c:ASLsemantics}


In this section, we give a semantic treatment of ASL programs. We will use
{\em dynamic typechecking}, i.e. we will test the type correctness of
programs during their interpretation.

\section{Semantic values}
%

We need a type for ASL semantic values (representing results of computations).
A semantic value will be either an integer, or a Caml functional value
from ASL values to ASL values.
\begin{caml_example}
type semval = Constval of int
            | Funval of (semval -> semval);;
\end{caml_example}
We now define two exceptions. The first one will be used when we
encounter an ill-typed program and will represent run-time type
errors. The other one is helpful  for debugging: it will be raised when
our interpreter (semantic function) goes into an illegal situation.


The following two exceptions will be raised in case of run-time
ASL type error, and in case of bug of our semantic treatment:
\begin{caml_example}
exception Illtyped;;
exception SemantBug of string;;
\end{caml_example}
We must give a semantic value to our basic functions (equality and
arithmetic operations). The next function transforms a Caml function
into an ASL value.
%
\begin{caml_example}
let init_semantics caml_fun =
    Funval
      (function Constval n ->
         Funval(function Constval m -> Constval(caml_fun n m)
                        | _ -> raise Illtyped)
              | _ -> raise Illtyped);;
\end{caml_example}


Now, associate a Caml Light function to each ASL predefined function:
\begin{caml_example}
let caml_function = function
    "+" -> prefix +
  | "-" -> prefix -
  | "*" -> prefix *
  | "/" -> prefix /
  | "=" -> (fun n m -> if n=m then 1 else 0)
  | s -> raise (SemantBug "Unknown primitive");;
\end{caml_example}
In the same way as, for parsing, we needed a global environment from
which the binding depth of identifiers was computed, we need a
semantic environment from which the interpreter will fetch the value
represented by identifiers.
The global semantic environment will be a reference on the list of
predefined ASL values.
\begin{caml_example}
let init_sem =  map (fun x -> init_semantics(caml_function x))
                    init_env;;
let global_sem = ref init_sem;;
\end{caml_example}

\section{Semantic functions}

The semantic function is the interpreter itself. There is one for
expressions and one for declarations. The one for expressions
computes the value of an ASL
expression from an environment {\tt rho}. The environment will contain
the values of globally defined ASL values or of temporary ASL values.
It is organized as a list, and the numbers representing variable
occurrences will be used as indices into the environment.
%
\begin{caml_example}
let rec nth n = function
     []  -> raise (Failure "nth")
  | x::l -> if n=1 then x else nth (n-1) l;;
let rec semant rho =
  let rec sem = function
      Const n -> Constval n
    | Var(n) -> nth n rho
    | Cond(e1,e2,e3) ->
        (match sem e1 with Constval 0 -> sem e3
                         | Constval n -> sem e2
                         | _ -> raise Illtyped)
    | Abs(_,e') -> Funval(fun x -> semant (x::rho) e')
    | App(e1,e2) -> (match sem e1
                      with Funval(f) -> f (sem e2)
                         | _ -> raise Illtyped)
  in sem
;;
\end{caml_example}
%
The main function must be able to treat an ASL declaration, evaluate it, and
update the global environments (\verb"global_env" and \verb"global_sem").
\begin{caml_example}
let semantics = function Decl(s,e) ->
    let result = semant !global_sem e
    in global_env := s::!global_env;
       global_sem := result::!global_sem;
       print_string "ASL Value of ";
       print_string s;
       print_string " is ";
       (match result with
         Constval n -> print_int n
       | Funval f -> print_string "<fun>");
       print_newline();;
\end{caml_example}

\section{Examples}


\begin{caml_example}
semantics (parse_top "let f be \\x. + x 1;");;
semantics (parse_top "let i be \\x. x;");;
semantics (parse_top "let x be i (f 2);");;
semantics (parse_top "let y be if x then (\\x.x) else 2 fi 0;");;
\end{caml_example}

\chapter{Encoding recursion}
\label{c:ASLuntypedrecursion}

\section{Fixpoint combinators}

We have seen that we do not have recursion in ASL. However, it is possible to
encode recursion by defining a {\em fixpoint combinator}.
A fixpoint combinator is a function $F$ such that:
\[
F~M \mbox{ is equivalent to } M~(F~M) \mbox{ modulo the evaluation rules.}
\]
for any expression $M$. A consequence of the equivalence given above
is that fixpoint combinators can encode recursion. Let us note $M
\equiv N$ if expressions $M$ and $N$ are equivalent modulo the
evaluation rules. Then, consider {\tt ffact} to be the functional
obtained from the body of the factorial function by abstracting (i.e.
using as a parameter) the {\tt fact} identifier, and {\tt fix} an
arbitrary fixpoint combinator. We have:
\begin{itemize}
\item[] \verb|ffact| is \verb|\fact.(\n. if = n 0 then 1 else * n (fact (- n 1)) fi)|
\end{itemize}
Now, let us consider the expression $E=\verb|(fix ffact) 3|$. Using our
intuition about the evaluation rules, and the definition of a fixpoint
combinator, we obtain:\\
\hspace{\parindent}\hspace{\parindent}
{\tt $E \equiv$ ffact (fix ffact) 3}\\
Replacing {\tt ffact} by its definition, we obtain:\\
\hspace{\parindent}\hspace{\parindent}
{\tt $E \equiv$  \verb|(\fact.(\n. if = n 0 then 1 else * n (fact (- n 1)) fi))| (fix ffact) 3}\\
We can now pass the two arguments to the first abstraction,
instantiating {\tt fact} and {\tt n} respectively to {\tt fix ffact}
and {\tt 3}:\\
\hspace{\parindent}\hspace{\parindent}
{\tt $E \equiv$  \verb|if = 3 0 then 1 else * 3 (fix ffact (- 3 1)) fi|}\\
We can now reduce the conditional into its {\tt else} branch:\\
\hspace{\parindent}\hspace{\parindent}
{\tt $E \equiv$  \verb|* 3 (fix ffact (- 3 1))|}\\
Continuing this way, we eventually compute:\\
\hspace{\parindent}\hspace{\parindent}
$E\equiv$ {\tt * 3 (* 2 (* 1 1))} $\equiv$ {\tt 6}

This is the expected behavior of the factorial function. Given an
appropriate fixpoint combinator {\tt fix}, we could define the
factorial function as {\tt fix ffact}, where {\tt ffact} is the
expression above.

Unfortunately, when using call-by-value, any application of a fixpoint
combinator $F$ such that:
\[
F~M \mbox{ evaluates to } M~(F~M)
\]
leads to non-termination of the evaluation (because evaluation of
$(F~M)$ leads to evaluating $(M~(F~M))$, and thus $(F~M)$ again).

We will use the $Z$ fixpoint combinator defined by:
\[
Z = \lambda f.((\lambda x.~f~(\lambda y.~(x ~x)~y)) (\lambda
x.~f~(\lambda y.~(x ~x)~y)))
\]
The fixpoint combinator $Z$ has the particularity of being usable
under call-by-value evaluation regime (in order to check that fact, it
is necessary to know the evaluation rules of $\lambda$-calculus).
Since the name {\tt z} looks more like an ordinary parameter name, we
will call {\tt fix} the ASL expression corresponding to the $Z$
fixpoint combinator.
%
\begin{caml_example}
semantics (parse_top
        "let fix be \\f.((\\x.f(\\y.(x x) y))(\\x.f(\\y.(x x) y)));");;
\end{caml_example}
%
We are now able to define the ASL factorial function:
\begin{caml_example}
semantics (parse_top
        "let fact be fix (\\f.(\\n. if = n 0 then 1
                                    else * n (f (- n 1)) fi));");;
semantics (parse_top "fact 8;");;
\end{caml_example}
and the ASL Fibonacci function:
\begin{caml_example}
semantics (parse_top
        "let fib be fix (\\f.(\\n. if = n 1 then 1
                                   else if = n 2 then 1
                                        else + (f (- n 1)) (f (- n 2)) fi fi));");;
semantics (parse_top "fib 9;");;
\end{caml_example}

\section{Recursion as a primitive construct}

Of course, in a more realistic prototype, we would extend the concrete
and abstract syntaxes of ASL in order to support recursion as a
primitive construct. We do not do it here because we want to keep ASL
simple.  This is an interesting non trivial exercise!

\chapter{Static typing, polymorphism and type synthesis}
\label{c:ASLtyping}
%

We now want to perform static typechecking of ASL programs, that is,
to complete typechecking {\em before} evaluation, making run-time type
tests unnecessary during evaluation of ASL programs.

Furthermore, we want to have {\em polymorphism} (i.e. allow the
identity function, for example, to be applicable to any kind of data).

Type synthesis may be seen as a game. When learning a game, we must:
\begin{itemize}
\item learn the rules (what is allowed, and what is forbidden);
\item learn a winning strategy.
\end{itemize}

In type synthesis, the rules of the game are called a {\em type
system}, and the winning strategy is the typechecking algorithm.

In the following sections, we give the ASL type system, the algorithm
and an implementation of that algorithm. Most of this presentation is
borrowed from
\cite{MiniML}.

\section{The type system}

We study in this section a type system for the ASL language. Then, we
present an algorithm performing the type synthesis of ASL programs,
and its Caml Light implementation. Because of subtle aspects of the
notation used (inference rules), and since some important mathematical
notions, such as unification of first-order terms, are not presented
here, this chapter may seem obscure at first reading.

The type system we will consider for ASL has been first given by
Milner \cite{Milner78} for a subset of the ML language (in fact, a
superset of $\lambda$-calculus).
A {\em type} is either:
\begin{itemize}
\item the type Number;
\item or a type variable ($\alpha$, $\beta$, \ldots);
\item or $\tau_1 \rightarrow \tau_2$, where $\tau_1$ and $\tau_2$ are types.
\end{itemize}
In a type, a type variable is an {\em unknown}, i.e. a type that we
are computing.  We will use $\tau$, $\tau'$, $\tau_1$, \ldots, as {\em
metavariables}\footnote{A metavariable should not be confused with a
{\em variable} or a {\em type variable}.} representing types. This
notation is important: we shall use other greek letters to denote
other notions in the following sections.

\Example
$(\alpha \rightarrow \mbox{Number}) \rightarrow \beta \rightarrow \beta$
is a type.
\End
%
A {\em type scheme}, is a type where some variables are distinguished as
being {\em generic}. We can represent type schemes by:
\[
\forall \alpha_1,  \ldots, \alpha_n . \tau \mbox{ where } \tau \mbox{ is a
type.}
\]

\Example
$\forall \alpha . (\alpha \rightarrow \mbox{Number}) \rightarrow \beta \rightarrow \beta$
and
$(\alpha \rightarrow \mbox{Number}) \rightarrow \beta \rightarrow \beta$
are type schemes.
\End
We will use $\sigma$, $\sigma'$, $\sigma_1$, \ldots, as metavariables
representing type schemes.
We may also write type schemes as $\forall
\vec{\alpha} . \tau$. In this case, $\vec{\alpha}$ represent a (possibly
empty) set of generic type variables. When the set of generic variables is
empty, we write $\forall . \tau$ or simply $\tau$.

We will write $FV(\sigma)$ for the set of {\em unknowns} occurring in the type
scheme $\sigma$. Unknowns are also called {\em free variables} (they are not
bound by a $\forall$ quantifier).
%
%

We also write $BV(\sigma)$ ({\em bound type variables of $\sigma$}) for the
set of type variables occurring in $\sigma$ which are not free (i.e.
the set of variables universally quantified). Bound type variables are
also said to be {\em generic}.

\Example
If $\sigma$ denotes the type scheme
$\forall \alpha . (\alpha \rightarrow \mbox{Number}) \rightarrow \beta \rightarrow \beta$,
then we have:
\[
FV(\sigma) = \{\beta\}
\]
and
\[
BV(\sigma) = \{\alpha\}
\]
\End
%
A {\em substitution instance} $\sigma'$ of a type scheme $\sigma$ is
the type scheme $S(\sigma)$ where $S$ is a substitution of types for
{\em free} type variables appearing in $\sigma$.
When applying a substitution to a type scheme, a renaming of some
bound type variables may become necessary, in order to avoid the capture
of a free type variable by a quantifier.

\Example
\begin{itemize}
\item 
If $\sigma$ denotes $\forall \beta . (\beta \rightarrow \alpha) \rightarrow \alpha$
and
$\sigma'$ is $\forall \beta . (\beta \rightarrow (\gamma \rightarrow \gamma))
                                 \rightarrow (\gamma \rightarrow \gamma)$,
then $\sigma'$
is a substitution instance of $\sigma$ because $\sigma' = S(\sigma)$
where $S=\{\alpha \leftarrow (\gamma \rightarrow \gamma)\}$, i.e. $S$
substitutes the type $\gamma \rightarrow \gamma$ for the variable $\alpha$.
\item 
If $\sigma$ denotes $\forall \beta . (\beta \rightarrow \alpha) \rightarrow \alpha$
and
$\sigma'$ is $\forall \delta . (\delta \rightarrow (\beta \rightarrow \beta))
                                 \rightarrow (\beta \rightarrow \beta)$,
then $\sigma'$ is a substitution instance of $\sigma$ because $\sigma' = S(\sigma)$
where $S=\{\alpha \leftarrow (\beta \rightarrow \beta)\}$. In this case,
the renaming of $\beta$ into $\delta$ was necessary: we did not want the
variable $\beta$ introduced by $S$ to be captured by the universal
quantification $\forall \beta$.
\end{itemize}
\End
%
The type scheme $\sigma' = \forall \beta_1 \ldots \beta_m . \tau'$ is
said to be a {\em generic instance} of
$\sigma = \forall \alpha_1 \ldots \alpha_n . \tau$ if there exists a
substitution $S$ such that:
\begin{itemize}
\item the domain of $S$ is included in $\{\alpha_1, \ldots, \alpha_n\}$;
\item $\tau' = S(\tau)$;
\item no $\beta_i$ occurs free in $\sigma$.
\end{itemize}
In other words, a generic instance of a type scheme is obtained by giving more
precise values to some generic variables, and (possibly) quantifying some of
the new type variables introduced.

\Example
If $\sigma = \forall \beta . (\beta \rightarrow \alpha) \rightarrow \alpha$,
then $\sigma' = \forall \gamma . ((\gamma \rightarrow \gamma)
                                      \rightarrow \alpha) \rightarrow \alpha$
is a generic instance of $\sigma$. We changed
$\beta$ into $(\gamma \rightarrow \gamma)$, and we universally quantified
on the newly introduced type variable $\gamma$.
\End
%
We express this type system by means of {\em inference rules}. An
inference rule is written as a fraction:
\begin{itemize}
\item the numerator is called the {\em premisses};
\item the denominator is called the {\em conclusion}.
\end{itemize}
An inference rule:
\[\frac{P_1~\ldots~P_n}{C}\]
may be read in two ways:
\begin{itemize}
\item ``{\bf If} $P_1$, \ldots {\bf and} $P_n$, {\bf then} $C$''.
\item ``{\bf In order to prove} $C$, {\bf it is sufficient to prove} $P_1$, \ldots {\bf and} $P_n$''.
\end{itemize}
An inference rule may have no premise: such a rule will be called an
{\em axiom}. A complete proof will be represented by a {\em proof tree}
of the following form:
\[
\begin{array}{ccccc}
 P_1^m       &   \ldots    &              & \ldots       & P_l^k\\
 \cline{1-1}                                               \cline{5-5}\\
             &             & \vdots       &              & \\
             &             & \ldots       &              &\\
               \cline{2-2}                  \cline{4-4}  & \\
             &   P_1^1     &  \ldots      & P_n^1        &\\
                              \cline{2-4}                 \\
             &             &     C        &              &
\end{array}
\]
where the leaves of the tree ($P_1^m$, \ldots, $P_l^k$) are instances of axioms.

In the premisses and the conclusions appear {\em judgements} having the form:
\[
\Gamma \vdash e:\sigma
\]
Such a judgement should be read as ``under the typing environment
$\Gamma$, the expression $e$ has type scheme $\sigma$''. Typing
environments are sets of {\em typing hypotheses} of the form $x:\sigma$
where $x$ is an identifier name and $\sigma$ is a type scheme: typing
environments give types to the variables occurring free (i.e. unbound)
in the expression.

When typing $\lambda$-calculus terms, the typing environment is
managed as a {\em stack} (because identifiers possess local scopes).
We represent that fact in the presentation of the type system by {\em
removing} the typing hypothesis concerning an identifier name $x$ (if
such a typing hypothesis exists) before adding a new typing hypothesis
concerning $x$.

We write $\Gamma-\Gamma(x)$ for the set ot typing hypotheses obtained from
$\Gamma$ by removing the typing hypothesis concerning $x$ (if it exists).

%
\par\noindent
Any numeric constant is of type Number:
\[
\frac{}
%--------------------------------------------------%
{\Gamma \vdash \mbox{\tt Const } n : \mbox{Number}}\mbox{\qquad(NUM)}
\]
We obtain type schemes for variables from the typing environment 
$\Gamma$:
\[
\frac{}
%--------------------------------------------------%
{\Gamma \cup \{x:\sigma\} \vdash \mbox{\tt Var }x:\sigma}\mbox{\qquad(TAUT)}
\]
It is possible to instantiate type schemes.
The ``GenInstance'' relation represents generic instantiation.
\[
\frac{\Gamma \vdash e:\sigma\qquad \sigma' = \mbox{GenInstance}(\sigma)}
%------------------------------------------------------------------%
{\Gamma \vdash e:\sigma'}\mbox{\qquad(INST)}
\]
It is possible to generalize type schemes with respect to variables that do
not occur free in the set of hypotheses:
\[
\frac{ \Gamma \vdash e:\sigma \qquad \alpha \notin FV(\Gamma)}
%---------------------------------------------------------------- %
{\Gamma  \vdash e:\forall \alpha . \sigma}\mbox{\qquad(GEN)}
\]
Typing a conditional:
\[\frac{\Gamma \vdash e_1:\mbox{Number} \qquad
      \Gamma \vdash e_2:\tau \qquad
      \Gamma \vdash e_3:\tau}
%------------------------------------------------------------%
{\Gamma \vdash (\mbox{\tt if}~ e_1 ~\mbox{\tt then}~e_2~
             \mbox{\tt else}~ e_3~\mbox{\tt fi}) : \tau}\mbox{\qquad(IF)}
\]
Typing an application:
\[
\frac{\Gamma \vdash e_1: \tau \rightarrow \tau' \qquad
      \Gamma \vdash e_2: \tau}
%---------------------------------------------%
{        \Gamma \vdash (e_1~e_2) : \tau'}\mbox{\qquad(APP)}
\]
Typing an abstraction:
\[
\frac{(\Gamma-\Gamma(x)) \cup \{x:\tau\} \vdash e : \tau'}
%------------------------------------------------------------%
{\Gamma \vdash (\lambda  x~ \mbox{\tt .} e) : \tau \rightarrow \tau'}\mbox{\qquad(ABS)}
\]
%
The special rule below is the one that introduces polymorphism: this
corresponds to the ML {\tt let} construct.
\[
\frac{\Gamma \vdash e:\sigma\qquad
      (\Gamma-\Gamma(x)) \cup \{x:\sigma\}
                  \vdash e':\tau}
%------------------------------------------------------------%
{\Gamma \vdash (\lambda  x~ \mbox{\tt .} e')~ e : \tau}\mbox{\qquad(LET)}
\]
%
This type system has been proven to be {\em semantically sound}, i.e.
the semantic value of a well-typed expression (an expression admitting a
type) cannot be an {\em error value} due to a type error. This is usually expressed as:
\begin{quote}
Well-typed programs cannot go wrong.
\end{quote}
This fact implies that a clever compiler may produce code without any
dynamic type test for a well-typed expression.

\def\th{\vdash}
\def\fun{\rightarrow}
\def\infrule{\begin{array}[b]{c}}
\def\endinfrule{\end{array}}
\def\imply#1{\\[-1.2ex]\hrulefill\hbox to 0pt{~\tiny (#1)\hss}\\}
\def\topimply#1{\hrulefill\hbox to 0pt{~\tiny (#1)\hss}\\}
\def\andalso{\qquad\quad}

\Example
Let us check, using the set of rules above, that the following is true:
$$ \emptyset \th \tt let~f = \lambda x. x ~ in ~ f ~ f : \beta \fun \beta $$
In order to do so, we will use the equivalence between the {\tt let}
construct and an  application of an immediate abstraction (i.e. an 
expression having the following shape: $(\lambda v . M) N$. The (LET) rule 
will be crucial: without it, we could not check the judgement above.
{\small
$$
\begin{infrule}
\begin{infrule}
\topimply{TAUT}
\tt \{x:\alpha\} \th x : \alpha
\imply{ABS}
\tt \emptyset \th (\lambda x.x) : \alpha \fun \alpha
\imply{GEN}
\tt \emptyset \th (\lambda x.x) : \forall \alpha. \alpha \fun \alpha
\end{infrule}
\andalso
\begin{infrule}
\begin{infrule}
\topimply{TAUT}
\tt \Gamma \th f : \forall \alpha. \alpha \fun \alpha
\imply{INST}
\tt \Gamma \th f : (\beta \fun \beta) \fun (\beta \fun \beta)
\end{infrule}
\andalso
\begin{infrule}
\topimply{TAUT}
\tt \Gamma \th f : \forall \alpha. \alpha \fun \alpha
\imply{INST}
\tt \Gamma \th f : \beta \fun \beta
\end{infrule}
\imply{APP}
\tt \Gamma = \{f : \forall \alpha. \alpha \fun \alpha\} \th f~f: \beta \fun \beta
\end{infrule}
\imply{LET}
\tt \emptyset \th \tt let~f = \lambda x. x ~ in ~ f ~ f : \beta \fun \beta
\end{infrule}
$$
}
\End
This type system does not tell us how to find the best type for an
expression. But what is the best type for an expression? It must be such
that any other possible type for that expression is more specific; in
other words, the best type is the {\em most general}.

\section{The algorithm}
%

How do we find the most general type for an expression of our language?
The problem with the set of rules above, is that we could instantiate and
generalize types at any time, introducing type schemes, while the most
important rules (application and abstraction) used only types.

Let us write a new set of inference rules that we will read as an algorithm
(close to a Prolog program):
\par\noindent
Any numeric constant is of type Number:
\[
\frac{}
%--------------------------------------------------%
{\Gamma \vdash \mbox{\tt Const } n : \mbox{Number}}\mbox{\qquad(NUM)}
\]
The types of identifiers are obtained by taking generic instances of
type schemes appearing in the typing environment. These generic
instances will be {\em types} and not type schemes: this restriction
appears in the rule below, where the type $\tau$ is expected to be a
generic instance of the type scheme $\sigma$.

As it is presented (belonging to a deduction system), the following
rule will have to anticipate the effect of the equality constraints
between types in the other rules (multiple occurrences of a type
metavariable), when choosing the instance $\tau$.
\[
\frac{ \tau = \mbox{GenInstance}(\sigma)}
%---------------------------------------------------------------- %
{\Gamma \cup \{x :\sigma\} \vdash \mbox{\tt Var } x : \tau}\mbox{\qquad(INST)}
\]
When we read
this set of inference rules as an algorithm, the (INST) rule will be
implemented by:
\begin{enumerate}
\item taking as $\tau$ the ``most general generic instance'' of $\sigma$
that is a type (the rule requires $\tau$ to be a type and not a type scheme),
\item making $\tau$ more specific by {\em unification} \cite{Unification}
when encountering equality constraints.
\end{enumerate}
Typing a conditional requires only the test part to be of type Number,
and both alternatives to be of the same type $\tau$. This is an
example of equality constraint between the types of two expressions.
\[\frac{\Gamma \vdash e_1:\mbox{Number} \qquad
      \Gamma \vdash e_2:\tau \qquad
      \Gamma \vdash e_3:\tau}
%------------------------------------------------------------%
{\Gamma \vdash (\mbox{\tt if}~ e_1 ~\mbox{\tt then}~e_2~
           \mbox{\tt else}~ e_3~\mbox{\tt fi}) : \tau}\mbox{\qquad(COND)}
\]
Typing an application produces also equality constraints that are to
be solved by unification:
\[
\frac{\Gamma \vdash e_1: \tau \rightarrow \tau' \qquad
      \Gamma \vdash e_2: \tau}
%---------------------------------------------%
{        \Gamma \vdash (e_1~e_2) : \tau'}\mbox{\qquad(APP)}
\]
Typing an abstraction ``pushes'' a typing hypothesis for the
abstracted identifier: unification will make it more precise during
the typing of the abstraction body:
\[
\frac{(\Gamma-\Gamma(x)) \cup \{x:\forall.\tau\} \vdash e : \tau'}
%------------------------------------------------------------%
{\Gamma \vdash (\lambda  x~ \mbox{\tt .} e) : \tau \rightarrow \tau'}
                                                             \mbox{\qquad(ABS)}
\]
Typing a {\tt let} construct involves a generalization step: we generalize
as much as possible.
\[
\frac{
        \Gamma \vdash e:\tau' \qquad
        \{\alpha_1,\ldots,\alpha_n\} = FV(\tau')-FV(\Gamma) \qquad
      (\Gamma-\Gamma(x)) \cup \{x:\forall \alpha_1\ldots\alpha_n. \tau'\}
                  \vdash e':\tau}
%------------------------------------------------------------%
{\Gamma \vdash (\lambda  x~ \mbox{\tt .} e')~ e : \tau}\mbox{\qquad(LET)}
\]

%
This set of inference rules represents an algorithm because there is
exactly one conclusion for each syntactic ASL construct (giving priority to
the (LET) rule over the regular application rule). This set of rules
may be read as a Prolog program.

%
This algorithm has been proven to be:
\begin{itemize}
\item {\em syntactically sound}\/: when the algorithm succeeds on an
expression $e$ and returns a type $\tau$, then $e:\tau$.
\item {\em complete}\/: if an expression $e$ possesses a type $\tau$, then the
algorithm will find a type $\tau'$ such that $\tau$ is an instance of
$\tau'$. The returned type $\tau'$ is thus the most general type of $e$.
\end{itemize}

%
\Example
We compute the type that we simply checked in our last example. What is drawn
below is the result of the type synthesis: in fact, we run our algorithm with
type variables representing unknowns, modifying the previous applications of
the (INST) rule when necessary (i.e. when encountering an equality
constraint). This is valid, since it can be proved that the correction of the
whole deduction tree is preserved by substitution of types for type variables.
In a real implementation of the algorithm, the data structures representing 
types will be submitted to a unification mechanism.
{\small
$$
\begin{infrule}
\begin{infrule}
\topimply{INST}
\tt \{x:\alpha\} \th x : \alpha
\imply{ABS}
\tt \emptyset \th (\lambda x.x) : \alpha \fun \alpha
\end{infrule}
\andalso
\begin{infrule}
\begin{infrule}
\topimply{INST}
\tt \Gamma \th f : (\beta \fun \beta) \fun (\beta \fun \beta)
\end{infrule}
\andalso
\begin{infrule}
\topimply{INST}
\tt \Gamma \th f : \beta \fun \beta
\end{infrule}
\imply{APP}
\tt \Gamma = \{f : \forall \alpha. \alpha \fun \alpha\} \th f~f: \beta \fun \beta
\end{infrule}
\imply{LET}
\tt \emptyset \th \tt let~f = \lambda x. x ~ in ~ f ~ f : \beta \fun \beta
\end{infrule}
$$
}
Once again, this expression is not typable without the use of the (LET) 
rule: an error would occur because of the type equality constraints between 
all occurrences of a variable bound by a ``$\lambda$''. In an effective 
implementation, a unification error would occur.
\End
%
We may notice from the example above that the algorithm is {\em
syntax-directed}: since, for a given expression, a type deduction for
that expression uses exactly one rule per sub-expression, the
deduction possesses the same structure as the expression.  We can thus
reconstruct the ASL expression from its type deduction tree. From the
deduction tree above, if we write upper rules as being ``arguments''
of the ones below and if we annotate the applications of the (INST)
and (ABS) rules by the name of the subject variable, we obtain:
%
\[
\mbox{LET}_f(\mbox{ABS}_x(\mbox{INST}_x),
           ~\mbox{APP}(\mbox{INST}_f,~\mbox{INST}_f))
\]
This is an illustration of the ``types-as-propositions and
programs-as-proofs'' paradigm, also known as the ``Curry-Howard
isomorphism'' (cf. \cite{CurryHoward}). In this example, we can see
the type of the considered expression as a proposition and the
expression itself as the proof, and, indeed, we recognize the
expression as the deduction tree.
%
 
\section{The ASL type-synthesizer}
%

We now implement the set of inference rules given above.

We need:
\begin{itemize}
\item a Caml representation of ASL types and type schemes,
\item a management of type environments,
\item a unification procedure,
\item a typing algorithm.
\end{itemize}

\subsection{Representation of ASL types and type schemes}
%

We first need to
define a Caml type for our ASL type data structure:
\begin{caml_example}
type asl_type = Unknown
              | Number
              | TypeVar of vartype
              | Arrow of asl_type * asl_type
and vartype = {Index:int; mutable Value:asl_type}
and asl_type_scheme = Forall of int list * asl_type ;;
\end{caml_example}
The {\tt Unknown} ASL type is not really a type: it is the initial value of
fresh ASL type variables. We will consider as abnormal a situation where
{\tt Unknown} appears in place of a regular ASL type. In such situations, we
will raise the following exception:
%
\begin{caml_example}
exception TypingBug of string;;
\end{caml_example}
Type variables are allocated by the \verb"new_vartype" function, and their
global counter (a local reference) is reset by \verb"reset_vartypes".
\begin{caml_example}
let new_vartype, reset_vartypes =
(* Generating and resetting unknowns *)
    let counter = ref 0
    in (function () -> counter:=!counter+1;
                       {Index=!counter; Value=Unknown}),
       (function () -> counter:=0; ());;
\end{caml_example}

\subsection{Destructive unification of ASL types}
%

We will need to ``shorten'' type variables: since they are indirections to ASL
types, we need to follow these indirections in order to obtain the
type that they represent. For the sake of efficiency, we take
advantage of this operation to replace multiple indirections by single
indirections (shortening).
\begin{caml_example}
let rec shorten t =
    match t with
     TypeVar {Index=_; Value=Unknown} -> t
   | TypeVar ({Index=_;
                Value=TypeVar {Index=_;
                               Value=Unknown} as tv}) -> tv
   | TypeVar ({Index=_; Value=TypeVar tv1} as tv2)
            -> tv2.Value <- tv1.Value; shorten t
   | TypeVar {Index=_; Value=t'} -> t'
   | Unknown -> raise (TypingBug "shorten")
   | t' -> t';;
\end{caml_example}
An ASL type error will be represented by the following exception:
\begin{caml_example}
exception TypeClash of asl_type * asl_type;;
\end{caml_example}
%
We will need unification on ASL types with {\em occur-check}.
The following function implements occur-check:
\begin{caml_example}
let occurs {Index=n;Value=_} =
  let rec occrec = function
        TypeVar{Index=m;Value=_} -> (n=m)
      | Number -> false
      | Arrow(t1,t2) -> (occrec t1) or (occrec t2)
      | Unknown -> raise (TypingBug "occurs")
  in occrec
;;
\end{caml_example}
%
The unification function: implements destructive unification. Instead of
returning the most general unifier, it returns the unificand of two types
(their most general common instance). The two arguments are physically
modified in order to represent the same type. The unification function will
detect type clashes.
\begin{caml_example}
let rec unify (tau1,tau2) =
  match (shorten tau1, shorten tau2)
  with (* type variable n and type variable m *)
       (TypeVar({Index=n; Value=Unknown} as tv1) as t1),
       (TypeVar({Index=m; Value=Unknown} as tv2) as t2)
           -> if n <> m then tv1.Value <- t2
     | (* type t1 and type variable *)
      t1, (TypeVar ({Index=_;Value=Unknown} as tv) as t2)
            -> if not(occurs tv t1) then tv.Value <- t1
               else raise (TypeClash (t1,t2))
     | (* type variable and type t2 *)
       (TypeVar ({Index=_;Value=Unknown} as tv) as t1), t2
            -> if not(occurs tv t2) then tv.Value <- t2
               else raise (TypeClash (t1,t2))
     | Number, Number -> ()
     | Arrow(t1,t2), (Arrow(t'1,t'2) as t)
            -> unify(t1,t'1); unify(t2,t'2)
     | (t1,t2) -> raise (TypeClash (t1,t2));;
\end{caml_example}

\subsection{Representation of typing environments}
%

We use \verb"asl_type_scheme list" as typing environments, and we will use
the encoding of variables as indices into the environment.

The initial environment is a list of types %
\verb"(Number -> (Number -> Number))", which are the types of the ASL
primitive functions.
\begin{caml_example*}
let init_typing_env =
    map (function s ->
            Forall([],Arrow(Number,
                              Arrow(Number,Number))))
        init_env;;
\end{caml_example*}
%
The global typing environment is initialized to the initial
typing environment, and will be updated with the type of each ASL
declaration, after they are type-checked.
\begin{caml_example*}
let global_typing_env = ref init_typing_env;;
\end{caml_example*}

\subsection{From types to type schemes: generalization}
%

In order to implement generalization, we will need some functions collecting
types variables occurring in ASL types.

The following function computes the list of type variables of its argument.
\begin{caml_example}
let vars_of_type tau =
 let rec vars vs = function
       Number -> vs
     | TypeVar {Index=n; Value=Unknown}
                 -> if mem n vs then vs else n::vs
     | TypeVar {Index=_; Value= t} -> vars vs t
     | Arrow(t1,t2) -> vars (vars vs t1) t2
     | Unknown -> raise (TypingBug "vars_of_type")
  in vars [] tau
;;
\end{caml_example}
%
The \verb"unknowns_of_type(bv,t)" application returns the list of variables
occurring in {\tt t} that do not appear in {\tt bv}. The {\tt 
subtract} function returns the difference of two lists.
\begin{caml_example}
let unknowns_of_type (bv,t) =
    subtract (vars_of_type t) bv;;
\end{caml_example}
We need to compute the list of unknowns of a type environment for the
generalization process (unknowns belonging to that list cannot become
generic).
The set of unknowns of a  type environment is the union of the unknowns
of each type. The {\tt flat} function flattens a list of lists.
\begin{caml_example}
let flat = it_list (prefix @) [];;
let unknowns_of_type_env env =
    flat (map (function Forall(gv,t) -> unknowns_of_type (gv,t)) env);;
\end{caml_example}
%
The generalization of a type is relative to a typing environment.
The \verb"make_set" function eliminates duplicates in its list argument.
\begin{caml_example}
let rec make_set = function
     []  -> []
  | x::l -> if mem x l then make_set l else x :: make_set l;;
let generalise_type (gamma, tau) =
  let genvars =
        make_set (subtract (vars_of_type tau)
                           (unknowns_of_type_env gamma))
  in Forall(genvars, tau)
;;
\end{caml_example}

\subsection{From type schemes to types: generic instantiation}
%

The following function returns a generic instance of its type scheme argument.
A generic instance is obtained by replacing all generic type variables by new
unknowns:
\begin{caml_example}
let gen_instance (Forall(gv,tau)) = 
  (* We associate a new unknown to each generic variable *)
  let unknowns = map (function n -> n, TypeVar(new_vartype())) gv in
  let rec ginstance = function
        (TypeVar {Index=n; Value=Unknown} as t) ->
                    (try assoc n unknowns
                     with Not_found -> t)
      | TypeVar {Index=_; Value= t} -> ginstance t
      | Number -> Number
      | Arrow(t1,t2) -> Arrow(ginstance t1, ginstance t2)
      | Unknown -> raise (TypingBug "gen_instance")
  in ginstance tau
;;
\end{caml_example}

\subsection{The ASL type synthesizer}
%

The type synthesizer is the \verb"asl_typing_expr" function. Each of its
match cases corresponds to an inference rule given above.
\begin{caml_example}
let rec asl_typing_expr gamma =
  let rec type_rec = function
      Const _ -> Number
    | Var n ->
        let sigma =
          try nth n gamma
          with Failure _ -> raise (TypingBug "Unbound")
        in gen_instance sigma
    | Cond (e1,e2,e3) ->
        unify(Number, type_rec e1);
        let t2 = type_rec e2 and t3 = type_rec e3
        in unify(t2, t3); t3
    | App((Abs(x,e2) as f), e1) -> (* LET case *)
        let t1 = type_rec e1 in
          let sigma = generalise_type (gamma,t1)
        in asl_typing_expr (sigma::gamma) e2
    | App(e1,e2) ->
        let u = TypeVar(new_vartype())
        in unify(type_rec e1,Arrow(type_rec e2,u)); u
    | Abs(x,e) ->
        let u = TypeVar(new_vartype()) in
        let s = Forall([],u)
        in Arrow(u,asl_typing_expr (s::gamma) e)
  in type_rec;;
\end{caml_example}

\subsection{Typing, trapping type clashes and printing ASL types}
%

Now, we define some auxiliary functions in order to build a ``good-looking''
type synthesizer. First of all, a printing routine for ASL type schemes is
defined (using a function \verb"tvar_name" which computes a decent name for
type variables).
\begin{caml_example}
let tvar_name n =
 (* Computes a name "'a", ... for type variables, *)
 (* given an integer n representing the position  *)
 (* of the type variable in the list of generic   *)
 (* type variables                                *)
 let rec name_of n =
    let q,r = (n / 26), (n mod 26) in
    let s = make_string 1 (char_of_int (96+r)) in
    if q=0 then s else (name_of q)^s
 in "'"^(name_of n)
;;
\end{caml_example}
Then a printing function for type schemes.
%
\begin{caml_example}
let print_type_scheme (Forall(gv,t)) =
 (* Prints a type scheme.               *)
 (* Fails when it encounters an unknown *)
 let names = let rec names_of = function
                   (n,[]) -> []
                 | (n,(v1::Lv)) -> (tvar_name n)::(names_of (n+1, Lv))
             in names_of (1,gv) in
 let tvar_names = combine (rev gv, names) in
 let rec print_rec = function
      TypeVar{Index=n; Value=Unknown} ->
         let name = try assoc n tvar_names
                    with Not_found ->
                             raise (TypingBug "Non generic variable")
         in print_string name
    | TypeVar{Index=_;Value=t} -> print_rec t
    | Number -> print_string "Number"
    | Arrow(t1,t2) ->
           print_string "("; print_rec t1;
           print_string " -> "; print_rec t2;
           print_string ")"
    | Unknown -> raise (TypingBug "print_type_scheme")
 in print_rec t
;;
\end{caml_example}
%
Now, the main function which resets the type variables counter, calls the type
synthesizer, traps ASL type clashes and prints the resulting types. At the
end, the global environments are updated.
\begin{caml_example}
let typing (Decl(s,e)) =
 reset_vartypes();
 let tau = (* TYPING *)
     try asl_typing_expr !global_typing_env e
     with TypeClash(t1,t2) -> (* A typing error *)  
           let vars=vars_of_type(t1)@vars_of_type(t2) in
           print_string "ASL Type clash between ";
           print_type_scheme (Forall(vars,t1));
           print_string " and ";
           print_type_scheme (Forall(vars,t2));
           print_newline();
           raise (Failure "ASL typing") in                    
 let sigma = generalise_type (!global_typing_env,tau) in
 (* UPDATING ENVIRONMENTS *)
 global_env := s::!global_env;
 global_typing_env := sigma::!global_typing_env;
 reset_vartypes ();
 (* PRINTING RESULTING TYPE *)
 print_string "ASL Type of ";
 print_string s;
 print_string " is ";
 print_type_scheme sigma; print_newline();;
\end{caml_example}

\subsection{Typing ASL programs}
%

We reinitialize the parsing environment:
\begin{caml_example}
global_env:=init_env; ();;
\end{caml_example}
Now, let us run some examples through the ASL type checker:
\begin{caml_example}
typing (parse_top "let x be 1;");;
typing (parse_top "+ 2 ((\\x.x) 3);");;
typing (parse_top "if + 0 1 then 1 else 0 fi;");;
typing (parse_top "let id be \\x.x;");;
typing (parse_top "+ (id 1) (id id 2);");;
typing (parse_top "let f be (\\x.x x) (\\x.x);");;
typing (parse_top "+ (\\x.x) 1;");;
\end{caml_example}

\subsection{Typing and recursion}
%

The $Z$ fixpoint combinator does not have a type in Milner's type system:
\begin{caml_example}
typing (parse_top
  "let fix be \\f.((\\x.f(\\z.(x x)z)) (\\x.f(\\z.(x x)z)));");;
\end{caml_example}
This is because we try to apply {\tt x} to itself, and the type of {\tt x}
is not polymorphic.
In fact, no fixpoint combinator is typable in ASL.
This is why we need a special primitive or syntactic construct in order
to express recursivity.

If we want to assign types to recursive programs, we have to predefine
the $Z$ fixpoint combinator. Its type scheme should be
$\forall \alpha . ((\alpha \rightarrow \alpha) \rightarrow \alpha)$,
because we take fixpoints of functions.
\begin{caml_example}
global_env := "fix"::init_env;
global_typing_env:=
    (Forall([1],
     Arrow(Arrow(TypeVar{Index=1;Value=Unknown},
                   TypeVar{Index=1;Value=Unknown}),
            TypeVar{Index=1;Value=Unknown})))
   ::init_typing_env;
();;
\end{caml_example}
We can now define our favorite functions as:
\begin{caml_example}
typing (parse_top
    "let fact be fix (\\f.(\\n. if = n 0 then 1
                            else * n (f (- n 1))
                            fi));");;
typing (parse_top "fact 8;");;
typing (parse_top
  "let fib be fix (\\f.(\\n. if = n 1 then 1
                             else if = n 2 then 1
                                  else + (f(- n 1)) (f(- n 2))
                                  fi
                             fi));");;
typing (parse_top "fib 9;");;
\end{caml_example}

\chapter{Compiling ASL to an abstract machine code}
\label{c:ASLcompiling}
%

In order to fully take advantage of the static typing of ASL programs, we have
to:
\begin{itemize}
\item either write a new interpreter without type tests (difficult, because we used pattern-matching in order to realize type tests);
\item or design an untyped machine and produce code for it.
\end{itemize}
We choose here the second solution: it will permit us to give some
intuition about the compiling process of functional languages, and to
describe a typical execution model for (strict) functional languages.
The machine that we will use is a simplified version the {\em
Categorical Abstract Machine} (CAM, for short).

We will call CAM our abstract machine, despite its differences with
the original CAM. For more informations on the CAM, see
\cite{Cousineau,MaunyConfLisp}.

\section{The Abstract Machine}
%

The execution model is a {\em stack machine} (i.e. a machine using a
stack).  In this section, we define in Caml the {\em states} of the
CAM and its instructions.
%

A state is composed of:
\begin{itemize}
\item a {\em register} (holding values and environments),
\item a {\em program counter}, represented here as a list of instructions
whose first element is the current instruction being executed,
\item and a {\em stack} represented as a list of code addresses (instruction
lists), values and environments.
\end{itemize}
The first Caml type that we need is the type for CAM instructions. We
will study later the effect of each instruction.

%
\begin{caml_example}
type instruction =
  Quote of int               (* Integer constants *)
| Plus | Minus               (* Arithmetic operations *)
| Divide | Equal | Times 
| Nth of int                 (* Variable accesses *)
| Branch of instruction list * instruction list
                             (* Conditional execution *)
| Push                       (* Pushes onto the stack *)
| Swap                       (* Exch. top of stack and register *)
| Clos of instruction list   (* Builds a closure with the current environment *)
| Apply                      (* Function application *)
;;
\end{caml_example}
We need a new type for semantic values since instruction lists have now
replaced abstract syntax trees. The semantic values are merged in a
type {\tt object}.  The type {\tt object} behaves as data in a
computer memory: we need higher-level information (such as type
information) in order to interpret them. Furthermore, some data do not
correspond to anything (for example an environment composed of
environments represents neither an ASL value nor an intermediate
data in a legal computation process).
%
\begin{caml_example}
type object = Constant of int
            | Closure of object * object
            | Address of instruction list
            | Environment of object list
;;
\end{caml_example}
The type {\tt state} is a product type with mutable components.
\begin{caml_example}
type state = {mutable Reg: object;
              mutable PC: instruction list;
              mutable Stack: object list}
;;
\end{caml_example}
Now, we give the {\em operational semantics} of CAM instructions. The
effect of an instruction is to change the state configuration. This is
what we describe now with the {\tt step} function. Code executions
will be arbitrary iterations of this function.
\begin{caml_example}
exception CAMbug of string;;
exception CAM_End of object;;
let step state = match state with
  {Reg=_; PC=Quote(n)::code; Stack=s} ->
                state.Reg <- Constant(n); state.PC <- code

| {Reg=Constant(m); PC=Plus::code; Stack=Constant(n)::s} ->
                state.Reg <- Constant(n+m); state.Stack <- s;
                state.PC <- code

| {Reg=Constant(m); PC=Minus::code; Stack=Constant(n)::s} ->
                state.Reg <- Constant(n-m); state.Stack <- s;
                state.PC <- code

| {Reg=Constant(m); PC=Times::code; Stack=Constant(n)::s} ->
                state.Reg <- Constant(n*m); state.Stack <- s;
                state.PC <- code

| {Reg=Constant(m); PC=Divide::code; Stack=Constant(n)::s} ->
                state.Reg <- Constant(n/m); state.Stack <- s;
                state.PC <- code

| {Reg=Constant(m); PC=Equal::code; Stack=Constant(n)::s} ->
                state.Reg <- Constant(if n=m then 1 else 0);
                state.Stack <- s; state.PC <- code

| {Reg=Constant(m); PC=Branch(code1,code2)::code; Stack=r::s} ->
                state.Reg <- r;
                state.Stack <- Address(code)::s;
                state.PC <- (if m=0 then code2 else code1)

| {Reg=r; PC=Push::code; Stack=s} ->
                state.Stack <- r::s; state.PC <- code

| {Reg=r1; PC=Swap::code; Stack=r2::s} ->
                state.Reg <- r2; state.Stack <- r1::s;
                state.PC <- code

| {Reg=r; PC=Clos(code1)::code; Stack=s} ->
                state.Reg <- Closure(Address(code1),r);
                state.PC <- code

| {Reg=_; PC=[]; Stack=Address(code)::s} ->
                state.Stack <- s; state.PC <- code

| {Reg=v; PC=Apply::code;
          Stack=Closure(Address(code1),Environment(e))::s} ->
                state.Reg <- Environment(v::e);
                state.Stack <- Address(code)::s;
                state.PC <- code1

| {Reg=v; PC=[]; Stack=[]} ->
                raise (CAM_End v)
| {Reg=_; PC=(Plus|Minus|Times|Divide|Equal)::code; Stack=_::_} ->
                raise (CAMbug "IllTyped")

| {Reg=Environment(e); PC=Nth(n)::code; Stack=_} ->
                state.Reg <- (try nth n e
                              with Failure _ -> raise (CAMbug "IllTyped"));
                state.PC <- code
| _ -> raise (CAMbug "Wrong configuration")
;;
\end{caml_example}
We may notice that the empty code sequence denotes a (possibly final)
{\em return} instruction.

We could argue that pattern-matching in the {\tt Camlstep} function
implements a kind of dynamic typechecking. In fact, in a concrete
(low-level) implementation of the machine (expansion of the CAM
instructions in assembly code, for example), these tests would not
appear. They are useless since we trust the typechecker and the
compiler.  So, any execution error in a real implementation comes from
a {\em bug} in one of the above processes and would lead to memory
errors or illegal instructions (usually detected by the computer's
operating system).

\section{Compiling ASL programs into CAM code}
%

We give in this section a compiling function taking the abstract
syntax tree of an ASL expression and producing CAM code.  The
compilation scheme is very simple:
\begin{itemize}
\item the code of a constant is {\tt Quote};
\item a variable is compiled as an access to the appropriate component
of the current environment ({\tt Nth});
\item the code of a conditional expression will save the current environment
({\tt Push}),
evaluate the condition part, and, according to the boolean value obtained,
select the appropriate code to execute ({\tt Branch});
\item the code of an application will also save the environment on the
stack ({\tt Push}), execute the function part of the application, then exchange
the functional value and the saved environment ({\tt Swap}), evaluate the
argument and, finally, apply the functional value (which is at the top of the
stack) to the argument held in the register with the {\tt Apply} instruction;
\item the code of an abstraction simply consists in building a closure
representing the functional value: the closure is composed of the code of the
function and the current environment.
\end{itemize}
Here is the compiling function:
\begin{caml_example}
let rec code_of = function
  Const(n) -> [Quote(n)]
| Var n -> [Nth(n)]
| Cond(e_test,e_t,e_f) ->
         Push::(code_of e_test)
       @[Branch(code_of e_t, code_of e_f)]
| App(e1,e2) -> Push::(code_of e1)
               @[Swap]@(code_of e2)
               @[Apply]
| Abs(_,e) -> [Clos(code_of e)];;
\end{caml_example}
A global environment is needed in order to maintain already defined
values. Any CAM execution will start in a state whose register part
contains this global environment.
%
\begin{caml_example}
let init_CAM_env =
  let basic_instruction = function
        "+" -> Plus
      | "-" -> Minus
      | "*" -> Times
      | "/" -> Divide
      | "=" -> Equal
      | s -> raise (CAMbug "Unknown primitive")
  in map (function s ->
           Closure(Address[Clos(Push::Nth(2)
                                 ::Swap::Nth(1)
                                 ::[basic_instruction s])],
                    Environment []))
         init_env;;
let global_CAM_env = ref init_CAM_env;;
\end{caml_example}

%
As an example, here is the code for some ASL expressions.
\begin{caml_example}
code_of (expression(lexer(stream_of_string "1;")));;
code_of (expression(lexer(stream_of_string "+ 1 2;")));;
code_of (expression(lexer(stream_of_string "(\\x.x) ((\\x.x) 0);")));;
code_of (expression(lexer(stream_of_string
                "+ 1 (if 0 then 2 else 3 fi);")));;
\end{caml_example}

\section{Execution of CAM code}
%

The main function for executing compiled ASL manages the global
environment until execution has succeeded.
\begin{caml_example}
let run (Decl(s,e)) =
  (* TYPING *)
    reset_vartypes();
    let tau =
        try asl_typing_expr !global_typing_env e
        with TypeClash(t1,t2) ->
              let vars=vars_of_type(t1) @ vars_of_type(t2) in
              print_string "ASL Type clash between ";
              print_type_scheme (Forall(vars,t1));
              print_string " and ";
              print_type_scheme (Forall(vars,t2));
              raise (Failure "ASL typing")
           | Unbound s -> raise (TypingBug ("Unbound: "^s)) in
    let sigma = generalise_type (!global_typing_env,tau) in
  (* PRINTING TYPE INFORMATION *)
    print_string "ASL Type of ";
    print_string s; print_string " is ";
    print_type_scheme sigma; print_newline();
  (* COMPILING *)
    let code = code_of e in
    let state = {Reg=Environment(!global_CAM_env); PC=code; Stack=[]} in
  (* EXECUTING *)
    let result = try while true do step state done; state.Reg 
                 with CAM_End v -> v in
  (* UPDATING ENVIRONMENTS *)
    global_env := s::!global_env;
    global_typing_env := sigma::!global_typing_env;
    global_CAM_env := result::!global_CAM_env;
  (* PRINTING RESULT *)
    (match result
     with Constant(n) -> print_int n
        | Closure(_,_) -> print_string "<funval>"
        | _ -> raise (CAMbug "Wrong state configuration"));
    print_newline();;
\end{caml_example}
Now, let us run some examples:
%
\begin{caml_example}
(* Reinitializing environments *)
global_env:=init_env;
global_typing_env:=init_typing_env;
global_CAM_env:=init_CAM_env;
();;
run (parse_top "1;");;
run (parse_top "+ 1 2;");;
run (parse_top "(\\f.(\\x.f x)) (\\x. + x 1) 3;");;
\end{caml_example}
We may now introduce the $Z$ fixpoint combinator as a predefined 
function {\tt fix}.
%
\begin{caml_example}
begin
  global_env:="fix"::init_env;
  global_typing_env:=
    (Forall([1],
            Arrow(Arrow(TypeVar{Index=1;Value=Unknown},
                        TypeVar{Index=1;Value=Unknown}),
                  TypeVar{Index=1;Value=Unknown})))
   ::init_typing_env;
  global_CAM_env:=
   (match code_of (expression(lexer(stream_of_string
           "\\f.((\\x.f(\\z.(x x)z)) (\\x.f(\\z.(x x)z)));")))
    with [Clos(C)] -> Closure(Address(C), Environment [])
       | _ -> raise (CAMbug "Wrong code for fix"))
   ::init_CAM_env
end;;
\end{caml_example}
%
\begin{caml_example}
run (parse_top
    "let fact be fix (\\f.(\\n. if = n 0 then 1
                           else * n (f (- n 1))
                           fi));");;
run (parse_top
  "let fib be fix (\\f.(\\n. if = n 1 then 1
                             else if = n 2 then 1
                                  else + (f(- n 1)) (f(- n 2))
                                  fi
                             fi));");;
run (parse_top "fact 8;");;
run (parse_top "fib 9;");;
\end{caml_example}

It is of course possible (and desirable) to introduce recursion by
using a specific syntactic construct, special instructions and a
dedicated case to the compiling function. See \cite{MaunyConfLisp} for
efficient compilation of recursion, data structures etc.

\begin{exo}
Interesting exercises for which we won't give solutions consist in
enriching according to your taste the ASL language. Also, building a
standalone ASL interpreter is a good exercise in modular programming.
\end{exo}

\chapter{Answers to exercises}\label{c:ans}
We give in this chapter one possible solution for each exercise
contained in this document. Exercises are referred to by their number
and the page where they have been proposed: for example, ``2.1, p.
15'' refers to the first exercise in chapter 2; this exercise is
located on page 15.

\newcommand{\exoref}[1]{\subsection*{\ref{#1}, p. \pageref{#1}}}

\exoref{Fund:1} The following (anonymous) functions have the required types:
\begin{enumerate}
\item
\begin{caml_example}
function f -> (f 2)+1;;
\end{caml_example}
\item
\begin{caml_example}
function m -> (function n -> n+m+1);;
\end{caml_example}
\item
\begin{caml_example}
(function f -> (function m -> f(m+1) / 2));;
\end{caml_example}
\end{enumerate}

\exoref{Fund:2} We must first rename {\tt y} to {\tt z}, obtaining:
\begin{verbatim}
(function x -> (function z -> x+z))
\end{verbatim}
and finally:
\begin{verbatim}
(function y -> (function z -> y+z))
\end{verbatim}
Without renaming, we would have obtained:
\begin{verbatim}
(function y -> (function y -> y+y))
\end{verbatim}
which does not denote the same function.

\exoref{Fund:3}
We write successively the reduction steps for each expressions, and
then we use Caml in order to check the result.
\begin{itemize}
\item
\begin{verbatim}
let x=1+2 in ((function y -> y+x) x);;
(function y -> y+(1+2)) (1+2);;
(function y -> y+(1+2)) 3;;
3+(1+2);;
3+3;;
6;;
\end{verbatim}
Caml says:
\begin{caml_example}
let x=1+2 in ((function y -> y+x) x);;
\end{caml_example}
\item
\begin{verbatim}
let x=1+2 in ((function x -> x+x) x);;
(function x -> x+x) (1+2);;
3+3;;
6;;
\end{verbatim}
Caml says:
\begin{caml_example}
let x=1+2 in ((function x -> x+x) x);;
\end{caml_example}
\item
\begin{verbatim}
let f1 = function f2 -> (function x -> f2 x)
in let g = function x -> x+1
   in f1 g 2;;
let g = function x -> x+1
in function f2 -> (function x -> f2 x) g 2;;
(function f2 -> (function x -> f2 x)) (function x -> x+1) 2;;
(function x -> (function x -> x+1) x)  2;;
(function x -> x+1) 2;;
2+1;;
3;;
\end{verbatim}
Caml says:
\begin{caml_example}
let f1 = function f2 -> (function x -> f2 x)
in let g = function x -> x+1
   in f1 g 2;;
\end{caml_example}
\end{itemize}

\exoref{Basic:1} To compute the surface area of a rectangle and the volume
of a sphere:
\begin{caml_example}
let surface_rect len wid = len * wid;;
let pi = 4.0 *. atan 1.0;;
let volume_sphere r = 4.0 /. 3.0 *. pi *. (power r 3.);;
\end{caml_example}

\exoref{Basic:2} In a call-by-value language without conditional construct
(and without sum types), all programs involving a recursive definition
never terminate.

\exoref{Basic:3}
\begin{caml_example}
let rec factorial n = if n=1 then 1 else n*(factorial(n-1));;
factorial 5;;
let tail_recursive_factorial n =
  let rec fact n m = if n=1 then m else fact (n-1) (n*m)
  in fact n 1;;
tail_recursive_factorial 5;;
\end{caml_example}

\exoref{Basic:5}
\begin{caml_example}
let rec fibonacci n =
  if n=1 then 1
         else if n=2 then 1
                     else fibonacci(n-1) + fibonacci(n-2);;
fibonacci 20;;
\end{caml_example}

\exoref{Basic:6}
\begin{caml_example}
let compose f g = function x -> f (g (x));;
let curry f = function x -> (function y -> f(x,y));;
let uncurry f = function (x,y) -> f x y;;
uncurry compose;;
compose curry uncurry;;
compose uncurry curry;;
\end{caml_example}

\exoref{Lists:1}
\begin{caml_example}
let rec combine =
  function [],[] -> []
         | (x1::l1),(x2::l2) -> (x1,x2)::(combine(l1,l2))
         | _ -> raise (Failure "combine: lists of different length");;
combine ([1;2;3],["a";"b";"c"]);;
combine ([1;2;3],["a";"b"]);;
\end{caml_example}

\exoref{Lists:2}
\begin{caml_example}
let rec sublists =
    function [] -> [[]]
           | x::l -> let sl = sublists l
                     in sl @ (map (fun l -> x::l) sl);;
sublists [];;
sublists [1;2;3];;
sublists ["a"];;
\end{caml_example}

\exoref{Types:1}
\begin{caml_example}
type ('a,'b) btree = Leaf of 'b
                   | Btree of ('a,'b) node
and ('a,'b) node = {Op:'a;
                    Son1: ('a,'b) btree;
                    Son2: ('a,'b) btree};;
let rec nodes_and_leaves =
    function Leaf x -> ([],[x])
           | Btree {Op=x; Son1=s1; Son2=s2} ->
                let (nodes1,leaves1) = nodes_and_leaves s1
                and (nodes2,leaves2) = nodes_and_leaves s2
                in (x::nodes1@nodes2, leaves1@leaves2);;
nodes_and_leaves (Btree {Op="+"; Son1=Leaf 1; Son2=Leaf 2});;
\end{caml_example}

\exoref{Types:2}
\begin{caml_example}
let rec map_btree f g = function
      Leaf x -> Leaf (f x)
    | Btree {Op=op; Son1=s1; Son2=s2}
               -> Btree {Op=g op; Son1=map_btree f g s1;
                                  Son2=map_btree f g s2};;
\end{caml_example}

\exoref{Types:3}
We need to give a functional interpretation to {\tt btree} data constructors.
We use {\tt f} (resp. {\tt g}) to denote the function associated to
the {\tt Leaf} (resp. {\tt Btree}) data constructor, obtaining the
following Caml definition:
\begin{caml_example}
let rec btree_it f g = function
      Leaf x -> f x
    | Btree{Op=op; Son1=s1; Son2=s2}
              -> g op (btree_it f g s1) (btree_it f g s2)
;;
btree_it (function x -> x)
         (function "+" -> prefix +
                 | _ -> raise (Failure "Unknown op"))
         (Btree {Op="+"; Son1=Leaf 1; Son2=Leaf 2});;
\end{caml_example}

\exoref{Annot:4}
\begin{caml_example}
type ('a,'b) lisp_cons = {mutable Car:'a; mutable Cdr:'b};;
let car p = p.Car
and cdr p = p.Cdr
and rplaca p v = p.Car <- v
and rplacd p v = p.Cdr <- v;;
let p = {Car=1; Cdr=true};;
rplaca p 2;;
p;;
\end{caml_example}

\exoref{Annot:5}
\begin{caml_example}
let stamp_counter = ref 0;;
let stamp () =
  stamp_counter := 1 + !stamp_counter; !stamp_counter;;
stamp();;
stamp();;
\end{caml_example}

\exoref{Annot:6}
\begin{caml_example}
let exchange t i j =
  let v = t.(i) in vect_assign t i t.(j); vect_assign t j v
;;
let quick_sort t =
  let rec quick lo hi =
    if lo < hi
    then begin
          let i = ref lo
          and j = ref hi
          and p = t.(hi) in
          while !i < !j
           do
            while !i < hi & t.(!i) <=. p do incr i done;
            while !j > lo & p <=. t.(!j) do decr j done;
            if !i < !j then exchange t !i !j
           done;
          exchange t hi !i;
          quick lo (!i - 1);
          quick (!i + 1) hi
         end
     else ()
  in quick 0 (vect_length t - 1)
;;
let a = [| 2.0; 1.5; 4.0; 0.0; 10.0; 1.0 |];;
quick_sort a;;
a;;
\end{caml_example}

\exoref{Exc:1}
\begin{caml_example}
let rec find_succeed f = function
      [] -> raise (Failure "find_succeed")
    | x::l -> try f x; x with _ -> find_succeed f l
;;
let hd = function [] -> raise (Failure "empty") | x::l -> x;;
find_succeed hd [[];[];[1;2];[3;4]];;
\end{caml_example}

\exoref{Exc:2}
\begin{caml_example}
let rec map_succeed f = function
      [] -> []
    | h::t -> try (f h)::(map_succeed f t) 
              with _ ->  map_succeed f t;;
map_succeed hd [[];[1];[2;3];[4;5;6]];;
\end{caml_example}

\exoref{IO:1}
The first function ({\tt copy}) that we define assumes that its
arguments are respectively the input and the output channel. They are
assumed to be already opened.
\begin{caml_example}
let copy inch outch =
  (* inch and outch are supposed to be opened channels *)
  try (* actual copying *)
      while true
      do output_char outch (input_char inch)
      done
   with End_of_file -> (* Normal termination *)
             raise End_of_file
      | sys__Sys_error msg -> (* Abnormal termination *)
             prerr_endline msg;
             raise (Failure "cp")
      | _ -> (* Unknow exception, maybe interruption? *)
             prerr_endline "Unknown error while copying";
             raise (Failure "cp")
;;
\end{caml_example}
The next function opens channels connected to its filename arguments,
and calls {\tt copy} on these channels. The advantage of dividing the
code into two functions is that {\tt copy} performs the actual work,
and can be reused in different applications, while the role of {\tt
cp} is more ``administrative'' in the sense that it does nothing but
opening and closing channels and printing possible error messages.
\begin{caml_example}
let cp f1 f2 =
  (* Opening channels, f1 first, then f2 *)
  let inch =
      try open_in f1
      with sys__Sys_error msg -> 
                prerr_endline (f1^": "^msg);
                raise (Failure "cp")
         | _ -> prerr_endline ("Unknown exception while opening "^f1);
                raise (Failure "cp")
  in
  let outch =
      try open_out f2
      with sys__Sys_error msg -> 
                close_in inch;
                prerr_endline (f2^": "^msg);
                raise (Failure "cp")
         | _ -> close_in inch;
                prerr_endline ("Unknown exception while opening "^f2);
                raise (Failure "cp")
  in (* Copying and then closing *)
     try copy inch outch
     with End_of_file -> close_in inch; close_out outch
                        (* close_out flushes *)
        | exc -> close_in inch; close_out outch; raise exc
;;
\end{caml_example}
Let us try {\tt cp}:
\begin{caml_eval}
try sys__remove "/tmp/foo" with _ -> ();;
\end{caml_eval}
\begin{caml_example}
cp "/etc/passwd" "/tmp/foo";;
cp "/tmp/foo" "/foo";;
\end{caml_example}
The last example failed because a regular user is not allowed to write
at the root of the file system.
\begin{caml_eval}
try sys__remove "/tmp/foo" with _ -> ();;
\end{caml_eval}

\exoref{IO:2}
As in the previous exercise, the function {\tt count} performs the
actual counting. It works on an input channel and returns a pair of integers.
\begin{caml_example}
let count inch =
    let chars = ref 0
    and lines = ref 0 in
    try
      while true do
        let c = input_char inch in
          chars := !chars + 1;
          if c = `\n` then lines := !lines + 1 else ()
      done;
      (!chars, !lines)
    with End_of_file -> (!chars, !lines)
;;
\end{caml_example}
The function {\tt wc} opens a channel on its filename argument, calls
{\tt count} and prints the result.
\begin{caml_example}
let wc f =
    let inch =
       try open_in f
       with sys__Sys_error msg -> 
                prerr_endline (f^": "^msg);
                raise (Failure "wc")
         | _ -> prerr_endline ("Unknown exception while opening "^f);
                raise (Failure "wc")
  in let (chars,lines) = count inch
     in   print_int chars;
          print_string " characters, ";
          print_int lines;
          print_string " lines.\n"
;;
\end{caml_example}
Counting {\tt /etc/passwd} gives:
\begin{caml_example}
wc "/etc/passwd";;
\end{caml_example}

\exoref{Streams:1}
Let us recall the definitions of the type {\tt token} and of the
lexical analyzer:
\begin{caml_example}
type token =
  PLUS | MINUS | TIMES | DIV | LPAR | RPAR
| INT of int;;
(* Spaces *)
let rec spaces = function
  [< '` `|`\t`|`\n`; spaces _ >] -> ()
| [< >] -> ();;
(* Integers *)
let int_of_digit = function
  `0`..`9` as c -> (int_of_char c) - (int_of_char `0`)
| _ -> raise (Failure "not a digit");;
let rec integer n = function
  [< ' `0`..`9` as c; (integer (10*n + int_of_digit c)) r >] -> r
| [< >] -> n;;
(* The lexical analyzer *)
let rec lexer s = match s with
  [< '`(`; spaces _ >] -> [< 'LPAR; lexer s >]
| [< '`)`; spaces _ >] -> [< 'RPAR; lexer s >]
| [< '`+`; spaces _ >] -> [< 'PLUS; lexer s >]
| [< '`-`; spaces _ >] -> [< 'MINUS; lexer s >]
| [< '`*`; spaces _ >] -> [< 'TIMES; lexer s >]
| [< '`/`; spaces _ >] -> [< 'DIV; lexer s >]
| [< '`0`..`9` as c; (integer (int_of_digit c)) n; spaces _ >]
                                -> [< 'INT n; lexer s >];;
\end{caml_example}
The parser has the same shape as the grammar:
\begin{caml_example}
let rec expr = function
  [< 'INT n >] -> n
| [< 'PLUS; expr e1; expr e2 >] -> e1+e2
| [< 'MINUS; expr e1; expr e2 >] -> e1-e2
| [< 'TIMES; expr e1; expr e2 >] -> e1*e2
| [< 'DIV; expr e1; expr e2 >] -> e1/e2;;
expr (lexer (stream_of_string "1"));;
expr (lexer (stream_of_string "+ 1 * 2 4"));;
\end{caml_example}

\exoref{Streams:2}
The only new function that we need is a function taking as argument a
character stream, and returning the first identifier of that stream.
It could be written as:
\begin{caml_example}
let ident_buf = make_string 8 ` `;;
let rec ident len = function
  [< ' `a`..`z`|`A`..`Z` as c;
     (if len >= 8 then ident len
      else begin
            set_nth_char ident_buf len c;
            ident (succ len)
           end) s >] -> s
| [< >] -> sub_string ident_buf 0 len;;
\end{caml_example}
The lexical analyzer will first try to recognize an alphabetic
character $c$, then put $c$ at position 0 of \verb|ident_buf|, and
call {\tt ident 1} on the rest of the character stream.  Alphabetic
characters encountered will be stored in the string buffer
\verb|ident_buf|, up to the 8th. Further alphabetic characters will be
skipped. Finally, a substring of the buffer will be returned as
result.
\begin{caml_example}
let s = stream_of_string "toto 1";;
ident 0 s;;
(* Let us see what remains in the stream *)
match s with [< 'c >] -> c;;
let s = stream_of_string "LongIdentifier ";;
ident 0 s;;
match s with [< 'c >] -> c;;
\end{caml_example}

The definitions of the new {\tt token} type and of the lexical analyzer
is trivial, and we shall omit them. A slightly more complex lexical
analyzer recognizing identifiers (lowercase only) is given in
section~\ref{s:ASLlexing} in this part.

\exoref{Modules:1}
\begin{verbatim}
(* main.ml *)
let chars = counter__new 0;;
let lines = counter__new 0;;

let count_file filename =
  let in_chan = open_in filename in
    try
      while true do
        let c = input_char in_chan in
          counter__incr chars;
          if c = `\n` then counter__incr lines
      done
    with End_of_file ->
      close_in in_chan
;;

for i = 1 to vect_length sys__command_line - 1 do
  count_file sys__command_line.(i)
done;
print_int (counter__read chars);
print_string " characters, ";
print_int (counter__read lines);
print_string " lines.\n";
exit 0;;
\end{verbatim}



\chapter{Conclusions and further reading}

We have not been exhaustive in the description of the Caml Light features.
We only introduced general concepts in functional programming, and we
have insisted on the features used in the prototyping of ASL:
a tiny model of Caml Light typing and semantics.

The reference manual \cite{CamlLightDoc} provides an exhaustive
description of the Caml Light language, its libraries, commands and
extensions.

Those who read French are referred to \cite{Weis-Leroy}, a
progressive, but thorough introduction to programming in Caml Light,
with many interesting examples, and to \cite{Leroy-Weis}, the French
edition of the Caml Light reference manual.

Description about ``Caml Strong'' and useful information about
programming in Caml can be found in \cite{CAMLPrimer} and
\cite{CAMLRefMan}.

An introduction to lambda-calculus and type systems can be found in
\cite{Krivine}, \cite{Hindley} and \cite{Barendregt}.

The description of the implementation of call-by-value functional
programming languages can be found in \cite{ZINC}.

The implementation of lazy functional languages is described in
\cite{Peyton} (translated in French as \cite{FPeyton}).
An introduction to programming in lazy functional languages can be
found in \cite{Bird}.

\bibliographystyle{plain}
\bibliography{refs}

\end{document}
