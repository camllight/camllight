\chapter{Basic input/output}
\label{c:basicio}

We describe in this chapter the Caml Light input/output model and some
of its primitive operations. More complete information about IO can
be found in the Caml Light manual \cite{CamlLightDoc}.

Caml Light has an imperative input/output model: an IO operation
should be considered as a side-effect, and is thus dependent on the
order of evaluation. IOs are performed onto {\em channels} with types
\verb|in_channel| and \verb|out_channel|. These types are {\em
abstract}, i.e. their representation is not accessible.

Three channels are predefined:
\begin{caml_example}
std_in;;
std_out;;
std_err;;
\end{caml_example}
They are the ``standard'' IO channels: \verb|std_in| is usually
connected to the keyboard, and printing onto \verb|std_out| and
\verb|std_err| usually appears on the screen.

\section{Printable types}

It is not possible to print and read every value. Functions, for
example, are typically not readable, unless a suitable string
representation is designed and reading such a representation
is followed by an interpretation computing the desired function.

We call {\em printable type} a type for which there are input/output
primitives implemented in Caml Light. The main printable types are:
\begin{itemize}
\item characters: type \verb|char|;
\item strings: type \verb|string|;
\item integers: type \verb|int|;
\item floating point numbers: type \verb|float|.
\end{itemize}
We know all these types from the previous chapters. Strings and
characters support a notation for escaping to ASCII codes or to
denote special characters such as newline:
\begin{caml_example}
`A`;;
`\065`;;
`\\`;;
`\n`;;
"string with\na newline inside";;
\end{caml_example}
The ``\verb|\|'' character is used as an escape and is
useful for non-printable or special characters.

Of course, character constants can be used as constant patterns:
\begin{caml_example}
function `a` -> 0 | _ -> 1;;
\end{caml_example}
On types such as {\tt char} that have a finite number of constant
elements, it may be useful to use {\em or-patterns}, gathering
constants in the same matching rule:
\begin{caml_example}
let is_vowel = function
  `a` | `e` | `i` | `o` | `u` | `y` -> true
| _ -> false;;
\end{caml_example}
The first rule is chosen if the argument matches one of the cases.
Since there is a total ordering on characters, the syntax of character
patterns is enriched with a ``\verb|..|'' notation:
\begin{caml_example}
let is_lower_case_letter = function
  `a`..`z` -> true
| _ -> false;;
\end{caml_example}
Of course, or-patterns and this notation can be mixed, as in:
\begin{caml_example}
let is_letter = function
  `a`..`z` | `A`..`Z` -> true
| _ -> false;;
\end{caml_example}

In the next sections, we give the most commonly used IO primitives on
these printable types. A complete listing of predefined IO operations
is given in \cite{CamlLightDoc}.

\section{Output}

Printing on standard output is performed by the following
functions:
\begin{caml_example}
print_char;;
print_string;;
print_int;;
print_float;;
\end{caml_example}
Printing is {\em buffered}, i.e. the effect of a call to a printing
function may not be seen immediately: {\em flushing} explicitly the
output buffer is sometimes required, unless a printing function
flushes it implicitly. Flushing is done with the {\tt flush}
function:
\begin{caml_example}
flush;;
print_string "Hello!"; flush std_out;;
\end{caml_example}
The \verb|print_newline| function prints a newline character and
flushes the standard output:
\begin{caml_example}
print_newline;;
\end{caml_example}
Flushing is required when writing standalone applications, in which
the application may terminate without all printing being done.
Standalone applications should terminate by a call to the {\tt exit}
function (from the {\tt io} module), which flushes all pending output on
\verb|std_out| and
\verb|std_err|.

Printing on the standard error channel \verb|std_err| is done with the
following functions:
\begin{caml_example}
prerr_char;;
prerr_string;;
prerr_int;;
prerr_float;;
\end{caml_example}
The following function prints its string argument followed by a
newline character to \verb|std_err| and then flushes \verb|std_err|.
\begin{caml_example}
prerr_endline;;
\end{caml_example}

\section{Input}

These input primitives flush the standard output and read from the
standard input:
\begin{caml_example}
read_line;;
read_int;;
read_float;;
\end{caml_example}
Because of their names and types, these functions do not need further
explanation.

\section{Channels on files}

When programs have to read from or print to files, it is necessary to
open and close channels on these files.

\subsection{Opening and closing channels}

Opening and closing is performed with the following functions:
\begin{caml_example}
open_in;;
open_out;;
close_in;;
close_out;;
\end{caml_example}
The \verb|open_in| function checks the existence of its filename
argument, and returns a new input channel on that file;
\verb|open_out| creates a new file (or truncates it to zero length if
it exists) and returns an output channel on that file. Both functions
fail if permissions are not sufficient for reading or writing.

\noindent
{\bf Warning:}
\begin{itemize}
\item Closing functions close their channel argument. Since
their behavior is unspecified on already closed channels, anything
can happen in this case!
\item Closing one of the standard IO channels (\verb|std_in|,
\verb|std_out|, \verb|std_err|) have unpredictable effects!
\end{itemize}

\subsection{Reading or writing from/to specified channels}

Some of the functions on standard input/output have corresponding
functions working on channels:
\begin{caml_example}
output_char;;
output_string;;
input_char;;
input_line;;
\end{caml_example}

\subsection{Failures}

The exception \verb|End_of_file| is raised when an input operation
cannot complete because the end of the file has been reached.
\begin{caml_example}
End_of_file;;
\end{caml_example}

The exception \verb|sys__Sys_error| (\verb|Sys_error| from the module
{\tt sys}) is raised when some manipulation of files is forbidden by
the operating system:
\begin{caml_example}
open_in "abracadabra";;
\end{caml_example}

The functions that we have seen in this chapter are sufficient for our
needs. Many more exist (useful mainly when working with files) and are
described in \cite{CamlLightDoc}.

\section*{Exercises}

\begin{exo}\label{IO:1}
Define a function \verb|copy_file| taking two filenames (of type {\tt
string}) as arguments, and copying the contents of the first file on
the second one. Error messages must be printed on \verb|std_err|.
\end{exo}

\begin{exo}\label{IO:2}
Define a function {\tt wc} taking a filename as argument and printing
on the standard output the number of characters and lines appearing
in the file. Error messages must be printed on \verb|std_err|.
\end{exo}

\noindent
{\bf Note:} it is good practice to develop a program in defining small
functions. A single function doing the whole work is usually harder to
debug and to read. With small functions, one can trace them and see
the arguments they are called on and the result they produce.
