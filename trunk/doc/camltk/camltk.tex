\documentstyle[11pt,alltt,html,fullpage]{report}

% Various notations
\newcommand{\tcl}	{{\sf Tcl}}
\newcommand{\tk}	{{\sf Tk}}
\newcommand{\caml}	{{\sf Caml Light}}
\newcommand{\camltk}	{{\sf CamlTk}}
\newcommand{\wish}	{{\sf wish}}
\newcommand{\thisrelease}{{\tt beta1 (for 0.7)}}


\title{The \camltk\ interface \\
       Release \thisrelease}

\author{Fran\c{c}ois Pessaux and Fran\c{c}ois Rouaix \\
           Projet Cristal, INRIA Rocquencourt \\
        {\small Francois.Pessaux@inria.fr, Francois.Rouaix@inria.fr}
        }
\date{December 1994}

\begin{document}
\maketitle
\tableofcontents
\newpage

\section*{Introduction}
\addcontentsline{toc}{section}{Introduction}

\tk\ is a GUI (graphical user-interface) library for the \tcl\ language.
Normally, \tk\ is controlled from \tcl. The purpose of the \camltk\
interface is to provide a \caml\ library to control \tk\ from \caml\
programs. Thus, \tk\ can be used to program user-interfaces in \caml\
without knowledge of the \tcl\ language.

\section*{Installation}
\addcontentsline{toc}{section}{Installation}

\begin{itemize}
\item This release runs only on Unix systems.
\item You need Caml Light 0.7.
\item You need to have \tcl\ and \tk\ installed. This release requires that
you have installed the includes and libraries, and \wish.
This release has been tested with \tcl\ 7.3 and \tk\ 3.6, but {\bf not} with
beta versions of \tk\ 4.0.
\end{itemize} 

\begin{htmlonly}
Check 
\htmladdnormallink{here}
                  {http://pauillac.inria.fr/\%7Erouaix/camltk-porting.html}
for problems reported since the distribution.
\end{htmlonly}

\section*{This documentation}
\addcontentsline{toc}{section}{This documentation}

{\bf The present documentation is still sketchy, and some basic background
about toolkits, event-driven programming, and \tk\ is assumed.}
Documentation for \tk\ is currently available in \cite{ouster94},
\cite{welch94}, and in the {\tt man} pages of the \tk\ distribution.
What you will find here is:
\begin{description}
\item[The user's manual] : a short tutorial describing how to write \caml\
programs using the \tk\ interface and how to compile them. 
\item[The implementor's manual] : describing how the interface library is
generated from a high-level description. This part is of interest mainly for
expert \tk\ users who wish to extend the interface to support more
functions, more widgets, or simply to fix bugs in test versions.
\item[The Tentative reference manual] : the collection of interface files
from the library.
\end{description}

The \camltk\ distribution also includes a first shot at a \caml\ browser,
an {\em hypertext} tool that may help in finding out the types of functions
available in the library as well as the type definitions, 

\section*{Bug reports}
\addcontentsline{toc}{section}{Bug reports}

Before reporting a bug, please check the \verb|KnownBugs| file in the
\verb|doc| directory of the distribution.
Please report problems and bugs directly to \verb|Francois.Rouaix@inria.fr|
instead of \caml\ mailing lists.
Don't hesitate to make comments and suggestions, or ask questions.
Contributions (description of missing stuff) are welcome.

\section*{Acknowledgments}
Thanks to Loic Prylli \verb|<Loic.Prylli@lip.ens-lyon.fr>| \newline
      and Christophe Raffalli \verb|<raffalli@cs.chalmers.se>|
for bug reports.

\chapter{User's Manual}
{\bf This manual is a simple tutorial for \caml\tk}.  When writing your own
programs, you should consult the partial reference manual , as well as the
\tk\ man pages.  Whenever the reference manual is unsufficient, you can
check the \verb|Widgets.src| file (which syntax is given in
chapter~\ref{chap:impl}) since it describes the mappings between
\tk\ and \caml\ for functions and data. 

\section{Modules}
All modules required by the \tk\ interface are in the \verb|tklib.zo| 
library archive. These modules are
\begin{itemize}
\item \verb|tk|, containing the initialisation functions, some frequently
used functions, and all types manipulated by the library.

\item a module for each widget class (e.g. \verb|button|). The creation
function and the commands specific to this class are contained in the
module.

\item various other modules grouping functions of a same family
(\verb|focus|, \verb|winfo|). In general, you should not \verb|#open| these
files since many functions bear the same names in different modules.
Instead, you should use the \verb|modulename__symbol| syntax.

\item the \verb|textvariable| module, for the \verb|-textvariable| option
support.

\item \verb|protocol| and \verb|support|, which contain the internals
of the interface, and are of no interest to the programmer.
\end{itemize} 

\section{Basics}
As usual with toolkits, the simplest way to learn how to write programs is
by copy-paste-ing from available source code. In general, a \caml\ program
using the \tk\ toolkit will have the following structure :
\begin{verbatim}
(* Initialisation of the interface )
let top = tk__OpenTk ();;
(* top is now the toplevel widget *)
(* Widget initialisation *)
let b = button__create top ...
;;
(* Mainloop for events *)
tk__MainLoop()
;;
\end{verbatim} 

The first step is to initialize the interface with \verb|OpenTk|.
Then, the program creates some widgets, forming the user interface of your
application. Finally, the program enters the \verb|MainLoop|, and
user-interaction takes place.

\section{Example}
Here is a complete working example with a few widgets and callbacks.
This program may be found in the \verb|test| directory of the distribution,
under the name \verb|addition.ml|.

\begin{verbatim}
(* Make main functions and types available *)
#open "tk";;

let main () =
  let top = OpenTk ()  in
  (* The widgets. They all have "top" as parent widget.  *)
  let en1 = entry__create top [TextWidth 6; Relief Sunken] in
  let lab1 = label__create top [Text "plus"] in
  let en2 = entry__create top [TextWidth 6 ; Relief Sunken] in
  let lab2 = label__create top [Text "="] in
  let result_display = label__create top [] in
  (* References holding values of entry widgets *)
  let n1 = ref 0
  and n2 = ref 0  in
  (* Refresh result *)
  let refresh () =
    label__configure result_display [Text (string_of_int (!n1 + !n2))]  in
  (* Electric *)
  let get_and_refresh (w,r) =
    fun _ _ _ _ ->
      try
       r := int_of_string (entry__get w);
       refresh ()
      with
        Failure "int_of_string" ->
          label__configure result_display [Text "error"]
  in
  (* Set the callbacks *)
  entry__configure en1 [ScrollCommand (get_and_refresh (en1,n1)) ];
  entry__configure en2 [ScrollCommand (get_and_refresh (en2,n2)) ];
  (* Map the widgets *)
  pack [en1;lab1;en2;lab2;result_display] [];
  (* Make the window resizable *)
  wm__minsize_set top 1 1;
  (* Start interaction (event-driven program) *)
  MainLoop ()
;;

printexc__f main () ;;
\end{verbatim}


\section{Widgets}
The full description of widgets, their creation options, their commands, is
out of the scope of this document. The reader should refer to the \verb|man|
pages of the distribution. The commands for a given widget class are in 
a module bearing the same name as the class (i.e. \verb|button|), except
for {\em toplevel} widgets, which are defined is \verb|toplevelw| (due to a 
technical problem with the Caml Light toplevel).
As of this release, all widgets of \tk3.6 are supported, although some
commands might not be implemented. The most glaring omission is the {\tt
placer} geometry manager.

\section{Bindings}
\tk\ offers, in addition to widget-specific callbacks, a general facility
for binding functions to events. (bind(n)).
In the \caml\tk\ interface, this is implemented by the function
\begin{verbatim}
bind: Widget -> (Modifier list * XEvent) list -> BindAction -> unit
\end{verbatim} 
with the following types
\begin{verbatim}
type Modifier =
    Control
  | Shift
  | Lock
  | Button1
  | Button2
  | Button3
  | Button4
  | Button5
  | Any
  | Double
  | Triple
  | Mod1
  | Mod2
  | Mod3
  | Mod4
  | Mod5
  | Meta
  | Alt 
;;

type XEvent =
    XKey of string	(* /usr/include/X11/keysymdef.h *)
  | ButtonPress
  | Button
  | ButtonRelease
  | Circulate
  | CirculateRequest
  | ColorMap
  | Configure
  | ConfigureRequest
  | Destroy
  | Enter
  | Expose
  | FocusIn
  | FocusOut
  | Gravity
  | Keymap
  | KeyPress
  | Key
  | KeyRelease
  | MapRequest
  | Motion
  | Leave
  | Map
  | Property
  | Reparent
  | ResizeRequest
  | Unmap
  | Visibility 
  | WhatButton of int
;;

type EventField =
    Above
  | ButtonNumber
  | Count
  | Detail
  | Focus
  | Height
  | KeyCode
  | Mode
  | State
  | ValueMask
  | Width
  | MouseX
  | MouseY
  | WidgetName
  | Time 
  | OverrideRedirect
  | Place
  | Char
  | BorderWidth
  | Display
  | SendEvent
  | KeySymString
  | KeySymInt
  | RootWindow
  | SubWindow
  | Type
  | RootX
  | RootY
;;

type EventInfo = {
  mutable Ev_Above : int;
  mutable Ev_ButtonNumber : int;
  mutable Ev_Count : int;
  mutable Ev_Detail : string;
  mutable Ev_Focus : bool;
  mutable Ev_Height : int;
  mutable Ev_KeyCode : int;
  mutable Ev_Mode : string;
  mutable Ev_State : string;
  mutable Ev_ValueMask : int;
  mutable Ev_Width : int;
  mutable Ev_MouseX : int;
  mutable Ev_MouseY : int;
  mutable Ev_WidgetName : string;
  mutable Ev_Time : string;
  mutable Ev_OverrideRedirect : string;
  mutable Ev_Place : string;
  mutable Ev_Char : string;
  mutable Ev_BorderWidth : int;
  mutable Ev_Display : int;
  mutable Ev_SendEvent : int;
  mutable Ev_KeySymString : string;
  mutable Ev_KeySymInt : int;
  mutable Ev_RootWindow : int;
  mutable Ev_SubWindow : int;
  mutable Ev_Type : int;
  mutable Ev_RootX : int;
  mutable Ev_RootY : int }
;;

type BindAction =
   BindSet of EventField list *  (EventInfo -> unit)
 | BindRemove
;;
\end{verbatim} 
The first argument of type \verb|Widget| is naturally the widget for which
this binding is defined (there is no support yet for bindings on whole
widget classes).
The second argument of type \verb|(Modifier list * XEvent) list| is the 
succession of events that will trigger the callback.
The last argument is the action : \verb|BindRemove| will remove a binding.
\verb|BindSet| will set a binding.
The argument of \verb|BindSet| are an \verb|EventField list| and the
callback itself. 
In bindings, the callback needs some information about the event that
triggered it. This information is contained in the \verb|EventInfo| passed
to the callback, but the user has to specify which fields in this record are
relevant for this particular callback. The other fields do {\bf not} contain
accurate information. Check also the \tk\ documentation for finding out
which field is valid for a given event.

\paragraph{Example}
\begin{verbatim}
bind MyCan [[],Motion] 
   (BindSet([MouseX; MouseY;Place;SendEvent;KeySymString], BindCmd)) ;;
\end{verbatim} 
will trigger the callback \verb|BindCmd| when the event \verb|Motion| occurs
in the \verb|MyCan| widget. The command \verb|BindCmd| may assume that the 
information in the fields {\tt Ev\_MouseX, Ev\_MouseY, Ev\_Place, Ev\_Place,
Ev\_SendEvent, Ev\_KeySymString} of its \verb|EventInfo| argument contain
accurate information.

Bindings for \verb|text| and \verb|canvas| widgets obey the same rules,
and the functions are 
\begin{verbatim}
tk__text_tag_bind : Widget -> TextTag -> 
      (Modifier list * XEvent) list -> BindAction -> unit
tk__canvas_bind : Widget -> TagOrId -> 
      (Modifier list * XEvent) list -> BindAction -> unit
\end{verbatim} 
Additional arguments are respectively the {\tt tag} and the {\tt tag or id}
for which the binding is defined.

\section{Exiting properly}
You should always try to call \verb|CloseTk| before exiting. Otherwise, the
slave {\tt wish} may stay around for a while, as well as named pipes created
by the communication protocol in \verb|/tmp|. In particular, something like 
\begin{verbatim}
signal SIGINT (Signal_handle (function _ -> CloseTk(); exit 0))
\end{verbatim} 
should help your application behave nicely. Naturally your own cleanup
should be inserted also.

\section{More examples}
The \verb|test| directory of the distribution contains several other
examples. Read them to learn more about \caml\tk\ programming.
The \verb|tktop| directory contains a more advanced example, although
nor very robust.

{\bf All examples included in this distribution  should not be considered as
working applications. They are provided for documentation purposes only}.

\section{The Caml Browser}
In the directory {\tt browser} you fill find a yet more complete example,
featuring many different kinds of widgets. The browser mimicks the Caml
Light toplevel behaviour: you can add open and close modules ({\tt \#open} and
{\tt \#close} directives), or add a library path ({\tt \#directory} directive).
You can enter a symbol (value, type constructor, value constructor,
exception), and see its definition. You can also see the complete interface
of a module by double-clicking on it.
The browser works on compiled interfaces ({\tt .zi} files) only, not on
source files.

\section{Compilation}
The usual commands for compiling programs using the \caml\tk interface are
\begin{verbatim}
$ camlc -c -I $CAMLTKDIR addition.ml
$ camlc -custom -I $CAMLTKDIR -o addition unix.zo tklib.zo addition.zo -lunix
\end{verbatim} 
The \verb|CAMLTKDIR| environment variable should point to the directory where
the interface library has been installed, that is normally
\verb|/usr/local/lib/caml-light/tk|. \verb|unix.zo| comes from the
\verb|libunix| contrib, and is normally installed in the same directory as
standard \caml\ libraries. 

The distribution \verb|Makefile| also contains an entry to produce a \caml\
toplevel featuring the \tk\ interface : \verb|camltktop|.
This toplevel may be run with: 
\begin{verbatim}
$ camllight -I $CAMLTKDIR $CAMLTKDIR/camltktop
\end{verbatim} 

\section{Extensions}
The alpha versions have already one extension (i.e. a feature not
normally available in \tk\ 3.6). This extension allows the association of
callbacks to Unix input file descriptors. 

\subsection{File descriptor callbacks}
A callback can be associated to a file descriptor using the following
primitive:
\begin{verbatim}
tk__add_fileinput : file_descr -> (unit -> unit) -> unit
\end{verbatim} 
When some input is available on the file descriptor specified by the first
argument, then the callback (second argument) is called.
\begin{verbatim}
tk__remove_fileinput : file_descr -> unit
\end{verbatim} 
Removes the file descriptor callback. It is the programmer's responsability
to check for end of file, and to remove the callback if the file descriptor
is closed.

\section{Translating \tk\ idioms}
If you are a seasoned \tk\ programmer, you will find out that some \tk\
idioms have a different form in \camltk\ .

\subsection{Widgets}
First of all, widget creation is more ``functional''. One does not specify
the name of the created widget (except for \verb|toplevel| widgets), but
only its parent. The name is allocated by the library, and a handle to the
widget is returned to you. Then, widgets are not ``procedures''. They are
objects, and must be passed to widget manipulation functions.
For example,
\begin{verbatim}
button .myf.bok -title "Ok" -relief raised
\end{verbatim} 
is translated by
\begin{verbatim}
let b = button__create myf [Title "Ok"; Relief Raised] in
...
\end{verbatim} 
assuming that \verb|myf| is the parent widget \verb|.myf|.
Then, 
\begin{verbatim}
.myf.bok configure -state disabled
\end{verbatim} 
would be in \caml:
\begin{verbatim}
button__configure b [State Disabled]
\end{verbatim} 

When widgets are mutually recursive (through callbacks, for example when
linking a scrollbar to a text widget), one should first create the widgets,
and then set up the callbacks by calling \verb|configure| on the widgets.
Otherwise, partially applied widget commands (such as redisplay commands)
translate quite well to \caml, with possibly some wrapping required due to
value constructors.

\subsection{Text variables}
\verb|TextVariables| are now available in \caml\tk. However,
\verb|textvariables| are not physically shared between \caml\ and \tk.
Instead, setting or consulting a \verb|textvariable| requires a transaction
between \caml\ and \tk. However, if the \verb|textvariable| is used to
physically share the content of an {\tt entry} with the text of a {\tt
label}, then this sharing will effectively occur in the \tk\ world, and
\caml\ is not concerned with the actual contents of the \verb|textvariable|.

\subsection{How to find out the \caml\ name of \tk\ keywords and functions}
Since this documentation is not a reference manual, you have unfortunately
to make some efforts to find out what functions are available, and how the
datatypes are defined. For the time being the best methods are:
\begin{itemize}
\item Check the .mli files in the library directory (they also appear in
chapter~\ref{chap:ref}). This will give you at
least the functions for each widget class.
\item The types are almost all defined in the \verb|tk| module. Instead of
looking at the source, you can use the {\tt mletags} utility in conjunction
with the {\tt tags} mechanism of Emacs. 
\item Use the browser
\end{itemize} 

\section{Debugging}
Since this is an alpha release, some \tk\ functions may have been improperly
implemented in the \caml\ library. This may cause undue \tk\ errors
(exception \verb|TkError|). To facilitate the debugging, you can set the
Unix environment variable \verb|CAMLTKDEBUG| to any value before launching
your program. This will allow you to see all
data transferred between the \caml\ and the \tk\ processes. Since this data is
essentially \tcl\tk\ code, you need a basic knowledge of this language to
understand what is going on.
It is also possible to trigger debugging by setting the boolean reference
\verb|protocol__debug| to true (or false to remove debugging).



\chapter{Implementor's Manual}
\label{chap:impl}
\tk\ is still evolving, and also offers numerous user-contributed
extensions. Thus, we chose to generate the library from a high-level
description of available widgets and functions. 
We describe in this chapter
the syntax and semantics of the description language.

\section{The interface description language}
The language allows the specification of
\begin{description}
\item[widgets] of the standard \tk\ library, with options and associated commands
\item[toplevel functions,] not associated with a specific widget class
\item[modules,] grouping families of functions.
\end{description} 

A source file is simply a collection of entries, each entry describing an
element of the interface. The {\tt tkcompiler} provided in this distribution
takes the source file and produces \caml\ modules forming the interface
library.

\section{Types}
A type entry contains information to translate data from \caml\ to \tk\ 
and the converse. Each data type that needs to be exchanged by \caml\ and
\tk\ must be described by a type entry. For each type, a conversion function
from \caml\ to \tk\ is generated, as well as a parsing function from \tk\ to
\caml\ if required (because values of the type are returned by some
function in the library).

A type is given by a set of value constructors, as in ML. For each
constructor, one must give the ML name, and optionally
\begin{itemize}
\item the \tk\ name of the constructor
\item the type of its arguments
\end{itemize} 

\paragraph{Example}
\begin{verbatim}
type State { 
   Normal "normal"
   Active "active"
   Disabled "disabled"
}
\end{verbatim} 
defines the \caml\ type
\begin{verbatim}
type State =
          Disabled
        | Active
        | Normal
;;
\end{verbatim} 
and the conversion function
\begin{verbatim}
let CAMLtoTKState = function
          Disabled -> "disabled"
        | Active -> "active"
        | Normal -> "normal"
;;
\end{verbatim} 

The \tk\ source-code produced from the ML value is the \tk\ name (if any)
followed by the \tk\ version of the argument (if any).
The parser can be generated only in restricted cases: zeroary constructors,
and at most one anonymous \verb|int| constructor and one anonymous
\verb|string| constructor (anonymous means that the \tk\ name is an empty
string). Otherwise, the compiler will produce a warning
message, and you have to write the parser yourself. 

\section{Subtypes}
As customary, types will be statically verified in ML programs. However, in
order to reduce the number of value constructors required by the interface
(and associated naming problems),
we introduced a notion of {\em subtypes}. A subtype is a named subset of the
set of constructors of a type. In this case, type-checking (i.e. verifiying
that a constructor belongs to a subtype) will occur at run-time.

If a type requires subtyping, then one should not declare the type itself,
but instead define each of its subtypes. The compiler will consider the
whole set of constructors for the definition of the type. Note also that the
definition of a constructor may be omitted if it has  already been declared.

\paragraph{Example}
\begin{verbatim}
subtype Index(entry) {
   Number (int)
   End "end"
   Insert "insert"
   SelFirst "sel.first"
   SelLast "sel.last" 
}
subtype Index(listbox) {
   Number End Insert 
}
\end{verbatim}
defines two subtypes (\verb|entry| and \verb|listbox|) of the type
\verb|Index|.  

Subtypes may be referred to (in argument types of functions) with the same
syntax: \verb|type(subtype)|.
\paragraph{Example}
\begin{verbatim}
   function () delete "delete" (Index(listbox), Index(listbox))
\end{verbatim} 

\section{Widgets}
A widget class description is composed of its name, the set of valid options
for the creation of widgets of this class, and the set of functions and
commands associated to the class.

Options descriptions follow the same syntax as value constructors in type or
subtype declarations, except for the presence of the \verb|option| keyword.
Actually, the options declared inside a widget description form a subtype
(with the name of the class) of the \verb|option| type. 
\verb|option| is not really hardwired in the compiler, in that it obeys the
same rules for subtyping as user-defined types. However the \verb|option|
type is implicitly used when producing the widget creation functions.

Function declarations are formed of
\begin{itemize}
\item the \verb|function| keyword
\item the result type of the function (inside parenthesis)
\item the ML name of the function
\item the corresponding \tk\ name
\item the type of the arguments (inside parenthesis)
\end{itemize} 

The ML function produced from this description always takes a widget as
first argument. The function is curryfied, although the notation seems to
imply that the function has only one tuple argument.


\paragraph{Example}
\begin{verbatim}
widget message {
   option Anchor
   option Background
   option Borderwidth
   option Cursor
   option Font
   option Foreground
   option PadX
   option PadY
   option Relief
   option Text
   option Width

   option Aspect "-aspect" (int)
   option Justify "-justify" (Justification)

   function () configure "configure" (option(message) list)
   function (string) configure_get "configure" ()
   }
\end{verbatim}
Note that the first options are given implicitly, since they belong to
``standard'' options, defined elsewhere. The two others are given in their
full form. 

Each widget class is compiled into a separate module (bearing the name of the
widget class). Besides the functions described in the entry, the compiler
produces a creation function of type \verb|Widget -> option list -> Widget|
(with, in fact, run-time verification of options, who must belong to the
subtype \verb|option(widgetname)|. The first argument is the parent of the
created widget.


\section{Modules}
A module is simply a set of functions to be grouped in a separate module.
\paragraph{Example}
\begin{verbatim}
module selection {
   function () clear "selection clear" (widget)
   function (string) get "selection get" ()
   }
\end{verbatim} 

\section{Builtins}
Some datatypes cannot be described in the syntax of the interface
description language, because they require custom pretty-printers and
parsers. Thus it is possible to write them directly in \caml\ and place
them in a \verb|builtin_*.ml| file.
To avoid a warning message from the compiler, one may also declare the type
as external
\paragraph{Example}
\begin{verbatim}
type Units external     # builtin_GetPixel
\end{verbatim} 
For each builtin type \verb|foo|, one should provide a \verb|CAMLtoTKfoo|
function, of type \verb|foo -> string|, and, if data of this type is to be
returned by a function, a parser \verb|TKtoCAMLfoo|.
For example, the file \verb|builtin_GetPixel.ml| contains the \verb|Units|
type, used extensively in \tk\ for specifying distances or coordinates.
It requires both a pretty-printer and a parser (which emulates the
\verb|tkGetPixel| function in the \tk\ library).


\section{Compiling}
The interface description source is compiled with
\begin{verbatim}
$ tkcompiler filename
\end{verbatim}
(the file compiled defaults to \verb|Widgets.src| if \verb|file| is
omitted).
The compiler requires the existence of a \verb|lib| subdirectory, where it
will produce \verb|tkgen.ml|, \verb|modules|, and various \caml\ files
corresponding to entries in the interface description source.

The compiler will report the following errors
\begin{description}
\item[lexical errors]
\item[syntax errors] see the grammar of the language below
\item[duplicate definitions]
\item[illegal implicit use of constructors]
\item[cyclic dependancy on types]
\end{description} 

Producing the library requires the presence of several files in the
\verb|lib| directory. For more details, check the \verb|Makefile| in
\verb|lib|. 

\section{The architecture}
In this version (\thisrelease), the interface is implemented as a
communication protocol between a \caml\ process and the {\tt wish}
interpreter ({\tt wish} is a toplevel interactive interpreter for 
\tcl\ and \tk).
{\bf This implementation is subject to change.}
A \caml\ function of the library produces \tcl\tk\ code, and sends it to the
\wish\ interpreter. \wish\ sends callback information and results from
function calls to the \caml\ process.
The support layer is implemented in the \verb|protocol.ml| and
\verb|support.ml| files.



% Not included in html version, by command line option
\chapter{Reference Manual (incomplete)}
\label{chap:ref}
\section*{Module {\tt tk}}
Builtin functions are:
\begin{alltt}
\input{tkbuiltin.mli}
\end{alltt}

\noindent
The public functions generated from {\tt Widgets.src} are:
\begin{alltt}
\input{../lib/tkgen.mli}
\end{alltt}

%% The generated modules
\include{inclmod}

%% Other builtin modules
\section*{Module {\tt textvariable}}
\begin{alltt}
\input{../libsupport/textvariable.mli}
\end{alltt}


\begin{htmlonly}
\chapter{Reference manual}
{\Large WARNING: The on-line reference manual corresponds to the version
being developped NOW. It may not be accurate for the last released version}

\htmladdnormallink{This}{/bin/psearch/doc-camltk/library}
is the on-line version.
\end{htmlonly}

\bibliography{tk}
\bibliographystyle{plain}
\appendix
\chapter*{BNF syntax of the interface description language}
\begin{verbatim}
Type0 :
      TYINT
    | TYFLOAT
    | TYBOOL
    | TYCHAR
    | TYSTRING
    | WIDGET
    | IDENT

Type01 :
       Type0
     | IDENT LPAREN IDENT RPAREN
     | WIDGET LPAREN IDENT RPAREN
     | OPTION LPAREN IDENT RPAREN

Type1 :
      Type01
    | Type01 LIST

Type1list :
      Type1 COMMA Type1list
    | Type1

Typearg :
      LPAREN RPAREN
    | LPAREN Type1 RPAREN
    | LPAREN Type1list RPAREN

Type :
     Typearg
   | LPAREN FUNCTION Typearg RPAREN

Constructor :
     IDENT STRING
   | IDENT Type
   | IDENT STRING Type

AbbrevConstructor :
     Constructor
   | IDENT

Constructors :
     Constructor Constructors
   | Constructor

AbbrevConstructors :
     AbbrevConstructor AbbrevConstructors
   | AbbrevConstructor

Command :
     FUNCTION Typearg IDENT STRING Type

Option :
     OPTION IDENT STRING Type
   | OPTION IDENT

WidgetComponent :
     Command
   | Option

WidgetComponents :
     /* epsilon */
   | WidgetComponent WidgetComponents

ModuleComponents :
     /* epsilon */
   | Command ModuleComponents

Entry :
     WIDGET IDENT LBRACE WidgetComponents RBRACE
   | Command
   | TYPE IDENT LBRACE Constructors RBRACE
   | TYPE IDENT EXTERNAL
   | SUBTYPE OPTION LPAREN IDENT RPAREN LBRACE AbbrevConstructors RBRACE
   | SUBTYPE IDENT LPAREN IDENT RPAREN LBRACE AbbrevConstructors RBRACE  
   | MODULE IDENT LBRACE ModuleComponents RBRACE
   | EOF
\end{verbatim} 

\end{document}

