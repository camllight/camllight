\chapter{Implementor's Manual}
\label{chap:impl}
\tk\ is still evolving, and also offers numerous user-contributed
extensions. To ensure the compatibility of \camltk\ with this evolution,
we chose to divide the library in two parts:
\begin{itemize}
\item a protocol layer for communication between \caml\ and \tcl,
\item implementation of \tk\ commands generated from  a high-level
description of available widgets and functions.
\end{itemize} 

This chapter defines the syntax and semantics of the \camltk\ description
language, as well as some technical details of the protocol.

\section{The \camltk\ description language}
The language allows the specification of
\begin{description}
\item[widgets] of the standard \tk\ library, with options and associated
commands,
\item[functions] not associated with a specific widget class
\item[modules] grouping families of functions.
\end{description} 

A source file is simply a collection of entries, each entry describing an
element of the interface. The {\tt tkcompiler} provided in this distribution
takes the source file and produces \caml\ modules forming the interface
library.

\subsection{Tk tokens}
{\bf Note: this is new to \thisrelease, and reflects a complete rewrite of
the low level communication between \caml\ and \tcl\tk.}

In the heart of the interface, \caml\ needs to call \tcl\tk\ procedures with
arguments. Essentially, arguments are lists of tokens (strings). In some
cases an argument will itself be a list of strings. It is mandatory that the
descriptions of types and procedure respects the ``tokenization'' of
arguments. 

The syntax for describing translations respecting tokens is called a {\em
template}. A {\em template} is a list of tokens, each token being
\begin{description}
\item[a string,] e.g. \verb|"foobar"|, used as is by \tk.
\item[a type,] e.g. Color. The translation of a value of this type may
result in several tokens
\item[a list of tokens]
\end{description} 

\subsection{Types}
A type entry contains information to translate data from \caml\ to \tk\ 
and the converse. Each data type that needs to be exchanged by \caml\ and
\tk\ must be described by a type entry. For each type, a conversion function
from \caml\ to \tk\ is generated, as well as a parsing function from \tk\ to
\caml\ if required (because values of the type are returned by some
function in the library) and feasible.

A type is given by a set of value constructors, as in \caml. For each
constructor, one must give the \caml\ name, and its template.
\paragraph{Example}
\begin{verbatim}
type State { 
   Normal ["normal"]
   Active ["active"]
   Disabled ["disabled"]
}
\end{verbatim} 
defines the \caml\ type
\begin{verbatim}
type State =
          Disabled
        | Active
        | Normal
;;
\end{verbatim} 
and the conversion function
\begin{verbatim}
let CAMLtoTKState = function
	Disabled -> TkToken"disabled"
	| Active -> TkToken"active"
	| Normal -> TkToken"normal"
;;
\end{verbatim} 

The parser can be generated only in restricted cases: zeroary constructors,
and at most one anonymous \verb|int| constructor and one anonymous
\verb|string| constructor (anonymous means that the template is composed
only with this token). Otherwise, the compiler will produce a warning
message, and you have to write the parser yourself. 

\subsection{Subtypes}
As customary, types will be statically verified in \caml\ programs. However, in
order to reduce the number of value constructors required by the interface
(and associated naming problems),
we introduced a notion of {\em subtypes}. A subtype is a named subset of the
set of constructors of a type. In this case, type-checking (i.e. verifiying
that a constructor belongs to a subtype) will occur at run-time.

If a type requires subtyping, then one should not declare the type itself,
but instead define each of its subtypes. The compiler will consider the
whole set of constructors for the definition of the type. Note also that the
definition of a constructor may be omitted if it has  already been declared.

\paragraph{Example}
\begin{verbatim}
subtype option(standard) {
   ActiveBackground		["-activebackground"; Color]
   ActiveBorderWidth		["-activeborderwidth"; Units]
   ActiveForeground		["-activeforeground"; Color]
   Anchor			["-anchor"; Anchor]
   ...
   }
subtype option (bitmap) {
   Anchor
   Background
   ...
   }
\end{verbatim}
defines two subtypes (\verb|standard| and \verb|bitmap|) of the type
\verb|option|.

Subtypes may be referred to (in argument types of functions) with the same
syntax: \verb|type(subtype)|.
\paragraph{Example}
\begin{verbatim}
   function () delete [widget(listbox); "delete"; Index(listbox); Index(listbox)]
\end{verbatim} 

\subsection{Widgets}
A widget class description is composed of its name, the set of valid options
for the creation of widgets of this class, and the set of functions and
commands associated to the class.

Options descriptions follow the same syntax as value constructors in type or
subtype declarations, except for the presence of the \verb|option| keyword.
Actually, the options declared inside a widget description form a subtype
(with the name of the class) of the \verb|option| type. 
\verb|option| is not really hardwired in the compiler, in that it obeys the
same rules for subtyping as user-defined types. However the \verb|option|
type is implicitly used when producing the widget creation functions.

Function declarations are formed of
\begin{itemize}
\item the \verb|function| keyword
\item the result type of the function (inside parenthesis)
\item the \caml\ name of the function
\item the template of arguments
\end{itemize} 

The \caml\ function will have as many arguments as types appearing in the
template, in the same order.

\paragraph{Example}
\begin{verbatim}
widget message {
   option Anchor
   option Background
   option Borderwidth
   option Cursor
   option Font
   option Foreground
   option PadX
   option PadY
   option Relief
   option Text
   option TextVariable
   option Width

   option Aspect ["-aspect"; int]
   option Justify ["-justify"; Justification]

   function () configure [widget(message); "configure"; option(message) list]
   function (string) configure_get [widget(message); "configure"]
   }
\end{verbatim}
Note that the first options are given implicitly, since they belong to
``standard'' options, defined elsewhere. The two others are given in their
full form. 

Each widget class is compiled into a separate module (bearing the name of the
widget class). Besides the functions described in the entry, the compiler
produces two creation functions
\begin{verbatim}
value create : Widget -> option list -> Widget ;;
value create_named : Widget -> string -> option list -> Widget ;;
\end{verbatim} 
with, in fact, run-time verification of options, who must belong to the
subtype \verb|option(widgetname)|. The first argument of \verb|create| is
the parent of the created widget. The second argument of \verb|create_named|
is the new path component of the widget.


\subsection{Modules}
A module is simply a set of functions to be grouped in a separate module.
\paragraph{Example}
\begin{verbatim}
module selection {
   function () clear ["selection"; "clear"; widget]
   function (string) get ["selection"; "get"]
   }
\end{verbatim} 

\subsection{Builtins}
Some datatypes cannot be described in the syntax of the interface
description language, because they require custom converters. Thus it is
possible to write them directly in \caml\ and place 
them in a \verb|builtin_*.ml| file.
To avoid a warning message from the compiler, one may also declare the type
as external
\paragraph{Example}
\begin{verbatim}
type Units external     # builtin_GetPixel
\end{verbatim} 
For each builtin type \verb|foo|, one should provide a \verb|CAMLtoTKfoo|
function, of type \verb|foo -> protocol__TkArgs|, respecting the
tokenization. If data of this type is to be returned by a function, one
should also write a parser \verb|TKtoCAMLfoo|.
For example, the file \verb|builtin_GetPixel.ml| contains the \verb|Units|
type, used extensively in \tk\ for specifying distances or coordinates.
It requires both a pretty-printer and a parser (which emulates the
\verb|tkGetPixel| function in the \tk\ library).


\section{Compiling}
The interface description source is compiled with
\begin{verbatim}
$ tkcompiler filename
\end{verbatim}
(the file compiled defaults to \verb|Widgets.src| if \verb|filename| is
omitted).
The compiler requires the existence of a \verb|lib| subdirectory, where it
will produce \verb|tkgen.ml|, \verb|modules|, and various \caml\ files
corresponding to entries in the interface description source.

The compiler will report the following errors
\begin{description}
\item[lexical errors]
\item[syntax errors] see the grammar of the language below
\item[duplicate definitions]
\item[illegal implicit use of constructors]
\item[cyclic dependancy on types]
\end{description} 

Producing the library requires the presence of several files in the
\verb|lib| directory. For more details, check the \verb|Makefile| in
\verb|lib|. 

\section{Run-time support}
Apart from the modules produced by compiling the widget description file,
the \camltk\ interface uses low-level support for some basic data types and
communication protocol between \caml\ and \tcl.

\subsection{Widget support}
\begin{description}
\item[Widget naming] :
in the contrary of the {\sf Tcl/Tk} approach, \camltk\ keeps the naming of
widgets internal to the library. Moreover, there is no particular data
associated to widgets in the \caml\ half of the world. A widget is
essentially its \tk\ path. The public functions are: 

\begin{verbatim}
value default_toplevel_widget : Widget
      	(* The default toplevel widget is ".", of type "toplevel" *)
;;
value new_toplevel_widget : string -> Widget
       (* the argument must be a valid window name: no 8bit char, start
          with a lowercase, ... *)
;;
\end{verbatim} 
And some functions used internally are
\begin{verbatim}
value widget_name : Widget -> string
      (* Return the name (tk "path") of a widget *)
and  widget_class : Widget-> string
      (* Returns the class of a widget *)
;;
value new_widget_atom : string -> Widget -> Widget
and   new_named_widget : string -> Widget -> string -> Widget
      (* Abstract creation functions *)
;;
\end{verbatim} 

\item[Widget typing] :
\camltk\ attempts to check widget types before sending code to \tcl.
We chose to have a unique \caml\ type for widgets, \verb|Widget|, and
to check dynamically when necessary if a widget is of the required class.
Thus, internally, 
\begin{verbatim}
type Widget =
  Untyped of string
| Typed of string * string
;;
\end{verbatim} 
We also maintain a table associating {\em widget paths} to {\em Widgets}.
The reason for this table is that we may have to dynamically retype a widget
path returned by a \tcl\ function.

\end{description}


\subsection{Transferring control}
\caml\ has been extended with primitives for evaluating \tcl\ procedures.
\begin{description}
\item[tcl\_eval] evaluates a \tcl\ source phrase. The implementation of
\verb|tcl_eval| is straightforward.
During this evaluation, \tcl\ will perform all customary substitutions and
argument parsing. Thus, one needs to be careful with \tcl\ special
characters. This function is deprecated, and {\tt tcl\_direct\_eval} is
preferred. 

\item[tcl\_direct\_eval] evaluates a \tcl\ procedure call. It takes as
argument a value of type {\tt TkArgs vect}, which could be thought of as
an {\tt argv} parameter. Note however that the vector may still contain
token lists that have to be expanded.
\end{description} 

\tcl\ has been extended with a new command \verb|camlcb| to call \caml\ (or
more precisely a given \caml\ function). 
\verb|camlcb| relies on the \verb|callback| primitive of the \caml\
runtime. Currently, the \caml\ runtime does not allow storage of \caml\
values in C global variables, because global variables cannot be roots for
the garbage collector. The implementation of \verb|camlcb| uses the same
trick as signal handling.

\subsection{Other \camltk\ primitives}
The other primitives, written in C, added to the \caml\ runtime deal with
\begin{description}
\item[initialising a \tcl\ interpreter] :  straightforward
\item[file descriptor callbacks] : the mechanism of file descriptor
callbacks is available in \tk, although not as \tcl\ commands. We simply
provide a \caml\ primitive to deal with this form of callbacks.
\end{description} 

\subsection{Transferring data}
We provide {\em extensible buffers} to simplify the production of the \tcl\
source code to be evaluated by \tcl.

\subsection{Callback handling}
Callbacks are \caml\ functions. They have to be invoked by \tcl, using the
\verb|camlcb| command. The link between \tcl\ and \caml\ is made through
a hash-table, associating an identifier (string) to a \caml\ function.

In order to avoid space leaks due to storing callback closures in the table,
we implement a memo-mechanism associating callbacks to widgets.
Except for class wide callbacks, each callback is defined for a given
widget. When this widget is destroyed, the callback cannot be invoked any
more, and we can remove it from the table. We thus provide a general binding
for all widgets, on event \verb|Destroy|, that will remove all callbacks
associated to a widget being destroyed.
If a user bindings masks the default binding for the \verb|Destroy| event,
its callback should call the function \verb|protocol__remove_callbacks| with
argument the widget being destroyed.
